% !TEX root = ../Diplombericht.tex
Das Testprotokoll dient dazu, die vordefinierten Testfälle zu prüfen. Die Testergebnisse dienen zur Abnahme des Systems und können das weitere Vorgehen beinflussen.

\subsection{Testfälle}
Diee Testfälle wurden im Testkonzept definiert und sind dort zu entnehmen.

\subsection{Testhilfsmittel}
Folgedene Hilfsmittel wurden für die Tests eingesetzt.

\begin{table}[H]
\centering
\begin{tabular}{p{6cm}p{10cm}}
\hline
\cellcolor{heading}\textbf{Testhilfsmittel Name} & Test Notebook \\\hline
\cellcolor{heading}\textbf{Gerät} & Dell XPS 13 \\\hline
\cellcolor{heading}\textbf{CPU} & Intel Core i7-5600U, 2.60GHz 4 cores \\\hline
\cellcolor{heading}\textbf{RAM} & 8GB \\\hline
\cellcolor{heading}\textbf{Betriebssystem} & Fedora Release 27, 64-Bit \\\hline
\cellcolor{heading}\textbf{Installierte Software / Pakete} & nmap, bind-utils, Google Chrome, Version 63.0.3239.132  \\\hline
\end{tabular}
\caption{Testhilfsmittel Notebook}
\end{table}

Falls das Test Notebook ausfallen sollte, wäre ein zweites Testgerät bereit.
\begin{table}[H]
\centering
\begin{tabular}{p{6cm}p{10cm}}
\hline
\cellcolor{heading}\textbf{Testhilfsmittel Name} & Test PC \\\hline
\cellcolor{heading}\textbf{Gerät} & Eigenbau \\\hline
\cellcolor{heading}\textbf{CPU} & Intel Core i7-4770, 3.40GHz 8 Cores \\\hline
\cellcolor{heading}\textbf{RAM} & 32GB \\\hline
\cellcolor{heading}\textbf{Betriebssystem} & Microsoft Windows 10 Education, Version 10.0.16299 Build 16299 \\\hline
\cellcolor{heading}\textbf{Installierte Software} & Zenmap, Google Chrome, Version 66.0.3359.139  \\\hline
\end{tabular}
\caption{Testhilfsmittel PC}
\end{table}
