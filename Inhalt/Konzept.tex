% !TEX root = ../Diplombericht.tex
\section{Konzept} 
\label{sec:Konzept}
Das Konzept beschreibt das Vorgehen und Vorhaben der Realisierung. 

\subsection{Systemlandschaft}
\subsubsection{Grafische Übersicht}
\begin{figure}[htb]
\centering
\includegraphics[scale=0.6]{Systemlandschaft.jpg}
\caption{Systemlandschaft}
\label{fig:Systemlandschaft}
\end{figure} 

\begin{table}[H]
\centering
\begin{tabular}{p{1cm}p{5cm}p{5cm}p{5cm}}
\hline
\rowcolor{heading} \textbf{Nr.} & \textbf{Protokoll} & \textbf{Protokollfamilie} & Ports \\\hline
1 & SSH & TCP & 22 \\\hline
2 & SMB & TCP & 445 \\\hline
3 & DHCP & UDP & 67 / 78 \\\hline
4 & TFTP & UDP & 69 \\\hline
5 & HTTP & TCP & 80 \\\hline
\end{tabular}
\caption{Systemlandschaft}
\end{table}


\subsection{Design der Lösung}
\begin{itemize}
	\item Grob- und Detaildesign, Prozesse, Abläufe etc. erstellen. Alle Zusaamenhänge müssen nachvollziehbar und transparent sein!
\end{itemize}

