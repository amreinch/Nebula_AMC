% !TEX root = ../Projektdokumentation.tex
\section{Studie} 
\label{sec:Studie}
Hier werden Grundsatzentscheide gefällt, es werden keine Details beschrieben!
\begin{itemize}
	\item Informationsbeschaffung
	\item Anforderungskatalog
	\item Schwachstellenkatalog
	\item Pflichtenheft
	\item Analyse welche Varianten möglich sind
	\item Evaluation der möglichen Varianten
	\item Entscheid mit welcher Varianten das Konzept realisiert wird
	\item Wirtschaftlichkeit
\end{itemize}


\subsubsection{Anforderungskatalog}
Es wurden folgende Anforderungen an die Cluster Software gestellt.

\begin{itemize}
\item Die Software soll auf einer 64 Bit Architektur betrieben werden können.
\item Es müssen stabile Releases vorhanden sein
\item Die Software soll eine skalierbare Funktion beinhalten.
\item Die Software soll die CPU Ressourcen für Berechnungsaufgaben zu mindestens 90\% auslasten.
\item Es sollen Monitoring Tools bereits in der Softwarelösung vorhanden sein
\item Es sollen Logging Tools bereits in der Softwarelösung vorhanden sein
\item Die Softwarelösung soll mit einem Netzwerkboot zurecht kommen.
\item Die Software muss gratis sein.
\item Die Installation soll einfach und nachvollziehbar sein.
\item Es sollen mehrere Befehlssatzarchitekturen unterstützt werden.
\item Die Softwarelösung soll eine Community für Fragen anbieten.
\item Der Letzte Release soll nicht lange zurückliegen.
\item Die Software soll bereits einen bekanntheitsgrad vorweisen.
\end{itemize}



