% !TEX root = ../Diplombericht.tex
\subsection{Situationsanalyse}
Für das Schürfen von Kryptowährungen werden folgende Komponenten eingesetzt:
\begin{table}[H]
\begin{tabular}[t]{p{0.5cm}p{0.8cm}p{6.7cm}p{6.7cm}}
\hline
\rowcolor{heading}\textbf{Nr.} & \textbf{Typ} & \textbf{Komponente} & \textbf{Modell \/Version} \\\hline
1 & HW & Prozessor & Intel Core i7-4700, 3.40 GHz Quad Core \\\hline
2 & HW & Grafikkarte & NVIDIA GeForce GTX 1070 Ti  \\\hline
3 & HW & Festplatte & TOSHIBA DT01ACA200  \\\hline
4 & SW & Schürf-Software & Minergate, Version 7.2  \\\hline
5 & SW & Betriebssystem & Windows 10 EDU, Version 1709 \\\hline
\end{tabular}
\caption{Situationsanalyse Komponenten}
\end{table}


\textbf{Legende:} HW = Hardware, SW = Software

\subsubsection{Stärken}
\begin{table}[H]
\begin{tabular}[t]{p{0.5cm}p{4.1cm}p{10.1cm}}
\hline
\rowcolor{heading}\textbf{Nr.} & \textbf{Kategorie} & \textbf{Beschreibung} \\\hline
1 & Bedienbarkeit & Das Schürfen der Währungen kann über eine grafische Benutzeroberfläche (GUI) gestartet werden. \\\hline
2 & Wartung & Es existieren keine Umsysteme  \\\hline
\end{tabular}
\caption{Situationsanalyse Stärken}
\end{table}

\subsubsection{Schwächen}
\begin{table}[H]
\begin{tabular}[t]{p{0.5cm}p{4.1cm}p{10.1cm}}
\hline
\rowcolor{heading}\textbf{Nr.} & \textbf{Kategorie} & \textbf{Beschreibung} \\\hline
1 & Flexibilität & Während des Schürfens ist der Computer für andere Tätigkeiten blockiert. \\\hline
2 & Kosten & Die Betriebskosten sind höher als der Ertrag.  \\\hline
3 & Betriebszeit & Es können nicht durchgehend Kryptowährungen geschürft werden.  \\\hline
\end{tabular}
\caption{Situationsanalyse Schwächen}
\end{table}