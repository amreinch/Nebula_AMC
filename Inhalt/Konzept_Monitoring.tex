% !TEX root = ../Diplombericht.tex
\subsection{Monitoring}
Als Monitoring Lösungen werden die Applikationen Nagios und Ganglia eingesetzt. Die Einsatzgebiete sind wiefolgt definiert:
\begin{itemize} 
\item {Nagios = Service Monitoring}
\item {Ganglia = Performance Monitoring}
\end{itemize}

\subsubsection{Service Monitoring - Nagios}
Sämtliche Service Tests werden vom Managementnode aus, automatisiert in definierten Intervallen ausgeführt.  Fehlgeschlagene Tests, sowie Statusänderungen der Überwachungsstatis generieren eine Benachrichtigungs E-Mail welche an den Systemadministrator versendet wird. Nagios ist über den Browser via http://nebula/nagios zu erreichen.\newline

Die folgenden Überwachungen sollen während der Realisierungsphase implementiert werden. Weitere Überwachungen können nach dem Projektabschluss implementiert werden.


\begin{table}[H]
\begin{tabular}[t]{p{0.6cm}p{2.5cm}p{2.2cm}p{1.5cm}p{8.8cm}}
\hline
\rowcolor{heading}\textbf{Nr.} & \textbf{Überwachung} & \textbf{Schweregrad} & \textbf{Intervall} &\textbf{Beschreibung} \\\hline
1 & Erreichbarkeit & Kritisch & 5 & Es wird mittels Ping eine Statusüberprüfung der Nodes durchgeführt \\\hline
2 & SSH Zugriff & Mittel & 60 & Die Zugriffe auf die Computenodes sollen über den Managementnode stattfinden  \\\hline
3 & CPU Last & Hoch & 5 & Die CPU's ständig ausgelastet sein  \\\hline
\end{tabular}
\caption{Service Monitoring}
\end{table}

\subsubsection{Performance Monitoring - Ganglia}
Die Ganglia Applikation ist auf dem Managementnode installiert und kommuniziert mit den Ganglia Daemons auf den Computenodes. Dabei werden die übermittelten Daten als Grafen dargestellt. Ganglia ist über http://nebula/ganglia aufrufbar.


