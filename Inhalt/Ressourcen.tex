% !TEX root = ../Projektdokumentation.tex
\subsection{Ressourcen}

\subsubsection{Budget}
Dem privaten Projekt steht ein Budget von 3'250 CHF zu. Die Aufwände werden hierbei nicht berücksichtigt, da keine Löhne bezahlt werden müssen.
\begin{table}[H]
\centering
\begin{tabular}[t]{p{1cm}p{10cm}p{5cm}}
\hline
\rowcolor{heading}\textbf{Nr.} & \textbf{Verwendungszweck} &\hfill \textbf{Budget in CHF} \\\hline
1 & Beschaffungen & \hfill 3'000.00 \\\hline
2 & Apéro & \hfill 150.00 \\\hline
3 & Drucken \& Binden &\hfill 100.00 \\\hline
\textbf{} & \textbf{Total} &\hfill \textbf{3'250.00}  \\\hline
\end{tabular}
\caption{Projektbudget}
\end{table}

\subsubsection{Sachmittel}
Die aufgelisteten Komponenten werden für die Lösung benötigt.

\begin{table}[H]
\centering
\begin{tabular}[t]{p{1cm}p{1.2cm}p{6.9cm}p{6.9cm}}
\hline
\rowcolor{heading}\textbf{Nr.} & \textbf{Anzahl} & \textbf{Komponenten} & \textbf{Modell / Spezifikationen}\\\hline
1 & 40 & Mini Computer & Raspberry PI 3 Model B+\\\hline
2 & 1 & Schaltnetzteil & RSP-750-5, Mean Well\\\hline
3 & 1 & USB zu TTL Serial-Kabel & Adafruit USB zu TTL Seriel Kabel, 75cm \\\hline
4 & 40 & Ethernetkabel & FTP Cat.5e Patchkabel \\\hline
5 & 1 & Switch & TL-SL3452 48-Port 10/100, TP-Link \\\hline
6 & 1 & Datenspeicher & Synology NAS DS216\\\hline
7 & * & Diverse Kabel \& Befestigungsmaterialien & *\\\hline
\end{tabular}
\caption{Sachmittel}
\end{table}
* Anzahl und Hersteller unbekannt. Die Artikel wurden in lokalen Baumärkten eingekauft.