% !TEX root = ../Diplombericht.tex

\subsection{Kosten}

\subsubsection{Einmalige Kosten}
\textbf{Beschaffungskosten}
\begin{table}[H]
\centering
\begin{tabular}{p{2cm}p{5cm}p{4cm}p{4cm}}
\hline
\rowcolor{heading} \textbf{Anzahl} & \textbf{Komponente} & \hfill \textbf{Stückpreis(CHF)} & \hfill \textbf{Gesamtwert(CHF)} \\\hline
40 & Raspberry PI Model B & \hfill 33.00 & \hfill 1'320.00 \\\hline
1 & Schaltnetzteil & \hfill  229.00 & \hfill 229.00 \\\hline
1 & TTL Serial Kabel & \hfill 30.00 & \hfill 30.00 \\\hline
40 & Patchkabel Cat. 5e & \hfill 1.00 & \hfill 40.00 \\\hline
1 & TP-Link Switch & \hfill 220.00 & \hfill 220.00 \\\hline
1 & Synology NAS DS216 & \hfill 600.00 & \hfill 600.00 \\\hline
1 & Diverse Kabel, Schrauben & \hfill 50.00 & \hfill 50.00 \\\hline
\rowcolor{subheading}\textbf{-} & \textbf{Total} & \hfill \textbf{-} & \hfill \textbf{2'489.00} \\\hline
\end{tabular}
\caption{Beschaffungskosten}
\end{table}

\textbf{Aufwandskosten}
\begin{table}[H]
\centering
\begin{tabular}{p{2cm}p{5cm}p{4cm}p{4cm}}
\hline
\rowcolor{heading} \textbf{Stunden} & \textbf{Phase} & \textbf{Stundenansatz(CHF)} &\hfill \textbf{Gesamtkosten(CHF)} \\\hline
30 & Initialisierung & \hfill 120.00 & \hfill 3'600.00 \\\hline
41 & Konzept & \hfill 120.00 & \hfill  6'000.00 \\\hline
171 & Realisierung & \hfill 120.00 & \hfill  17'040.00 \\\hline
25.5 & Einführung & \hfill 120.00 & \hfill 2'640.00 \\\hline
25 & Periodische Arbeiten & \hfill 120.00 & \hfill  3'000.00 \\\hline
1.5 & Reserve / Administration & \hfill 120.00 & \hfill 3'000.00 \\\hline
\rowcolor{subheading}\textbf{294} & \textbf{Total} &\hfill  \textbf{120.00} & \hfill \textbf{35'280.00} \\\hline
\end{tabular}
\caption{Aufwandskosten}
\end{table}

\subsubsection{Betriebskosten (repetitiv)}

Folgende Voraussetzungen sind für die folgenden Berechnungen definiert:
\begin{itemize}
	\item 1000 Watt die Stunde kostet 0.2894 CHF
	\item Ein Monat hat 30 Tage
	\item Ein Jahr hat 360 Tage
\end{itemize}

\textbf{Wartungskosten}
\newline
Pro Monat sind 3 Stunden Wartungsaufwand einzuberechnen, dadurch ergeben sich mit dem definierten Stundenansatz jährliche Wartungskosten von \textbf{4'320.00 CHF}. \newline
\newline
\textbf{Stromkosten}\newline
Der Strom wird durch die BKW über den Vertrag Energy Blue bezogen.

\begin{table}[H]
\centering
\begin{tabular}{p{1.5cm}p{2cm}|p{2.75cm}p{2.75cm}p{2.75cm}p{2.75cm}}
\hline
\rowcolor{heading} \textbf{Anzahl} & \textbf{Leistung} & \multicolumn{4}{l}{\textbf{Kosten in CHF}} \\\hline
\rowcolor{subheading} \textbf{RPI} & \textbf{kW} & \textbf{Stunde} &\textbf{Tag} & \textbf{Monat} &\textbf{Jahr} \\\hline
40 & 0.4 & 0.11576 & 2.78 & 83.35 & 1'000.20 \\\hline
\end{tabular}
\caption{Stromkostenrechnung}
\end{table}

\subsubsection{Gesamtkosten}

\textbf{Jährliche Kosten} \newline
Das Projekt wird Anfang Juni abgeschlossen sein. Deshalb belaufen sich die Wartungs- und Stromkosten im 1. Jahr auf die Hälfte gegenüber den Folgejahren.
Der Cluster soll innerhalb von 3 Jahren gewinnbringend wirken. Die jährlichen sowie täglichen Kosten (in CHF) sind der nachfolgenden Tabelle zu entnehmen: 

\begin{table}[H]
\centering
\begin{tabular}{p{4cm}p{4cm}p{4cm}p{4cm}}
\hline
\rowcolor{heading} \textbf{Kostengrund} & \hfill \textbf{Kosten 1. Jahr} & \hfill\textbf{Kosten 2. Jahr} & \hfill \textbf{Kosten 3. Jahr}\\\hline
Beschaffung & \hfill 2'489.00 &\hfill - &\hfill - \\\hline
Aufwand & \hfill 35'280.00 & \hfill- & \hfill- \\\hline
Wartungskosten & \hfill 2'160.00 & \hfill 4'320.00 & \hfill 4'320.00 \\\hline
Stromkosten & \hfill 500.00 & \hfill 1'000.00 & \hfill 1'000.00 \\\hline
\rowcolor{subheading}\textbf{Total} & \hfill \textbf{40'429.00} & \hfill \textbf{45'749.00} & \hfill \textbf{51'069.00} \\\hline
\end{tabular}
\caption{Gesamtkosten}
\end{table}

\textbf{Tägliche Kosten}\newline
Auf 3 Jahre ausgerechnet, muss täglich ein Ertrag von 47.29 CHF erwirtschaftet werden, um die investierten Aufwände und Kosten zu decken.

