% !TEX root = ../Diplombericht.tex

\section{Kosten}

\subsection{Einmalige Kosten}
\textbf{Beschaffungskosten}
\begin{table}[H]
\centering
\begin{tabular}{p{2cm}p{5cm}p{4cm}p{4cm}}
\hline
\rowcolor{heading} \textbf{Anzahl} & \textbf{Komponente} & \textbf{Stückpreis(CHF)} &\textbf{Gesamtwert(CHF)} \\\hline
40 & Raspberry PI Model B+ & 33.00 & 1320.00 \\\hline
1 & Schaltnetzteil & 229.00 & 229.00 \\\hline
1 & Midi Tower & 80.00 & 80.00 \\\hline
1 & TTL Serial Kabel & 30.00 & 30.00 \\\hline
40 & Pazchkabel Cat. 5e & 1.00 & 40.00 \\\hline
1 & TP-Link Switch & 220.00 & 220.00 \\\hline
1 & Synology NAS DS216 & 600.00 & 600.00 \\\hline
1 & Diverse Kabel, Schrauben & 50.00 & 50.00 \\\hline
\rowcolor{subheading}\textbf{-} & \textbf{Total} & \textbf{-} & \textbf{2'569.00} \\\hline
\end{tabular}
\caption{Beschaffungskosten}
\end{table}

\textbf{Aufwandskosten}
\begin{table}[H]
\centering
\begin{tabular}{p{2cm}p{5cm}p{4cm}p{4cm}}
\hline
\rowcolor{heading} \textbf{Stunden} & \textbf{Phase} & \textbf{Stundenansatz(CHF)} &\textbf{Gesamtkosten(CHF)} \\\hline
30 & Initialisierung & 120.00 & 3'600.00 \\\hline
50 & Konzept & 120.00 & 6'000.00 \\\hline
142 & Realisierung & 120.00 & 17'040.00 \\\hline
22 & Einführung & 120.00 & 2'640 \\\hline
25 & Periodische Arbeiten & 120.00 & 3'000.00 \\\hline
25 & Reserve & 120.00 & 3'000.00 \\\hline
\rowcolor{subheading}\textbf{294} & \textbf{Total} & \textbf{120.00} & \textbf{35'280.00} \\\hline
\end{tabular}
\caption{Aufwandskosten}
\end{table}

\subsection{Betriebskosten (Repetitiv)}

Folgende Voraussetzungen sind für die folgenden Berechnungen definiert.
\begin{itemize}
	\item 1000 Watt die Stunde kosten 0.2894 CHF.
	\item Ein Monat hat 30 Tage
	\item Ein Jahr hat 360 Tage
\end{itemize}

\textbf{Wartungskosten}
\newline
Pro Monat sind 3 Stunden Wartungsaufwand einzuberechnen, dadurch ergeben sich mit dem definierten Stundenansatz jährliche Wartungskosten von \textbf{4'320.00 CHF}. \newline
\newline
\textbf{Stromkosten}
Der Strom wird durch die BKW über den Vertrag Energy Blue bezogen.

\begin{table}[H]
\centering
\begin{tabular}{p{1.5cm}p{2cm}|p{2.75cm}p{2.75cm}p{2.75cm}p{2.75cm}}
\hline
\rowcolor{heading} \textbf{Anzahl} & \textbf{Leistung} & \multicolumn{4}{l}{\textbf{Kosten in CHF}} \\\hline
\rowcolor{subheading} \textbf{RPI} & \textbf{kW} & \textbf{Stunde} &\textbf{Tag} & \textbf{Monat} &\textbf{Jahr} \\\hline
40 & 0.4 & 0.11576 & 2.78 & 83.35 & 1000.17 \\\hline
\end{tabular}
\caption{Stromkostenrechnung}
\end{table}

\subsection{Gesamtkosten}

\textbf{Jährliche Kosten} \newline
Das Projekt ist Anfangs Juni abgeschlossen. Deshalb belaufen sich die Wartungs- und Stromkosten auf die Hälfte gegenüber den Folgejahren.
Der Cluster soll innerhalb von 3 Jahren gewinnbringend wirken. Die jährlichen sowie täglichen Kosten sind unten zu entnehmen. 

\begin{table}[H]
\centering
\begin{tabular}{p{4cm}p{4cm}p{4cm}p{4cm}}
\hline
\rowcolor{heading} \textbf{Kostengrund} & \textbf{Kosten 1.Jahr} & \textbf{Kosten 2.Jahr} & \textbf{Kosten 3.Jahr}\\\hline
Beschaffung & 2'569.00 & - & - \\\hline
Aufwand & 35'280.00 & - & - \\\hline
Wartungskosten & 2'160.00 & 4'320.00 & 4'320.00 \\\hline
Stromkosten & 500.00 & 1000.00 & 1000.00 \\\hline
\rowcolor{subheading}\textbf{Total} & \textbf{40'509.00} & \textbf{45'829.00} & \textbf{51'149.00} \\\hline
\end{tabular}
\caption{Gesamtkosten}
\end{table}

\textbf{Tägliche Kosten}

\begin{table}[H]
\centering
\begin{tabular}{p{4cm}p{4cm}}
\hline
\rowcolor{heading} & \textbf{3.Jahr}\\\hline
Kosten (CHF) & 53'809.00 \\\hline
Dauer (Tage) & 1080 \\\hline
\rowcolor{subheading}\textbf{Kosten pro Tag} & \textbf{47.36} \\\hline
\end{tabular}
\caption{tägliche Kosten}
\end{table}


