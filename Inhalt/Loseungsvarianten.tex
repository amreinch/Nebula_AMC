% !TEX root = ../Diplombericht.tex

\subsection{Lösungsvarianten}
\subsubsection{Variantenübersicht}

Die Informationen wurden über die folgenden Produkte gesammelt und zusammengestellt:

\begin{table}[H]
\centering
\begin{tabular}{p{1cm}p{4cm}p{11cm}}
\hline
\rowcolor{heading} \textbf{Nr.} & \textbf{Variante} & \textbf{Bezeichnung} \\\hline
01 & OpenHPC & HPC Lösung entwickelt von der Linux Foundation \\\hline
02 & TinyTitan & Open Source Lösung entwickelt von Oak Ridge Leadership Computing Facility \\\hline
03 & Minimale Lösung & Simple 32-Bit Architekturlösung \\\hline
\end{tabular}
\caption{Variantenübersicht}
\end{table}

\subsubsection{Variante V1 \flqq OpenHPC\frqq}

\textbf{Beschreibung}\newline
OpenHPC gilt als vorangeschrittenes OpenSource Projekt der Linux Foundation. Das Produkt steht in direkter Verbindung mit diversen grossen IT Unternehmen weltweit. Das Ziel der Linux Foundation ist es, durch OpenHPC eine kostengünstige sowie schnell zu installierende HPC-Umgebung aufzubauen. Durch viele zusätzliche OpenSource Tools rundet sich das Produkt ab und gilt als ernstzunehmender Konkurrent gegenüber kostenpflichtiger Software.

\textbf{Installation und Betrieb}\newline
Es existieren diverse Guides, Foren und Chats sowie eine E-Mail-Liste zu OpenHPC. Dadurch scheint die Unterstützung bei allfälligen Problemen vorhanden zu sein. Die Installationsanleitung, welche von der Linux Foundation geschrieben wurde, liest sich sehr gut und ist absolut ausreichend für die Installation. Der Betriebsaufwand wird als gering eingeschätzt, da es ein sehr ausgereiftes Produkt, welches stetig weiterentwickelt wird, ist.

\textbf{Voraussetzungen, Abhängigkeiten}\newline
Für die Cluster Software werden mindestens ein Masternode und 4 Computenodes vorausgesetzt. Das Betriebssystem bezieht sich hierbei auf ein CentOS7x. Jeder Computenode benötigt 2 Netzwerkschnittstellen. Das eine Interface wird für den Standard Ethernet Zugriff verwendet und das zweite Interface wird für die Kommunikation zu jedem BMC Host verwendet. Es werden zusätzliche Intel Bibliotheken benötigt. Dazu müssen Lizenzen für Parallel Studio XE von Intel besorgt werden. Die Lizenzen können mit einer offiziellen E-Mail-Adresse der Schule gratis bezogen werden. Die Linux Foundation erwähnt in ihrem Guide, dass sie die «Bring your own Licence» Strategie verfolgt.

\subsubsection{Variante V2 \flqq TinyTitan\frqq}
\textbf{Beschreibung}\newline
Das Produkt wurde von der Firma «Oak Ridge Facility» entwickelt. Die Software ist unter anderem für RPI’s entwickelt worden. TinyTitan wurde für das Durchführen wissenschaftlicher Berechnungen entworfen. Jedoch wurde seit geraumer Zeit an dem Produkt nicht mehr weitergearbeitet, wie dem offiziellen GitHub Repository zu entnehmen ist. Die Community selbst erweist sich ebenfalls als sehr klein. 

\textbf{Installation und Betrieb}\newline
Für die Installation des Produktes wird ein XServer vorausgesetzt, da empfohlen wird Tiny-Titan über ein GUI zu installieren. Der Installationsanleitung ist ebenfalls zu entnehmen, dass sich die Entwickler viele Gedanken über das Look a Like des Clusters gemacht haben, zum Beispiel wird ein Thema dem Einbinden von LED’s gewidmet. Die Installation findet ausschliesslich durch vordefinierte Scripts statt. Durch die kleine Community und nicht mehr gepflegte Software kann nichts über den Betriebsaufwand in Erfahrung gebracht werden.

\textbf{Voraussetzungen, Abhängigkeiten}\newline
Laut Guides werden lediglich 2 RPI’s benötigt.

\subsubsection{Variante V3 \flqq Minimale Lösung\frqq}
\textbf{Beschreibung}\newline
Die Minimale Installation ist eine zum Teil Eigenbau Lösung, welche sich nahe an diverse Guides aus dem Internet bezieht. Jedoch wird diese auf eigene Bedürfnisse angepasst.

\textbf{Installation und Betrieb}\newline
Da es bei dieser Lösung selber zu entscheiden gilt, was und wie die Lösung installiert und umgesetzt werden soll, kann während der Installation darauf geachtet werden, was den grössten Vorteil für den Betrieb danach mit sich bringt. Während dem Projekt soll die Installation aber klein gehalten werden und nur das nötigste wird umgesetzt. 

\textbf{Voraussetzungen, Abhängigkeiten}\newline
Es werden 2 RPI's benötigt.

\subsubsection{Anforderungsabdeckung der Varianten}

\begin{table}[H]
\centering
\begin{tabular}{p{1cm}p{2.5cm}p{2.2cm}p{10.3cm}}
\hline
\rowcolor{heading} \textbf{Nr.} & \textbf{Kriterium} & \textbf{Gewichtung} &\textbf{Begründung} \\\hline
1 & Installation & 40\% & Die Installation soll keine Hürden aufweisen, da der Zeitplan ansonsten nicht eingehalten werden kann. \\\hline
2 & Partner & 10\% & Je mehr Partner vorhanden sind, desto grösser und innovativer ist die Software. Die Software hat dadurch einen fixen Standpunkt auf dem Markt und wird weiterentwickelt. \\\hline
3 & Aktualität & 20\% & Fördert den LifeCycle und die Sicherheit der Cluster-Software. \\\hline
4 & Tools & 30\% & Mitgelieferte Tools. \\\hline
\end{tabular}
\caption{Anforderungsabdeckung}
\end{table}

\subsubsection{Notenskala der Kriterien}
\begin{table}[H]
\centering
\begin{tabular}{p{0.6cm}p{2.2cm}p{1.cm}p{12.2cm}}
\hline
\rowcolor{heading} \textbf{Nr.} & \textbf{Kriterium} & \textbf{Note} &\textbf{Begründung} \\\hline
1 & Installation & 0/3/5 & 5 = Kann gemäss Anleitung direkt installiert werden. \newline 3 = Veraltete Anleitung. Komplikationsprobleme möglich. \newline
0 = Keine Anleitung vorhanden.
 \\\hline
2 & Partner & 0/2 & 2 = Viele Partner vorhanden. \newline
0 = Keine Partner vorhanden \\\hline
3 & Aktualität & 0/3/5 & 5 = Releases in den letzten 2 Monaten. \newline 3 = Releases in den letzten 6 Monaten \newline 0 = Keine Releases seit einem Jahr. \\\hline
4 & Tools & 0/5 & 5 = Es werden Tools angeboten \newline 0 = Es werden keine Tools angeboten. \\\hline
\end{tabular}
\caption{Notenskala der Kriterien}
\end{table}

\subsection{Bewertung der Varianten}
\begin{table}[H]
\centering
\begin{tabular}{p{2cm}p{2cm}p{4cm}p{4cm}p{4cm}}
\hline
\rowcolor{heading} \textbf{Kriterium} & \textbf{Gewicht} & \textbf{Variante 1} & \textbf{Variante 2}& \textbf{Variante 3} \\\hline
Installation & 40\% & 5x40 = 200 & 4x40 = 160 & 5x40 = 200 \\\hline
Partner & 10\% & 2x10 = 20 & 0x10 = 0 & 0x10 = 0 \\\hline
Aktualität & 20\% & 5x20 = 100 & 0x20 = 0 & 5x20 = 100 \\\hline
Tools & 30\% & 5x30 = 150 & 0x30 = 0 & 0x30 = 0\\\hline
\textbf{Total} & \textbf{100\%} & \ \ \ \ \ \ \ \ \ \ \ \textbf{470} & \ \ \ \ \ \ \ \ \ \ \textbf{160} & \ \ \ \ \ \ \ \ \ \ \ \textbf{300} \\\hline
\end{tabular}
\caption{Bewertung der Varianten}
\end{table}

\subsection{Variantenentscheid}
Anhand der Bewertung wird empfohlen, die OpenHPC Lösung der Linux Foundation zu verwenden. Die Installation kann gemäss Anleitung in kürzester Zeit umgesetzt werden. Die Releases können mit kleinerem Aufwand installiert werden. Zudem runden die Möglichkeiten der Schnittstellen und Komponenten den Entscheid ab. Es ist möglich, Administrations- sowie Performance Monitoring Tools einzusetzen, welche mit der Lösung harmonieren. Als Hürde sehe ich die möglichen anfallenden Lizenzen und das zweite Netzwerk Interface, welches man für die Kommunikation unter den RPI’s benötigt.