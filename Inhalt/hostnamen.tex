% !TEX root = ../Diplombericht.tex
\subsection{Hostnamen}
Die Compute Node Namen wurden nach einem überschaulichen Konzept definiert. Jeder Compute Node trägt den Prefix "c". Dies soll bei der Behebung von Problemen auf physicher Ebene, z.B. Austauschen eines Nodes, dienen. Zudem werden alle Hostnamen immer in kleinen Buchstaben geschrieben. 

\subsubsection{Management Node Name}
\begin{table}[H]
\centering
\begin{tabular}{p{5cm}p{5.5cm}p{5.5cm}}
\hline
\rowcolor{heading} \textbf{Name} & \textbf{IP} & \textbf{MAC} \\\hline
nebula & 192.168.1.10 & B8:27:EB:32:A9:1C \\\hline
\end{tabular}
\caption{Management Node Name}
\end{table}

\subsubsection{Compute Node Namen}
\begin{longtable}{p{1cm}p{2cm}p{6cm}p{6cm}}
\hline
\rowcolor{heading} \textbf{Nr.} & \textbf{Name} & \textbf{IP} & \textbf{MAC} \\\hline
01 & c1 & 192.168.1.11 & B8:27:EB:32:39:A7\\\hline
02 & c2 & 192.168.1.12 & B8:27:EB:2E:A3:D1\\\hline
03 & c3 & 192.168.1.13 & B8:27:EB:50:45:3F\\\hline
04 & c4 & 192.168.1.14 & B8:27:EB:0D:E6:25\\\hline
05 & c5 & 192.168.1.15 & B8:27:EB:3E:96:B5\\\hline
06 & c6 & 192.168.1.16 & B8:27:EB:EE:77:DA\\\hline
07 & c7 & 192.168.1.17 & B8:27:EB:21:63:E6\\\hline
08 & c8 & 192.168.1.18 & B8:27:EB:2E:2E:CC\\\hline
09 & c9 & 192.168.1.19 & B8:27:EB:17:32:96\\\hline
10 & c10 & 192.168.1.20 & B8:27:EB:B2:1C:A9\\\hline
11 & c11 & 192.168.1.21 & B8:27:EB:AF:63:1F\\\hline
12 & c12 & 192.168.1.22 & B8:27:EB:43:00:2C\\\hline
13 & c13 & 192.168.1.23 & B8:27:EB:13:7B:18\\\hline
14 & c14 & 192.168.1.24 & B8:27:EB:43:CD:29\\\hline
15 & c15 & 192.168.1.25 & B8:27:EB:FF:C7:56\\\hline
16 & c16 & 192.168.1.26 & B8:27:EB:CE:98:66\\\hline
17 & c17 & 192.168.1.27 & B8:27:EB:5D:63:34\\\hline
18 & c18 & 192.168.1.28 & B8:27:EB:91:3E:0F\\\hline
19 & c19 & 192.168.1.29 & B8:27:EB:F4:65:EC\\\hline
20 & c20 & 192.168.1.30 & B8:27:EB:3E:AB:DC\\\hline
21 & c21 & 192.168.1.31 & B8:27:EB:66:60:F6\\\hline
22 & c22 & 192.168.1.32 & B8:27:EB:37:3F:74\\\hline
23 & c23 & 192.168.1.33 & B8:27:EB:18:5E:F0\\\hline
24 & c24 & 192.168.1.34 & B8:27:EB:B0:23:B8\\\hline
25 & c25 & 192.168.1.35 & B8:27:EB:BE:C4:94\\\hline
26 & c26 & 192.168.1.36 & B8:27:EB:FB:FF:57\\\hline
27 & c27 & 192.168.1.37 & B8:27:EB:4E:EC:CE\\\hline
28 & c28 & 192.168.1.38 & B8:27:EB:43:1C:35\\\hline
29 & c29 & 192.168.1.39 & B8:27:EB:DC:74:5F\\\hline
30 & c30 & 192.168.1.40 & B8:27:EB:D1:DE:2F\\\hline
31 & c31 & 192.168.1.41 & B8:27:EB:5E:90:34\\\hline
32 & c32 & 192.168.1.42 & B8:27:EB:DE:80:24\\\hline
33 & c33 & 192.168.1.43 & B8:27:EB:A4:79:6F\\\hline
34 & c34 & 192.168.1.44 & B8:27:EB:0A:4D:C7\\\hline
35 & c35 & 192.168.1.45 & B8:27:EB:5C:53:5F\\\hline
36 & c36 & 192.168.1.46 & B8:27:EB:F7:AF:C2\\\hline
37 & c37 & 192.168.1.47 & B8:27:EB:CE:BA:ED\\\hline
38 & c38 & 192.168.1.48 & B8:27:EB:59:38:3C\\\hline
39 & c39 & 192.168.1.49 & B8:27:EB:99:BB:8E\\\hline
40 & c40 & 192.168.1.50 & B8:27:EB:8F:7A:0D\\\hline
\caption{Compute Node Namen}
\end{longtable}


\subsubsection{Reserve Node Name}
Die Reservenodes sind als Fallback für ausgefallene Compute Nodes vorgesehen.
\begin{table}[H]
\centering
\begin{tabular}{p{1cm}p{2cm}p{6cm}p{6cm}}
\hline
\rowcolor{heading} \textbf{Nr.} & \textbf{Name} & \textbf{IP} & \textbf{MAC} \\\hline
1 & c41 & 192.168.1.51 & B8:27:EB:DE:C9:69 \\\hline
2 & c42 & 192.168.1.52 & B8:27:EB:7E:6F:48 \\\hline
3 & c43 & 192.168.1.53 & B8:27:EB:5D:DD:FE \\\hline
4 & c44 & 192.168.1.54 & B8:27:EB:A6:6D:4D \\\hline
5 & c45 & 192.168.1.55 & B8:27:EB:0C:63:10 \\\hline
\end{tabular}
\caption{Reserve Node Namen}
\end{table}