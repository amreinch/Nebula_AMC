% !TEX root = ../Diplombericht.tex

\section{Schlussbetrachtung} 
\label{sec:Schlussbetrachtung}
Der Cluster kann parallele Berechnungen durchführen. Leider ist es nicht möglich, Kryptowährungen mit nur einem Prozess über den Cluster hinweg zu schürfen. Jedoch ergibt es hierbei keinen Unterschied, ob jeweils neue Prozesse pro Node gestartet werden oder ob diese in einen Prozess zusammengeführt sind. Dies würde erst eine Wichtigkeit erlangen, wenn es darum ginge, Währungen zu schürfen, welche in den Blockchains diverse Bountys bereithalten, welche Belohnungen für das Schürfen beinhalten.

\subsection{Abweichungen}
Leider konnte keine Software installiert werden, welche die Logdateien der Komponenten sammelt und zur Analyse bereitstellt. Dazu haben die Raspberry PI's schlichtweg zu wenig RAM-Ressourcen. Ich habe jedoch ein Compute Modul für ein Raspberry PI bestellt und werde es nach dem Projekt versuchen, zu installieren. Der Flaschenhals bei der Installation liegt darin, dass die Verwendung von Elastic Search unumgänglich ist und dafür die RAM-Ressourcen nicht ausreichen.

\subsection{Arbeiten nach dem Projekt}
Generell gilt es den Cluster in Betrieb zu nehmen und Kryptowährungen nach dem Projektabschluss zu schürfen. Die festgestellten Mangel werden innerhalb von 6 Monaten beseitigt. Zudem soll die Stromversorgung optimiert werden. Die Compute Nodes sollen erst den Strom erhalten, nachdem der Management Node gestartet ist. Dieser soll dann über einen Stromstoss-Schalter das Signal an das Netzeil der Compute Nodes senden. Weiterhin gilt es ein zentrales Log Analyse Tool zu installieren, welches bei der Analyse von Problemen Zeit einsparen soll.

\subsection{Persönliche Betrachtung}
Generell ist es mir gelungen, eine grössere Anzahl von verschiedenen Komponenten zu einem einheitlichen Produkt zu verbinden. Dadurch habe ich nun private CPU Ressourcen, welche abgekoppelt von meinem PC sind. Den Cluster kann ich für meine Vorhaben in naher Zukunft benutzen und ich erhoffe mir durch Spekulationen beim Schürfen von Kryptowährungen ein zusätzliches Einkommen.
 
\subsection{Danksagung}
An dieser Stelle möchte ich mich speziell bei den unten aufgeführten Personen für die Unterstützung meiner Diplomarbeit bedanken:

\textbf{Monika Amrein}\newline
Vielen Dank für die Überprüfung der Satzstellungen und das Korrigieren der Schreibfehler.\newline
\textbf{Stefan Räz}\newline
Vielen Dank für die Beratung der Stromversorgung der Compute Nodes.

\newpage
\section{Authentizität}
Mit meiner Unterschrift bestätige ich, die vorliegende Diplomarbeit selbstständig, ohne Hilfe Dritter und nur unter Benutzung der angegebenen Quellen ohne Copyright-Verletzung, erstellt zu haben.

Schüpfen, 01.06.2018\newline
\newline
\newline
Christoph Amrein

