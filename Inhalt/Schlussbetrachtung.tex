% !TEX root = ../Projektdokumentation.tex
\section{Schlussbetrachtung} 
\label{sec:Schlussbetrachtung}
Der Cluster kann parallele Berechnungen durchführen. Jedoch ist es nicht möglich Kryptowährungen mit nur einem Prozess über den Cluster hinweg zu schürfen. Jedoch ergibt es hierbei keinen unterschied ob jeweils neue Prozesse pro Node gestartet sind, oder ob diese in einen Prozess zusammengeführt sind. Dies würde erst eine Wichtigkeit erlangen, wenn es darum geht Währungen zu schürfen welche in den Blockchains diverse Bountys beinhalten, welche Belohnungen für das schürfen beinhalten.

\subsection{Arbeiten nach dem Projekt}
Der Cluster ist nicht rentabel und es ist vorgesehen das Anwendungsgebiet zu wechseln, dabei soll der Cluster als Entwicklerumgebung für Webanwendungen eingesetzt werden. Worauf gleichzeitig eine Virtualisierung stattfinden soll.

\subsection{Persönliche Betrachtung}
Generell ist es mir gelungen eine grössere Anzahl von verschiedenen Komponenten zu einem einheitlichen Produkt zu verbinden. Dadurch habe ich nun private CPU Ressourcen welche abgekoppelt von meinem PC sind. Das Produkt kann ich für meine nächsten persönlichen Vorhaben weiterhin benutzen und muss mir keine Webserver mieten.
 
\subsection{Danksagung}
An dieser Stelle möchte ich mich speziell den unten aufgeführten Personen für die Unterstützung meiner Diplomarbeit bedanken:

\begin{itemize} 
	\item Vielen Dank für die Überprüfung der Satzstellungen und das Korrigieren der Schreibfehler
 \end{itemize}
 \begin{itemize} 
	\item Vielen Dank für das anpassen der Schürfsoftware, welches mir ermöglicht Satistiken einzelner Nodes besser auszulesen.
 \end{itemize}

\section{Authentizität}
Mit meiner Unterschrift bestätige ich, die vorliegende Diplomarbeit selbstständig, ohne Hilfe Dritter und nur unter Benutzung der angegebenen Quellen ohne Copyright-Verletzung, erstellt zu haben.

Schüpfen, 27.05.2018\newline
\newline
\newline
Christoph Amrein

