% !TEX root = ../Diplombericht.tex
\section{Schlussbetrachtung} 
\label{sec:Schlussbetrachtung}
Der Cluster kann parallele Berechnungen durchführen. Leider ist es nicht möglich, Kryptowährungen mit nur einem Prozess über den Cluster hinweg zu schürfen. Jedoch ergibt es hierbei keinen Unterschied, ob jeweils neue Prozesse pro Node gestartet werden oder ob diese in einen Prozess zusammengeführt sind. Dies würde erst eine Wichtigkeit erlangen, wenn es darum ginge, Währungen zu schürfen, welche in den Blockchains diverse Bountys bereithalten, welche Belohnungen für das Schürfen beinhalten.

\subsection{Arbeiten nach dem Projekt}
Es gibt noch kleinere Bugs, welche behoben werden müssen. Diese verhindern aber die Inbetriebnahme des Clusters nicht. Zudem wird nach einer Virtualisierungssoftware gesucht, welche es ermöglicht, verschiedene Instanzen für andere Anwendungsgebiete in Betrieb zu nehmen.

\subsection{Persönliche Betrachtung}
Generell ist es mir gelungen, eine grössere Anzahl von verschiedenen Komponenten zu einem einheitlichen Produkt zu verbinden. Dadurch habe ich nun private CPU Ressourcen, welche abgekoppelt von meinem PC sind. Das Produkt kann ich für meine nächsten persönlichen Vorhaben weiterhin benutzen und muss mir keine Webserver mieten.
 
\subsection{Danksagung}
An dieser Stelle möchte ich mich speziell bei den unten aufgeführten Personen für die Unterstützung meiner Diplomarbeit bedanken:

\textbf{Monika Amrein}\newline
Vielen Dank für die Überprüfung der Satzstellungen und das Korrigieren der Schreibfehler.\newline
\textbf{Stefan Räz}\newline
Vielen Dank für die Beratung der Stromversorgung der Compute Nodes.

\newpage
\section{Authentizität}
Mit meiner Unterschrift bestätige ich, die vorliegende Diplomarbeit selbstständig, ohne Hilfe Dritter und nur unter Benutzung der angegebenen Quellen ohne Copyright-Verletzung, erstellt zu haben.

Schüpfen, 27.05.2018\newline
\newline
\newline
Christoph Amrein

