% !TEX root = ../Diplombericht.tex

\subsection{Wirtschaftlichkeit}
Betreffend der Wirtschaftlichkeit des Produktes ist auf folgendes zu achten.

\subsubsection{Infrastruktur}
Es wurde darauf geachtet, dass die Komponenten durch eine zentrale Stelle versorgt werden. Dabei werden die Raspberry PI's mit nur einem Netzteil versorgt und das Betriebssystem wird über das Netzwerk verteilt welches Speicherkarten einspart.

\begin{table}[H]
\centering
\begin{tabular}{p{2cm}p{5cm}p{4cm}p{4cm}}
\hline
\rowcolor{heading} \textbf{Anzahl} & \textbf{Komponente} & \textbf{Stückpreis in CHF} &\textbf{Gesamtwert in CHF} \\\hline
\rowcolor{subheading}\multicolumn{3}{l}{\textbf{Standardlösung}} & \textbf{700.00} \\\hline
4 & USB-HUB 10 Ports & 35.00 & 140.00 \\\hline
40 & Mini-USB Kabel & 6.00 & 240.00 \\\hline
40 & MicroSD Karte & 8.00 & 320.00 \\\hline
\rowcolor{subheading}\multicolumn{3}{l}{\textbf{Projektlösung}} & \textbf{268.00} \\\hline
1 & Netzteil & 230.00 & 230.00 \\\hline
1 & MicroSD Karte & 8.00 & 8.00 \\\hline
- & Diverse Stromkabel & 30.00 & 30.00 \\\hline
\rowcolor{subheading}\multicolumn{3}{l}{\textbf{Differenz der Lösungen}} & \cellcolor{asparagus}\textbf{432.00} \\\hline
\end{tabular}
\caption{Wirtschaftlichkeit Hardware}
\end{table}

Durch die vorgesehene Hardwarelösung können 432.00 CHF eingespart werden. Dies entspricht einer Einsparung von \textbf{261\%}.

\subsubsection{Return of Investment}

