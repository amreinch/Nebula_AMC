% !TEX root = ../Diplombericht.tex

\subsection{Wirtschaftlichkeit}
Das Projekt wird für den privaten Nutzen und aus eigenem Interesse aufgebaut. Aus diesem Grunde ist die Wirtschaftlichkeit kein Kernpunkt des Clusters.

\subsubsection{Spekulation}
Das Ziel des Clusters ist es, täglich \textbf{30 CHF} zu erwirtschaften. Dieser Wert ist nicht deckungsgleich mit den täglichen Kosten, soll aber über Marktspekulationen gedeckt werden. Durch die volatilen Märkte sind Kursschwankungen in beide Richtungen möglich. Es ist jedoch davon auszugehen, dass die Währungen in Zukunft noch an Wert zulegen werden, sobald diese einmal als geltende Zahlungsmittel aufgenommen werden. Durch reine Betriebskosten ohne Spekulation \textbf{47.29 CHF} ergibt sich ein tägliches Defizit von \textbf{17.29 CHF}, welches einem Verlust von \textbf{36.56\%} entspricht.

\subsubsection{Infrastruktur}
Beim Projekt wird der Fokus der Wirtschaftlichkeit hauptsächlich auf den Aufbau gelegt. Hier gilt es, möglichst wenige überflüssige Komponenten zu benutzen. Es wurde darauf geachtet, dass die Komponenten durch eine zentrale Stelle versorgt werden. Dabei werden die Raspberry PI's mit nur einem Netzteil versorgt und das Betriebssystem wird über das Netzwerk verteilt, was Speicherkarten einspart.

\begin{table}[H]
\centering
\begin{tabular}{p{2cm}p{5cm}p{4cm}p{4cm}}
\hline
\rowcolor{heading} \textbf{Anzahl} & \textbf{Komponente} & \textbf{Stückpreis in CHF} &\textbf{Gesamtwert in CHF} \\\hline
\rowcolor{subheading}\multicolumn{3}{l}{\textbf{Standardlösung}} & \hfill \textbf{700.00} \\\hline
4 & Universal-Serial-Bus(USB)- HUB 10 Ports & \hfill 35.00 & \hfill 140.00 \\\hline
40 & Mini-USB Kabel & \hfill 6.00 & \hfill 240.00 \\\hline
40 & MicroSD Karten & \hfill 8.00 & \hfill 320.00 \\\hline
\rowcolor{subheading}\multicolumn{3}{l}{\textbf{Projektlösung}} & \hfill \textbf{268.00} \\\hline
1 & Netzteil & \hfill 230.00 & \hfill 230.00 \\\hline
1 & MicroSD Karte & \hfill 8.00 & \hfill 8.00 \\\hline
- & Diverse Stromkabel & \hfill 30.00 & \hfill 30.00 \\\hline
\rowcolor{subheading}\multicolumn{3}{l}{\textbf{Differenz der Lösungen}} & \cellcolor{asparagus}\hfill \textbf{432.00} \\\hline
\end{tabular}
\caption{Wirtschaftlichkeit Hardware}
\end{table}

Durch die vorgesehene Hardwarelösung können 432.00 CHF eingespart werden. Dies entspricht einer Einsparung von \textbf{261\%}.

