% !TEX root = ../Diplombericht.tex
\subsection{Cluster-Software Evaluation}
\subsubsection{Cluster Software Kriterien}
Es wurden drei Cluster Software Produkte evaluiert. Dabei mussten die Muss-Kriterien erfüllt werden, um in die Auswahl zu kommen. Diese Kriterien werden für den Entscheid der Software nicht berücksichtigt, grenzen aber die Auswahlmöglichkeit der Produkte ein.

\begin{table}[H]
\begin{tabular}[t]{p{0.7cm}|p{14cm}c}
\hline
\rowcolor{heading}\textbf{Nr.} & \textbf{Anforderung} & \textbf{Prio.} \\\hline
01 & Ist die Software HPC tauglich? & M \\\hline
02 & Kann das Produkt innerhalb des vorgesehenen Zeitraumes installiert werden? & M \\\hline
03 & Ist die Lösung skalierbar? &  M \\\hline
04 & Existieren Dokumentationen? & S \\\hline
05 & Kann Support beansprucht und bezogen werden? & S \\\hline
06 & Ist die Lösung benutzerfreundlich? & S \\\hline
07 & Existieren Verwaltungstools? & S \\\hline
08 & Fallen zusätzliche Kosten an? & S \\\hline
\end{tabular}
\caption{Software Kriterien}
\end{table}

\subsubsection{Informationsbeschaffung}
Es wurde nach einer Lösung gemäss der oben definierten Kriterien gesucht. Dabei bin ich auf einen Wikipedia Eintrag\footnote{\url{https://en.wikipedia.org/wiki/Comparison\_of\_cluster\_software}} gestossen, welcher die verschiedenen Cluster Software Angebote auflistet. Dabei wurde nur nach einer HPC Lösung gefiltert. Durch diese Analyse hat sich die OpenHPC Lösung der Linux Foundation herauskristalisiert. Weiterhin wurden Suchbegriffe wie \grqq HPC Raspberry PI\grqq \ über Suchmaschinen eingegeben, da die Compute Nodes des Clusters Raspberry PI's sein sollen. Den Artikel über TinyTitan\footnote{\url{http://www.hpctoday.com/best-practices/tinytitan-a-raspberry-pi-computing-based-cluster/}} habe ich als interessant erachtet und deshalb genauer untersucht.
Mit einer weiteren Suche (hpc cluster software raspberry) bin ich auf einen Guide\footnote{\url{http://thundaxsoftware.blogspot.ch/2016/07/creating-raspberry-pi-3-cluster.html}} gestossen, der relativ simpel aussieht und einfach umzusetzen ist. Es gab durchaus noch weitere Guides und Softwarelösungen, welche ich aber nach einer genaueren Analyse der Installationsanleitung verworfen habe, da diese mir zum Teil zu wenig Informationen lieferten. Während der Informationsbeschaffung wurden alle Installationsskripte und Anleitungen sorgfältig durchgelesen, um diese als mögliche Varianten zu empfehlen. Während der Informationsbeschaffung bin ich auf zwei Fachbegriffe Message Passing Interface (MPI) \& Simple Linux Utility for Resource Management (SLURM), welche meistens in Zusammenhang mit HPC stehen, gestossen. Diese musste ich ebenfalls noch in Erfahrung bringen.





