% !TEX root = ../Diplombericht.tex
\subsection{Physikalische Verbindungen}

\subsubsection{Stromversorgung Managementnode}
Der Managementnode wird über den Micro USB Anschluss mit Strom versorgt. Dabei muss darauf geachtet werden, dass ein mindest Strom von 2 Ampere und ein maximaler Strom von 2.5 Ampere fliesst. Zudem wird eine konstante Spannung von 5 Volt benötigt. Deshalb wird ein Netzteil mit einer Leistung von 12.5 Watt benötigt. Das Netzteil wird über eine Stromschiene an das Stromnetz angeschlossen.

\subsubsection{Computenodes}
Die Managementnodes werden über die GPIO Pins via Jumperkabel über ein gemeinsames Netzteil mit Strom versorgt . Da es sich hierbei um eine Anzahl von mindestens 45 Raspberry's handelt ist ein Netzteil mit einer Leistung von 500W vorgesehen. Das Netzteil wird über die Stromschiene an das Stomnetz angeschlossen.


\subsubsection{übrige Geräte}
Die übrigen Geräte werden über den herkömmlichen Weg mit Strom über eine Stromschiene versorgt.

\subsubsection{Netzwerkverbindungen}
Die folgenden Komponenten sind über den Switch in das lokale Netzwerk eingebunden, die nicht aufgelisteten Geräte werden direkt über Powerline oder WLAN mit dem Router verbunden.
\begin{itemize}
\item Managementnode
\item Computenodes
\item NAS
\end{itemize}


