% !TEX root = ../Diplombericht.tex
\subsection{Physikalische Verbindungen}

\subsubsection{Stromversorgung Management Node}
Der Management Node wird über den Micro USB Anschluss mit Strom versorgt. Dabei muss darauf geachtet werden, dass ein mindest Strom von 2 Ampere fliesst. Zudem wird eine konstante Spannung von 5 Volt benötigt. Deshalb wird ein Netzteil mit einer Leistung von 10 Watt verwendet. Das Netzteil wird über eine Stromschiene an das Stromnetz angeschlossen.

\subsubsection{Compute Nodes}
Die Compute Nodes werden über die General Purpose Input/Output (GPIO) Pins via Jumperkabel über ein gemeinsames Netzteil mit Strom versorgt. Da es sich hierbei um eine Anzahl von mindestens 45 Raspberry's handelt, ist ein Netzteil mit einer Leistung von 500W vorgesehen. Das Netzteil wird über die Stromschiene an das Stomnetz angeschlossen.


\subsubsection{Übrige Geräte}
Die übrigen Geräte werden über den herkömmlichen Weg mit Strom über eine Stromschiene versorgt.

\subsubsection{Netzwerkverbindungen}
Die folgenden Komponenten sind über den Switch in das lokale Netzwerk eingebunden. Die nicht aufgelisteten Geräte werden direkt über Powerline oder WLAN mit dem Router verbunden:
\begin{itemize}
\item Management Node
\item Compute Nodes
\item NAS
\end{itemize}


