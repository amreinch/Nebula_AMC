% !TEX root = ../Diplombericht.tex
\subsection{Mining}
Es soll die geforkte cpuminer-multi Version von tkinjo1985 verwendet werden. Diese ist mit ARM Prozessoren kompatibel und muss selbst kompiliert werden. Die Version kann von github.com bezogen werden: https://github.com/tkinjo1985/cpuminer-multi.git \newline
Die cpuminer-multi Version unterstützt alle gängigen Algorithmen welche für das Schürfen der Währungen mit CPU's benötigt werden.

\subsubsection{Algorithmen}
Die zu schürfenden Währungen benutzen folgende Algorithmen, welche von cpuminer-multi Unterstützt werden:
\begin{itemize}
\item CryptoNight
\item X11
\item scrypt
\end{itemize}


\subsubsection{Kryptowährungen}
Für das Projekt werden folgende Kryptowährungen berücksichtigt und müssen geschürft werden können.
\begin{table}[H]
\begin{tabular}[t]{p{0.6cm}p{7.65cm}p{7.35cm}}
\hline
\rowcolor{heading}\textbf{Nr.} & \textbf{Währung} & \textbf{Algorithmus} \\\hline
1 & Verium & scrypt \\\hline
2 & AEON & CryptoNight \\\hline
3 & Monero & CryptoNight & 5 & Die CPU's ständig ausgelastet sein  \\\hline
\end{tabular}
\caption{Service Monitoring}
\end{table}

\subsubsection{Performance Monitoring - Ganglia}
Die Ganglia Applikation ist auf dem Managementnode installiert und kommuniziert mit den Ganglia Daemons auf den Computenodes. Dabei werden die übermittelten Daten als Grafen dargestellt. Ganglia ist über http://nebula/ganglia aufrufbar.


