% !TEX root = ../Diplombericht.tex
\subsection{Mining}
Die Kryptowährungen werden über die Miningpools von Minergate.com geschürft. Dafür wird die cpuminer Version von tkinjo1985 verwendet. Diese Version unterstützt die ARMv8 Prozessoren und bietet alle gängigen Algorithmen für das Schürfen der Währungen an.
Zudem werden nur Währungen geschürft, welche auf Börsen resp. Märkten gehandelt werden können.


\subsubsection{Kryptowährungen}
Folgende Kryptowährungen werden über die Minergate Pools mit dem CryptoNight Algorithmus geschürft:

\begin{table}[H]
\centering
\begin{tabular}{p{1cm}p{3.5cm}p{2cm}p{9.5cm}}
\hline
\rowcolor{heading} \textbf{Nr.} & \textbf{Währung} & \textbf{Kürzel} &\textbf{Märkte } \\\hline
1 & Bytecoin & BCN & HitBTC, Poloniex \\\hline
2 & Monero & XMR & HitBTC, Binance, Bitfinex, Poloniex \\\hline
3 & Monero Original & XMO & HitBTC \\\hline
4 & DigitalNote & XDN & HitBTC, Bittrex \\\hline
5 & Quazar Coin & QCN & HitBTC \\\hline
6 & DashCoin & DSH & HitBTC \\\hline
7 & FantomCoin & FCN & HitBTC \\\hline
\end{tabular}
\caption{Kryptowährungen}
\end{table}


\subsection{Automatisiertes Schürfen}
Das zu entwickelnde Skript, welches automatisch die gewinnbringendste Kryptowährung schürfen soll, muss per Curl auf  die API von Coinmarketcap\footnote{\url{https://api.coinmarketcap.com/}} der jeweiligen Währung zugreifen. Dabei soll die Antwort für das Ermitteln der gewinnbringendsten Währung dienen.