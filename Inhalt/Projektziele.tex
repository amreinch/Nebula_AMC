% !TEX root = ../Diplombericht.tex
\newpage
\subsubsection{Projektziele} 
\label{sec:Projektziele}

\begin{table}[H]
\begin{tabular}[t]{p{0.7cm}|p{6.1cm}p{6.1cm} >{\centering}p{0.6cm}c}
\hline
\rowcolor{heading}\textbf{Nr.} & \textbf{Ziel} & \textbf{Messgrösse} & \textbf{Kat.} & \textbf{Prio.} \\\hline
01 & Die CPU des Clusters soll zu 90\% zum Schürfen von Kryptowährung  beansprucht werden. & Log und Monitoring Auswertungen nach dem Testlauf. & LZ & \textbf{M} \\\hline
02 & Die Daten werden auf einem Netzwerkshare (NAS) mit redundanten Festplatten (RAID I) gesichert. & Die Festplatten werden einzeln überprüft, der Datenbestand muss identisch sein. & BZ \newline TZ & \textbf{M} \\\hline
03 & Der Cluster soll eine Verfügbarkeit von 98\% aufweisen. & Dies kann erst nach dem Testlauf durch ein Monitoring der Laufzeit gemessen werden. & LZ \newline BZ \newline TZ & \textbf{M} \\\hline
04 & Es können während des Betriebs neue Compute Nodes hinzugefügt werden \& ausfallende Compute Nodes verursachen keinen Unterbruch des Betriebs. & Während der Testphase werden neue Compute Nodes hinzugefügt und Compute Nodes vom Cluster getrennt.  & LZ \newline BZ \newline TZ & \textbf{M} \\\hline 
05 & Der Cluster kann für verschiedene Anwendungsgebiete eingesetzt werden. & Während der Testphase werden andere Applikationen, welche die Cluster Ressourcen verwenden sollen, installiert. & BZ & \textbf{M} \\\hline
06 & Das Betriebssystem soll über das Netzwerk an die Compute Nodes verteilt werden, um SD-Karten zu sparen und ein Betriebssystem zentral verwalten zu können. & Wird während der Installation über Systemlogdateien ausgelesen und mit SSH-Zugriffen getestet. & WZ \newline BZ \newline TZ & \textbf{M} \\\hline
07 & Das Schürfprogramm soll automatisch die gewinnbringendste Währung abbauen. & Nach der Testphase werden die Logdateien und Wallets ausgewertet und mit Daten der Währungskurse abgeglichen. & LZ \newline WZ & \textbf{K} \\\hline
08 & Mit der geschürften Währung soll auf Börsen gehandelt werden können. & Kann nach der Realisierung durch Transaktionslogdaten gemessen werden. & LZ \newline WZ & \textbf{K} \\\hline
09 & Die Wartungsarbeiten sollen pro Monat nicht mehr als 3 Stunden betragen. & Wird durch ein Eingriffsprotokoll nach der Realisierungsphase festgehalten. & BZ & \textbf{K} \\\hline
10 & Der Cluster soll einfach transportierbar und wiederaufbaubar sein. & Der Cluster wird nach der Testphase physisch verschoben und neu aufgebaut, dabei wird die Zeit des Wiederaufbaus gemessen. & TZ & \textbf{K}\\\hline
\end{tabular}
\caption{Projektziele}
\end{table}

\textbf{Legende:} LZ = Leistungsziel, WZ = Wirtschaftsziel, BZ = Betriebsziel, TZ = Technisches Ziel, \newline M = Muss-Kriterium, K = Kann-Kriterium

\subsubsection{Lieferobjekte}
Folgende Dokumente werden während des Projektes erstellt und geliefert:

\begin{table}[H]
\begin{tabular}[t]{p{0.5cm}p{6.5cm}p{6.5cm}p{1.8cm}}
\hline
\rowcolor{heading}\textbf{Nr.} & \textbf{Dokument} & \textbf{Phase} & \textbf{Termin} \\\hline
1 & Projektplan & Initialisierung & 15.02.2018 \\\hline
2 & Projektlogo & Initialisierung & 20.02.2018 \\\hline
3 & Projektauftrag & Initialisierung & 20.02.2018 \\\hline
4 & Studie & Initialisierung & 25.02.2018 \\\hline
5 & Detailkonzept & Konzept & 17.03.2018 \\\hline
6 & Testkonzept & Konzept & 22.03.2018 \\\hline
7 & Testprotokoll & Realisierung & 03.05.2018 \\\hline
8 & Installationshandbuch & Realisierung & 10.05.2018 \\\hline
9 & Betriebshandbuch & Realisierung & 13.05.2018 \\\hline
\end{tabular}
\caption{Lieferobjekte}
\end{table}


 
