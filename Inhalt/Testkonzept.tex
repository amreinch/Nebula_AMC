% !TEX root = ../Diplombericht.tex
\subsection{Tests}
\subsubsection{Testobjekte}
Die folgende Hardware ist für die Tests der Funktionsfähigkeit des Clusters im Scope vorgesehen.
\begin{table}[H]
\centering
\begin{tabular}{p{1cm}p{7.cm}p{7.5cm}}
\hline
\rowcolor{heading} \textbf{Nr.} & \textbf{Objekt} & \textbf{Beschreibung} \\\hline
1 & Managementnodes & Raspberry PI 3  \\\hline
2 & Computenodes & Raspberry PI 3 \\\hline
3 & NAS & Synology NAS DS216+ \\\hline
4 & Switch & TP-Link TL-SL3428 \\\hline
\end{tabular}
\caption{Testobjekte}
\end{table}

\subsubsection{Testarten}
Die Tests werden in folgende Kategorien eingestuft:

\begin{table}[H]
\centering
\begin{tabular}{p{1cm}p{3cm}p{12cm}}
\hline
\rowcolor{heading} \textbf{Nr.} & \textbf{Testart} & \textbf{Beschreibung} \\\hline
1 & Komponentetest & Die Lauffähigkeit und Erreichbarkeit der einzelnen Hardware Komponenten wird überprüft.  \\\hline
2 & Integrationstest & Es wird die Zusammenarbeit der aktiven und neu Integrierten abhängigen Komponenten überprüft. \\\hline
3 & Systemtest & Das System wird als Komplettlösung getestet. Hierbei soll geprüft werden ob die Lösung den Anforderungen der Anwendbarkeit und Nutzbarkeit dem Auftrag entspricht.  \\\hline
\end{tabular}
\caption{Testarten}
\end{table}

\subsubsection{Testvoraussetzungen}
\textbf{Startbedingungen}\newline
Für den Start der Tests muss der Cluster aufgebaut sein und die einzelnen Komponenten müssen mit Strom versorgt sein. 

\textbf{Abbruchbedingungen}\newline
Die Tests werden abgebrochen sobald Fehler auftauchen welche Folgetests verhindern.

\subsubsection{Fehlerklassen}
\begin{table}[H]
\centering
\begin{tabular}{p{1cm}p{4cm}p{11cm}}
\hline
\rowcolor{heading} \textbf{Nr.} & \textbf{Fehlerklassen} & \textbf{Beschreibung} \\\hline
1 & Fehlerfrei & Die Erwartungen sind erfüllt.  \\\hline
2 & Harmloser Mangel & Es sind keine Betriebsverhinderungen zu erkennen. Die Erwartungen sind erfüllt. \\\hline
3 & Kleiner Mangel & Der Betrieb kann aufgenommen werden, das Problem sollte aber über einen Zeitraum von 6 Monaten behoben werden.  \\\hline
4 & Schwerer Mangel & Der Cluster kann nur teilweise in Betrieb genommen werden, der Mangel muss innerhalb zwei Wochen behoben werden. \\\hline
5 & Kritischer Mangel & Der Cluster kann nicht in Betrieb genommen werden. Die Mängel müssen umgehend behoben werden. \\\hline
\end{tabular}
\caption{Fehlerklassen}
\end{table}

\subsubsection{Testhilfsmittel}
Die Dokumentation der Tests wird im Testprotokoll nachgeführt. Damit die Tests durchgeführt werden können wird ein PC oder Notebook welches auf mit Linux betrieben wird benötigt. Dieser Client muss sich im selben Netzwerk wie der Cluster befinden.

\subsubsection{Testfälle}
\begin{table}[H]
\centering
\begin{tabular}{|p{4cm}|p{4cm}|p{4cm}|p{4cm}|}
\hline
Bezeichnung & \textbf{K-001} & Managementnode & Hostnamen / IP / MAC \\ \hline
Beschreibung & \multicolumn{3}{p{12cm}|}{Der Management Node wird auf die zugewiesene IP Adresse, MAC Adresse und den Hostnamen geprüft.} \\ \hline
Testvoraussetzung & \multicolumn{3}{p{12cm}|}{Der Managementnode und der Testclient befinden sich im selben Netzwerk.} \\ \hline
Testschritte & \multicolumn{3}{p{12cm}|}{
- Management Node starten (Strom anschliessen)\newline
- 30 Sekunden warten\newline
- Auf dem Testclient über Putty oder Shell mit dem Befehl \newline \grqq nmap -sn 192.168.1.10 \grqq eingeben\newline
- Prüfen ob der die Zuweisung gemäss Hostnamenkonzept richtig ist.} \\ \hline
Erwartetes Ergebnis & \multicolumn{3}{p{12cm}|}{Die Rückgabewerte (Hostnamen, IP- und MAC-Adresse) sollen deckungsleich mit deren im Hostnamenkonzept sein.} \\\hline
\end{tabular}
\caption{Testfall K-001}
\label{Testfall K-001}
\end{table}

\begin{table}[H]
\centering
\begin{tabular}{|p{4cm}|p{4cm}|p{4cm}|p{4cm}|}
\hline
Bezeichnung & \textbf{K-002} & Management Node & Anmelden \\ \hline
Beschreibung & \multicolumn{3}{p{12cm}|}{Es wird getestet ob der Management Node über das SSH Protkoll erreichbar ist.} \\ \hline
Testvoraussetzung & \multicolumn{3}{p{12cm}|}{Der Managementnode und der Testclient befinden sich im selben Netzwerk.} \\ \hline
Testschritte & \multicolumn{3}{p{12cm}|}{
- Management Node starten (Strom anschliessen)\newline
- 30 Sekunden warten\newline
- Auf dem Testclient über Putty oder Shell den Befehl \grqq ssh root@nebula \grqq eingeben \newline
- Passwort eingeben} \\ \hline
Erwartetes Ergebnis & \multicolumn{3}{p{12cm}|}{Der Zugriff auf den Management Node funktioniert} \\\hline
\end{tabular}
\caption{Testfall K-002}
\label{Testfall K-002}
\end{table}

\begin{table}[H]
\centering
\begin{tabular}{|p{4cm}|p{4cm}|p{4cm}|p{4cm}|}
\hline
Bezeichnung & \textbf{K-003} & Compute Node c1 & Hostnamen / IP / MAC \\ \hline
Beschreibung & \multicolumn{3}{p{12cm}|}{Der Compute Node wird auf die zugewiesene IP Adresse, MAC Adresse und den Hostnamen geprüft.} \\ \hline
Testvoraussetzung & \multicolumn{3}{p{12cm}|}{Der Compute Node und der Testclient befinden sich im selben Netzwerk.} \\ \hline
Testschritte & \multicolumn{3}{p{12cm}|}{
- Compute Node starten (Strom anschliessen)\newline
- 3 Minuten Sekunden warten\newline
- Auf dem Testclient über Putty oder Shell mit dem Befehl \newline \grqq nmap -sn 192.168.1.11 \grqq eingeben\newline
- Prüfen ob der die Zuweisung gemäss Hostnamenkonzept richtig ist.} \\ \hline
Erwartetes Ergebnis & \multicolumn{3}{p{12cm}|}{Die Rückgabewerte (Hostnamen, IP- und MAC-Adresse) sollen deckungsleich mit deren im Hostnamenkonzept sein.} \\\hline
\end{tabular}
\caption{Testfall K-003}
\label{Testfall K-003}
\end{table}


\begin{table}[H]
\centering
\begin{tabular}{|p{4cm}|p{4cm}|p{4cm}|p{4cm}|}
\hline
Bezeichnung & \textbf{K-004} & Compute Node c1 & Hostnamen / IP / MAC \\ \hline
Beschreibung & \multicolumn{3}{p{12cm}|}{Der Compute Node wird auf die zugewiesene IP Adresse, MAC Adresse und den Hostnamen geprüft.} \\ \hline
Testvoraussetzung & \multicolumn{3}{p{12cm}|}{Der Compute Node und der Testclient befinden sich im selben Netzwerk.} \\ \hline
Testschritte & \multicolumn{3}{p{12cm}|}{
- Compute Node starten (Strom anschliessen)\newline
- 3 Minuten Sekunden warten\newline
- Auf dem Testclient über Putty oder Shell mit dem Befehl \newline \grqq nmap -sn 192.168.1.11 \grqq eingeben\newline
- Prüfen ob der die Zuweisung gemäss Hostnamenkonzept richtig ist.} \\ \hline
Erwartetes Ergebnis & \multicolumn{3}{p{12cm}|}{Die Rückgabewerte (Hostnamen, IP- und MAC-Adresse) sollen deckungsleich mit deren im Hostnamenkonzept sein.} \\\hline
\end{tabular}
\caption{Testfall K-004}
\label{Testfall K-004}
\end{table}


\begin{table}[H]
\centering
\begin{tabular}{|p{4cm}|p{4cm}|p{4cm}|p{4cm}|}
\hline
Bezeichnung & \textbf{K-005} & Compute Node c2 & Hostnamen / IP / MAC \\ \hline
Beschreibung & \multicolumn{3}{p{12cm}|}{Der Compute Node wird auf die zugewiesene IP Adresse, MAC Adresse und den Hostnamen geprüft.} \\ \hline
Testvoraussetzung & \multicolumn{3}{p{12cm}|}{Der Compute Node und der Testclient befinden sich im selben Netzwerk.} \\ \hline
Testschritte & \multicolumn{3}{p{12cm}|}{
- Compute Node starten (Strom anschliessen)\newline
- 3 Minuten Sekunden warten\newline
- Auf dem Testclient über Putty oder Shell mit dem Befehl \newline \grqq nmap -sn 192.168.1.12 \grqq eingeben\newline
- Prüfen ob der die Zuweisung gemäss Hostnamenkonzept richtig ist.} \\ \hline
Erwartetes Ergebnis & \multicolumn{3}{p{12cm}|}{Die Rückgabewerte (Hostnamen, IP- und MAC-Adresse) sollen deckungsleich mit deren im Hostnamenkonzept sein.} \\\hline
\end{tabular}
\caption{Testfall K-005}
\label{Testfall K-005}
\end{table}


\begin{table}[H]
\centering
\begin{tabular}{|p{4cm}|p{4cm}|p{4cm}|p{4cm}|}
\hline
Bezeichnung & \textbf{K-006} & Compute Node c3 & Hostnamen / IP / MAC \\ \hline
Beschreibung & \multicolumn{3}{p{12cm}|}{Der Compute Node wird auf die zugewiesene IP Adresse, MAC Adresse und den Hostnamen geprüft.} \\ \hline
Testvoraussetzung & \multicolumn{3}{p{12cm}|}{Der Compute Node und der Testclient befinden sich im selben Netzwerk.} \\ \hline
Testschritte & \multicolumn{3}{p{12cm}|}{
- Compute Node starten (Strom anschliessen)\newline
- 3 Minuten Sekunden warten\newline
- Auf dem Testclient über Putty oder Shell mit dem Befehl \newline \grqq nmap -sn 192.168.1.13 \grqq eingeben\newline
- Prüfen ob der die Zuweisung gemäss Hostnamenkonzept richtig ist.} \\ \hline
Erwartetes Ergebnis & \multicolumn{3}{p{12cm}|}{Die Rückgabewerte (Hostnamen, IP- und MAC-Adresse) sollen deckungsleich mit deren im Hostnamenkonzept sein.} \\\hline
\end{tabular}
\caption{Testfall K-006}
\label{Testfall K-006}
\end{table}


\begin{table}[H]
\centering
\begin{tabular}{|p{4cm}|p{4cm}|p{4cm}|p{4cm}|}
\hline
Bezeichnung & \textbf{K-007} & Compute Node c4 & Hostnamen / IP / MAC \\ \hline
Beschreibung & \multicolumn{3}{p{12cm}|}{Der Compute Node wird auf die zugewiesene IP Adresse, MAC Adresse und den Hostnamen geprüft.} \\ \hline
Testvoraussetzung & \multicolumn{3}{p{12cm}|}{Der Compute Node und der Testclient befinden sich im selben Netzwerk.} \\ \hline
Testschritte & \multicolumn{3}{p{12cm}|}{
- Compute Node starten (Strom anschliessen)\newline
- 3 Minuten Sekunden warten\newline
- Auf dem Testclient über Putty oder Shell mit dem Befehl \newline \grqq nmap -sn 192.168.1.14 \grqq eingeben\newline
- Prüfen ob der die Zuweisung gemäss Hostnamenkonzept richtig ist.} \\ \hline
Erwartetes Ergebnis & \multicolumn{3}{p{12cm}|}{Die Rückgabewerte (Hostnamen, IP- und MAC-Adresse) sollen deckungsleich mit deren im Hostnamenkonzept sein.} \\\hline
\end{tabular}
\caption{Testfall K-007}
\label{Testfall K-007}
\end{table}


\begin{table}[H]
\centering
\begin{tabular}{|p{4cm}|p{4cm}|p{4cm}|p{4cm}|}
\hline
Bezeichnung & \textbf{K-008} & Compute Node c5 & Hostnamen / IP / MAC \\ \hline
Beschreibung & \multicolumn{3}{p{12cm}|}{Der Compute Node wird auf die zugewiesene IP Adresse, MAC Adresse und den Hostnamen geprüft.} \\ \hline
Testvoraussetzung & \multicolumn{3}{p{12cm}|}{Der Compute Node und der Testclient befinden sich im selben Netzwerk.} \\ \hline
Testschritte & \multicolumn{3}{p{12cm}|}{
- Compute Node starten (Strom anschliessen)\newline
- 3 Minuten Sekunden warten\newline
- Auf dem Testclient über Putty oder Shell mit dem Befehl \newline \grqq nmap -sn 192.168.1.15 \grqq eingeben\newline
- Prüfen ob der die Zuweisung gemäss Hostnamenkonzept richtig ist.} \\ \hline
Erwartetes Ergebnis & \multicolumn{3}{p{12cm}|}{Die Rückgabewerte (Hostnamen, IP- und MAC-Adresse) sollen deckungsleich mit deren im Hostnamenkonzept sein.} \\\hline
\end{tabular}
\caption{Testfall K-008}
\label{Testfall K-008}
\end{table}


\begin{table}[H]
\centering
\begin{tabular}{|p{4cm}|p{4cm}|p{4cm}|p{4cm}|}
\hline
Bezeichnung & \textbf{K-009} & Compute Node c6 & Hostnamen / IP / MAC \\ \hline
Beschreibung & \multicolumn{3}{p{12cm}|}{Der Compute Node wird auf die zugewiesene IP Adresse, MAC Adresse und den Hostnamen geprüft.} \\ \hline
Testvoraussetzung & \multicolumn{3}{p{12cm}|}{Der Compute Node und der Testclient befinden sich im selben Netzwerk.} \\ \hline
Testschritte & \multicolumn{3}{p{12cm}|}{
- Compute Node starten (Strom anschliessen)\newline
- 3 Minuten Sekunden warten\newline
- Auf dem Testclient über Putty oder Shell mit dem Befehl \newline \grqq nmap -sn 192.168.1.16 \grqq eingeben\newline
- Prüfen ob der die Zuweisung gemäss Hostnamenkonzept richtig ist.} \\ \hline
Erwartetes Ergebnis & \multicolumn{3}{p{12cm}|}{Die Rückgabewerte (Hostnamen, IP- und MAC-Adresse) sollen deckungsleich mit deren im Hostnamenkonzept sein.} \\\hline
\end{tabular}
\caption{Testfall K-009}
\label{Testfall K-009}
\end{table}


\begin{table}[H]
\centering
\begin{tabular}{|p{4cm}|p{4cm}|p{4cm}|p{4cm}|}
\hline
Bezeichnung & \textbf{K-010} & Compute Node c7 & Hostnamen / IP / MAC \\ \hline
Beschreibung & \multicolumn{3}{p{12cm}|}{Der Compute Node wird auf die zugewiesene IP Adresse, MAC Adresse und den Hostnamen geprüft.} \\ \hline
Testvoraussetzung & \multicolumn{3}{p{12cm}|}{Der Compute Node und der Testclient befinden sich im selben Netzwerk.} \\ \hline
Testschritte & \multicolumn{3}{p{12cm}|}{
- Compute Node starten (Strom anschliessen)\newline
- 3 Minuten Sekunden warten\newline
- Auf dem Testclient über Putty oder Shell mit dem Befehl \newline \grqq nmap -sn 192.168.1.17 \grqq eingeben\newline
- Prüfen ob der die Zuweisung gemäss Hostnamenkonzept richtig ist.} \\ \hline
Erwartetes Ergebnis & \multicolumn{3}{p{12cm}|}{Die Rückgabewerte (Hostnamen, IP- und MAC-Adresse) sollen deckungsleich mit deren im Hostnamenkonzept sein.} \\\hline
\end{tabular}
\caption{Testfall K-010}
\label{Testfall K-010}
\end{table}


\begin{table}[H]
\centering
\begin{tabular}{|p{4cm}|p{4cm}|p{4cm}|p{4cm}|}
\hline
Bezeichnung & \textbf{K-011} & Compute Node c8 & Hostnamen / IP / MAC \\ \hline
Beschreibung & \multicolumn{3}{p{12cm}|}{Der Compute Node wird auf die zugewiesene IP Adresse, MAC Adresse und den Hostnamen geprüft.} \\ \hline
Testvoraussetzung & \multicolumn{3}{p{12cm}|}{Der Compute Node und der Testclient befinden sich im selben Netzwerk.} \\ \hline
Testschritte & \multicolumn{3}{p{12cm}|}{
- Compute Node starten (Strom anschliessen)\newline
- 3 Minuten Sekunden warten\newline
- Auf dem Testclient über Putty oder Shell mit dem Befehl \newline \grqq nmap -sn 192.168.1.18 \grqq eingeben\newline
- Prüfen ob der die Zuweisung gemäss Hostnamenkonzept richtig ist.} \\ \hline
Erwartetes Ergebnis & \multicolumn{3}{p{12cm}|}{Die Rückgabewerte (Hostnamen, IP- und MAC-Adresse) sollen deckungsleich mit deren im Hostnamenkonzept sein.} \\\hline
\end{tabular}
\caption{Testfall K-011}
\label{Testfall K-011}
\end{table}



\begin{table}[H]
\centering
\begin{tabular}{|p{4cm}|p{4cm}|p{4cm}|p{4cm}|}
\hline
Bezeichnung & \textbf{K-012} & Compute Node c9 & Hostnamen / IP / MAC \\ \hline
Beschreibung & \multicolumn{3}{p{12cm}|}{Der Compute Node wird auf die zugewiesene IP Adresse, MAC Adresse und den Hostnamen geprüft.} \\ \hline
Testvoraussetzung & \multicolumn{3}{p{12cm}|}{Der Compute Node und der Testclient befinden sich im selben Netzwerk.} \\ \hline
Testschritte & \multicolumn{3}{p{12cm}|}{
- Compute Node starten (Strom anschliessen)\newline
- 3 Minuten Sekunden warten\newline
- Auf dem Testclient über Putty oder Shell mit dem Befehl \newline \grqq nmap -sn 192.168.1.19 \grqq eingeben\newline
- Prüfen ob der die Zuweisung gemäss Hostnamenkonzept richtig ist.} \\ \hline
Erwartetes Ergebnis & \multicolumn{3}{p{12cm}|}{Die Rückgabewerte (Hostnamen, IP- und MAC-Adresse) sollen deckungsleich mit deren im Hostnamenkonzept sein.} \\\hline
\end{tabular}
\caption{Testfall K-012}
\label{Testfall K-012}
\end{table}


\begin{table}[H]
\centering
\begin{tabular}{|p{4cm}|p{4cm}|p{4cm}|p{4cm}|}
\hline
Bezeichnung & \textbf{K-013} & Compute Node c10 & Hostnamen / IP / MAC \\ \hline
Beschreibung & \multicolumn{3}{p{12cm}|}{Der Compute Node wird auf die zugewiesene IP Adresse, MAC Adresse und den Hostnamen geprüft.} \\ \hline
Testvoraussetzung & \multicolumn{3}{p{12cm}|}{Der Compute Node und der Testclient befinden sich im selben Netzwerk.} \\ \hline
Testschritte & \multicolumn{3}{p{12cm}|}{
- Compute Node starten (Strom anschliessen)\newline
- 3 Minuten Sekunden warten\newline
- Auf dem Testclient über Putty oder Shell mit dem Befehl \newline \grqq nmap -sn 192.168.1.20 \grqq eingeben\newline
- Prüfen ob der die Zuweisung gemäss Hostnamenkonzept richtig ist.} \\ \hline
Erwartetes Ergebnis & \multicolumn{3}{p{12cm}|}{Die Rückgabewerte (Hostnamen, IP- und MAC-Adresse) sollen deckungsleich mit deren im Hostnamenkonzept sein.} \\\hline
\end{tabular}
\caption{Testfall K-013}
\label{Testfall K-013}
\end{table}


\begin{table}[H]
\centering
\begin{tabular}{|p{4cm}|p{4cm}|p{4cm}|p{4cm}|}
\hline
Bezeichnung & \textbf{K-014} & Compute Node c11 & Hostnamen / IP / MAC \\ \hline
Beschreibung & \multicolumn{3}{p{12cm}|}{Der Compute Node wird auf die zugewiesene IP Adresse, MAC Adresse und den Hostnamen geprüft.} \\ \hline
Testvoraussetzung & \multicolumn{3}{p{12cm}|}{Der Compute Node und der Testclient befinden sich im selben Netzwerk.} \\ \hline
Testschritte & \multicolumn{3}{p{12cm}|}{
- Compute Node starten (Strom anschliessen)\newline
- 3 Minuten Sekunden warten\newline
- Auf dem Testclient über Putty oder Shell mit dem Befehl \newline \grqq nmap -sn 192.168.1.21 \grqq eingeben\newline
- Prüfen ob der die Zuweisung gemäss Hostnamenkonzept richtig ist.} \\ \hline
Erwartetes Ergebnis & \multicolumn{3}{p{12cm}|}{Die Rückgabewerte (Hostnamen, IP- und MAC-Adresse) sollen deckungsleich mit deren im Hostnamenkonzept sein.} \\\hline
\end{tabular}
\caption{Testfall K-014}
\label{Testfall K-014}
\end{table}


\begin{table}[H]
\centering
\begin{tabular}{|p{4cm}|p{4cm}|p{4cm}|p{4cm}|}
\hline
Bezeichnung & \textbf{K-015} & Compute Node c12 & Hostnamen / IP / MAC \\ \hline
Beschreibung & \multicolumn{3}{p{12cm}|}{Der Compute Node wird auf die zugewiesene IP Adresse, MAC Adresse und den Hostnamen geprüft.} \\ \hline
Testvoraussetzung & \multicolumn{3}{p{12cm}|}{Der Compute Node und der Testclient befinden sich im selben Netzwerk.} \\ \hline
Testschritte & \multicolumn{3}{p{12cm}|}{
- Compute Node starten (Strom anschliessen)\newline
- 3 Minuten Sekunden warten\newline
- Auf dem Testclient über Putty oder Shell mit dem Befehl \newline \grqq nmap -sn 192.168.1.22 \grqq eingeben\newline
- Prüfen ob der die Zuweisung gemäss Hostnamenkonzept richtig ist.} \\ \hline
Erwartetes Ergebnis & \multicolumn{3}{p{12cm}|}{Die Rückgabewerte (Hostnamen, IP- und MAC-Adresse) sollen deckungsleich mit deren im Hostnamenkonzept sein.} \\\hline
\end{tabular}
\caption{Testfall K-015}
\label{Testfall K-015}
\end{table}


\begin{table}[H]
\centering
\begin{tabular}{|p{4cm}|p{4cm}|p{4cm}|p{4cm}|}
\hline
Bezeichnung & \textbf{K-016} & Compute Node c13 & Hostnamen / IP / MAC \\ \hline
Beschreibung & \multicolumn{3}{p{12cm}|}{Der Compute Node wird auf die zugewiesene IP Adresse, MAC Adresse und den Hostnamen geprüft.} \\ \hline
Testvoraussetzung & \multicolumn{3}{p{12cm}|}{Der Compute Node und der Testclient befinden sich im selben Netzwerk.} \\ \hline
Testschritte & \multicolumn{3}{p{12cm}|}{
- Compute Node starten (Strom anschliessen)\newline
- 3 Minuten Sekunden warten\newline
- Auf dem Testclient über Putty oder Shell mit dem Befehl \newline \grqq nmap -sn 192.168.1.23 \grqq eingeben\newline
- Prüfen ob der die Zuweisung gemäss Hostnamenkonzept richtig ist.} \\ \hline
Erwartetes Ergebnis & \multicolumn{3}{p{12cm}|}{Die Rückgabewerte (Hostnamen, IP- und MAC-Adresse) sollen deckungsleich mit deren im Hostnamenkonzept sein.} \\\hline
\end{tabular}
\caption{Testfall K-016}
\label{Testfall K-016}
\end{table}


\begin{table}[H]
\centering
\begin{tabular}{|p{4cm}|p{4cm}|p{4cm}|p{4cm}|}
\hline
Bezeichnung & \textbf{K-017} & Compute Node c14 & Hostnamen / IP / MAC \\ \hline
Beschreibung & \multicolumn{3}{p{12cm}|}{Der Compute Node wird auf die zugewiesene IP Adresse, MAC Adresse und den Hostnamen geprüft.} \\ \hline
Testvoraussetzung & \multicolumn{3}{p{12cm}|}{Der Compute Node und der Testclient befinden sich im selben Netzwerk.} \\ \hline
Testschritte & \multicolumn{3}{p{12cm}|}{
- Compute Node starten (Strom anschliessen)\newline
- 3 Minuten Sekunden warten\newline
- Auf dem Testclient über Putty oder Shell mit dem Befehl \newline \grqq nmap -sn 192.168.1.24 \grqq eingeben\newline
- Prüfen ob der die Zuweisung gemäss Hostnamenkonzept richtig ist.} \\ \hline
Erwartetes Ergebnis & \multicolumn{3}{p{12cm}|}{Die Rückgabewerte (Hostnamen, IP- und MAC-Adresse) sollen deckungsleich mit deren im Hostnamenkonzept sein.} \\\hline
\end{tabular}
\caption{Testfall K-017}
\label{Testfall K-017}
\end{table}


\begin{table}[H]
\centering
\begin{tabular}{|p{4cm}|p{4cm}|p{4cm}|p{4cm}|}
\hline
Bezeichnung & \textbf{K-018} & Compute Node c15 & Hostnamen / IP / MAC \\ \hline
Beschreibung & \multicolumn{3}{p{12cm}|}{Der Compute Node wird auf die zugewiesene IP Adresse, MAC Adresse und den Hostnamen geprüft.} \\ \hline
Testvoraussetzung & \multicolumn{3}{p{12cm}|}{Der Compute Node und der Testclient befinden sich im selben Netzwerk.} \\ \hline
Testschritte & \multicolumn{3}{p{12cm}|}{
- Compute Node starten (Strom anschliessen)\newline
- 3 Minuten Sekunden warten\newline
- Auf dem Testclient über Putty oder Shell mit dem Befehl \newline \grqq nmap -sn 192.168.1.25 \grqq eingeben\newline
- Prüfen ob der die Zuweisung gemäss Hostnamenkonzept richtig ist.} \\ \hline
Erwartetes Ergebnis & \multicolumn{3}{p{12cm}|}{Die Rückgabewerte (Hostnamen, IP- und MAC-Adresse) sollen deckungsleich mit deren im Hostnamenkonzept sein.} \\\hline
\end{tabular}
\caption{Testfall K-018}
\label{Testfall K-018}
\end{table}


\begin{table}[H]
\centering
\begin{tabular}{|p{4cm}|p{4cm}|p{4cm}|p{4cm}|}
\hline
Bezeichnung & \textbf{K-019} & Compute Node c16 & Hostnamen / IP / MAC \\ \hline
Beschreibung & \multicolumn{3}{p{12cm}|}{Der Compute Node wird auf die zugewiesene IP Adresse, MAC Adresse und den Hostnamen geprüft.} \\ \hline
Testvoraussetzung & \multicolumn{3}{p{12cm}|}{Der Compute Node und der Testclient befinden sich im selben Netzwerk.} \\ \hline
Testschritte & \multicolumn{3}{p{12cm}|}{
- Compute Node starten (Strom anschliessen)\newline
- 3 Minuten Sekunden warten\newline
- Auf dem Testclient über Putty oder Shell mit dem Befehl \newline \grqq nmap -sn 192.168.1.26 \grqq eingeben\newline
- Prüfen ob der die Zuweisung gemäss Hostnamenkonzept richtig ist.} \\ \hline
Erwartetes Ergebnis & \multicolumn{3}{p{12cm}|}{Die Rückgabewerte (Hostnamen, IP- und MAC-Adresse) sollen deckungsleich mit deren im Hostnamenkonzept sein.} \\\hline
\end{tabular}
\caption{Testfall K-019}
\label{Testfall K-019}
\end{table}


\begin{table}[H]
\centering
\begin{tabular}{|p{4cm}|p{4cm}|p{4cm}|p{4cm}|}
\hline
Bezeichnung & \textbf{K-020} & Compute Node c17 & Hostnamen / IP / MAC \\ \hline
Beschreibung & \multicolumn{3}{p{12cm}|}{Der Compute Node wird auf die zugewiesene IP Adresse, MAC Adresse und den Hostnamen geprüft.} \\ \hline
Testvoraussetzung & \multicolumn{3}{p{12cm}|}{Der Compute Node und der Testclient befinden sich im selben Netzwerk.} \\ \hline
Testschritte & \multicolumn{3}{p{12cm}|}{
- Compute Node starten (Strom anschliessen)\newline
- 3 Minuten Sekunden warten\newline
- Auf dem Testclient über Putty oder Shell mit dem Befehl \newline \grqq nmap -sn 192.168.1.27 \grqq eingeben\newline
- Prüfen ob der die Zuweisung gemäss Hostnamenkonzept richtig ist.} \\ \hline
Erwartetes Ergebnis & \multicolumn{3}{p{12cm}|}{Die Rückgabewerte (Hostnamen, IP- und MAC-Adresse) sollen deckungsleich mit deren im Hostnamenkonzept sein.} \\\hline
\end{tabular}
\caption{Testfall K-020}
\label{Testfall K-020}
\end{table}


\begin{table}[H]
\centering
\begin{tabular}{|p{4cm}|p{4cm}|p{4cm}|p{4cm}|}
\hline
Bezeichnung & \textbf{K-021} & Compute Node c18 & Hostnamen / IP / MAC \\ \hline
Beschreibung & \multicolumn{3}{p{12cm}|}{Der Compute Node wird auf die zugewiesene IP Adresse, MAC Adresse und den Hostnamen geprüft.} \\ \hline
Testvoraussetzung & \multicolumn{3}{p{12cm}|}{Der Compute Node und der Testclient befinden sich im selben Netzwerk.} \\ \hline
Testschritte & \multicolumn{3}{p{12cm}|}{
- Compute Node starten (Strom anschliessen)\newline
- 3 Minuten Sekunden warten\newline
- Auf dem Testclient über Putty oder Shell mit dem Befehl \newline \grqq nmap -sn 192.168.1.28 \grqq eingeben\newline
- Prüfen ob der die Zuweisung gemäss Hostnamenkonzept richtig ist.} \\ \hline
Erwartetes Ergebnis & \multicolumn{3}{p{12cm}|}{Die Rückgabewerte (Hostnamen, IP- und MAC-Adresse) sollen deckungsleich mit deren im Hostnamenkonzept sein.} \\\hline
\end{tabular}
\caption{Testfall K-021}
\label{Testfall K-021}
\end{table}


\begin{table}[H]
\centering
\begin{tabular}{|p{4cm}|p{4cm}|p{4cm}|p{4cm}|}
\hline
Bezeichnung & \textbf{K-022} & Compute Node c19 & Hostnamen / IP / MAC \\ \hline
Beschreibung & \multicolumn{3}{p{12cm}|}{Der Compute Node wird auf die zugewiesene IP Adresse, MAC Adresse und den Hostnamen geprüft.} \\ \hline
Testvoraussetzung & \multicolumn{3}{p{12cm}|}{Der Compute Node und der Testclient befinden sich im selben Netzwerk.} \\ \hline
Testschritte & \multicolumn{3}{p{12cm}|}{
- Compute Node starten (Strom anschliessen)\newline
- 3 Minuten Sekunden warten\newline
- Auf dem Testclient über Putty oder Shell mit dem Befehl \newline \grqq nmap -sn 192.168.1.29 \grqq eingeben\newline
- Prüfen ob der die Zuweisung gemäss Hostnamenkonzept richtig ist.} \\ \hline
Erwartetes Ergebnis & \multicolumn{3}{p{12cm}|}{Die Rückgabewerte (Hostnamen, IP- und MAC-Adresse) sollen deckungsleich mit deren im Hostnamenkonzept sein.} \\\hline
\end{tabular}
\caption{Testfall K-022}
\label{Testfall K-022}
\end{table}



\begin{table}[H]
\centering
\begin{tabular}{|p{4cm}|p{4cm}|p{4cm}|p{4cm}|}
\hline
Bezeichnung & \textbf{K-023} & Compute Node c20 & Hostnamen / IP / MAC \\ \hline
Beschreibung & \multicolumn{3}{p{12cm}|}{Der Compute Node wird auf die zugewiesene IP Adresse, MAC Adresse und den Hostnamen geprüft.} \\ \hline
Testvoraussetzung & \multicolumn{3}{p{12cm}|}{Der Compute Node und der Testclient befinden sich im selben Netzwerk.} \\ \hline
Testschritte & \multicolumn{3}{p{12cm}|}{
- Compute Node starten (Strom anschliessen)\newline
- 3 Minuten Sekunden warten\newline
- Auf dem Testclient über Putty oder Shell mit dem Befehl \newline \grqq nmap -sn 192.168.1.30 \grqq eingeben\newline
- Prüfen ob der die Zuweisung gemäss Hostnamenkonzept richtig ist.} \\ \hline
Erwartetes Ergebnis & \multicolumn{3}{p{12cm}|}{Die Rückgabewerte (Hostnamen, IP- und MAC-Adresse) sollen deckungsleich mit deren im Hostnamenkonzept sein.} \\\hline
\end{tabular}
\caption{Testfall K-023}
\label{Testfall K-023}
\end{table}


\begin{table}[H]
\centering
\begin{tabular}{|p{4cm}|p{4cm}|p{4cm}|p{4cm}|}
\hline
Bezeichnung & \textbf{K-024} & Compute Node c21 & Hostnamen / IP / MAC \\ \hline
Beschreibung & \multicolumn{3}{p{12cm}|}{Der Compute Node wird auf die zugewiesene IP Adresse, MAC Adresse und den Hostnamen geprüft.} \\ \hline
Testvoraussetzung & \multicolumn{3}{p{12cm}|}{Der Compute Node und der Testclient befinden sich im selben Netzwerk.} \\ \hline
Testschritte & \multicolumn{3}{p{12cm}|}{
- Compute Node starten (Strom anschliessen)\newline
- 3 Minuten Sekunden warten\newline
- Auf dem Testclient über Putty oder Shell mit dem Befehl \newline \grqq nmap -sn 192.168.1.31 \grqq eingeben\newline
- Prüfen ob der die Zuweisung gemäss Hostnamenkonzept richtig ist.} \\ \hline
Erwartetes Ergebnis & \multicolumn{3}{p{12cm}|}{Die Rückgabewerte (Hostnamen, IP- und MAC-Adresse) sollen deckungsleich mit deren im Hostnamenkonzept sein.} \\\hline
\end{tabular}
\caption{Testfall K-024}
\label{Testfall K-024}
\end{table}


\begin{table}[H]
\centering
\begin{tabular}{|p{4cm}|p{4cm}|p{4cm}|p{4cm}|}
\hline
Bezeichnung & \textbf{K-025} & Compute Node c22 & Hostnamen / IP / MAC \\ \hline
Beschreibung & \multicolumn{3}{p{12cm}|}{Der Compute Node wird auf die zugewiesene IP Adresse, MAC Adresse und den Hostnamen geprüft.} \\ \hline
Testvoraussetzung & \multicolumn{3}{p{12cm}|}{Der Compute Node und der Testclient befinden sich im selben Netzwerk.} \\ \hline
Testschritte & \multicolumn{3}{p{12cm}|}{
- Compute Node starten (Strom anschliessen)\newline
- 3 Minuten Sekunden warten\newline
- Auf dem Testclient über Putty oder Shell mit dem Befehl \newline \grqq nmap -sn 192.168.1.32 \grqq eingeben\newline
- Prüfen ob der die Zuweisung gemäss Hostnamenkonzept richtig ist.} \\ \hline
Erwartetes Ergebnis & \multicolumn{3}{p{12cm}|}{Die Rückgabewerte (Hostnamen, IP- und MAC-Adresse) sollen deckungsleich mit deren im Hostnamenkonzept sein.} \\\hline
\end{tabular}
\caption{Testfall K-025}
\label{Testfall K-025}
\end{table}


\begin{table}[H]
\centering
\begin{tabular}{|p{4cm}|p{4cm}|p{4cm}|p{4cm}|}
\hline
Bezeichnung & \textbf{K-026} & Compute Node c23 & Hostnamen / IP / MAC \\ \hline
Beschreibung & \multicolumn{3}{p{12cm}|}{Der Compute Node wird auf die zugewiesene IP Adresse, MAC Adresse und den Hostnamen geprüft.} \\ \hline
Testvoraussetzung & \multicolumn{3}{p{12cm}|}{Der Compute Node und der Testclient befinden sich im selben Netzwerk.} \\ \hline
Testschritte & \multicolumn{3}{p{12cm}|}{
- Compute Node starten (Strom anschliessen)\newline
- 3 Minuten Sekunden warten\newline
- Auf dem Testclient über Putty oder Shell mit dem Befehl \newline \grqq nmap -sn 192.168.1.33 \grqq eingeben\newline
- Prüfen ob der die Zuweisung gemäss Hostnamenkonzept richtig ist.} \\ \hline
Erwartetes Ergebnis & \multicolumn{3}{p{12cm}|}{Die Rückgabewerte (Hostnamen, IP- und MAC-Adresse) sollen deckungsleich mit deren im Hostnamenkonzept sein.} \\\hline
\end{tabular}
\caption{Testfall K-026}
\label{Testfall K-026}
\end{table}


\begin{table}[H]
\centering
\begin{tabular}{|p{4cm}|p{4cm}|p{4cm}|p{4cm}|}
\hline
Bezeichnung & \textbf{K-027} & Compute Node c24 & Hostnamen / IP / MAC \\ \hline
Beschreibung & \multicolumn{3}{p{12cm}|}{Der Compute Node wird auf die zugewiesene IP Adresse, MAC Adresse und den Hostnamen geprüft.} \\ \hline
Testvoraussetzung & \multicolumn{3}{p{12cm}|}{Der Compute Node und der Testclient befinden sich im selben Netzwerk.} \\ \hline
Testschritte & \multicolumn{3}{p{12cm}|}{
- Compute Node starten (Strom anschliessen)\newline
- 3 Minuten Sekunden warten\newline
- Auf dem Testclient über Putty oder Shell mit dem Befehl \newline \grqq nmap -sn 192.168.1.34 \grqq eingeben\newline
- Prüfen ob der die Zuweisung gemäss Hostnamenkonzept richtig ist.} \\ \hline
Erwartetes Ergebnis & \multicolumn{3}{p{12cm}|}{Die Rückgabewerte (Hostnamen, IP- und MAC-Adresse) sollen deckungsleich mit deren im Hostnamenkonzept sein.} \\\hline
\end{tabular}
\caption{Testfall K-027}
\label{Testfall K-027}
\end{table}


\begin{table}[H]
\centering
\begin{tabular}{|p{4cm}|p{4cm}|p{4cm}|p{4cm}|}
\hline
Bezeichnung & \textbf{K-028} & Compute Node c25 & Hostnamen / IP / MAC \\ \hline
Beschreibung & \multicolumn{3}{p{12cm}|}{Der Compute Node wird auf die zugewiesene IP Adresse, MAC Adresse und den Hostnamen geprüft.} \\ \hline
Testvoraussetzung & \multicolumn{3}{p{12cm}|}{Der Compute Node und der Testclient befinden sich im selben Netzwerk.} \\ \hline
Testschritte & \multicolumn{3}{p{12cm}|}{
- Compute Node starten (Strom anschliessen)\newline
- 3 Minuten Sekunden warten\newline
- Auf dem Testclient über Putty oder Shell mit dem Befehl \newline \grqq nmap -sn 192.168.1.35 \grqq eingeben\newline
- Prüfen ob der die Zuweisung gemäss Hostnamenkonzept richtig ist.} \\ \hline
Erwartetes Ergebnis & \multicolumn{3}{p{12cm}|}{Die Rückgabewerte (Hostnamen, IP- und MAC-Adresse) sollen deckungsleich mit deren im Hostnamenkonzept sein.} \\\hline
\end{tabular}
\caption{Testfall K-028}
\label{Testfall K-028}
\end{table}


\begin{table}[H]
\centering
\begin{tabular}{|p{4cm}|p{4cm}|p{4cm}|p{4cm}|}
\hline
Bezeichnung & \textbf{K-029} & Compute Node c26 & Hostnamen / IP / MAC \\ \hline
Beschreibung & \multicolumn{3}{p{12cm}|}{Der Compute Node wird auf die zugewiesene IP Adresse, MAC Adresse und den Hostnamen geprüft.} \\ \hline
Testvoraussetzung & \multicolumn{3}{p{12cm}|}{Der Compute Node und der Testclient befinden sich im selben Netzwerk.} \\ \hline
Testschritte & \multicolumn{3}{p{12cm}|}{
- Compute Node starten (Strom anschliessen)\newline
- 3 Minuten Sekunden warten\newline
- Auf dem Testclient über Putty oder Shell mit dem Befehl \newline \grqq nmap -sn 192.168.1.36 \grqq eingeben\newline
- Prüfen ob der die Zuweisung gemäss Hostnamenkonzept richtig ist.} \\ \hline
Erwartetes Ergebnis & \multicolumn{3}{p{12cm}|}{Die Rückgabewerte (Hostnamen, IP- und MAC-Adresse) sollen deckungsleich mit deren im Hostnamenkonzept sein.} \\\hline
\end{tabular}
\caption{Testfall K-029}
\label{Testfall K-029}
\end{table}


\begin{table}[H]
\centering
\begin{tabular}{|p{4cm}|p{4cm}|p{4cm}|p{4cm}|}
\hline
Bezeichnung & \textbf{K-030} & Compute Node c27 & Hostnamen / IP / MAC \\ \hline
Beschreibung & \multicolumn{3}{p{12cm}|}{Der Compute Node wird auf die zugewiesene IP Adresse, MAC Adresse und den Hostnamen geprüft.} \\ \hline
Testvoraussetzung & \multicolumn{3}{p{12cm}|}{Der Compute Node und der Testclient befinden sich im selben Netzwerk.} \\ \hline
Testschritte & \multicolumn{3}{p{12cm}|}{
- Compute Node starten (Strom anschliessen)\newline
- 3 Minuten Sekunden warten\newline
- Auf dem Testclient über Putty oder Shell mit dem Befehl \newline \grqq nmap -sn 192.168.1.37 \grqq eingeben\newline
- Prüfen ob der die Zuweisung gemäss Hostnamenkonzept richtig ist.} \\ \hline
Erwartetes Ergebnis & \multicolumn{3}{p{12cm}|}{Die Rückgabewerte (Hostnamen, IP- und MAC-Adresse) sollen deckungsleich mit deren im Hostnamenkonzept sein.} \\\hline
\end{tabular}
\caption{Testfall K-030}
\label{Testfall K-030}
\end{table}


\begin{table}[H]
\centering
\begin{tabular}{|p{4cm}|p{4cm}|p{4cm}|p{4cm}|}
\hline
Bezeichnung & \textbf{K-031} & Compute Node c28 & Hostnamen / IP / MAC \\ \hline
Beschreibung & \multicolumn{3}{p{12cm}|}{Der Compute Node wird auf die zugewiesene IP Adresse, MAC Adresse und den Hostnamen geprüft.} \\ \hline
Testvoraussetzung & \multicolumn{3}{p{12cm}|}{Der Compute Node und der Testclient befinden sich im selben Netzwerk.} \\ \hline
Testschritte & \multicolumn{3}{p{12cm}|}{
- Compute Node starten (Strom anschliessen)\newline
- 3 Minuten Sekunden warten\newline
- Auf dem Testclient über Putty oder Shell mit dem Befehl \newline \grqq nmap -sn 192.168.1.38 \grqq eingeben\newline
- Prüfen ob der die Zuweisung gemäss Hostnamenkonzept richtig ist.} \\ \hline
Erwartetes Ergebnis & \multicolumn{3}{p{12cm}|}{Die Rückgabewerte (Hostnamen, IP- und MAC-Adresse) sollen deckungsleich mit deren im Hostnamenkonzept sein.} \\\hline
\end{tabular}
\caption{Testfall K-031}
\label{Testfall K-031}
\end{table}


\begin{table}[H]
\centering
\begin{tabular}{|p{4cm}|p{4cm}|p{4cm}|p{4cm}|}
\hline
Bezeichnung & \textbf{K-032} & Compute Node c29 & Hostnamen / IP / MAC \\ \hline
Beschreibung & \multicolumn{3}{p{12cm}|}{Der Compute Node wird auf die zugewiesene IP Adresse, MAC Adresse und den Hostnamen geprüft.} \\ \hline
Testvoraussetzung & \multicolumn{3}{p{12cm}|}{Der Compute Node und der Testclient befinden sich im selben Netzwerk.} \\ \hline
Testschritte & \multicolumn{3}{p{12cm}|}{
- Compute Node starten (Strom anschliessen)\newline
- 3 Minuten Sekunden warten\newline
- Auf dem Testclient über Putty oder Shell mit dem Befehl \newline \grqq nmap -sn 192.168.1.39 \grqq eingeben\newline
- Prüfen ob der die Zuweisung gemäss Hostnamenkonzept richtig ist.} \\ \hline
Erwartetes Ergebnis & \multicolumn{3}{p{12cm}|}{Die Rückgabewerte (Hostnamen, IP- und MAC-Adresse) sollen deckungsleich mit deren im Hostnamenkonzept sein.} \\\hline
\end{tabular}
\caption{Testfall K-032}
\label{Testfall K-032}
\end{table}


\begin{table}[H]
\centering
\begin{tabular}{|p{4cm}|p{4cm}|p{4cm}|p{4cm}|}
\hline
Bezeichnung & \textbf{K-033} & Compute Node c30 & Hostnamen / IP / MAC \\ \hline
Beschreibung & \multicolumn{3}{p{12cm}|}{Der Compute Node wird auf die zugewiesene IP Adresse, MAC Adresse und den Hostnamen geprüft.} \\ \hline
Testvoraussetzung & \multicolumn{3}{p{12cm}|}{Der Compute Node und der Testclient befinden sich im selben Netzwerk.} \\ \hline
Testschritte & \multicolumn{3}{p{12cm}|}{
- Compute Node starten (Strom anschliessen)\newline
- 3 Minuten Sekunden warten\newline
- Auf dem Testclient über Putty oder Shell mit dem Befehl \newline \grqq nmap -sn 192.168.1.40 \grqq eingeben\newline
- Prüfen ob der die Zuweisung gemäss Hostnamenkonzept richtig ist.} \\ \hline
Erwartetes Ergebnis & \multicolumn{3}{p{12cm}|}{Die Rückgabewerte (Hostnamen, IP- und MAC-Adresse) sollen deckungsleich mit deren im Hostnamenkonzept sein.} \\\hline
\end{tabular}
\caption{Testfall K-033}
\label{Testfall K-033}
\end{table}



\begin{table}[H]
\centering
\begin{tabular}{|p{4cm}|p{4cm}|p{4cm}|p{4cm}|}
\hline
Bezeichnung & \textbf{K-034} & Compute Node c31 & Hostnamen / IP / MAC \\ \hline
Beschreibung & \multicolumn{3}{p{12cm}|}{Der Compute Node wird auf die zugewiesene IP Adresse, MAC Adresse und den Hostnamen geprüft.} \\ \hline
Testvoraussetzung & \multicolumn{3}{p{12cm}|}{Der Compute Node und der Testclient befinden sich im selben Netzwerk.} \\ \hline
Testschritte & \multicolumn{3}{p{12cm}|}{
- Compute Node starten (Strom anschliessen)\newline
- 3 Minuten Sekunden warten\newline
- Auf dem Testclient über Putty oder Shell mit dem Befehl \newline \grqq nmap -sn 192.168.1.41 \grqq eingeben\newline
- Prüfen ob der die Zuweisung gemäss Hostnamenkonzept richtig ist.} \\ \hline
Erwartetes Ergebnis & \multicolumn{3}{p{12cm}|}{Die Rückgabewerte (Hostnamen, IP- und MAC-Adresse) sollen deckungsleich mit deren im Hostnamenkonzept sein.} \\\hline
\end{tabular}
\caption{Testfall K-034}
\label{Testfall K-034}
\end{table}


\begin{table}[H]
\centering
\begin{tabular}{|p{4cm}|p{4cm}|p{4cm}|p{4cm}|}
\hline
Bezeichnung & \textbf{K-035} & Compute Node c32 & Hostnamen / IP / MAC \\ \hline
Beschreibung & \multicolumn{3}{p{12cm}|}{Der Compute Node wird auf die zugewiesene IP Adresse, MAC Adresse und den Hostnamen geprüft.} \\ \hline
Testvoraussetzung & \multicolumn{3}{p{12cm}|}{Der Compute Node und der Testclient befinden sich im selben Netzwerk.} \\ \hline
Testschritte & \multicolumn{3}{p{12cm}|}{
- Compute Node starten (Strom anschliessen)\newline
- 3 Minuten Sekunden warten\newline
- Auf dem Testclient über Putty oder Shell mit dem Befehl \newline \grqq nmap -sn 192.168.1.42 \grqq eingeben\newline
- Prüfen ob der die Zuweisung gemäss Hostnamenkonzept richtig ist.} \\ \hline
Erwartetes Ergebnis & \multicolumn{3}{p{12cm}|}{Die Rückgabewerte (Hostnamen, IP- und MAC-Adresse) sollen deckungsleich mit deren im Hostnamenkonzept sein.} \\\hline
\end{tabular}
\caption{Testfall K-035}
\label{Testfall K-035}
\end{table}


\begin{table}[H]
\centering
\begin{tabular}{|p{4cm}|p{4cm}|p{4cm}|p{4cm}|}
\hline
Bezeichnung & \textbf{K-036} & Compute Node c33 & Hostnamen / IP / MAC \\ \hline
Beschreibung & \multicolumn{3}{p{12cm}|}{Der Compute Node wird auf die zugewiesene IP Adresse, MAC Adresse und den Hostnamen geprüft.} \\ \hline
Testvoraussetzung & \multicolumn{3}{p{12cm}|}{Der Compute Node und der Testclient befinden sich im selben Netzwerk.} \\ \hline
Testschritte & \multicolumn{3}{p{12cm}|}{
- Compute Node starten (Strom anschliessen)\newline
- 3 Minuten Sekunden warten\newline
- Auf dem Testclient über Putty oder Shell mit dem Befehl \newline \grqq nmap -sn 192.168.1.43 \grqq eingeben\newline
- Prüfen ob der die Zuweisung gemäss Hostnamenkonzept richtig ist.} \\ \hline
Erwartetes Ergebnis & \multicolumn{3}{p{12cm}|}{Die Rückgabewerte (Hostnamen, IP- und MAC-Adresse) sollen deckungsleich mit deren im Hostnamenkonzept sein.} \\\hline
\end{tabular}
\caption{Testfall K-036}
\label{Testfall K-036}
\end{table}


\begin{table}[H]
\centering
\begin{tabular}{|p{4cm}|p{4cm}|p{4cm}|p{4cm}|}
\hline
Bezeichnung & \textbf{K-037} & Compute Node c34 & Hostnamen / IP / MAC \\ \hline
Beschreibung & \multicolumn{3}{p{12cm}|}{Der Compute Node wird auf die zugewiesene IP Adresse, MAC Adresse und den Hostnamen geprüft.} \\ \hline
Testvoraussetzung & \multicolumn{3}{p{12cm}|}{Der Compute Node und der Testclient befinden sich im selben Netzwerk.} \\ \hline
Testschritte & \multicolumn{3}{p{12cm}|}{
- Compute Node starten (Strom anschliessen)\newline
- 3 Minuten Sekunden warten\newline
- Auf dem Testclient über Putty oder Shell mit dem Befehl \newline \grqq nmap -sn 192.168.1.44 \grqq eingeben\newline
- Prüfen ob der die Zuweisung gemäss Hostnamenkonzept richtig ist.} \\ \hline
Erwartetes Ergebnis & \multicolumn{3}{p{12cm}|}{Die Rückgabewerte (Hostnamen, IP- und MAC-Adresse) sollen deckungsleich mit deren im Hostnamenkonzept sein.} \\\hline
\end{tabular}
\caption{Testfall K-037}
\label{Testfall K-037}
\end{table}


\begin{table}[H]
\centering
\begin{tabular}{|p{4cm}|p{4cm}|p{4cm}|p{4cm}|}
\hline
Bezeichnung & \textbf{K-038} & Compute Node c35 & Hostnamen / IP / MAC \\ \hline
Beschreibung & \multicolumn{3}{p{12cm}|}{Der Compute Node wird auf die zugewiesene IP Adresse, MAC Adresse und den Hostnamen geprüft.} \\ \hline
Testvoraussetzung & \multicolumn{3}{p{12cm}|}{Der Compute Node und der Testclient befinden sich im selben Netzwerk.} \\ \hline
Testschritte & \multicolumn{3}{p{12cm}|}{
- Compute Node starten (Strom anschliessen)\newline
- 3 Minuten Sekunden warten\newline
- Auf dem Testclient über Putty oder Shell mit dem Befehl \newline \grqq nmap -sn 192.168.1.45 \grqq eingeben\newline
- Prüfen ob der die Zuweisung gemäss Hostnamenkonzept richtig ist.} \\ \hline
Erwartetes Ergebnis & \multicolumn{3}{p{12cm}|}{Die Rückgabewerte (Hostnamen, IP- und MAC-Adresse) sollen deckungsleich mit deren im Hostnamenkonzept sein.} \\\hline
\end{tabular}
\caption{Testfall K-038}
\label{Testfall K-038}
\end{table}


\begin{table}[H]
\centering
\begin{tabular}{|p{4cm}|p{4cm}|p{4cm}|p{4cm}|}
\hline
Bezeichnung & \textbf{K-039} & Compute Node c36 & Hostnamen / IP / MAC \\ \hline
Beschreibung & \multicolumn{3}{p{12cm}|}{Der Compute Node wird auf die zugewiesene IP Adresse, MAC Adresse und den Hostnamen geprüft.} \\ \hline
Testvoraussetzung & \multicolumn{3}{p{12cm}|}{Der Compute Node und der Testclient befinden sich im selben Netzwerk.} \\ \hline
Testschritte & \multicolumn{3}{p{12cm}|}{
- Compute Node starten (Strom anschliessen)\newline
- 3 Minuten Sekunden warten\newline
- Auf dem Testclient über Putty oder Shell mit dem Befehl \newline \grqq nmap -sn 192.168.1.46 \grqq eingeben\newline
- Prüfen ob der die Zuweisung gemäss Hostnamenkonzept richtig ist.} \\ \hline
Erwartetes Ergebnis & \multicolumn{3}{p{12cm}|}{Die Rückgabewerte (Hostnamen, IP- und MAC-Adresse) sollen deckungsleich mit deren im Hostnamenkonzept sein.} \\\hline
\end{tabular}
\caption{Testfall K-039}
\label{Testfall K-039}
\end{table}


\begin{table}[H]
\centering
\begin{tabular}{|p{4cm}|p{4cm}|p{4cm}|p{4cm}|}
\hline
Bezeichnung & \textbf{K-040} & Compute Node c37 & Hostnamen / IP / MAC \\ \hline
Beschreibung & \multicolumn{3}{p{12cm}|}{Der Compute Node wird auf die zugewiesene IP Adresse, MAC Adresse und den Hostnamen geprüft.} \\ \hline
Testvoraussetzung & \multicolumn{3}{p{12cm}|}{Der Compute Node und der Testclient befinden sich im selben Netzwerk.} \\ \hline
Testschritte & \multicolumn{3}{p{12cm}|}{
- Compute Node starten (Strom anschliessen)\newline
- 3 Minuten Sekunden warten\newline
- Auf dem Testclient über Putty oder Shell mit dem Befehl \newline \grqq nmap -sn 192.168.1.47 \grqq eingeben\newline
- Prüfen ob der die Zuweisung gemäss Hostnamenkonzept richtig ist.} \\ \hline
Erwartetes Ergebnis & \multicolumn{3}{p{12cm}|}{Die Rückgabewerte (Hostnamen, IP- und MAC-Adresse) sollen deckungsleich mit deren im Hostnamenkonzept sein.} \\\hline
\end{tabular}
\caption{Testfall K-040}
\label{Testfall K-040}
\end{table}


\begin{table}[H]
\centering
\begin{tabular}{|p{4cm}|p{4cm}|p{4cm}|p{4cm}|}
\hline
Bezeichnung & \textbf{K-041} & Compute Node c38 & Hostnamen / IP / MAC \\ \hline
Beschreibung & \multicolumn{3}{p{12cm}|}{Der Compute Node wird auf die zugewiesene IP Adresse, MAC Adresse und den Hostnamen geprüft.} \\ \hline
Testvoraussetzung & \multicolumn{3}{p{12cm}|}{Der Compute Node und der Testclient befinden sich im selben Netzwerk.} \\ \hline
Testschritte & \multicolumn{3}{p{12cm}|}{
- Compute Node starten (Strom anschliessen)\newline
- 3 Minuten Sekunden warten\newline
- Auf dem Testclient über Putty oder Shell mit dem Befehl \newline \grqq nmap -sn 192.168.1.48 \grqq eingeben\newline
- Prüfen ob der die Zuweisung gemäss Hostnamenkonzept richtig ist.} \\ \hline
Erwartetes Ergebnis & \multicolumn{3}{p{12cm}|}{Die Rückgabewerte (Hostnamen, IP- und MAC-Adresse) sollen deckungsleich mit deren im Hostnamenkonzept sein.} \\\hline
\end{tabular}
\caption{Testfall K-041}
\label{Testfall K-041}
\end{table}


\begin{table}[H]
\centering
\begin{tabular}{|p{4cm}|p{4cm}|p{4cm}|p{4cm}|}
\hline
Bezeichnung & \textbf{K-042} & Compute Node c39 & Hostnamen / IP / MAC \\ \hline
Beschreibung & \multicolumn{3}{p{12cm}|}{Der Compute Node wird auf die zugewiesene IP Adresse, MAC Adresse und den Hostnamen geprüft.} \\ \hline
Testvoraussetzung & \multicolumn{3}{p{12cm}|}{Der Compute Node und der Testclient befinden sich im selben Netzwerk.} \\ \hline
Testschritte & \multicolumn{3}{p{12cm}|}{
- Compute Node starten (Strom anschliessen)\newline
- 3 Minuten Sekunden warten\newline
- Auf dem Testclient über Putty oder Shell mit dem Befehl \newline \grqq nmap -sn 192.168.1.49 \grqq eingeben\newline
- Prüfen ob der die Zuweisung gemäss Hostnamenkonzept richtig ist.} \\ \hline
Erwartetes Ergebnis & \multicolumn{3}{p{12cm}|}{Die Rückgabewerte (Hostnamen, IP- und MAC-Adresse) sollen deckungsleich mit deren im Hostnamenkonzept sein.} \\\hline
\end{tabular}
\caption{Testfall K-042}
\label{Testfall K-042}
\end{table}


\begin{table}[H]
\centering
\begin{tabular}{|p{4cm}|p{4cm}|p{4cm}|p{4cm}|}
\hline
Bezeichnung & \textbf{K-043} & Compute Node c40 & Hostnamen / IP / MAC \\ \hline
Beschreibung & \multicolumn{3}{p{12cm}|}{Der Compute Node wird auf die zugewiesene IP Adresse, MAC Adresse und den Hostnamen geprüft.} \\ \hline
Testvoraussetzung & \multicolumn{3}{p{12cm}|}{Der Compute Node und der Testclient befinden sich im selben Netzwerk.} \\ \hline
Testschritte & \multicolumn{3}{p{12cm}|}{
- Compute Node starten (Strom anschliessen)\newline
- 3 Minuten Sekunden warten\newline
- Auf dem Testclient über Putty oder Shell mit dem Befehl \newline \grqq nmap -sn 192.168.1.50 \grqq eingeben\newline
- Prüfen ob der die Zuweisung gemäss Hostnamenkonzept richtig ist.} \\ \hline
Erwartetes Ergebnis & \multicolumn{3}{p{12cm}|}{Die Rückgabewerte (Hostnamen, IP- und MAC-Adresse) sollen deckungsleich mit deren im Hostnamenkonzept sein.} \\\hline
\end{tabular}
\caption{Testfall K-043}
\label{Testfall K-043}
\end{table}



\begin{table}[H]
\centering
\begin{tabular}{|p{4cm}|p{4cm}|p{4cm}|p{4cm}|}
\hline
Bezeichnung & \textbf{K-044} & Compute Node c41 & Hostnamen / IP / MAC \\ \hline
Beschreibung & \multicolumn{3}{p{12cm}|}{Der Compute Node wird auf die zugewiesene IP Adresse, MAC Adresse und den Hostnamen geprüft.} \\ \hline
Testvoraussetzung & \multicolumn{3}{p{12cm}|}{Der Compute Node und der Testclient befinden sich im selben Netzwerk.} \\ \hline
Testschritte & \multicolumn{3}{p{12cm}|}{
- Compute Node starten (Strom anschliessen)\newline
- 3 Minuten Sekunden warten\newline
- Auf dem Testclient über Putty oder Shell mit dem Befehl \newline \grqq nmap -sn 192.168.1.51 \grqq eingeben\newline
- Prüfen ob der die Zuweisung gemäss Hostnamenkonzept richtig ist.} \\ \hline
Erwartetes Ergebnis & \multicolumn{3}{p{12cm}|}{Die Rückgabewerte (Hostnamen, IP- und MAC-Adresse) sollen deckungsleich mit deren im Hostnamenkonzept sein.} \\\hline
\end{tabular}
\caption{Testfall K-044}
\label{Testfall K-044}
\end{table}


\begin{table}[H]
\centering
\begin{tabular}{|p{4cm}|p{4cm}|p{4cm}|p{4cm}|}
\hline
Bezeichnung & \textbf{K-045} & Compute Node c42 & Hostnamen / IP / MAC \\ \hline
Beschreibung & \multicolumn{3}{p{12cm}|}{Der Compute Node wird auf die zugewiesene IP Adresse, MAC Adresse und den Hostnamen geprüft.} \\ \hline
Testvoraussetzung & \multicolumn{3}{p{12cm}|}{Der Compute Node und der Testclient befinden sich im selben Netzwerk.} \\ \hline
Testschritte & \multicolumn{3}{p{12cm}|}{
- Compute Node starten (Strom anschliessen)\newline
- 3 Minuten Sekunden warten\newline
- Auf dem Testclient über Putty oder Shell mit dem Befehl \newline \grqq nmap -sn 192.168.1.52 \grqq eingeben\newline
- Prüfen ob der die Zuweisung gemäss Hostnamenkonzept richtig ist.} \\ \hline
Erwartetes Ergebnis & \multicolumn{3}{p{12cm}|}{Die Rückgabewerte (Hostnamen, IP- und MAC-Adresse) sollen deckungsleich mit deren im Hostnamenkonzept sein.} \\\hline
\end{tabular}
\caption{Testfall K-045}
\label{Testfall K-045}
\end{table}


\begin{table}[H]
\centering
\begin{tabular}{|p{4cm}|p{4cm}|p{4cm}|p{4cm}|}
\hline
Bezeichnung & \textbf{K-046} & Compute Node c43 & Hostnamen / IP / MAC \\ \hline
Beschreibung & \multicolumn{3}{p{12cm}|}{Der Compute Node wird auf die zugewiesene IP Adresse, MAC Adresse und den Hostnamen geprüft.} \\ \hline
Testvoraussetzung & \multicolumn{3}{p{12cm}|}{Der Compute Node und der Testclient befinden sich im selben Netzwerk.} \\ \hline
Testschritte & \multicolumn{3}{p{12cm}|}{
- Compute Node starten (Strom anschliessen)\newline
- 3 Minuten Sekunden warten\newline
- Auf dem Testclient über Putty oder Shell mit dem Befehl \newline \grqq nmap -sn 192.168.1.53 \grqq eingeben\newline
- Prüfen ob der die Zuweisung gemäss Hostnamenkonzept richtig ist.} \\ \hline
Erwartetes Ergebnis & \multicolumn{3}{p{12cm}|}{Die Rückgabewerte (Hostnamen, IP- und MAC-Adresse) sollen deckungsleich mit deren im Hostnamenkonzept sein.} \\\hline
\end{tabular}
\caption{Testfall K-046}
\label{Testfall K-046}
\end{table}


\begin{table}[H]
\centering
\begin{tabular}{|p{4cm}|p{4cm}|p{4cm}|p{4cm}|}
\hline
Bezeichnung & \textbf{K-047} & Compute Node c44 & Hostnamen / IP / MAC \\ \hline
Beschreibung & \multicolumn{3}{p{12cm}|}{Der Compute Node wird auf die zugewiesene IP Adresse, MAC Adresse und den Hostnamen geprüft.} \\ \hline
Testvoraussetzung & \multicolumn{3}{p{12cm}|}{Der Compute Node und der Testclient befinden sich im selben Netzwerk.} \\ \hline
Testschritte & \multicolumn{3}{p{12cm}|}{
- Compute Node starten (Strom anschliessen)\newline
- 3 Minuten Sekunden warten\newline
- Auf dem Testclient über Putty oder Shell mit dem Befehl \newline \grqq nmap -sn 192.168.1.54 \grqq eingeben\newline
- Prüfen ob der die Zuweisung gemäss Hostnamenkonzept richtig ist.} \\ \hline
Erwartetes Ergebnis & \multicolumn{3}{p{12cm}|}{Die Rückgabewerte (Hostnamen, IP- und MAC-Adresse) sollen deckungsleich mit deren im Hostnamenkonzept sein.} \\\hline
\end{tabular}
\caption{Testfall K-047}
\label{Testfall K-047}
\end{table}


\begin{table}[H]
\centering
\begin{tabular}{|p{4cm}|p{4cm}|p{4cm}|p{4cm}|}
\hline
Bezeichnung & \textbf{K-048} & Compute Node c45 & Hostnamen / IP / MAC \\ \hline
Beschreibung & \multicolumn{3}{p{12cm}|}{Der Compute Node wird auf die zugewiesene IP Adresse, MAC Adresse und den Hostnamen geprüft.} \\ \hline
Testvoraussetzung & \multicolumn{3}{p{12cm}|}{Der Compute Node und der Testclient befinden sich im selben Netzwerk.} \\ \hline
Testschritte & \multicolumn{3}{p{12cm}|}{
- Compute Node starten (Strom anschliessen)\newline
- 3 Minuten Sekunden warten\newline
- Auf dem Testclient über Putty oder Shell mit dem Befehl \newline \grqq nmap -sn 192.168.1.55 \grqq eingeben\newline
- Prüfen ob der die Zuweisung gemäss Hostnamenkonzept richtig ist.} \\ \hline
Erwartetes Ergebnis & \multicolumn{3}{p{12cm}|}{Die Rückgabewerte (Hostnamen, IP- und MAC-Adresse) sollen deckungsleich mit deren im Hostnamenkonzept sein.} \\\hline
\end{tabular}
\caption{Testfall K-048}
\label{Testfall K-048}
\end{table}

