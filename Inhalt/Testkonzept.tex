% !TEX root = ../Diplombericht.tex
\subsection{Tests}
\subsubsection{Testobjekte}
Die folgende Hardware ist für die Tests der Funktionsfähigkeit des Clusters im Scope vorgesehen:
\begin{table}[H]
\centering
\begin{tabular}{p{1cm}p{7.cm}p{7.5cm}}
\hline
\rowcolor{heading} \textbf{Nr.} & \textbf{Objekt} & \textbf{Beschreibung} \\\hline
1 & Management Node & Raspberry PI 3  \\\hline
2 & Compute Nodes & Raspberry PI 3 \\\hline
3 & NAS & Synology NAS DS216 \\\hline
4 & Switch & TP-Link TL-SL3428 \\\hline
\end{tabular}
\caption{Testobjekte}
\end{table}

\subsubsection{Testarten}
Die Tests werden in folgende Kategorien eingestuft:

\begin{table}[H]
\centering
\begin{tabular}{p{1cm}p{3cm}p{12cm}}
\hline
\rowcolor{heading} \textbf{Nr.} & \textbf{Testart} & \textbf{Beschreibung} \\\hline
1 & Komponententest & Die Lauffähigkeit und Erreichbarkeit der einzelnen Hardware Komponenten wird überprüft.  \\\hline
2 & Integrationstest & Es wird die Zusammenarbeit der aktiven und neu integrierten abhängigen Komponenten überprüft. \\\hline
3 & Systemtest & Das System wird als Komplettlösung getestet. Hierbei soll geprüft werden, ob die Lösung den Anforderungen der Anwendbarkeit und Nutzbarkeit dem Auftrag entspricht.  \\\hline
\end{tabular}
\caption{Testarten}
\end{table}

\subsubsection{Testvoraussetzungen}
\textbf{Startbedingungen}\newline
Für den Start der Tests muss der Cluster aufgebaut sein und die einzelnen Komponenten müssen mit Strom versorgt sein. 

\textbf{Abbruchbedingungen}\newline
Die Tests werden abgebrochen, sobald Fehler auftauchen, welche Folgetests verhindern.

\subsubsection{Fehlerklassen}
\begin{table}[H]
\centering
\begin{tabular}{p{1cm}p{4cm}p{11cm}}
\hline
\rowcolor{heading} \textbf{Nr.} & \textbf{Fehlerklassen} & \textbf{Beschreibung} \\\hline
1 & Fehlerfrei & Die Erwartungen sind erfüllt.  \\\hline
2 & Harmloser Mangel & Es sind keine Betriebsverhinderungen zu erkennen. Die Erwartungen sind erfüllt. \\\hline
3 & Kleiner Mangel & Der Betrieb kann aufgenommen werden. Das Problem sollte aber über einen Zeitraum von 6 Monaten behoben werden.  \\\hline
4 & Schwerer Mangel & Der Cluster kann nur teilweise in Betrieb genommen werden. Der Mangel muss innerhalb zwei Wochen behoben werden. \\\hline
5 & Kritischer Mangel & Der Cluster kann nicht in Betrieb genommen werden. Die Mängel müssen umgehend behoben werden. \\\hline
\end{tabular}
\caption{Fehlerklassen}
\end{table}

\subsubsection{Testhilfsmittel}
Die Dokumentation der Tests wird im Testprotokoll nachgeführt. Damit die Tests durchgeführt werden können wird ein PC oder Notebook als Testclient benötigt. Dieser Client muss sich im selben Netzwerk wie der Cluster befinden.

