% !TEX root = ../Diplombericht.tex
\subsection{Tests}
\subsubsection{Testobjekte}
Die folgende Hardware ist für die Tests der Funktionsfähigkeit des Clusters im Scope vorgesehen.
\begin{table}[H]
\centering
\begin{tabular}{p{1cm}p{7.cm}p{7.5cm}}
\hline
\rowcolor{heading} \textbf{Nr.} & \textbf{Objekt} & \textbf{Beschreibung} \\\hline
1 & Managementnodes & Raspberry PI 3  \\\hline
2 & Computenodes & Raspberry PI 3 \\\hline
3 & NAS & Synology NAS DS216+ \\\hline
4 & Switch & TP-Link TL-SL3428 \\\hline
\end{tabular}
\caption{Testobjekte}
\end{table}

\subsubsection{Testarten}
Die Tests werden in folgende Kategorien eingestuft:

\begin{table}[H]
\centering
\begin{tabular}{p{1cm}p{3cm}p{12cm}}
\hline
\rowcolor{heading} \textbf{Nr.} & \textbf{Testart} & \textbf{Beschreibung} \\\hline
1 & Komponentetest & Die Lauffähigkeit und Erreichbarkeit der einzelnen Hardware Komponenten wird überprüft.  \\\hline
2 & Integrationstest & Es wird die Zusammenarbeit der aktiven und neu Integrierten abhängigen Komponenten überprüft. \\\hline
3 & Systemtest & Das System wird als Komplettlösung getestet. Hierbei soll geprüft werden ob die Lösung den Anforderungen der Anwendbarkeit und Nutzbarkeit dem Auftrag entspricht.  \\\hline
\end{tabular}
\caption{Testarten}
\end{table}

\subsubsection{Testvoraussetzungen}
\textbf{Startbedingungen}\newline
Für den Start der Tests muss der Cluster aufgebaut sein und die einzelnen Komponenten müssen mit Strom versorgt sein. 

\textbf{Abbruchbedingungen}\newline
Die Tests werden abgebrochen sobald Fehler auftauchen welche Folgetests verhindern.

\subsubsection{Fehlerklassen}
\begin{table}[H]
\centering
\begin{tabular}{p{1cm}p{4cm}p{11cm}}
\hline
\rowcolor{heading} \textbf{Nr.} & \textbf{Fehlerklassen} & \textbf{Beschreibung} \\\hline
1 & Fehlerfrei & Die Erwartungen sind erfüllt.  \\\hline
2 & Harmloser Mangel & Es sind keine Betriebsverhinderungen zu erkennen. Die Erwartungen sind erfüllt. \\\hline
3 & Kleiner Mangel & Der Betrieb kann aufgenommen werden, das Problem sollte aber über einen Zeitraum von 6 Monaten behoben werden.  \\\hline
4 & Schwerer Mangel & Der Cluster kann nur teilweise in Betrieb genommen werden, der Mangel muss innerhalb zwei Wochen behoben werden. \\\hline
5 & Kritischer Mangel & Der Cluster kann nicht in Betrieb genommen werden. Die Mängel müssen umgehend behoben werden. \\\hline
\end{tabular}
\caption{Fehlerklassen}
\end{table}

\subsubsection{Testhilfsmittel}
Die Dokumentation der Tests wird im Testprotokoll nachgeführt. Damit die Tests durchgeführt werden können wird ein PC oder Notebook welches auf mit Linux betrieben wird benötigt. Dieser Client muss sich im selben Netzwerk wie der Cluster befinden.

\subsubsection{Testfälle}
\begin{table}[H]
\centering
\begin{tabular}{|p{4cm}|p{4cm}|p{4cm}|p{4cm}|}
\hline
Bezeichnung & \textbf{K-001} & Start Managementnode & Systemstart \\ \hline
Beschreibung & \multicolumn{3}{p{12cm}|}{Der Managementnode wird auf die Erreichbarkeit nach einem Systemstart der Komponenten überprüft} \\ \hline
Testvoraussetzung & \multicolumn{3}{p{12cm}|}{Der Managementnode und der Test PC befinden sich im selben Netzwerk} \\ \hline
Testschritte & \multicolumn{3}{p{12cm}|}{\begin{itemize}
\item Managementnode starten (Strom anschliessen)
\item 30 Sekunden warten
\item auf Test PC mit dem Befehl \grqq nmap -sn 192.168.1.10 | grep Nmap scan report\grqq{}  auf erreichbare Geräte mit der IP 192.168.1.10 prüfen.
\end{itemize}} \\ \hline
Erwartetes Ergebnis & \multicolumn{3}{p{12cm}|}{Nmap scan report for nebula (192.168.1.10)} \\\hline
\end{tabular}
\caption{Testfall K-001}
\label{Testfall K-001}
\end{table}

\begin{table}[H]
\centering
\begin{tabular}{|p{4cm}|p{4cm}|p{4cm}|p{4cm}|}
\hline
Bezeichnung & \textbf{K-002} & Start Computenodes & Systemstart \\ \hline
Beschreibung & \multicolumn{3}{p{12cm}|}{Die Computenodes werden auf die Erreichbarkeit nach einem Systemstart überprüft} \\ \hline
Testvoraussetzung & \multicolumn{3}{p{12cm}|}{Die Nodes und der Test PC befinden sich im selben Netzwerk, der Managementnode muss bereits in Betrieb sein} \\\hline
Testschritte & \multicolumn{3}{p{12cm}|}{\begin{itemize}
\item Computenodes an Strom anschliessen
\item 5 Minuten warten
\item auf Test PC mit dem Befehl  \textit{nmap -sn 192.168.1.11-55 | grep Nmap scan report} auf erreichbare Geräte mit der IP Range 192.168.1.11 - 192.168.1.55 prüfen
\end{itemize}} \\ \hline
Erwartetes Ergebnis & \multicolumn{3}{p{12cm}|}{
Nmap scan report for c1 (192.168.1.11) \newline
...}\\ \hline
\end{tabular}
\caption{Testfall K-002}
\label{Testfall K-002}
\end{table}


\begin{table}[H]
\centering
\begin{tabular}{|p{4cm}|p{4cm}|p{4cm}|p{4cm}|}
\hline
Bezeichnung & \textbf{K-003} & Hostnamen \& IP Nodes & DHCP \\\hline
Beschreibung & \multicolumn{3}{p{12cm}|}{Es wird geprüft ob alle Nodes die richtige IP Adresse und den richtigen Hostnamen zugewiesen erhalten haben} \\\hline
Testvoraussetzung & \multicolumn{3}{p{12cm}|}{Die Nodes und der Test PC befinden sich im selben Netzwerk, alle Nodes müssen gestartet sein. Dieser Test ist abhängig von K-001 und K-002} \\\hline
Testschritte & \multicolumn{3}{p{12cm}|}{\begin{itemize}
\item Auf dem Test PC mit dem Befehl \textit{nmap -sn 192.168.1.10-55 | \grq grep Nmap scan report|MAC Address\grq} die IP, den Hostnamen und die MAC Adressen auslesen.
\item Die Ausgabe mit der Hostnamenliste im Anhang vergleichen, diese muss identisch sein
\end{itemize}} \\ \hline
Erwartetes Ergebnis & \multicolumn{3}{p{12cm}|}{Nmap scan report for c1.home (192.168.1.11) \newline
MAC Address: B8:27:EB:32:39:A7 (Raspberry Pi Foundation)\newline
... \newline
Es müssen 46 Einträge vorhanden sein}\\\hline
\end{tabular}
\caption{Testfall K-003}
\label{Testfall K-003}
\end{table}

\begin{table}[H]
\centering
\begin{tabular}{|p{4cm}|p{4cm}|p{4cm}|p{4cm}|}
\hline
Bezeichnung & \textbf{K-004} & NAS-Share & SMB \\\hline
Beschreibung & \multicolumn{3}{p{12cm}|}{Es wird getest ob das NAS erreichbar ist und der Samba Dienst läuft} \\\hline
Testvoraussetzung & \multicolumn{3}{p{12cm}|}{Das NAS und der Managementnode sind gestartet} \\\hline
Testschritte & \multicolumn{3}{p{12cm}|}{Auf dem Managementnode mit dem Befehl nc -v nasbox 445 in der Shell die Verbindung prüfen.} \\ \hline
Erwartetes Ergebnis & \multicolumn{3}{p{12cm}|}{Ncat: Version 6.40 ( http:\/\/nmap.org\/ncat )\newline
Ncat: Connected to 2a02:1205:5012:90f0:211:32ff:fe54:1e69:445.}\\\hline
\end{tabular}
\caption{Testfall K-004}
\label{Testfall K-004}
\end{table}

\begin{table}[H]
\centering
\begin{tabular}{|p{4cm}|p{4cm}|p{4cm}|p{4cm}|}
\hline
Bezeichnung & \textbf{I-001} & Installation & Skript \\\hline
Beschreibung & \multicolumn{3}{p{12cm}|}{Das Installationsskript soll automatisiert durchlaufen. Der Cluster soll danach direkt einsetzbar sein ohne zusätzliche Konfigurationen vornehmen zu müssen} \\\hline
Testvoraussetzung & \multicolumn{3}{p{12cm}|}{Alle Komponententests müssen fehlerfrei durchgelaufen sein} \\\hline
Testschritte & \multicolumn{3}{p{12cm}|}{\begin{itemize}
\item Das Installationsskript aus dem Git Repository herunterladen
\item Das Skript ausführen
\item Installationsdurchlauf abwarten, ca. 1 Stunde
\item Auf den Managementnode per SSH über den PC verbinden
\item Den Befehl sinfo eingeben
\end{itemize}} \\ \hline
Erwartetes Ergebnis & \multicolumn{3}{p{12cm}|}{Die Ausgabe muss die Computenodes im idle Status ausgeben.\newline
PARTITION AVAIL  TIMELIMIT  NODES  STATE NODELIST\newline
normal*      up   infinite      1   idle c[1-45]}\\\hline
\end{tabular}
\caption{Testfall I-001}
\label{Testfall I-001}
\end{table}

\begin{table}[H]
\centering
\begin{tabular}{|p{4cm}|p{4cm}|p{4cm}|p{4cm}|}
\hline
Bezeichnung & \textbf{I-002} & Netzerkboot & Betriebssystem \\\hline
Beschreibung & \multicolumn{3}{p{12cm}|}{Der Managementnode verteilt das Betriebssystem an alle Computenodes} \\\hline
Testvoraussetzung & \multicolumn{3}{p{12cm}|}{Dnsmasq und die Netzerkboot Verzeichnisse sind angelegt. Der Masternode ist erreichbar} \\\hline
Testschritte & \multicolumn{3}{p{12cm}|}{\begin{itemize}
\item Die Computenodes mit Strom versorgen
\item 5 Minuten warten
\item Auf dem Managementnode in der Konsole folgendes eingeben: for ((i=1; i<=45; i++)); do ssh c\$i hostname; done eingeben
\item Abwarten bis alle Computenodes durchgegangen wurden
\end{itemize}} \\ \hline
Erwartetes Ergebnis & \multicolumn{3}{p{12cm}|}{Für jeden Computenode wird der Hostname ausgegeben. c1, c2, c3, ... c45.}\\\hline
\end{tabular}
\caption{Testfall I-002}
\label{Testfall I-002}
\end{table}

\begin{table}[H]
\centering
\begin{tabular}{|p{4cm}|p{4cm}|p{4cm}|p{4cm}|}
\hline
Bezeichnung & \textbf{I-003} & NAS Share & Verbindung \\\hline
Beschreibung & \multicolumn{3}{p{12cm}|}{Der NAS-Share ist auf dem Masternode und den Computenodes angehängt} \\\hline
Testvoraussetzung & \multicolumn{3}{p{12cm}|}{Es ist ein NAS-Share eingerichtet} \\\hline
Testschritte & \multicolumn{3}{p{12cm}|}{\begin{itemize}
\item Auf den Managementnode per SSH verbinden
\item Den Befehl mount eingeben
\item Den Mountpoint auf das NAS auslesen
\item mit cd Dir in das Verzeichnis wechseln
\end{itemize}} \\ \hline
Erwartetes Ergebnis & \multicolumn{3}{p{12cm}|}{Der Mountpoint wird bei der Eingabe von mount angezeigt \newline
Es kann in das Verzeichnis gewechselt werden \newline
Falls bereits Daten auf dem Share vorhanden sind werden diese mit dem Befehl ls -lrtha angezeigt.}\\\hline
\end{tabular}
\caption{Testfall I-003}
\label{Testfall I-003}
\end{table}

\begin{table}[H]
\centering
\begin{tabular}{|p{4cm}|p{4cm}|p{4cm}|p{4cm}|}
\hline
Bezeichnung & \textbf{I-004} & Computenodes Internet & Konnektivität \\\hline
Beschreibung & \multicolumn{3}{p{12cm}|}{Es wird überprüft ob die Computenodes eine Verbindung zum Internet aufbauen können.} \\\hline
Testvoraussetzung & \multicolumn{3}{p{12cm}|}{Managementnode und Computenodes sind gestartet} \\\hline
Testschritte & \multicolumn{3}{p{12cm}|}{Auf jedem Computenode muss in der Kommandozeile ping google.com eingeben werden} \\\hline
Erwartetes Ergebnis & \multicolumn{3}{p{12cm}|}{PING google.com (216.58.205.46) 56(84) bytes of data.
64 bytes from mil04s24-in-f46.1e100.net (216.58.205.46)
64 bytes from mil04s24-in-f46.1e100.net (216.58.205.46)}\\\hline
\end{tabular}
\caption{Testfall I-004}
\label{Testfall I-004}
\end{table}

\begin{table}[H]
\centering
\begin{tabular}{|p{4cm}|p{4cm}|p{4cm}|p{4cm}|}
\hline
Bezeichnung & \textbf{S-001} & NAS Share & Existenz \\\hline
Beschreibung & \multicolumn{3}{p{12cm}|}{Es wird überprüft ob der NAS-Share nebula vorhanden ist} \\\hline
Testvoraussetzung & \multicolumn{3}{p{12cm}|}{Das NAS ist erreichbar und der Share ist eingerichtet} \\\hline
Testschritte & \multicolumn{3}{p{12cm}|}{\begin{itemize}
\item Über den Browser auf dem PC auf http://nasbox:5000/ anmelden
\item Logindaten eingeben
\item Dem Pfad folgen Desktop/File Station/nebula
\end{itemize}} \\ \hline
Erwartetes Ergebnis & \multicolumn{3}{p{12cm}|}{Das Verzeichnis ist vorhanden und kann gelesen und beschrieben werden}\\\hline
\end{tabular}
\caption{Testfall S-001}
\label{Testfall S-001}
\end{table}

\begin{table}[H]
\centering
\begin{tabular}{|p{4cm}|p{4cm}|p{4cm}|p{4cm}|}
\hline
Bezeichnung & \textbf{S-002} & Ausfall Computenode & Ausfallsicherheit \\\hline
Beschreibung & \multicolumn{3}{p{12cm}|}{Es soll getestet werden ob der Cluster weiterhin stabil läuft, nachdem ein Computenode vom Strom entfernt wird} \\\hline
Testvoraussetzung & \multicolumn{3}{p{12cm}|}{Der Cluster ist in Betrieb} \\\hline
Testschritte & \multicolumn{3}{p{12cm}|}{\begin{itemize}
\item Job in Auftrag geben (srun)
\item Die Stromzufuhr eines aktiven Computenodes unterbrechen
\item Den Job mit sinfo oder squeue abrufen
\end{itemize}} \\ \hline
Erwartetes Ergebnis & \multicolumn{3}{p{12cm}|}{Der Job läuft weiter und ist auf einen passiven Node ausgelagert worden}\\\hline
\end{tabular}
\caption{Testfall S-002}
\label{Testfall S-002}
\end{table}

\begin{table}[H]
\centering
\begin{tabular}{|p{4cm}|p{4cm}|p{4cm}|p{4cm}|}
\hline
Bezeichnung & \textbf{S-003} & Slurm & Jobverwaltung \\\hline
Beschreibung & \multicolumn{3}{p{12cm}|}{Es soll getestet werden, ob Slurm richtig konfiguriert wurde und Jobs auf dem Cluster gestartet werden können} \\\hline
Testvoraussetzung & \multicolumn{3}{p{12cm}|}{Managementnode und Computenodes sind gestartet} \\\hline
Testschritte & \multicolumn{3}{p{12cm}|}{\begin{itemize}
\item srun --nodes=40-40 --ntasks=40 --cpus-per-task=4 testslurm.sh in der Kommandozeile eingeben
\item squeue in der Kommandozeile eingeben
\end{itemize}} \\ \hline
Erwartetes Ergebnis & \multicolumn{3}{p{12cm}|}{Das testslrum.sh Script wird in der Queue angezeigt und der Status ist auf Running}\\\hline
\end{tabular}
\caption{Testfall S-003}
\label{Testfall S-003}
\end{table}

\begin{table}[H]
\centering
\begin{tabular}{|p{4cm}|p{4cm}|p{4cm}|p{4cm}|}
\hline
Bezeichnung & \textbf{S-004} & Nagios & Monitoring \\\hline
Beschreibung & \multicolumn{3}{p{12cm}|}{Die Installation und Konfiguration von Nagios soll überprüft werden} \\\hline
Testvoraussetzung & \multicolumn{3}{p{12cm}|}{Nagios ist installiert und Konfiguriert} \\\hline
Testschritte & \multicolumn{3}{p{12cm}|}{\begin{itemize}
\item http://nebula/nagios aufrufen
\item Anmeldedaten eingeben
\item 
\end{itemize}} \\ \hline
Erwartetes Ergebnis & \multicolumn{3}{p{12cm}|}{Das testslrum.sh Script wird in der Queue angezeigt und der Status ist auf Running}\\\hline
\end{tabular}
\caption{Testfall S-004}
\label{Testfall S-004}
\end{table}