% !TEX root = Projektdokumentation.tex
\section{Anhang}
Die folgenden Kapitel gehören zum Anhang und ergänzen den Diplombericht.
% !TEX root = ../Diplombericht.tex
\section{Vorbereitunge RPI's}
\subsection{Betriebssystem installieren}
Für die Installation des CentOS 7.4 ist kein boot fähiger Kernel vorhanden. Das RPI kann aber dennoch mit einem Centos 7.4 betrieben werden. Dafür sind die folgenden Schritte vorzunehmen. Die Installation des Betriebssystems ist von einem Fedora Linux Client aus beschrieben.

1. Gentoo 64 Bit Image herunterladen (https://github.com/sakaki-/gentoo-on-rpi3-64bit/releases/download/v1.2.1/genpi64.img.xz) \newline
2. Das Archiv wird mit dem Fedora Media Writer auf die SD Karte geschrieben. Dies ist anhand der Screenshots beschrieben. \newline
\begin{figure}[H]
	\centering
	\includegraphics[scale=0.3]{Bilder/fmw1.png}
	\caption{FMW: Fedora Media Writer starten}
\end{figure}
\begin{figure}[H]
	\centering
	\includegraphics[scale=0.2]{Bilder/fmw2.png}
	\caption{FMW: Archiv auswählen}
\end{figure}
\begin{figure}[H]
	\centering
	\includegraphics[scale=0.3]{Bilder/fmw3.png}
	\caption{FMW: Abbild schreiben}
\end{figure}
3. Durch das Schreiben des Archivs wurden zwei Partitionen (boot \& rootfs) auf der SD Karte erstellt. Diese werden wiefolgt ausgelesen:
\begin{lstlisting}
[camrein@wifibridge ~]\$ lsblk
NAME MAJ:MIN RM SIZE RO TYPE MOUNTPOINT
sda 8:0 0 238.5G 0 disk 
-sda1 8:1 0 200M 0 part /boot/efi
-sda2 8:2 0 1G 0 part /boot
-sda3 8:3 0 237.3G 0 part 
--fedora-root 253:0 0 50G 0 lvm /
--fedora-swap 253:1 0 7.8G 0 lvm [SWAP]
--fedora-home 253:2 0 179.5G 0 lvm /home
mmcblk0 179:0 0 29.7G 0 disk}
-mmcblk0p1 179:1 0 43.1M 0 part /run/media/camrein/boot
-mmcblk0p2 179:2 0 29.7G 0 part /run/media/camrein/rootfs
[camrein@wifibridge ~]\$
\end{lstlisting}
4. Die Dateisystem Partition muss mit der von CentOS überschrieben werden. Dazu wird die Partition auf dem Linux Client angehängt. Bei Schritt 5-7 wird überprüft ob die Partition wirklich leer ist.
\begin{lstlisting}
[root@wifibridge Downloads]# mkdir mnt
[root@wifibridge Downloads]# mount /dev/mmcblk0p2 mnt
[root@wifibridge Downloads]# cd mnt
[root@wifibridge mnt]# rm -rf *
[root@wifibridge mnt]# ls -lrtha
drwxr-xr-x. 5 camrein camrein 20K 15. Mai 17:39 ..
drwxr-xr-x. 2 root root 4.0K 16. Mai 17:58 .
\end{lstlisting}

Die Dateisystem Partition ist nun leer und kann mit der von CentOS überschrieben werden. \newline
5. Das Dateisystem von Centos beziehen. http://mirror.centos.org/altarch/7.4.1708/isos/aarch64/CentOS-7-aarch64-rootfs-7.4.1708.tar.xz \newline
6. Die geleerte Partition wird nun mit dem Dateisystem von Centos7.4 überschrieben, zugleich soll bei Schritt 2 überprüft werden, ob die Daten wirklich auf die Partition geschrieben wurden.
\begin{lstlisting}
[root@wifibridge mnt]# tar --numeric-owner -xpJf ../CentOS-7-aarch64-rootfs-7.4.1708.tar.xz
[root@wifibridge mnt]# ls -lrtha
insgesamt 84K
drwxr-xr-x. 2 root root 4.0K 23. Nov 2016 srv
drwxr-xr-x. 2 root root 4.0K 23. Nov 2016 opt
drwxr-xr-x. 2 root root 4.0K 23. Nov 2016 mnt
drwxr-xr-x. 2 root root 4.0K 23. Nov 2016 media
drwxr-xr-x. 2 root root 4.0K 23. Nov 2016 home
drwxr-xr-x. 2 root root 4.0K 12. Sep 2017 dev
drwxr-xr-x. 2 root root 4.0K 12. Sep 2017 proc
drwxr-xr-x. 2 root root 4.0K 12. Sep 2017 run
drwxr-xr-x. 2 root root 4.0K 12. Sep 2017 sys
lrwxrwxrwx. 1 root root 7 12. Sep 2017 bin -> usr/bin
lrwxrwxrwx. 1 root root 8 12. Sep 2017 sbin -> usr/sbin
lrwxrwxrwx. 1 root root 9 12. Sep 2017 lib64 -> usr/lib64
lrwxrwxrwx. 1 root root 7 12. Sep 2017 lib -> usr/lib
drwxr-xr-x. 13 root root 4.0K 12. Sep 2017 usr
drwxr-xr-x. 19 root root 4.0K 12. Sep 2017 var
dr-xr-xr-x. 17 root root 4.0K 12. Sep 2017 .
drwxr-xr-x. 82 root root 4.0K 12. Sep 2017 etc
dr-xr-xr-x. 3 root root 4.0K 12. Sep 2017 boot
drwxrwxrwt. 7 root root 4.0K 12. Sep 2017 tmp
dr-xr-x---. 2 root root 4.0K 12. Sep 2017 root
drwxr-xr-x. 5 camrein camrein 20K 15. Mai 17:39 ..
[root@wifibridge mnt]#
\end{lstlisting}

Die SD Karte kann nun mit den RPI's verwendet werden. Diese starten jeweils mit dem Hostnamen centos.


\subsection{RPI für den Netzwerkboot vorbereiten}
Für das Vorbereiten der Clients für den Netzwerkboot wurde der Guide NETWORK BOOT YOU RASPBERRY PI von raspberrypi.org verwendet. (https://www.raspberrypi.org/documentation/hardware/raspberrypi/bootmodes/net\_tutorial.md). Die RPI's werden wie folgt vorbereitet:

1. Die config.txt Datei im /boot Verzeichnis benötigt einen OTP Eintrag, dieser sagt aus, dass das RPI ohne SD Karte nach einem Betriebssystem anfragen soll. 
\begin{lstlisting}
echo program_usb_boot_mode=1 | sudo tee -a /boot/config.txt
\end{lstlisting}
2. RPI neustarten \newline
3. Prüfen ob die Änderung aktiv ist
\begin{lstlisting}
vcgencmd otp_dump | grep 17:
\end{lstlisting}
Erwartetes Ergebnis:
\begin{lstlisting}
17:3020000a
\end{lstlisting}
4. Den Eintrag in der /boot/config.txt wieder entfernen\newline
\textbf{MAC-Adressen auslesen}\newline
Um Zeit zu sparen können alle MAC-Adressen der RPI's während der Vorbereitung der Clients auf den Netzwerkboot ausgelesen werden.\newline
5. nmap Scan auf die IP Range 192.168.1.0-255 von einem Linux Client aus durchführen.
\begin{lstlisting}
	  nmap -sP 192.168.1.0/24  
\end{lstlisting}
Erwartetes Ergebnis für ein RPI:
\begin{lstlisting}
	  Nmap scan report for centos.home (192.168.1.11)
	  Host is up (0.00055s latency).
	  MAC Address: B8:27:EB:32:39:A7 (Raspberry Pi Foundation)   
\end{lstlisting}

Die Vorbereitung der RPI(Computenodes) ist somit abgeschlossen und es kann mit der Installation des Managementnode fortgefahren werden.

\subsection{Hostname und IP der MAC Adresse zuweisen}
über http://internetbox/ wird die IP Adresse sowie der Hostname der MAC Adresse der RPI's zugewiesen. Alle jemals angeschlossenen Geräte werden unter der Geräteliste aufgelistet. Dort können ebenfalls die Hostnamen den Geräten zugewiesen werden.

\begin{figure}[H]
	\centering
	\includegraphics[scale=0.8]{Bilder/hostnamen_mac_edit.png}
	\caption{Hostnamen editieren}
\end{figure}
\begin{figure}[H]
	\centering
	\includegraphics[scale=0.6]{Bilder/hostnamen_mac_overview.png}
	\caption{Übersicht der Hostnamen Zuweisung}
\end{figure}

Die fixe IP Adresse wird unter den Netzwerkeinstellungen des Routers direkt vorgenommen. Hierbei kann den bereits erkannten und definierten RPI's per Hostname eine IP zugewiesen werden.
\begin{figure}[H]
	\centering
	\includegraphics[scale=0.8]{Bilder/ip_mac_edit.png}
	\caption{Statische IP vergeben}
\end{figure}
% !TEX root = ../Diplombericht.tex
\section{Installation}
\subsection{Vorbereiten der Raspberry PI's}
Die RPI's müssen für den Netzwerkboot eine Anfrage an das Netzwerk senden. Dafür müssen diese jeweils einen OTP Eintrag für den Start des RPI's verwenden. Als Vorarbeiten für den PXE Boot sind folgende Schritte vorzunehmen.

\begin{enumerate}
      \item Betriebssystem
      \begin{enumerate}
         \item Zuerst muss über die offizielle RPI Webseite das NOOBS Betriebssystem heruntergeladen werden.
         \item Mit dem Win32DiskImager Tool wird das NOOBS Betriebssystem auf die SD Karte geschrieben.
         \item Die SD Karte muss in den dafür vorgesehenen RPI Slot eingeführt werden
         \item Das Raspberry muss über HDMI an einen Monitor angeschlossen sein und zugleich ist es notwendig, dass ein Patchkabel für die Netzwerkverbindung sowie eine Tastatur und Maus angeschlossen ist.
         \item Sobald das RPI gestartet ist, wird der SD Karte der Hostname testrpi zugewiesen und der SSH Zugriff aktiviert. Dies kann über ein Menu eingerichtet werden. Das Menü wird mit folgendem Befehl aufgerufen:
         \begin{verbatim}
pi@raspberry ~ $ sudo raspi-config
\end{verbatim}
       \end{enumerate}
      \item Die SD Karte ist nun bereit und es kann für jedes einzelne RPI folgende Konfiguration vorgenommen werden. Der folgende Ablauf schreibt voraus, dass die RPI's mit einem Patchkabel und Strom versorgt sind.
      \begin{enumerate}
      \item RPI starten (Mit SD Karte)
      \item Von einem PC oder Laptop via SSH auf testrpi verbinden.
      \item Auf dem RPI den USB Boot Modus aktivieren
      \begin{verbatim}
      echo program_usb_boot_mode=1 | sudo tee -a /boot/config.txt
      \end{verbatim}
      \item Die Hardware Adresse der RPI's kann ab diesem Zeitpunkt von einem anderen Linux Client aus bereits mit dem folgenden Befehl ausgelesen werden. Dieser wird für die IP und Hostnamenzuweisung via Router und dnsmasq verwendet.
      \begin{verbatim}
	  nmap -sP 192.168.1.0/24
      \end{verbatim}
      \item Das RPI neustarten und den erstellten OTP Eintrag testen.
      \begin{verbatim}
      vcgencmd otp_dump | grep 17:
      \end{verbatim}
      Es wird die Ausgabe 17:3020000a erwartet.
      \item Den Eintrag aus dem /boot/config.txt wieder entfernen.
      \item Das RPI frägt nun im Netzwerk bei einem Start nach einem Betriebssystem an.
      \end{enumerate}
   \end{enumerate}

% Literatur ------------------------------------------------------------------
\clearpage
\renewcommand{\refname}{Literaturverzeichnis}
\subsection{Quellenverzeichnis}
\bibliography{Bibliographie}
\bibliographystyle{Allgemein/natdin} % DIN-Stil des Literaturverzeichnisses
\newpage
\includepdf[pages=1,scale=.8,pagecommand={\section{Diplomeingabe}\label{pdf:Diplomeingabe}},linktodoc=true]{Anhang/Diplomarbeit_Eingabe.pdf}
\includepdf[pages={2,3},scale=0.9,pagecommand={},linktodoc=true]{Anhang/Diplomarbeit_Eingabe.pdf}
\section{Testprotokoll}
% !TEX root = ../Diplombericht.tex

\subsection{Testobjekte}
\begin{table}[H]
\centering
\begin{tabular}{p{1cm}p{3cm}p{3cm}p{4cm}p{4cm}}
\hline
\rowcolor{heading} \textbf{Nr.} & \textbf{Objekt} & \textbf{Hostname} & \textbf{IP} & \textbf{MAC} \\\hline
1 & Managementnode & nebula & 192.168.1.11 & B8:27:EB:32:39:A7\\\hline
2 & Computenode 1 & c1 & 192.168.1.11 & B8:27:EB:32:39:A7\\\hline
3 & Computenode 2 & c2 & 192.168.1.12 & B8:27:EB:2E:A3:D1\\\hline
4 & Computenode 3 & c3 & 192.168.1.13 & B8:27:EB:50:45:3F\\\hline
5 & Computenode 4 & c4 & 192.168.1.14 & B8:27:EB:0D:E6:25\\\hline
6 & Computenode 5 & c5 & 192.168.1.15 & B8:27:EB:3E:96:B5\\\hline
7 & Computenode 6 & c6 & 192.168.1.16 & B8:27:EB:EE:77:DA\\\hline
8 & Computenode 7 & c7 & 192.168.1.17 & B8:27:EB:21:63:E6\\\hline
9 & Computenode 8 & c8 & 192.168.1.18 & B8:27:EB:2E:2E:CC\\\hline
10 & Computenode 9 & c9 & 192.168.1.19 & B8:27:EB:17:32:96\\\hline
11 & Computenode 10 & c10 & 192.168.1.20 & B8:27:EB:B2:1C:A9\\\hline
12 & Computenode 11 & c11 & 192.168.1.21 & B8:27:EB:AF:63:1F\\\hline
13 & Computenode 12 & c12 & 192.168.1.22 & B8:27:EB:43:00:2C\\\hline
14 & Computenode 13 & c13 & 192.168.1.23 & B8:27:EB:13:7B:18\\\hline
15 & Computenode 14 & c14 & 192.168.1.24 & B8:27:EB:43:CD:29\\\hline
16 & Computenode 15 & c15 & 192.168.1.25 & B8:27:EB:FF:C7:56\\\hline
17 & Computenode 16 & c16 & 192.168.1.26 & B8:27:EB:CE:98:66\\\hline
18 & Computenode 17 & c17 & 192.168.1.27 & B8:27:EB:5D:63:34\\\hline
19 & Computenode 18 & c18 & 192.168.1.28 & B8:27:EB:91:3E:0F\\\hline
20 & Computenode 19 & c19 & 192.168.1.29 & B8:27:EB:F4:65:EC\\\hline
21 & Computenode 20 & c20 & 192.168.1.30 & B8:27:EB:3E:AB:DC\\\hline
22 & Computenode 21 & c21 & 192.168.1.31 & B8:27:EB:66:60:F6\\\hline
23 & Computenode 22 & c22 & 192.168.1.32 & B8:27:EB:37:3F:74\\\hline
24 & Computenode 23 & c23 & 192.168.1.33 & B8:27:EB:18:5E:F0\\\hline
25 & Computenode 24 & c24 & 192.168.1.34 & B8:27:EB:B0:23:B8\\\hline
26 & Computenode 25 & c25 & 192.168.1.35 & B8:27:EB:BE:C4:94\\\hline
27 & Computenode 26 & c26 & 192.168.1.36 & B8:27:EB:FB:FF:57\\\hline
28 & Computenode 27 & c27 & 192.168.1.37 & B8:27:EB:4E:EC:CE\\\hline
29 & Computenode 28 & c28 & 192.168.1.38 & B8:27:EB:43:1C:35\\\hline
30 & Computenode 29 & c29 & 192.168.1.39 & B8:27:EB:DC:74:5F\\\hline
31 & Computenode 30 & c30 & 192.168.1.40 & B8:27:EB:D1:DE:2F\\\hline
32 & Computenode 31 & c31 & 192.168.1.41 & B8:27:EB:5E:90:34\\\hline
33 & Computenode 32 & c32 & 192.168.1.42 & B8:27:EB:DE:80:24\\\hline
34 & Computenode 33 & c33 & 192.168.1.43 & B8:27:EB:A4:79:6F\\\hline
35 & Computenode 34 & c34 & 192.168.1.44 & B8:27:EB:0A:4D:C7\\\hline
36 & Computenode 35 & c35 & 192.168.1.45 & B8:27:EB:5C:53:5F\\\hline
37 & Computenode 36 & c36 & 192.168.1.46 & B8:27:EB:F7:AF:C2\\\hline
38 & Computenode 37 & c37 & 192.168.1.47 & B8:27:EB:CE:BA:ED\\\hline
39 & Computenode 38 & c38 & 192.168.1.48 & B8:27:EB:59:38:3C\\\hline
40 & Computenode 39 & c39 & 192.168.1.49 & B8:27:EB:99:BB:8E\\\hline
41 & Computenode 40 & c40 & 192.168.1.50 & B8:27:EB:8F:7A:0D\\\hline
42 & Reservenode 1 & c41 & 192.168.1.51 & B8:27:EB:DE:C9:69\\\hline
43 & Reservenode 2 & c42 & 192.168.1.52 & B8:27:EB:7E:6F:48\\\hline
44 & Reservenode 3 & c43 & 192.168.1.53 & B8:27:EB:5D:DD:FE\\\hline
45 & Reservenode 4 & c44 & 192.168.1.54 & B8:27:EB:A6:6D:4D\\\hline
46 & Reservenode 5 & c45 & 192.168.1.55 & B8:27:EB:0C:63:10\\\hline
\end{tabular}
\caption{Computenode Namen}
\end{table}

\subsection{Arbeitsjournal}
\label{app:Arbeitsjournal}

\tabelleAnhang{Arbeitsjournal}

\subsection{Protkolle}
\label{app:Protokolle}


%\subsection{Detaillierte Zeitplanung}
%\label{app:Zeitplanung}

%\tabelleAnhang{ZeitplanungKomplett}

%\input{Anhang/AnhangLastenheft.tex}
\clearpage

%\subsection{Use Case-Diagramm}
%\label{app:UseCase}
%Use Case-Diagramme und weitere \acs{UML}-Diagramme kann man auch direkt mit \LaTeX{} zeichnen, siehe \zB %%%%%\url{http://metauml.sourceforge.net/old/usecase-diagram.html}.
%\begin{figure}[htb]
%\centering
%\includegraphicsKeepAspectRatio{UseCase.pdf}{0.7}
%\caption{Use Case-Diagramm}
%\end{figure}

%\input{Anhang/AnhangPflichtenheft.tex}

%\subsection{Datenbankmodell}
%\label{app:Datenbankmodell}
%ER-Modelle kann man auch direkt mit \LaTeX{} zeichnen, siehe \zB \url{http://www.texample.net/tikz/examples/entity-relationship-diagram/}.
%\begin{figure}[htb]
%\centering
%\includegraphicsKeepAspectRatio{database.pdf}{1}
%\caption{Datenbankmodell}
%\end{figure}
%\clearpage

%\input{Anhang/AnhangEntwuerfe.tex}
%\clearpage
%\input{Anhang/AnhangScreenshots.tex}
%\input{Anhang/AnhangDoc.tex}
%\clearpage
%\input{Anhang/AnhangTest.tex}

%\subsection{Klasse: ComparedNaturalModuleInformation}
%\label{app:CNMI}
%Kommentare und simple Getter/Setter werden nicht angezeigt.
%\lstinputlisting[language=php, caption={Klasse: ComparedNaturalModuleInformation}]{Listings/cnmi.php}
%\clearpage

%\subsection{Klassendiagramm}
%\label{app:Klassendiagramm}
%Klassendiagramme und weitere \acs{UML}-Diagramme kann man auch direkt mit \LaTeX{} zeichnen, siehe \zB %\url{http://metauml.sourceforge.net/old/class-diagram.html}.
%\begin{figure}[htb]
%\centering
%\includegraphicsKeepAspectRatio{Klassendiagramm.pdf}{1}
%\caption{Klassendiagramm}
%\end{figure}
%\clearpage

\subsection{Datenblätter}
\label{app:Datenblätter}

\subsection{Produktinformationen}
\label{app:Produktinformationen}


\input{Anhang/AnhangBenutzerDoku.tex}
