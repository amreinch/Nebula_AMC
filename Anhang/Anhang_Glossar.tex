% !TEX root = ../Diplombericht.tex
\addcontentsline{toc}{section}{Glossar}
\section*{Glossar}
\textbf{API - Application Programming Interface} \newline
Schnittstelle zu einem Webservice, welcher Daten für eine Verarbeitung liefert.

\textbf{Cluster}\newline
Ein Cluster ist ein Verbund aus mehreren Recheneinheiten. In diesem Projekt ist der Cluster bestehend aus dem Management Node und den Compute Nodes.

\textbf{Compute Nodes}\newline
Compute Nodes sind die Ressourcen des Clusters. Auf diesen werden die Aufgaben des Clusters ausgeführt.

\textbf{Confluence}\newline
Dokumentations- und Ablagetool für die kollaborative Arbeit innerhalb eines Teams. Kann aber auch für Einzelanwender benutzt werden. 

\textbf{CPU - Central Processing Unit}\newline
Prozessor der Computer.

\textbf{DHCP - Dynamic Host Configuration Protocol}\newline
Kommunikationsprotokoll, welches für die Zuweisung von Netzwerkkonfigurationen an Clients benutzt wird.

\textbf{DHCPD - Dynamic Host Configuration Protocol Daemon}\newline
DHCPD ist ein Serverdaemon, welcher die Kommunikation mit DHCP ermöglicht.

\textbf{DNS - Domain Name System}\newline
Ist für die Namensauflösung auf IP Adressen und umgekehrt zuständig.

\textbf{GND - Ground}\newline
Wird in der Elektronik verwendet und dient als leitender Körper, welcher ein Potential von NULL zugewiesen hat. 

\textbf{GPIO - General Purpose Input/Output}\newline
Während des Projektes werden die Raspberry PI's über die GPIO Pins mit Strom versorgt. Kann aber generell auch Steuerungsbefehle entgegennehmen und dient als Hardware Schnittstelle.

\textbf{GPU - Graphics Processing Unit}\newline
Grafikprozessor der Grafikkarte.

\textbf{GUI - Graphical User Interface}\newline
Grafische Benutzeroberfläche für das Bedienen von Programmen.

\textbf{HERMES}\newline
HERMES ist eine Projektvorgehensweise.

\textbf{Hostname}\newline
Eindeutiger Name eines Gerätes in einem Netzwerk.

\textbf{HPC - High-Performance-Computing}\newline
Ist ein allgemeiner Begriff für das Hochleistungsrechnen in einem Cluster.

\textbf{HTTP - Hypertext Transfer Protocol}\newline
Zustandsloses Protokoll, welches für die Übertragung von Daten auf der Anwendungsschicht über ein Netzwerk verwendet wird.

\textbf{Interface}\newline
Ist eine Schnittstelle, welche bei IT-Systemen für den Datenaustausch steht.

\textbf{IP-Adresse}\newline
Die IP-Adresse ist eine eindeutige Adresse im Netzwerk. Durch diese können Datenpakete adressiert werden. 

\textbf{Kernel}\newline
Der Kernel ist der zentrale Bestandteil eines Betriebssystems. Die Prozess- und Datenorganisation ist darin festgelet.

\textbf{Kryptowährung}\newline
Umfasst virtuelle Währungen, welche für Bezahlungen über das Internet angedacht sind.

\textbf{LED - Light Emitting Diode}\newline
Eine Leuchtdiode wird mit Strom versorgt. Sie dient meistens als Statusanzeige auf Hardware-Ebene.

\textbf{MAC-Adresse}\newline 
Eindeutige Hardware Adresse eines beliebigen Gerätes.

\textbf{Management Node}\newline
Innerhalb des Clusters ist der Management Node die zentrale Komponente. Über den Management Node werden die Compute Nodes gesteuert.

\textbf{Mining}\newline
Der Ausdruck wird oft für das Schürfen, Generieren oder Abbauen von Kryptowährungen benutzt.

\textbf{Monitoring}\newline
Überwachung von Systemen und Diensten.

\textbf{Mounten}\newline
Anhängen von Netzwerkverzeichnissen auf Systemen.

\textbf{MPI - Message Passing Interface}\newline
Ist ein Standard für den Nachrichtenaustausch bei parallelen Berechnungen auf verschiedenen Systemen. 

\textbf{NAS - Network Attached Storage}\newline
Ist ein Netzwerkverzeichnis, welches auf Systemen eingebunden werden kann. Dient als gemeinsame Ablage von wichtigen Daten. Die Daten werden oft redundant gesichert. 

\textbf{Netzwerkboot}\newline
Das Betriebssystem wird über das Netzwerk an Geräte verteilt.

\textbf{NFS - Network File System}\newline
Ist ein Protokoll, welches den Zugriff auf Dateien und Verzeichnisse über das Netzwerk ermöglicht.

\textbf{NTP - Network Time Protocol}\newline
Ist ein Protokol,l welches zur Zeit-Synchronisierung dient. Dadurch werden Zeiten von Servern und Clients synchronisiert.

\textbf{OTP - One Time Programmable}\newline
Ist ein einmalig programmierbarer Eintrag. Er wird für den Netzwerkboot benötigt, damit die Compute Nodes nach einem Betriebssystem über das Netzwerk anfragen.

\textbf{Patch}\newline
Kleinere Veränderungen an Programmen und Paketen.

\textbf{RAID - Redundant Array of Independent Disks}\newline
Redundante Anordnung unabhängiger Festplatten.

\textbf{RAM - Random Access Memory}\newline
Arbeitsspeicher eines Servers oder Clients.

\textbf{Release}\newline
Veröffentlichung eines Programmes.

\textbf{Repository}\newline
Verwaltetes digitales Verzeichnis für das Speichern von Daten.

\textbf{RPM - Red Hat Packet Manager} \newline
Paket Manager von Red Hat.

\textbf{RPI - Raspberry PI}\newline
Mini Computer, wird oft für private Projekte oder schulische Ausbildungen eingesetzt.

\textbf{SLURM - Simple Linux Utility for Resource Management}\newline
Dient der Verwaltung von Ressourcen und Aufgaben über mehrere Systeme.

\textbf{SSH - Secure Shell}\newline
Protokoll für die sichere Verbindung auf Linux Systeme.

\textbf{SMTP - Simple Mail Transfer Protocol}\newline
Protokoll, welches in Zusammenhang mit dem Mailverkehr steht.

\textbf{Switch}\newline
Netzwerkverteiler für die Kommunikation unter den angeschlossenen Geräten.

\textbf{TCP - Transmission Control Protocol}\newline
Verbindungsorientiertes Netzwerkprotokoll.

\textbf{TFTP - Trivial File Transfer Protocol}\newline
Einfaches Datenübertragungsprotokoll.

\textbf{Token}\newline
Bezeichnet eine Währungseinheit.

\textbf{UDP - User Datagram Protocol}\newline
Verbindungsloses Netzwerkprotokoll.

\textbf{Wallets}\newline
Brieftasche für Kryptowährungen.



