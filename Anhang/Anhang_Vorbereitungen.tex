% !TEX root = ../Diplombericht.tex
\section{Vorbereitunge RPI's}
\subsection{Betriebssystem installieren}
Für die initialen Arbeiten der RPI's wird ein NOOBS Betriebssystem verwendet. Dieses wird wie folgt bezogen und installiert.

1. Das NOOBS Abbild von https://www.raspberrypi.org/downloads/ herunterladen.\newline
2. Mit dem Win32DiskImager Tool auf dem PC das heruntergeladene Abbild auf die SD Karte schreiben.

\subsection{Betriebssystem \& SD Karte konfigurieren}
Die RPI's werden über SSH und Hostnamen angesprochen. Beides wird über den Konfigurationsassistenten von NOOBS konfiguriert. Als Voraussetzung wird ein RPI benötigt, welches mit Strom beliefert wird und via HDMI-Kabel ein Bild auf den Monitor liefert. Zudem wird noch eine Tastatur benötigt welche direkt am RPI angeschlossen ist. Das Konfigurationsmenü wird wiefolgt aufgerufen:
\begin{lstlisting}
pi@raspberry ~ $ sudo raspi-config
\end{lstlisting}
Bei der Frage nach dem Hostnamen wird der Name testrpi eingegeben und der SSH Dienst wird aktiviert.

\subsection{RPI für den Netzwerkboot vorbereiten}
Für das Vorbereiten der Clients für den Netzwerkboot wurde der Guide NETWORK BOOT YOU RASPBERRY PI von raspberrypi.org verwendet. (https://www.raspberrypi.org/documentation/hardware/raspberrypi/bootmodes/net\_tutorial.md). Die RPI's werden wie folgt vorbereitet:

1. SD Karte in das RPI einschieben \newline
2. RPI starten \newline
3. Die config.txt Datei im /boot Verzeichnis benötigt einen OTP Eintrag, dieser sagt aus, dass das RPI ohne SD Karte nach einem Betriebssystem anfragen soll. 
\begin{lstlisting}
echo program_usb_boot_mode=1 | sudo tee -a /boot/config.txt
\end{lstlisting}
4. RPI neustarten \newline
5. Prüfen ob die Änderung aktiv ist
\begin{lstlisting}
vcgencmd otp_dump | grep 17:
\end{lstlisting}
Erwartetes Ergebnis:
\begin{lstlisting}
17:3020000a
\end{lstlisting}
6. Den Eintrag in der /boot/config.txt wieder entfernen\newline
\textbf{MAC-Adressen auslesen}\newline
Um Zeit zu sparen können alle MAC-Adressen der RPI's während der Vorbereitung der Clients auf den Netzwerkboot ausgelesen werden.\newline
7. nmap Scan auf die IP Range 192.168.1.0-255 von einem Linux Client aus durchführen.
\begin{lstlisting}
	  nmap -sP 192.168.1.0/24  
\end{lstlisting}
Erwartetes Ergebnis für ein RPI:
\begin{lstlisting}
	  Nmap scan report for testrpi.home (192.168.1.11)
	  Host is up (0.00055s latency).
	  MAC Address: B8:27:EB:32:39:A7 (Raspberry Pi Foundation)   
\end{lstlisting}

Die Vorbereitung der RPI(Computenodes) ist somit abgeschlossen und es kann mit der Installation des Managementnode fortgefahren werden.