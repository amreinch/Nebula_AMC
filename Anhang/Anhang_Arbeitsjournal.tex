% !TEX root = ../Diplombericht.tex
\section{Arbeitsjournal}
\begin{longtable}{|p{5cm}|p{5cm}|p{6cm}|}
\hline
\rowcolor{heading}\textbf{Tag:} 1 & \textbf{Datum:} 05.02.2018 & \textbf{Aufwand:} \\ \hline
\textbf{Erledigte Arbeit} & \multicolumn{2}{p{11cm}|}{- Es wurde ein Share für die Ablage der entstehenden Dokumente und Arbeiten erstellt \newline
- Projektinitialisierungsauftrag wurde geschrieben \newline
- Kick-Off Meeting geplant und durchgeführt} \\ \hline
\textbf{Aufgetretene Probleme} & \multicolumn{2}{p{11cm}|}{-} \\ \hline
\rowcolor{heading}\textbf{Tag:} 2 & \textbf{Datum:} 06.02.2018 & \textbf{Aufwand:} \\ \hline
\textbf{Erledigte Arbeit} & \multicolumn{2}{p{11cm}|}{Confluence für das Projekt eingerichtet (Wurde am Kick-Off Meeting entschieden) \newline
- Protokoll des Kick-Off Meetings verfeinert \newline
- Projektinitialisierungsauftrag gemäss Feedback aus Kick-Off Meeting angepasst \newline
- Projektnamen definiert \newline
- Beginn mit der Projektplanung} \\ \hline
\textbf{Aufgetretene Probleme} & \multicolumn{2}{p{11cm}|}{-} \\ \hline
\rowcolor{heading}\textbf{Tag:} 3 & \textbf{Datum:} 07.02.2018 & \textbf{Aufwand:} \\ \hline
\textbf{Erledigte Arbeit} & \multicolumn{2}{p{11cm}|}{- Zusammenfassung des Kick-Off Meetings an die Anwesenden versendet \newline
- Administrative Arbeiten (Organisation)} \\ \hline
\textbf{Aufgetretene Probleme} & \multicolumn{2}{p{11cm}|}{-} \\ \hline
\rowcolor{heading}\textbf{Tag:} 4 & \textbf{Datum:} 08.02.2018 & \textbf{Aufwand:} \\ \hline
\textbf{Erledigte Arbeit} & \multicolumn{2}{p{11cm}|}{- An der Projektplanung gearbeitet} \\ \hline
\textbf{Aufgetretene Probleme} & \multicolumn{2}{p{11cm}|}{-} \\ \hline
\rowcolor{heading}\textbf{Tag:} 5 & \textbf{Datum:} 09.02.2018 & \textbf{Aufwand:} \\ \hline
\textbf{Erledigte Arbeit} & \multicolumn{2}{p{11cm}|}{-} \\ \hline
\textbf{Aufgetretene Probleme} & \multicolumn{2}{p{11cm}|}{-} \\ \hline
\rowcolor{heading}\textbf{Tag:} 6 & \textbf{Datum:} 10.02.2018 & \textbf{Aufwand:} \\ \hline
\textbf{Erledigte Arbeit} & \multicolumn{2}{p{11cm}|}{-} \\ \hline
\textbf{Aufgetretene Probleme} & \multicolumn{2}{p{11cm}|}{-} \\ \hline
\rowcolor{heading}\textbf{Tag:} 7 & \textbf{Datum:} 11.02.2018 & \textbf{Aufwand:} \\ \hline
\textbf{Erledigte Arbeit} & \multicolumn{2}{p{11cm}|}{-} \\ \hline
\textbf{Aufgetretene Probleme} & \multicolumn{2}{p{11cm}|}{-} \\ \hline
\rowcolor{heading}\textbf{Tag:} 8 & \textbf{Datum:} 12.02.2018 & \textbf{Aufwand:} \\ \hline
\textbf{Erledigte Arbeit} & \multicolumn{2}{p{11cm}|}{- Terminplan und Projektübersicht im Confluence erstellen \newline
- An der Projektplanung weitergearbeitet \newline
- Beginn mit dem Projektauftrag} \\ \hline
\textbf{Aufgetretene Probleme} & \multicolumn{2}{p{11cm}|}{-} \\ \hline
\rowcolor{heading}\textbf{Tag:} 9 & \textbf{Datum:} 13.02.2018 & \textbf{Aufwand:} \\ \hline
\textbf{Erledigte Arbeit} & \multicolumn{2}{p{11cm}|}{- Am Projektauftrag weitergearbeitet} \\ \hline
\textbf{Aufgetretene Probleme} & \multicolumn{2}{p{11cm}|}{-} \\ \hline
\rowcolor{heading}\textbf{Tag:} 10 & \textbf{Datum:} 14.02.2018 & \textbf{Aufwand:} \\ \hline
\textbf{Erledigte Arbeit} & \multicolumn{2}{p{11cm}|}{-} \\ \hline
\textbf{Aufgetretene Probleme} & \multicolumn{2}{p{11cm}|}{-} \\ \hline
\rowcolor{heading}\textbf{Tag:} 11 & \textbf{Datum:} 15.02.2018 & \textbf{Aufwand:} \\ \hline
\textbf{Erledigte Arbeit} & \multicolumn{2}{p{11cm}|}{- Projektauftrag fertiggestellt \newline
- Administrative Arbeiten} \\ \hline
\textbf{Aufgetretene Probleme} & \multicolumn{2}{p{11cm}|}{-} \\ \hline
\rowcolor{heading}\textbf{Tag:} 12 & \textbf{Datum:} 16.02.2018 & \textbf{Aufwand:} \\ \hline
\textbf{Erledigte Arbeit} & \multicolumn{2}{p{11cm}|}{- Krank} \\ \hline
\textbf{Aufgetretene Probleme} & \multicolumn{2}{p{11cm}|}{-} \\ \hline
\rowcolor{heading}\textbf{Tag:} 13 & \textbf{Datum:} 17.02.2018 & \textbf{Aufwand:} \\ \hline
\textbf{Erledigte Arbeit} & \multicolumn{2}{p{11cm}|}{- Krank} \\ \hline
\textbf{Aufgetretene Probleme} & \multicolumn{2}{p{11cm}|}{-} \\ \hline
\rowcolor{heading}\textbf{Tag:} 14 & \textbf{Datum:} 18.02.2018 & \textbf{Aufwand:} \\ \hline
\textbf{Erledigte Arbeit} & \multicolumn{2}{p{11cm}|}{- Krank} \\ \hline
\textbf{Aufgetretene Probleme} & \multicolumn{2}{p{11cm}|}{-} \\ \hline
\rowcolor{heading}\textbf{Tag:} 15 & \textbf{Datum:} 19.02.2018 & \textbf{Aufwand:} \\ \hline
\textbf{Erledigte Arbeit} & \multicolumn{2}{p{11cm}|}{- Krank} \\ \hline
\textbf{Aufgetretene Probleme} & \multicolumn{2}{p{11cm}|}{-} \\ \hline
\rowcolor{heading}\textbf{Tag:} 16 & \textbf{Datum:} 20.02.2018 & \textbf{Aufwand:} \\ \hline
\textbf{Erledigte Arbeit} & \multicolumn{2}{p{11cm}|}{- Krank} \\ \hline
\textbf{Aufgetretene Probleme} & \multicolumn{2}{p{11cm}|}{-} \\ \hline
\rowcolor{heading}\textbf{Tag:} 17 & \textbf{Datum:} 21.02.2018 & \textbf{Aufwand:} \\ \hline
\textbf{Erledigte Arbeit} & \multicolumn{2}{p{11cm}|}{- Krank} \\ \hline
\textbf{Aufgetretene Probleme} & \multicolumn{2}{p{11cm}|}{-} \\ \hline
\rowcolor{heading}\textbf{Tag:} 18 & \textbf{Datum:} 22.02.2018 & \textbf{Aufwand:} \\ \hline
\textbf{Erledigte Arbeit} & \multicolumn{2}{p{11cm}|}{- Krank} \\ \hline
\textbf{Aufgetretene Probleme} & \multicolumn{2}{p{11cm}|}{-} \\ \hline
\rowcolor{heading}\textbf{Tag:} 19 & \textbf{Datum:} 23.02.2018 & \textbf{Aufwand:} \\ \hline
\textbf{Erledigte Arbeit} & \multicolumn{2}{p{11cm}|}{- Krank} \\ \hline
\textbf{Aufgetretene Probleme} & \multicolumn{2}{p{11cm}|}{-} \\ \hline
\rowcolor{heading}\textbf{Tag:} 20 & \textbf{Datum:} 24.02.2018 & \textbf{Aufwand:} \\ \hline
\textbf{Erledigte Arbeit} & \multicolumn{2}{p{11cm}|}{- Krank \newline
- Mit der Studie begonnen \newline
- Informationsbeschaffung der Varianten} \\ \hline
\textbf{Aufgetretene Probleme} & \multicolumn{2}{p{11cm}|}{-} \\ \hline
\rowcolor{heading}\textbf{Tag:} 21 & \textbf{Datum:} 25.02.2018 & \textbf{Aufwand:} \\ \hline
\textbf{Erledigte Arbeit} & \multicolumn{2}{p{11cm}|}{- Studie ausarbeiten \newline
- Informationsbeschaffung \newline
- IST Zustand beschreiben \newline
- Ziele beschreiben} \\ \hline
\textbf{Aufgetretene Probleme} & \multicolumn{2}{p{11cm}|}{-} \\ \hline
\rowcolor{heading}\textbf{Tag:} 22 & \textbf{Datum:} 26.02.2018 & \textbf{Aufwand:} \\ \hline
\textbf{Erledigte Arbeit} & \multicolumn{2}{p{11cm}|}{- Studie abschliessen \newline
- Ziele erweitern \newline
- Varianten beschreiben \newline
- Variantenentscheid fällen \newline
- Empfehlung der Variante beschreiben} \\ \hline
\textbf{Aufgetretene Probleme} & \multicolumn{2}{p{11cm}|}{-} \\ \hline
\rowcolor{heading}\textbf{Tag:} 23 & \textbf{Datum:} 27.02.2018 & \textbf{Aufwand:} \\ \hline
\textbf{Erledigte Arbeit} & \multicolumn{2}{p{11cm}|}{- Projektauftrag mit den neuen Erkenntnissen der Studie ergänzen} \\ \hline
\textbf{Aufgetretene Probleme} & \multicolumn{2}{p{11cm}|}{-} \\ \hline
\rowcolor{heading}\textbf{Tag:} 24 & \textbf{Datum:} 28.02.2018 & \textbf{Aufwand:} \\ \hline
\textbf{Erledigte Arbeit} & \multicolumn{2}{p{11cm}|}{- Projektauftrag fertigstellen \newline
- Studie fertigstellen
- Zwischenmeeting vorbereiten} \\ \hline
\textbf{Aufgetretene Probleme} & \multicolumn{2}{p{11cm}|}{Das vorgesehene Zeitkontingent der Studie wurde überschossen} \\ \hline
\rowcolor{heading}\textbf{Tag:} 25 & \textbf{Datum:} 01.03.2018 & \textbf{Aufwand:} \\ \hline
\textbf{Erledigte Arbeit} & \multicolumn{2}{p{11cm}|}{- Zwischenmeeting vorbereiten \newline
- Finale Version Projektauftrag \newline
- Finale Version Studie \newline
- Powerpoint Präsentation (Agenda) erstellen \newline
- Projektplan updaten } \\ \hline
\textbf{Aufgetretene Probleme} & \multicolumn{2}{p{11cm}|}{-} \\ \hline
\rowcolor{heading}\textbf{Tag:} 26 & \textbf{Datum:} 02.03.2018 & \textbf{Aufwand:} \\ \hline
\textbf{Erledigte Arbeit} & \multicolumn{2}{p{11cm}|}{-} \\ \hline
\textbf{Aufgetretene Probleme} & \multicolumn{2}{p{11cm}|}{-} \\ \hline
\rowcolor{heading}\textbf{Tag:} 27 & \textbf{Datum:} 03.03.2018 & \textbf{Aufwand:} \\ \hline
\textbf{Erledigte Arbeit} & \multicolumn{2}{p{11cm}|}{-} \\ \hline
\textbf{Aufgetretene Probleme} & \multicolumn{2}{p{11cm}|}{-} \\ \hline
\rowcolor{heading}\textbf{Tag:} 28 & \textbf{Datum:} 04.03.2018 & \textbf{Aufwand:} \\ \hline
\textbf{Erledigte Arbeit} & \multicolumn{2}{p{11cm}|}{-} \\ \hline
\textbf{Aufgetretene Probleme} & \multicolumn{2}{p{11cm}|}{-} \\ \hline
\rowcolor{heading}\textbf{Tag:} 29 & \textbf{Datum:} 05.03.2018 & \textbf{Aufwand:} \\ \hline
\textbf{Erledigte Arbeit} & \multicolumn{2}{p{11cm}|}{-} \\ \hline
\textbf{Aufgetretene Probleme} & \multicolumn{2}{p{11cm}|}{-} \\ \hline
\rowcolor{heading}\textbf{Tag:} 30 & \textbf{Datum:} 06.03.2018 & \textbf{Aufwand:} \\ \hline
\textbf{Erledigte Arbeit} & \multicolumn{2}{p{11cm}|}{-} \\ \hline
\textbf{Aufgetretene Probleme} & \multicolumn{2}{p{11cm}|}{-} \\ \hline
\rowcolor{heading}\textbf{Tag:} 31 & \textbf{Datum:} 07.03.2018 & \textbf{Aufwand:} \\ \hline
\textbf{Erledigte Arbeit} & \multicolumn{2}{p{11cm}|}{Erneute Informationsbeschaffung. Es sind Probleme bei der Installation der Variante OpenHPC aufgetreten. Als Zwischenlösung habe ich entschieden einen Laptop der keinen AMD Prozessor verwendet zu benutzen, da dort das Produkt einfach installiert werden kann. Die Computenodes sollen aber weiterhin via Raspberry PI betrieben werden} \\ \hline
\textbf{Aufgetretene Probleme} & \multicolumn{2}{p{11cm}|}{Zuwenige Informationen über OpenHPC gesammelt} \\ \hline
\rowcolor{heading}\textbf{Tag:} 32 & \textbf{Datum:} 08.03.2018 & \textbf{Aufwand:} \\ \hline
\textbf{Erledigte Arbeit} & \multicolumn{2}{p{11cm}|}{Hostnamen und IP's definiert und in Konzept aufgenommen} \\ \hline
\textbf{Aufgetretene Probleme} & \multicolumn{2}{p{11cm}|}{-} \\ \hline
\rowcolor{heading}\textbf{Tag:} 33 & \textbf{Datum:} 09.03.2018 & \textbf{Aufwand:} \\ \hline
\textbf{Erledigte Arbeit} & \multicolumn{2}{p{11cm}|}{- Ich habe versucht verschiedene CentOS Images auf die Raspberry PI's zu installieren, keines hat funktioniert. Es wird weiterhin nach einer alternative gesucht} \\ \hline
\textbf{Aufgetretene Probleme} & \multicolumn{2}{p{11cm}|}{CentOS Images können nicht auf Raspberry PI's installiert werden.} \\ \hline
\rowcolor{heading}\textbf{Tag:} 34 & \textbf{Datum:} 10.03.2018 & \textbf{Aufwand:} \\ \hline
\textbf{Erledigte Arbeit} & \multicolumn{2}{p{11cm}|}{- Neudefinition der Hostnamen \newline
- OpenHPC wurde auf einem Laptop installier \newline
- Die MAC Adressen der Nodes wurden noch nicht ausgelesen deshalb können die Computenodes noch nicht dem Cluster zugewiesen werden \newline
- Die Warewulf Komponente funktioniert noch nicht, somit kann das Betriebssystem noch nich auf die Computenodes verteilt werden} \\ \hline
\textbf{Aufgetretene Probleme} & \multicolumn{2}{p{11cm}|}{WareWulf Komponente blockiert die PXE Boot Installation} \\ \hline
\rowcolor{heading}\textbf{Tag:} 35 & \textbf{Datum:} 11.03.2018 & \textbf{Aufwand:} \\ \hline
\textbf{Erledigte Arbeit} & \multicolumn{2}{p{11cm}|}{- 10 MAC Adressen der RPI's ausgelesen, dafür musste ich eine SD Karte welche ein kompatibles OS für die RPI's besitzt verwenden. Danach habe ich über nmap die MAC Adressen jeweils ausgelesen} \\ \hline
\textbf{Aufgetretene Probleme} & \multicolumn{2}{p{11cm}|}{-} \\ \hline
\rowcolor{heading}\textbf{Tag:} 36 & \textbf{Datum:} 12.03.2018 & \textbf{Aufwand:} \\ \hline
\textbf{Erledigte Arbeit} & \multicolumn{2}{p{11cm}|}{- Ich habe mit dem physischen Aufbau des Clusters begonnen. \newline
- Distanzbolzen mit Raspberry PI's verbunden \newline
- Patchkabel and Raspberry PI's und Switch angeschlossen \newline
- Netzteil via Jumper Kabel an Raspberry PI's angeschlossen} \\ \hline
\textbf{Aufgetretene Probleme} & \multicolumn{2}{p{11cm}|}{Die Lösung sieht instabil aus} \\ \hline
\rowcolor{heading}\textbf{Tag:} 37 & \textbf{Datum:} 13.03.2018 & \textbf{Aufwand:} \\ \hline
\textbf{Erledigte Arbeit} & \multicolumn{2}{p{11cm}|}{- Stromtest des Clusters durchgeführt. 24 Stunden lang den Cluster mit Strom versorgt und beobachtet ob am Schluss Raspberry PI's ausgefallen sind. \newline
- Informationen über das Bedienen von OpenHPC eingeholt} \\ \hline
\textbf{Aufgetretene Probleme} & \multicolumn{2}{p{11cm}|}{} \\ \hline
\rowcolor{heading}\textbf{Tag:} 38 & \textbf{Datum:} 14.03.2018 & \textbf{Aufwand:} \\ \hline
\textbf{Erledigte Arbeit} & \multicolumn{2}{p{11cm}|}{- ein weiterer Versuch PXE Boot mit Warewulf umzusetzen. Dabei bin ich auf Informationen gestossen, dass dies mit den ARMv8 Prozessoren aufgrunde der Architektur nicht möglich ist} \\ \hline
\textbf{Aufgetretene Probleme} & \multicolumn{2}{p{11cm}|}{Warewulf ist nicht mit Raspberry PI's kompatibel} \\ \hline
\rowcolor{heading}\textbf{Tag:} 39 & \textbf{Datum:} 15.03.2018 & \textbf{Aufwand:} \\ \hline
\textbf{Erledigte Arbeit} & \multicolumn{2}{p{11cm}|}{- Weitere Versuche PXE Boot mit Warewulf einzurichten, da ich es nicht glauben kann, dass dies nicht gehen soll} \\ \hline
\textbf{Aufgetretene Probleme} & \multicolumn{2}{p{11cm}|}{Warewulf ist nicht mit Raspberry PI's kompatibel} \\ \hline
\rowcolor{heading}\textbf{Tag:} 40 & \textbf{Datum:} 16.03.2018 & \textbf{Aufwand:} \\ \hline
\textbf{Erledigte Arbeit} & \multicolumn{2}{p{11cm}|}{- Weitere Versuche PXE Boot mit Warewulf einzurichten, da ich es nicht glauben kann, dass dies nicht gehen soll} \\ \hline
\textbf{Aufgetretene Probleme} & \multicolumn{2}{p{11cm}|}{Warewulf ist nicht mit Raspberry PI's kompatibel} \\ \hline
\rowcolor{heading}\textbf{Tag:} 41 & \textbf{Datum:} 17.03.2018 & \textbf{Aufwand:} \\ \hline
\textbf{Erledigte Arbeit} & \multicolumn{2}{p{11cm}|}{- Weitere Versuche PXE Boot mit Warewulf einzurichten, da ich es nicht glauben kann, dass dies nicht gehen soll} \\ \hline
\textbf{Aufgetretene Probleme} & \multicolumn{2}{p{11cm}|}{Warewulf ist nicht mit Raspberry PI's kompatibel} \\ \hline
\rowcolor{heading}\textbf{Tag:} 42 & \textbf{Datum:} 18.03.2018 & \textbf{Aufwand:} \\ \hline
\textbf{Erledigte Arbeit} & \multicolumn{2}{p{11cm}|}{- Den PXE Boot mit einem NOOBS Betriebssystem aufgesetzt. Dies hat funktioniert \newline
- Dies habe ich versucht in Warewulf zu implementieren. Leider erfolglos \newline
Entschieden den TFTBOOT / PXE ohne WareWulf auf dem Managementnode einzurichten.} \\ \hline
\textbf{Aufgetretene Probleme} & \multicolumn{2}{p{11cm}|}{} \\ \hline
\rowcolor{heading}\textbf{Tag:} 43 & \textbf{Datum:} 19.03.2018 & \textbf{Aufwand:} \\ \hline
\textbf{Erledigte Arbeit} & \multicolumn{2}{p{11cm}|}{- Installationsscript geschrieben, welches alles automatisch installiert} \\ \hline
\textbf{Aufgetretene Probleme} & \multicolumn{2}{p{11cm}|}{Es bestehen noch Probleme mit der Installation von CentOS auf den Raspberry PI's} \\ \hline
\rowcolor{heading}\textbf{Tag:} 44 & \textbf{Datum:} 20.03.2018 & \textbf{Aufwand:} \\ \hline
\textbf{Erledigte Arbeit} & \multicolumn{2}{p{11cm}|}{- Den Managementnode mehrfach neu aufgesetzt \newline
- Weitere Abklärungen betreffend, PCE, CentOS und Warewulf} \\ \hline
\textbf{Aufgetretene Probleme} & \multicolumn{2}{p{11cm}|}{} \\ \hline
\rowcolor{heading}\textbf{Tag:} 45 & \textbf{Datum:} 21.03.2018 & \textbf{Aufwand:} \\ \hline
\textbf{Erledigte Arbeit} & \multicolumn{2}{p{11cm}|}{- Ich habe versucht diverse Images auf die SD Karte zu schreiben und die Raspberry PI's zu betreiben \newline
- Die Images stammen von http://mirror.centos.org/altarch/7/isos/aarch64/} \\ \hline
\textbf{Aufgetretene Probleme} & \multicolumn{2}{p{11cm}|}{- Leider hat kein Image funktioniert \newline
- Vermutlich kann der Kernel nicht richtig geladen werden oder ist nicht kompatibel} \\ \hline
\rowcolor{heading}\textbf{Tag:} 46 & \textbf{Datum:} 22.03.2018 & \textbf{Aufwand:} \\ \hline
\textbf{Erledigte Arbeit} & \multicolumn{2}{p{11cm}|}{- Weitere fremde Images versucht zu installieren, z.B. Gentoo 64 Bit für Raspberry PI's, Fedora, usw.} \\ \hline
\textbf{Aufgetretene Probleme} & \multicolumn{2}{p{11cm}|}{-} \\ \hline
\rowcolor{heading}\textbf{Tag:} 47 & \textbf{Datum:} 23.03.2018 & \textbf{Aufwand:} \\ \hline
\textbf{Erledigte Arbeit} & \multicolumn{2}{p{11cm}|}{- Ich habe nach einer alternativen 64 Bit Version für RPI's gesucht. Dabei bin ich auf diverse Images gestossen: \newline Fedora https://fedoraproject.org/wiki/Architectures/ARM, die Installation hat einwandfrei funktioniert. (CentOS nahe) \newline Gentoo https://github.com/sakaki-/gentoo-on-rpi3-64bit, die Installation haut auf anhieb funktioniert. Ich bin auf diesen Guide gestossen: https://github.com/umiddelb/aarch64/wiki/Install-CentOS-7-on-your-favourite-ARMv8-ARM64-AArch64-board, durch diesen Guide habe ich den Durchbruch geschafft. Die Installation habe ich mit der Fedora Boot Partition leider nicht geschafft, mit der Gentoo Lösung ging es aber. Ich bin dabei wie folgt vorgegangen: Beim Schreiben des Gentoo Images auf die SD Karte, werden zwei Partitionen erstellt (Boot \& FileSystem). Die Boot Partition habe ich nicht angepasst. Wichtig dabei ist, dass das RPI ein Kernel8.img für ARMv8 in der Boot Partition benötigt. Dies musste ich also stehen lassen. Als zweiten Schritt habe ich aus dem CentOS Repos das Archiv CentOS-7-aarch64-rootfs-7.4.1708.tar.xz heruntergeladen. Darin ist das komplette FileSystem enthalten. Dies habe ich auf der FileSystem Partition mit dem Befehl tar --numeric-owner -xpJf .../CentOS-7-aarch64-rootfs-7.4.1708.tar.xz -C /home/camrein/Downloads/mnt2 niedergschrieben. (Achtung die Partition wurde nach /home/camrein/Downloads/mnt2 gemountet) Danach hatte ich ein funktionsfaehiges CentOS 64 Bit auf dem Raspberry PI} \\ \hline
\textbf{Aufgetretene Probleme} & \multicolumn{2}{p{11cm}|}{-} \\ \hline
\rowcolor{heading}\textbf{Tag:} 48 & \textbf{Datum:} 24.03.2018 & \textbf{Aufwand:} \\ \hline
\textbf{Erledigte Arbeit} & \multicolumn{2}{p{11cm}|}{- 	
Das Das CentOS nun erfolgreich auf dem RPI betrieben werden kann, habe ich versucht in RPI als Management Node zu verwenden. Bei der Installation des Provisioning Progammes Warewulf bin ich jeweils auf Fehler gestossen. Leider konnte ich in keinem Log inkl Systemlogs keinen Eintrag zum Fehler finden. Das RPI ist jedoch immer wieder eingefrorern. Deshalb habe ich versucht einen üblichen PXE Boot mit TFT einzurichten. Dies hat einwandfrei funktioniert. Bei jedem Erfolg habe ich die SD Karte erneut kopiert, so dass beim Aufsetzen des RPI's wieder möglichst rasch mit einem stabilen Ausgangspunkt weitergemacht werden kann.} \\ \hline
\textbf{Aufgetretene Probleme} & \multicolumn{2}{p{11cm}|}{- Beim PXE Boot trat ich auf diverse Probleme. Jedoch habe ich durch lesen von mehreren Anleitungen und Guides dies beheben können. Ich hatte nicht alles Dateien von der Boot Partition im entsprechenden tftboot Ordner drin} \\ \hline
\rowcolor{heading}\textbf{Tag:} 49 & \textbf{Datum:} 25.03.2018 & \textbf{Aufwand:} \\ \hline
\textbf{Erledigte Arbeit} & \multicolumn{2}{p{11cm}|}{- Administratives, Dokumente nachführen. Entscheidungen treffen. Ich habe mich entschieden es nochmals mit dem Laptop als Managementnode zu versuchen. Dieser hat einen Intel Prozessor und Warewulf kann ohne Probleme auf dem Managementnode installiert werden. Ich habe mich dazu entschlossen als nächsten Schritt es nochmals zu versuchen das vorhandene tftboot Image in Warewulf zu implementieren. Dies würde die Installation enorm vereinfachen, da die Raspberry PI's direkt über die MAC Adresse eine IP und einen Hostnamen zugewiesen werden.} \\ \hline
\textbf{Aufgetretene Probleme} & \multicolumn{2}{p{11cm}|}{-} \\ \hline
\rowcolor{heading}\textbf{Tag:} 50 & \textbf{Datum:} 26.03.2018 & \textbf{Aufwand:} \\ \hline
\textbf{Erledigte Arbeit} & \multicolumn{2}{p{11cm}|}{- Zur Erkenntnis genommen, das die MS Office Tools für mich nicht zu gebrauchen sind. Ich habe mich entschieden die Dokumentation mit LaTeX zu schreiben} \\ \hline
\textbf{Aufgetretene Probleme} & \multicolumn{2}{p{11cm}|}{-} \\ \hline
\rowcolor{heading}\textbf{Tag:} 51 & \textbf{Datum:} 27.03.2018 & \textbf{Aufwand:} \\ \hline
\textbf{Erledigte Arbeit} & \multicolumn{2}{p{11cm}|}{- Studieren von LaTeX: Ich habe noch nie mit LaTeX gearbeitet, es begeistert mich aber, deshalb habe ich mir Beispiele von Dokumentationen und Befehlen angeschaut} \\ \hline
\textbf{Aufgetretene Probleme} & \multicolumn{2}{p{11cm}|}{-} \\ \hline
\rowcolor{heading}\textbf{Tag:} 52 & \textbf{Datum:} 28.03.2018 & \textbf{Aufwand:} \\ \hline
\textbf{Erledigte Arbeit} & \multicolumn{2}{p{11cm}|}{- Versuche mit LaTeX: Ich habe versucht selbst ein LaTeX Dokument zu erstellen und mich mit Freunden darüber unterhalten, dabei habe ich erfahren, dass es bereits sehr gute vordefinierte Templates gibt, welche frei im Internet beziehbar sind} \\ \hline
\textbf{Aufgetretene Probleme} & \multicolumn{2}{p{11cm}|}{-} \\ \hline
\rowcolor{heading}\textbf{Tag:} 53 & \textbf{Datum:} 29.03.2018 & \textbf{Aufwand:} \\ \hline
\textbf{Erledigte Arbeit} & \multicolumn{2}{p{11cm}|}{- Ich habe mich für ein LaTeX Template von Macke entschieden und mit der Migration von Word nach LaText begonnen} \\ \hline
\textbf{Aufgetretene Probleme} & \multicolumn{2}{p{11cm}|}{-} \\ \hline
\rowcolor{heading}\textbf{Tag:} 54 & \textbf{Datum:} 30.03.2018 & \textbf{Aufwand:} \\ \hline
\textbf{Erledigte Arbeit} & \multicolumn{2}{p{11cm}|}{- Ausgangslage und Projektziele nach LaTeX migriert} \\ \hline
\textbf{Aufgetretene Probleme} & \multicolumn{2}{p{11cm}|}{-} \\ \hline\rowcolor{heading}\textbf{Tag:} 55 & \textbf{Datum:} 31.03.2018 & \textbf{Aufwand:} \\ \hline
\textbf{Erledigte Arbeit} & \multicolumn{2}{p{11cm}|}{- Termine und Projektorganisation nach LaTeX migriert} \\ \hline
\textbf{Aufgetretene Probleme} & \multicolumn{2}{p{11cm}|}{-} \\ \hline
\rowcolor{heading}\textbf{Tag:} 56 & \textbf{Datum:} 01.04.2018 & \textbf{Aufwand:} \\ \hline
\textbf{Erledigte Arbeit} & \multicolumn{2}{p{11cm}|}{- Überarbeitung der Darstellung der bereits migrierten Inhalte} \\ \hline
\textbf{Aufgetretene Probleme} & \multicolumn{2}{p{11cm}|}{-} \\ \hline
\rowcolor{heading}\textbf{Tag:} 57 & \textbf{Datum:} 02.04.2018 & \textbf{Aufwand:} \\ \hline
\textbf{Erledigte Arbeit} & \multicolumn{2}{p{11cm}|}{- Kapitel Ressourcen nach LaTeX migriert} \\ \hline
\textbf{Aufgetretene Probleme} & \multicolumn{2}{p{11cm}|}{-} \\ \hline
\rowcolor{heading}\textbf{Tag:} 58 & \textbf{Datum:} 03.04.2018 & \textbf{Aufwand:} \\ \hline
\textbf{Erledigte Arbeit} & \multicolumn{2}{p{11cm}|}{- Überarbeitung bisheriger Dokumentation in LaTeX} \\ \hline
\textbf{Aufgetretene Probleme} & \multicolumn{2}{p{11cm}|}{-} \\ \hline
\rowcolor{heading}\textbf{Tag:} 59 & \textbf{Datum:} 04.04.2018 & \textbf{Aufwand:} \\ \hline
\textbf{Erledigte Arbeit} & \multicolumn{2}{p{11cm}|}{- Design Anpassungen des Diplomberichts} \\ \hline
\textbf{Aufgetretene Probleme} & \multicolumn{2}{p{11cm}|}{-} \\ \hline
\rowcolor{heading}\textbf{Tag:} 60 & \textbf{Datum:} 05.04.2018 & \textbf{Aufwand:} \\ \hline
\textbf{Erledigte Arbeit} & \multicolumn{2}{p{11cm}|}{-} \\ \hline
\textbf{Aufgetretene Probleme} & \multicolumn{2}{p{11cm}|}{-} \\ \hline
\rowcolor{heading}\textbf{Tag:} 61 & \textbf{Datum:} 06.04.2018 & \textbf{Aufwand:} \\ \hline
\textbf{Erledigte Arbeit} & \multicolumn{2}{p{11cm}|}{-} \\ \hline
\textbf{Aufgetretene Probleme} & \multicolumn{2}{p{11cm}|}{-} \\ \hline
\rowcolor{heading}\textbf{Tag:} 62 & \textbf{Datum:} 07.04.2018 & \textbf{Aufwand:} \\ \hline
\textbf{Erledigte Arbeit} & \multicolumn{2}{p{11cm}|}{- Abschliessen der Migration des Projektauftrags (LaTeX) \newline -Informationen über OpenHPC Warewulf und ARMv8 sammeln, da ich noch Probleme mit dem Provisioning des Betriebssystems habe.} \\ \hline
\textbf{Aufgetretene Probleme} & \multicolumn{2}{p{11cm}|}{-} \\ \hline
\rowcolor{heading}\textbf{Tag:} 62 & \textbf{Datum:} 08.04.2018 & \textbf{Aufwand:} \\ \hline
\textbf{Erledigte Arbeit} & \multicolumn{2}{p{11cm}|}{- Studie nach LaTex migrieren \newline
- Informationen über OpenHPC Warewulf und ARMv8 sammeln, da ich noch Probleme mit dem Provisioning des Betriebssystems habe} \\ \hline
\textbf{Aufgetretene Probleme} & \multicolumn{2}{p{11cm}|}{-} \\ \hline
\rowcolor{heading}\textbf{Tag:} 63 & \textbf{Datum:} 09.04.2018 & \textbf{Aufwand:} \\ \hline
\textbf{Erledigte Arbeit} & \multicolumn{2}{p{11cm}|}{- Überarbeitung bisheriger Dokumentation und Konzept Dokumentation migrieren.} \\ \hline
\textbf{Aufgetretene Probleme} & \multicolumn{2}{p{11cm}|}{-} \\ \hline
\rowcolor{heading}\textbf{Tag:} 64 & \textbf{Datum:} 10.04.2018 & \textbf{Aufwand:} \\ \hline
\textbf{Erledigte Arbeit} & \multicolumn{2}{p{11cm}|}{- Entschieden Warewulf auszulassen und mit dnsmasq und pxe boot fortzufahren. PXE Boot mit Centos eingerichtet, alle Raspberry PI's können gestartet werden und beziehen das Betriebssystem über das Netzwerk} \\ \hline
\textbf{Aufgetretene Probleme} & \multicolumn{2}{p{11cm}|}{-} \\ \hline
\rowcolor{heading}\textbf{Tag:} 65 & \textbf{Datum:} 11.04.2018 & \textbf{Aufwand:} \\ \hline
\textbf{Erledigte Arbeit} & \multicolumn{2}{p{11cm}|}{- Nagios einrichten. Es existiert ein Bug mit Centos 7.4. der Ordner /var/run/nagios wird nicht automatisch erstellt. Debuggen bislang erfolglos. Wenn ich in der Konfiguration ein anderes Verzeichnis für die PID Ablage erstelle funktioniert dies noch nicht. Standard Monitoring für alle Nodes eingerichtet. Es müssen aber noch spezifiziertere Überwachungen geschrieben werden} \\ \hline
\textbf{Aufgetretene Probleme} & \multicolumn{2}{p{11cm}|}{-} \\ \hline
\rowcolor{heading}\textbf{Tag:} 66 & \textbf{Datum:} 12.04.2018 & \textbf{Aufwand:} \\ \hline
\textbf{Erledigte Arbeit} & \multicolumn{2}{p{11cm}|}{- Installationsskript anpassen und updaten. Ich habe eine Kopie der SD Karte erstellt. Die SD Karte wurde danach gelöscht und ich habe mit dem automatisierten Installationsskript den Cluster wieder versucht zu installieren. Dabei sind folgende Probleme noch vorhanden: \newline - NTP Sync von Master zu Computenodes \newline
- Slurmd startet nicht automatisch auf den Computes \newline 
- Nagios PID Fehler \newline 
- Ganglia Errors in /var/log/messages ( Keine Nodes werden aufgeführt)} \\ \hline
\textbf{Aufgetretene Probleme} & \multicolumn{2}{p{11cm}|}{-} \\ \hline
\rowcolor{heading}\textbf{Tag:} 67 & \textbf{Datum:} 13.04.2018 & \textbf{Aufwand:} \\ \hline
\textbf{Erledigte Arbeit} & \multicolumn{2}{p{11cm}|}{- Slurm Konfiguration manuell angepasst. Fehler gefunden und behoben. Versucht ersten Job zu erstellen. Jedoch noch erfolglos.} \\ \hline
\textbf{Aufgetretene Probleme} & \multicolumn{2}{p{11cm}|}{- Jobs können in Slurm nocht nicht erstellt werden} \\ \hline
\rowcolor{heading}\textbf{Tag:} 68 & \textbf{Datum:} 14.04.2018 & \textbf{Aufwand:} \\ \hline
\textbf{Erledigte Arbeit} & \multicolumn{2}{p{11cm}|}{- Herausgefunden wie Jobs erstellt werden müssen. Anstatt ein sbatch Script muss mit dem Befehl srun gearbeitet werden. Job erstellt und einen Testlauf vollzogen} \\ \hline
\textbf{Aufgetretene Probleme} & \multicolumn{2}{p{11cm}|}{-} \\ \hline
\rowcolor{heading}\textbf{Tag:} 69 & \textbf{Datum:} 15.04.2018 & \textbf{Aufwand:} \\ \hline
\textbf{Erledigte Arbeit} & \multicolumn{2}{p{11cm}|}{- Dokumentation nachgeführt} \\ \hline
\textbf{Aufgetretene Probleme} & \multicolumn{2}{p{11cm}|}{-} \\ \hline
\rowcolor{heading}\textbf{Tag:} 70 & \textbf{Datum:} 16.04.2018 & \textbf{Aufwand:} \\ \hline
\textbf{Erledigte Arbeit} & \multicolumn{2}{p{11cm}|}{- Bei einem erneuten Testlauf ist ein Raspberry PI beschädigt worden. Ich wollte es austauschen und habe Bemerkt das der Aufwand für den Austausch zu viel Zeit kostet. Deshalb habe ich nochmals den Aufbau überdacht. \newline 
- Dokumentation nachführen, Heatsinks installieren} \\ \hline
\textbf{Aufgetretene Probleme} & \multicolumn{2}{p{11cm}|}{-} \\ \hline
\rowcolor{heading}\textbf{Tag:} 71 & \textbf{Datum:} 17.04.2018 & \textbf{Aufwand:} \\ \hline
\textbf{Erledigte Arbeit} & \multicolumn{2}{p{11cm}|}{- Neuen Aufbau des Clusters in Angriff genommen. Ich habe mir ein passendes Gerüst / Gestell in einem Warenhaus gekauft. } \\ \hline
\textbf{Aufgetretene Probleme} & \multicolumn{2}{p{11cm}|}{-} \\ \hline
\rowcolor{heading}\textbf{Tag:} 72 & \textbf{Datum:} 18.04.2018 & \textbf{Aufwand:} \\ \hline
\textbf{Erledigte Arbeit} & \multicolumn{2}{p{11cm}|}{- Schrauben für die Montage der Raspberry PI's bestellt \newline
- Dokumentation nachgeführt.} \\ \hline
\textbf{Aufgetretene Probleme} & \multicolumn{2}{p{11cm}|}{-} \\ \hline
\rowcolor{heading}\textbf{Tag:} 73 & \textbf{Datum:} 19.04.2018 & \textbf{Aufwand:} \\ \hline
\textbf{Erledigte Arbeit} & \multicolumn{2}{p{11cm}|}{- Überarbeitung des Diplomberichts} \\ \hline
\textbf{Aufgetretene Probleme} & \multicolumn{2}{p{11cm}|}{-} \\ \hline
\rowcolor{heading}\textbf{Tag:} 74 & \textbf{Datum:} 20.04.2018 & \textbf{Aufwand:} \\ \hline
\textbf{Erledigte Arbeit} & \multicolumn{2}{p{11cm}|}{- Diplombericht erweitern} \\ \hline
\textbf{Aufgetretene Probleme} & \multicolumn{2}{p{11cm}|}{-} \\ \hline
\rowcolor{heading}\textbf{Tag:} 75 & \textbf{Datum:} 21.04.2018 & \textbf{Aufwand:} \\ \hline
\textbf{Erledigte Arbeit} & \multicolumn{2}{p{11cm}|}{- Diplombericht erweitern} \\ \hline
\textbf{Aufgetretene Probleme} & \multicolumn{2}{p{11cm}|}{-} \\ \hline
\rowcolor{heading}\textbf{Tag:} 76 & \textbf{Datum:} 22.04.2018 & \textbf{Aufwand:} \\ \hline
\textbf{Erledigte Arbeit} & \multicolumn{2}{p{11cm}|}{- Diplombericht erweitern} \\ \hline
\textbf{Aufgetretene Probleme} & \multicolumn{2}{p{11cm}|}{-} \\ \hline
\rowcolor{heading}\textbf{Tag:} 77 & \textbf{Datum:} 23.04.2018 & \textbf{Aufwand:} \\ \hline
\textbf{Erledigte Arbeit} & \multicolumn{2}{p{11cm}|}{- Schrauben sind angekommen.\newline
- Ich habe den Cluster neu zusammengestellt.} \\ \hline
\textbf{Aufgetretene Probleme} & \multicolumn{2}{p{11cm}|}{-} \\ \hline
\rowcolor{heading}\textbf{Tag:} 78 & \textbf{Datum:} 24.04.2018 & \textbf{Aufwand:} \\ \hline
\textbf{Erledigte Arbeit} & \multicolumn{2}{p{11cm}|}{- Wackelkontakte waren vorhanden. Ich musste nochmals die Verkabelung stabiler gestalten. Die Raspberry PI's wurden nicht konstant mit 5 Volt versorgt} \\ \hline
\textbf{Aufgetretene Probleme} & \multicolumn{2}{p{11cm}|}{-} \\ \hline
\rowcolor{heading}\textbf{Tag:} 79 & \textbf{Datum:} 25.04.2018 & \textbf{Aufwand:} \\ \hline
\textbf{Erledigte Arbeit} & \multicolumn{2}{p{11cm}|}{-} \\ \hline
\textbf{Aufgetretene Probleme} & \multicolumn{2}{p{11cm}|}{-} \\ \hline
\rowcolor{heading}\textbf{Tag:} 80 & \textbf{Datum:} 26.04.2018 & \textbf{Aufwand:} \\ \hline
\textbf{Erledigte Arbeit} & \multicolumn{2}{p{11cm}|}{- Dokumentation erweitert} \\ \hline
\textbf{Aufgetretene Probleme} & \multicolumn{2}{p{11cm}|}{-} \\ \hline
\rowcolor{heading}\textbf{Tag:} 81 & \textbf{Datum:} 27.04.2018 & \textbf{Aufwand:} \\ \hline
\textbf{Erledigte Arbeit} & \multicolumn{2}{p{11cm}|}{- Monitoring Programme optimieren und einrichten. Nagios konnte auf nicht alle Ports eine Verbindung aufbauen. Dies wurde gefixt. Die Ursprüngliche Konfiguration war nicht für die Umgebung eingerichtet} \\ \hline
\textbf{Aufgetretene Probleme} & \multicolumn{2}{p{11cm}|}{-} \\ \hline
\rowcolor{heading}\textbf{Tag:} 82 & \textbf{Datum:} 28.04.2018 & \textbf{Aufwand:} \\ \hline
\textbf{Erledigte Arbeit} & \multicolumn{2}{p{11cm}|}{-} \\ \hline
\textbf{Aufgetretene Probleme} & \multicolumn{2}{p{11cm}|}{-} \\ \hline
\rowcolor{heading}\textbf{Tag:} 83 & \textbf{Datum:} 29.04.2018 & \textbf{Aufwand:} \\ \hline
\textbf{Erledigte Arbeit} & \multicolumn{2}{p{11cm}|}{-} \\ \hline
\textbf{Aufgetretene Probleme} & \multicolumn{2}{p{11cm}|}{-} \\ \hline
\rowcolor{heading}\textbf{Tag:} 84 & \textbf{Datum:} 30.04.2018 & \textbf{Aufwand:} \\ \hline
\textbf{Erledigte Arbeit} & \multicolumn{2}{p{11cm}|}{- Ganglia Monitoring konfiguriert, XMLParser Fehler waren vorhanden. Ganglia wurde für RPI Tests optimiert} \\ \hline
\textbf{Aufgetretene Probleme} & \multicolumn{2}{p{11cm}|}{-} \\ \hline
\rowcolor{heading}\textbf{Tag:} 85 & \textbf{Datum:} 01.05.2018 & \textbf{Aufwand:} \\ \hline
\textbf{Erledigte Arbeit} & \multicolumn{2}{p{11cm}|}{- Mining Tests absolviert. Alle gewünschten Währungen wurden für eine Testdauer von jeweils 30 Minuten geschürft \newline
- Die Raspberry PI's müssen noch übertaktet werden} \\ \hline
\textbf{Aufgetretene Probleme} & \multicolumn{2}{p{11cm}|}{-} \\ \hline
\rowcolor{heading}\textbf{Tag:} 86 & \textbf{Datum:} 02.05.2018 & \textbf{Aufwand:} \\ \hline
\textbf{Erledigte Arbeit} & \multicolumn{2}{p{11cm}|}{- Dokumentation nachgerführt \newline
- Raspberry PI's übertaktet (PXE Boot Image)} \\ \hline
\textbf{Aufgetretene Probleme} & \multicolumn{2}{p{11cm}|}{-} \\ \hline
\rowcolor{heading}\textbf{Tag:} 87 & \textbf{Datum:} 03.05.2018 & \textbf{Aufwand:} \\ \hline
\textbf{Erledigte Arbeit} & \multicolumn{2}{p{11cm}|}{- Dokumentation erweitert} \\ \hline
\textbf{Aufgetretene Probleme} & \multicolumn{2}{p{11cm}|}{-} \\ \hline
\rowcolor{heading}\textbf{Tag:} 88 & \textbf{Datum:} 04.05.2018 & \textbf{Aufwand:} \\ \hline
\textbf{Erledigte Arbeit} & \multicolumn{2}{p{11cm}|}{- Dokumentation erweitert} \\ \hline
\textbf{Aufgetretene Probleme} & \multicolumn{2}{p{11cm}|}{-} \\ \hline
\rowcolor{heading}\textbf{Tag:} 89 & \textbf{Datum:} 05.05.2018 & \textbf{Aufwand:} \\ \hline
\textbf{Erledigte Arbeit} & \multicolumn{2}{p{11cm}|}{-} \\ \hline
\textbf{Aufgetretene Probleme} & \multicolumn{2}{p{11cm}|}{-} \\ \hline
\rowcolor{heading}\textbf{Tag:} 90 & \textbf{Datum:} 06.05.2018 & \textbf{Aufwand:} \\ \hline
\textbf{Erledigte Arbeit} & \multicolumn{2}{p{11cm}|}{-} \\ \hline
\textbf{Aufgetretene Probleme} & \multicolumn{2}{p{11cm}|}{-} \\ \hline
\rowcolor{heading}\textbf{Tag:} 91 & \textbf{Datum:} 07.05.2018 & \textbf{Aufwand:} \\ \hline
\textbf{Erledigte Arbeit} & \multicolumn{2}{p{11cm}|}{-} \\ \hline
\textbf{Aufgetretene Probleme} & \multicolumn{2}{p{11cm}|}{-} \\ \hline
\rowcolor{heading}\textbf{Tag:} 92 & \textbf{Datum:} 08.05.2018 & \textbf{Aufwand:} \\ \hline
\textbf{Erledigte Arbeit} & \multicolumn{2}{p{11cm}|}{- Dokumentation erweitert} \\ \hline
\textbf{Aufgetretene Probleme} & \multicolumn{2}{p{11cm}|}{-} \\ \hline
\rowcolor{heading}\textbf{Tag:} 93 & \textbf{Datum:} 09.05.2018 & \textbf{Aufwand:} \\ \hline
\textbf{Erledigte Arbeit} & \multicolumn{2}{p{11cm}|}{- Dokumentation erweitert} \\ \hline
\textbf{Aufgetretene Probleme} & \multicolumn{2}{p{11cm}|}{-} \\ \hline
\rowcolor{heading}\textbf{Tag:} 94 & \textbf{Datum:} 10.05.2018 & \textbf{Aufwand:} \\ \hline
\textbf{Erledigte Arbeit} & \multicolumn{2}{p{11cm}|}{- Dokumentation erweitert} \\ \hline
\textbf{Aufgetretene Probleme} & \multicolumn{2}{p{11cm}|}{-} \\ \hline
\rowcolor{heading}\textbf{Tag:} 95 & \textbf{Datum:} 11.05.2018 & \textbf{Aufwand:} \\ \hline
\textbf{Erledigte Arbeit} & \multicolumn{2}{p{11cm}|}{- Dokumentation erweitert} \\ \hline
\textbf{Aufgetretene Probleme} & \multicolumn{2}{p{11cm}|}{-} \\ \hline
\rowcolor{heading}\textbf{Tag:} 96 & \textbf{Datum:} 12.05.2018 & \textbf{Aufwand:} \\ \hline
\textbf{Erledigte Arbeit} & \multicolumn{2}{p{11cm}|}{- Dokumentation erweitert} \\ \hline
\textbf{Aufgetretene Probleme} & \multicolumn{2}{p{11cm}|}{-} \\ \hline
\rowcolor{heading}\textbf{Tag:} 97 & \textbf{Datum:} 13.05.2018 & \textbf{Aufwand:} \\ \hline
\textbf{Erledigte Arbeit} & \multicolumn{2}{p{11cm}|}{- Dokumentation erweitert} \\ \hline
\textbf{Aufgetretene Probleme} & \multicolumn{2}{p{11cm}|}{-} \\ \hline
\rowcolor{heading}\textbf{Tag:} 98 & \textbf{Datum:} 14.05.2018 & \textbf{Aufwand:} \\ \hline
\textbf{Erledigte Arbeit} & \multicolumn{2}{p{11cm}|}{- Dokumentation erweitert} \\ \hline
\textbf{Aufgetretene Probleme} & \multicolumn{2}{p{11cm}|}{-} \\ \hline
\rowcolor{heading}\textbf{Tag:} 99 & \textbf{Datum:} 15.05.2018 & \textbf{Aufwand:} \\ \hline
\textbf{Erledigte Arbeit} & \multicolumn{2}{p{11cm}|}{- Dokumentation erweitert} \\ \hline
\textbf{Aufgetretene Probleme} & \multicolumn{2}{p{11cm}|}{-} \\ \hline
\rowcolor{heading}\textbf{Tag:} 100 & \textbf{Datum:} 16.05.2018 & \textbf{Aufwand:} \\ \hline
\textbf{Erledigte Arbeit} & \multicolumn{2}{p{11cm}|}{- Dokumentation erweitert} \\ \hline
\textbf{Aufgetretene Probleme} & \multicolumn{2}{p{11cm}|}{-} \\ \hline
\rowcolor{heading}\textbf{Tag:} 101 & \textbf{Datum:} 17.05.2018 & \textbf{Aufwand:} \\ \hline
\textbf{Erledigte Arbeit} & \multicolumn{2}{p{11cm}|}{- Dokumentation erweitert} \\ \hline
\textbf{Aufgetretene Probleme} & \multicolumn{2}{p{11cm}|}{-} \\ \hline
\rowcolor{heading}\textbf{Tag:} 102 & \textbf{Datum:} 18.05.2018 & \textbf{Aufwand:} \\ \hline
\textbf{Erledigte Arbeit} & \multicolumn{2}{p{11cm}|}{- Dokumentation erweitert} \\ \hline
\textbf{Aufgetretene Probleme} & \multicolumn{2}{p{11cm}|}{-} \\ \hline
\rowcolor{heading}\textbf{Tag:} 103 & \textbf{Datum:} 19.05.2018 & \textbf{Aufwand:} \\ \hline
\textbf{Erledigte Arbeit} & \multicolumn{2}{p{11cm}|}{- Dokumentation erweitert} \\ \hline
\textbf{Aufgetretene Probleme} & \multicolumn{2}{p{11cm}|}{-} \\ \hline
\rowcolor{heading}\textbf{Tag:} 104 & \textbf{Datum:} 20.05.2018 & \textbf{Aufwand:} \\ \hline
\textbf{Erledigte Arbeit} & \multicolumn{2}{p{11cm}|}{- Dokumentation abgeschlossen \newline
- Die Dokumentation ist nun bereit für Korrekturen} \\ \hline
\textbf{Aufgetretene Probleme} & \multicolumn{2}{p{11cm}|}{-} \\ \hline
\rowcolor{heading}\textbf{Tag:} 105 & \textbf{Datum:} 21.05.2018 & \textbf{Aufwand:} \\ \hline
\textbf{Erledigte Arbeit} & \multicolumn{2}{p{11cm}|}{- Überarbeitung der Dokumentation} \\ \hline
\textbf{Aufgetretene Probleme} & \multicolumn{2}{p{11cm}|}{-} \\ \hline
\caption{Arbeitsjournal}\\
\end{longtable}


