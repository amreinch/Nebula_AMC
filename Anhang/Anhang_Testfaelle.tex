% !TEX root = ../Diplombericht.tex
\section{Testfälle}
\subsection{Komponententests}
\begin{table}[H]
\centering
\begin{tabular}{|p{4cm}|p{4cm}|p{4cm}|p{4cm}|}
\hline
Bezeichnung & \textbf{K-001} & Management Node & Hostnamen / IP / MAC \\ \hline
Beschreibung & \multicolumn{3}{p{12cm}|}{Der Management Node wird auf die zugewiesene IP Adresse, MAC Adresse und den Hostnamen geprüft.} \\ \hline
Testvoraussetzung & \multicolumn{3}{p{12cm}|}{Der Management Node und der Testclient befinden sich im selben Netzwerk.} \\ \hline
Testschritte & \multicolumn{3}{p{12cm}|}{
- Management Node starten (Strom anschliessen).\newline
- 30 Sekunden warten.\newline
- Auf dem Testclient über Putty oder Shell mit dem Befehl \newline 
 \grqq nmap -sn 192.168.1.10\grqq \ eingeben.\newline
- Prüfen, ob der die Zuweisung gemäss Hostnamenkonzept richtig ist.} \\ \hline
Erwartetes Ergebnis & \multicolumn{3}{p{12cm}|}{Die Rückgabewerte (Hostnamen, IP- und MAC-Adresse) sollen identisch mit deren im Hostnamenkonzept sein.} \\\hline
\end{tabular}
\caption{Testfall K-001}
\label{Testfall K-001}
\end{table}

\begin{table}[H]
\centering
\begin{tabular}{|p{4cm}|p{4cm}|p{4cm}|p{4cm}|}
\hline
Bezeichnung & \textbf{K-002} & Management Node & Anmelden \\ \hline
Beschreibung & \multicolumn{3}{p{12cm}|}{Es wird getestet ob der Management Node über das SSH Protokoll erreichbar ist.} \\ \hline
Testvoraussetzung & \multicolumn{3}{p{12cm}|}{Der Management Node und der Testclient befinden sich im selben Netzwerk.} \\ \hline
Testschritte & \multicolumn{3}{p{12cm}|}{
- Management Node starten (Strom anschliessen).\newline
- 30 Sekunden warten.\newline
- Auf dem Testclient über Putty oder Shell den Befehl \grqq ssh root@nebula\grqq \ \ eingeben. \newline
- Passwort eingeben.} \\ \hline
Erwartetes Ergebnis & \multicolumn{3}{p{12cm}|}{Der Zugriff auf den Management Node funktioniert} \\\hline
\end{tabular}
\caption{Testfall K-002}
\label{Testfall K-002}
\end{table}

\begin{table}[H]
\centering
\begin{tabular}{|p{4cm}|p{4cm}|p{4cm}|p{4cm}|}
\hline
Bezeichnung & \textbf{K-003} & Compute Node c1 & Hostnamen / IP / MAC \\ \hline
Beschreibung & \multicolumn{3}{p{12cm}|}{Der Compute Node wird auf die zugewiesene IP Adresse, MAC Adresse und den Hostnamen geprüft.} \\ \hline
Testvoraussetzung & \multicolumn{3}{p{12cm}|}{Der Compute Node und der Testclient befinden sich im selben Netzwerk.} \\ \hline
Testschritte & \multicolumn{3}{p{12cm}|}{
- Compute Node starten (Strom anschliessen)\newline
- 3 Minuten warten\newline
- Auf dem Testclient über Putty oder Shell mit dem Befehl \newline \grqq nmap -sn 192.168.1.11\grqq \  eingeben\newline
- Prüfen ob die Zuweisung gemäss Hostnamenkonzept richtig ist.} \\ \hline
Erwartetes Ergebnis & \multicolumn{3}{p{12cm}|}{Die Rückgabewerte (Hostnamen, IP- und MAC-Adresse) sollen identisch mit deren im Hostnamenkonzept sein.} \\\hline
\end{tabular}
\caption{Testfall K-003}
\label{Testfall K-003}
\end{table}


\begin{table}[H]
\centering
\begin{tabular}{|p{4cm}|p{4cm}|p{4cm}|p{4cm}|}
\hline
Bezeichnung & \textbf{K-004} & Compute Node c2 & Hostnamen / IP / MAC \\ \hline
Beschreibung & \multicolumn{3}{p{12cm}|}{Der Compute Node wird auf die zugewiesene IP Adresse, MAC Adresse und den Hostnamen geprüft.} \\ \hline
Testvoraussetzung & \multicolumn{3}{p{12cm}|}{Der Compute Node und der Testclient befinden sich im selben Netzwerk.} \\ \hline
Testschritte & \multicolumn{3}{p{12cm}|}{
- Compute Node starten (Strom anschliessen)\newline
- 3 Minuten Sekunden warten\newline
- Auf dem Testclient über Putty oder Shell mit dem Befehl \newline \grqq nmap -sn 192.168.1.12\grqq \ eingeben\newline
- Prüfen ob die Zuweisung gemäss Hostnamenkonzept richtig ist.} \\ \hline
Erwartetes Ergebnis & \multicolumn{3}{p{12cm}|}{Die Rückgabewerte (Hostnamen, IP- und MAC-Adresse) sollen identisch mit deren im Hostnamenkonzept sein.} \\\hline
\end{tabular}
\caption{Testfall K-004}
\label{Testfall K-004}
\end{table}


\begin{table}[H]
\centering
\begin{tabular}{|p{4cm}|p{4cm}|p{4cm}|p{4cm}|}
\hline
Bezeichnung & \textbf{K-005} & Compute Node c3 & Hostnamen / IP / MAC \\ \hline
Beschreibung & \multicolumn{3}{p{12cm}|}{Der Compute Node wird auf die zugewiesene IP Adresse, MAC Adresse und den Hostnamen geprüft.} \\ \hline
Testvoraussetzung & \multicolumn{3}{p{12cm}|}{Der Compute Node und der Testclient befinden sich im selben Netzwerk.} \\ \hline
Testschritte & \multicolumn{3}{p{12cm}|}{
- Compute Node starten (Strom anschliessen)\newline
- 3 Minuten Sekunden warten\newline
- Auf dem Testclient über Putty oder Shell mit dem Befehl \newline \grqq nmap -sn 192.168.1.13\grqq \ eingeben\newline
- Prüfen ob die Zuweisung gemäss Hostnamenkonzept richtig ist.} \\ \hline
Erwartetes Ergebnis & \multicolumn{3}{p{12cm}|}{Die Rückgabewerte (Hostnamen, IP- und MAC-Adresse) sollen identisch mit deren im Hostnamenkonzept sein.} \\\hline
\end{tabular}
\caption{Testfall K-005}
\label{Testfall K-005}
\end{table}


\begin{table}[H]
\centering
\begin{tabular}{|p{4cm}|p{4cm}|p{4cm}|p{4cm}|}
\hline
Bezeichnung & \textbf{K-006} & Compute Node c4 & Hostnamen / IP / MAC \\ \hline
Beschreibung & \multicolumn{3}{p{12cm}|}{Der Compute Node wird auf die zugewiesene IP Adresse, MAC Adresse und den Hostnamen geprüft.} \\ \hline
Testvoraussetzung & \multicolumn{3}{p{12cm}|}{Der Compute Node und der Testclient befinden sich im selben Netzwerk.} \\ \hline
Testschritte & \multicolumn{3}{p{12cm}|}{
- Compute Node starten (Strom anschliessen)\newline
- 3 Minuten Sekunden warten\newline
- Auf dem Testclient über Putty oder Shell mit dem Befehl \newline \grqq nmap -sn 192.168.1.14\grqq \ eingeben\newline
- Prüfen ob die Zuweisung gemäss Hostnamenkonzept richtig ist.} \\ \hline
Erwartetes Ergebnis & \multicolumn{3}{p{12cm}|}{Die Rückgabewerte (Hostnamen, IP- und MAC-Adresse) sollen identisch mit deren im Hostnamenkonzept sein.} \\\hline
\end{tabular}
\caption{Testfall K-006}
\label{Testfall K-006}
\end{table}


\begin{table}[H]
\centering
\begin{tabular}{|p{4cm}|p{4cm}|p{4cm}|p{4cm}|}
\hline
Bezeichnung & \textbf{K-007} & Compute Node c5 & Hostnamen / IP / MAC \\ \hline
Beschreibung & \multicolumn{3}{p{12cm}|}{Der Compute Node wird auf die zugewiesene IP Adresse, MAC Adresse und den Hostnamen geprüft.} \\ \hline
Testvoraussetzung & \multicolumn{3}{p{12cm}|}{Der Compute Node und der Testclient befinden sich im selben Netzwerk.} \\ \hline
Testschritte & \multicolumn{3}{p{12cm}|}{
- Compute Node starten (Strom anschliessen)\newline
- 3 Minuten Sekunden warten\newline
- Auf dem Testclient über Putty oder Shell mit dem Befehl \newline \grqq nmap -sn 192.168.1.15\grqq \ eingeben\newline
- Prüfen ob die Zuweisung gemäss Hostnamenkonzept richtig ist.} \\ \hline
Erwartetes Ergebnis & \multicolumn{3}{p{12cm}|}{Die Rückgabewerte (Hostnamen, IP- und MAC-Adresse) sollen identisch mit deren im Hostnamenkonzept sein.} \\\hline
\end{tabular}
\caption{Testfall K-007}
\label{Testfall K-007}
\end{table}


\begin{table}[H]
\centering
\begin{tabular}{|p{4cm}|p{4cm}|p{4cm}|p{4cm}|}
\hline
Bezeichnung & \textbf{K-008} & Compute Node c6 & Hostnamen / IP / MAC \\ \hline
Beschreibung & \multicolumn{3}{p{12cm}|}{Der Compute Node wird auf die zugewiesene IP Adresse, MAC Adresse und den Hostnamen geprüft.} \\ \hline
Testvoraussetzung & \multicolumn{3}{p{12cm}|}{Der Compute Node und der Testclient befinden sich im selben Netzwerk.} \\ \hline
Testschritte & \multicolumn{3}{p{12cm}|}{
- Compute Node starten (Strom anschliessen)\newline
- 3 Minuten Sekunden warten\newline
- Auf dem Testclient über Putty oder Shell mit dem Befehl \newline \grqq nmap -sn 192.168.1.16\grqq \ eingeben\newline
- Prüfen ob die Zuweisung gemäss Hostnamenkonzept richtig ist.} \\ \hline
Erwartetes Ergebnis & \multicolumn{3}{p{12cm}|}{Die Rückgabewerte (Hostnamen, IP- und MAC-Adresse) sollen identisch mit deren im Hostnamenkonzept sein.} \\\hline
\end{tabular}
\caption{Testfall K-008}
\label{Testfall K-008}
\end{table}


\begin{table}[H]
\centering
\begin{tabular}{|p{4cm}|p{4cm}|p{4cm}|p{4cm}|}
\hline
Bezeichnung & \textbf{K-009} & Compute Node c7 & Hostnamen / IP / MAC \\ \hline
Beschreibung & \multicolumn{3}{p{12cm}|}{Der Compute Node wird auf die zugewiesene IP Adresse, MAC Adresse und den Hostnamen geprüft.} \\ \hline
Testvoraussetzung & \multicolumn{3}{p{12cm}|}{Der Compute Node und der Testclient befinden sich im selben Netzwerk.} \\ \hline
Testschritte & \multicolumn{3}{p{12cm}|}{
- Compute Node starten (Strom anschliessen)\newline
- 3 Minuten Sekunden warten\newline
- Auf dem Testclient über Putty oder Shell mit dem Befehl \newline \grqq nmap -sn 192.168.1.17\grqq \ eingeben\newline
- Prüfen ob die Zuweisung gemäss Hostnamenkonzept richtig ist.} \\ \hline
Erwartetes Ergebnis & \multicolumn{3}{p{12cm}|}{Die Rückgabewerte (Hostnamen, IP- und MAC-Adresse) sollen identisch mit deren im Hostnamenkonzept sein.} \\\hline
\end{tabular}
\caption{Testfall K-009}
\label{Testfall K-009}
\end{table}


\begin{table}[H]
\centering
\begin{tabular}{|p{4cm}|p{4cm}|p{4cm}|p{4cm}|}
\hline
Bezeichnung & \textbf{K-010} & Compute Node c8 & Hostnamen / IP / MAC \\ \hline
Beschreibung & \multicolumn{3}{p{12cm}|}{Der Compute Node wird auf die zugewiesene IP Adresse, MAC Adresse und den Hostnamen geprüft.} \\ \hline
Testvoraussetzung & \multicolumn{3}{p{12cm}|}{Der Compute Node und der Testclient befinden sich im selben Netzwerk.} \\ \hline
Testschritte & \multicolumn{3}{p{12cm}|}{
- Compute Node starten (Strom anschliessen)\newline
- 3 Minuten Sekunden warten\newline
- Auf dem Testclient über Putty oder Shell mit dem Befehl \newline \grqq nmap -sn 192.168.1.18\grqq \ eingeben\newline
- Prüfen ob die Zuweisung gemäss Hostnamenkonzept richtig ist.} \\ \hline
Erwartetes Ergebnis & \multicolumn{3}{p{12cm}|}{Die Rückgabewerte (Hostnamen, IP- und MAC-Adresse) sollen identisch mit deren im Hostnamenkonzept sein.} \\\hline
\end{tabular}
\caption{Testfall K-010}
\label{Testfall K-010}
\end{table}



\begin{table}[H]
\centering
\begin{tabular}{|p{4cm}|p{4cm}|p{4cm}|p{4cm}|}
\hline
Bezeichnung & \textbf{K-011} & Compute Node c9 & Hostnamen / IP / MAC \\ \hline
Beschreibung & \multicolumn{3}{p{12cm}|}{Der Compute Node wird auf die zugewiesene IP Adresse, MAC Adresse und den Hostnamen geprüft.} \\ \hline
Testvoraussetzung & \multicolumn{3}{p{12cm}|}{Der Compute Node und der Testclient befinden sich im selben Netzwerk.} \\ \hline
Testschritte & \multicolumn{3}{p{12cm}|}{
- Compute Node starten (Strom anschliessen)\newline
- 3 Minuten Sekunden warten\newline
- Auf dem Testclient über Putty oder Shell mit dem Befehl \newline \grqq nmap -sn 192.168.1.19\grqq \ eingeben\newline
- Prüfen ob die Zuweisung gemäss Hostnamenkonzept richtig ist.} \\ \hline
Erwartetes Ergebnis & \multicolumn{3}{p{12cm}|}{Die Rückgabewerte (Hostnamen, IP- und MAC-Adresse) sollen identisch mit deren im Hostnamenkonzept sein.} \\\hline
\end{tabular}
\caption{Testfall K-011}
\label{Testfall K-011}
\end{table}


\begin{table}[H]
\centering
\begin{tabular}{|p{4cm}|p{4cm}|p{4cm}|p{4cm}|}
\hline
Bezeichnung & \textbf{K-012} & Compute Node c10 & Hostnamen / IP / MAC \\ \hline
Beschreibung & \multicolumn{3}{p{12cm}|}{Der Compute Node wird auf die zugewiesene IP Adresse, MAC Adresse und den Hostnamen geprüft.} \\ \hline
Testvoraussetzung & \multicolumn{3}{p{12cm}|}{Der Compute Node und der Testclient befinden sich im selben Netzwerk.} \\ \hline
Testschritte & \multicolumn{3}{p{12cm}|}{
- Compute Node starten (Strom anschliessen)\newline
- 3 Minuten Sekunden warten\newline
- Auf dem Testclient über Putty oder Shell mit dem Befehl \newline \grqq nmap -sn 192.168.1.20\grqq \ eingeben\newline
- Prüfen ob die Zuweisung gemäss Hostnamenkonzept richtig ist.} \\ \hline
Erwartetes Ergebnis & \multicolumn{3}{p{12cm}|}{Die Rückgabewerte (Hostnamen, IP- und MAC-Adresse) sollen identisch mit deren im Hostnamenkonzept sein.} \\\hline
\end{tabular}
\caption{Testfall K-012}
\label{Testfall K-012}
\end{table}


\begin{table}[H]
\centering
\begin{tabular}{|p{4cm}|p{4cm}|p{4cm}|p{4cm}|}
\hline
Bezeichnung & \textbf{K-013} & Compute Node c11 & Hostnamen / IP / MAC \\ \hline
Beschreibung & \multicolumn{3}{p{12cm}|}{Der Compute Node wird auf die zugewiesene IP Adresse, MAC Adresse und den Hostnamen geprüft.} \\ \hline
Testvoraussetzung & \multicolumn{3}{p{12cm}|}{Der Compute Node und der Testclient befinden sich im selben Netzwerk.} \\ \hline
Testschritte & \multicolumn{3}{p{12cm}|}{
- Compute Node starten (Strom anschliessen)\newline
- 3 Minuten Sekunden warten\newline
- Auf dem Testclient über Putty oder Shell mit dem Befehl \newline \grqq nmap -sn 192.168.1.21\grqq \ eingeben\newline
- Prüfen ob die Zuweisung gemäss Hostnamenkonzept richtig ist.} \\ \hline
Erwartetes Ergebnis & \multicolumn{3}{p{12cm}|}{Die Rückgabewerte (Hostnamen, IP- und MAC-Adresse) sollen identisch mit deren im Hostnamenkonzept sein.} \\\hline
\end{tabular}
\caption{Testfall K-013}
\label{Testfall K-013}
\end{table}


\begin{table}[H]
\centering
\begin{tabular}{|p{4cm}|p{4cm}|p{4cm}|p{4cm}|}
\hline
Bezeichnung & \textbf{K-014} & Compute Node c12 & Hostnamen / IP / MAC \\ \hline
Beschreibung & \multicolumn{3}{p{12cm}|}{Der Compute Node wird auf die zugewiesene IP Adresse, MAC Adresse und den Hostnamen geprüft.} \\ \hline
Testvoraussetzung & \multicolumn{3}{p{12cm}|}{Der Compute Node und der Testclient befinden sich im selben Netzwerk.} \\ \hline
Testschritte & \multicolumn{3}{p{12cm}|}{
- Compute Node starten (Strom anschliessen)\newline
- 3 Minuten Sekunden warten\newline
- Auf dem Testclient über Putty oder Shell mit dem Befehl \newline \grqq nmap -sn 192.168.1.22\grqq \ eingeben\newline
- Prüfen ob die Zuweisung gemäss Hostnamenkonzept richtig ist.} \\ \hline
Erwartetes Ergebnis & \multicolumn{3}{p{12cm}|}{Die Rückgabewerte (Hostnamen, IP- und MAC-Adresse) sollen identisch mit deren im Hostnamenkonzept sein.} \\\hline
\end{tabular}
\caption{Testfall K-014}
\label{Testfall K-014}
\end{table}


\begin{table}[H]
\centering
\begin{tabular}{|p{4cm}|p{4cm}|p{4cm}|p{4cm}|}
\hline
Bezeichnung & \textbf{K-015} & Compute Node c13 & Hostnamen / IP / MAC \\ \hline
Beschreibung & \multicolumn{3}{p{12cm}|}{Der Compute Node wird auf die zugewiesene IP Adresse, MAC Adresse und den Hostnamen geprüft.} \\ \hline
Testvoraussetzung & \multicolumn{3}{p{12cm}|}{Der Compute Node und der Testclient befinden sich im selben Netzwerk.} \\ \hline
Testschritte & \multicolumn{3}{p{12cm}|}{
- Compute Node starten (Strom anschliessen)\newline
- 3 Minuten Sekunden warten\newline
- Auf dem Testclient über Putty oder Shell mit dem Befehl \newline \grqq nmap -sn 192.168.1.23\grqq \ eingeben\newline
- Prüfen ob die Zuweisung gemäss Hostnamenkonzept richtig ist.} \\ \hline
Erwartetes Ergebnis & \multicolumn{3}{p{12cm}|}{Die Rückgabewerte (Hostnamen, IP- und MAC-Adresse) sollen identisch mit deren im Hostnamenkonzept sein.} \\\hline
\end{tabular}
\caption{Testfall K-015}
\label{Testfall K-015}
\end{table}


\begin{table}[H]
\centering
\begin{tabular}{|p{4cm}|p{4cm}|p{4cm}|p{4cm}|}
\hline
Bezeichnung & \textbf{K-016} & Compute Node c14 & Hostnamen / IP / MAC \\ \hline
Beschreibung & \multicolumn{3}{p{12cm}|}{Der Compute Node wird auf die zugewiesene IP Adresse, MAC Adresse und den Hostnamen geprüft.} \\ \hline
Testvoraussetzung & \multicolumn{3}{p{12cm}|}{Der Compute Node und der Testclient befinden sich im selben Netzwerk.} \\ \hline
Testschritte & \multicolumn{3}{p{12cm}|}{
- Compute Node starten (Strom anschliessen)\newline
- 3 Minuten Sekunden warten\newline
- Auf dem Testclient über Putty oder Shell mit dem Befehl \newline \grqq nmap -sn 192.168.1.24\grqq \ eingeben\newline
- Prüfen ob die Zuweisung gemäss Hostnamenkonzept richtig ist.} \\ \hline
Erwartetes Ergebnis & \multicolumn{3}{p{12cm}|}{Die Rückgabewerte (Hostnamen, IP- und MAC-Adresse) sollen identisch mit deren im Hostnamenkonzept sein.} \\\hline
\end{tabular}
\caption{Testfall K-016}
\label{Testfall K-016}
\end{table}


\begin{table}[H]
\centering
\begin{tabular}{|p{4cm}|p{4cm}|p{4cm}|p{4cm}|}
\hline
Bezeichnung & \textbf{K-017} & Compute Node c15 & Hostnamen / IP / MAC \\ \hline
Beschreibung & \multicolumn{3}{p{12cm}|}{Der Compute Node wird auf die zugewiesene IP Adresse, MAC Adresse und den Hostnamen geprüft.} \\ \hline
Testvoraussetzung & \multicolumn{3}{p{12cm}|}{Der Compute Node und der Testclient befinden sich im selben Netzwerk.} \\ \hline
Testschritte & \multicolumn{3}{p{12cm}|}{
- Compute Node starten (Strom anschliessen)\newline
- 3 Minuten Sekunden warten\newline
- Auf dem Testclient über Putty oder Shell mit dem Befehl \newline \grqq nmap -sn 192.168.1.25\grqq \ eingeben\newline
- Prüfen ob die Zuweisung gemäss Hostnamenkonzept richtig ist.} \\ \hline
Erwartetes Ergebnis & \multicolumn{3}{p{12cm}|}{Die Rückgabewerte (Hostnamen, IP- und MAC-Adresse) sollen identisch mit deren im Hostnamenkonzept sein.} \\\hline
\end{tabular}
\caption{Testfall K-017}
\label{Testfall K-017}
\end{table}


\begin{table}[H]
\centering
\begin{tabular}{|p{4cm}|p{4cm}|p{4cm}|p{4cm}|}
\hline
Bezeichnung & \textbf{K-018} & Compute Node c16 & Hostnamen / IP / MAC \\ \hline
Beschreibung & \multicolumn{3}{p{12cm}|}{Der Compute Node wird auf die zugewiesene IP Adresse, MAC Adresse und den Hostnamen geprüft.} \\ \hline
Testvoraussetzung & \multicolumn{3}{p{12cm}|}{Der Compute Node und der Testclient befinden sich im selben Netzwerk.} \\ \hline
Testschritte & \multicolumn{3}{p{12cm}|}{
- Compute Node starten (Strom anschliessen)\newline
- 3 Minuten Sekunden warten\newline
- Auf dem Testclient über Putty oder Shell mit dem Befehl \newline \grqq nmap -sn 192.168.1.26\grqq \ eingeben\newline
- Prüfen ob die Zuweisung gemäss Hostnamenkonzept richtig ist.} \\ \hline
Erwartetes Ergebnis & \multicolumn{3}{p{12cm}|}{Die Rückgabewerte (Hostnamen, IP- und MAC-Adresse) sollen identisch mit deren im Hostnamenkonzept sein.} \\\hline
\end{tabular}
\caption{Testfall K-018}
\label{Testfall K-018}
\end{table}


\begin{table}[H]
\centering
\begin{tabular}{|p{4cm}|p{4cm}|p{4cm}|p{4cm}|}
\hline
Bezeichnung & \textbf{K-019} & Compute Node c17 & Hostnamen / IP / MAC \\ \hline
Beschreibung & \multicolumn{3}{p{12cm}|}{Der Compute Node wird auf die zugewiesene IP Adresse, MAC Adresse und den Hostnamen geprüft.} \\ \hline
Testvoraussetzung & \multicolumn{3}{p{12cm}|}{Der Compute Node und der Testclient befinden sich im selben Netzwerk.} \\ \hline
Testschritte & \multicolumn{3}{p{12cm}|}{
- Compute Node starten (Strom anschliessen)\newline
- 3 Minuten Sekunden warten\newline
- Auf dem Testclient über Putty oder Shell mit dem Befehl \newline \grqq nmap -sn 192.168.1.27\grqq \ eingeben\newline
- Prüfen ob die Zuweisung gemäss Hostnamenkonzept richtig ist.} \\ \hline
Erwartetes Ergebnis & \multicolumn{3}{p{12cm}|}{Die Rückgabewerte (Hostnamen, IP- und MAC-Adresse) sollen identisch mit deren im Hostnamenkonzept sein.} \\\hline
\end{tabular}
\caption{Testfall K-019}
\label{Testfall K-019}
\end{table}


\begin{table}[H]
\centering
\begin{tabular}{|p{4cm}|p{4cm}|p{4cm}|p{4cm}|}
\hline
Bezeichnung & \textbf{K-020} & Compute Node c18 & Hostnamen / IP / MAC \\ \hline
Beschreibung & \multicolumn{3}{p{12cm}|}{Der Compute Node wird auf die zugewiesene IP Adresse, MAC Adresse und den Hostnamen geprüft.} \\ \hline
Testvoraussetzung & \multicolumn{3}{p{12cm}|}{Der Compute Node und der Testclient befinden sich im selben Netzwerk.} \\ \hline
Testschritte & \multicolumn{3}{p{12cm}|}{
- Compute Node starten (Strom anschliessen)\newline
- 3 Minuten Sekunden warten\newline
- Auf dem Testclient über Putty oder Shell mit dem Befehl \newline \grqq nmap -sn 192.168.1.28\grqq \ eingeben\newline
- Prüfen ob die Zuweisung gemäss Hostnamenkonzept richtig ist.} \\ \hline
Erwartetes Ergebnis & \multicolumn{3}{p{12cm}|}{Die Rückgabewerte (Hostnamen, IP- und MAC-Adresse) sollen identisch mit deren im Hostnamenkonzept sein.} \\\hline
\end{tabular}
\caption{Testfall K-020}
\label{Testfall K-020}
\end{table}


\begin{table}[H]
\centering
\begin{tabular}{|p{4cm}|p{4cm}|p{4cm}|p{4cm}|}
\hline
Bezeichnung & \textbf{K-021} & Compute Node c19 & Hostnamen / IP / MAC \\ \hline
Beschreibung & \multicolumn{3}{p{12cm}|}{Der Compute Node wird auf die zugewiesene IP Adresse, MAC Adresse und den Hostnamen geprüft.} \\ \hline
Testvoraussetzung & \multicolumn{3}{p{12cm}|}{Der Compute Node und der Testclient befinden sich im selben Netzwerk.} \\ \hline
Testschritte & \multicolumn{3}{p{12cm}|}{
- Compute Node starten (Strom anschliessen)\newline
- 3 Minuten Sekunden warten\newline
- Auf dem Testclient über Putty oder Shell mit dem Befehl \newline \grqq nmap -sn 192.168.1.29\grqq \ eingeben\newline
- Prüfen ob die Zuweisung gemäss Hostnamenkonzept richtig ist.} \\ \hline
Erwartetes Ergebnis & \multicolumn{3}{p{12cm}|}{Die Rückgabewerte (Hostnamen, IP- und MAC-Adresse) sollen identisch mit deren im Hostnamenkonzept sein.} \\\hline
\end{tabular}
\caption{Testfall K-021}
\label{Testfall K-021}
\end{table}



\begin{table}[H]
\centering
\begin{tabular}{|p{4cm}|p{4cm}|p{4cm}|p{4cm}|}
\hline
Bezeichnung & \textbf{K-022} & Compute Node c20 & Hostnamen / IP / MAC \\ \hline
Beschreibung & \multicolumn{3}{p{12cm}|}{Der Compute Node wird auf die zugewiesene IP Adresse, MAC Adresse und den Hostnamen geprüft.} \\ \hline
Testvoraussetzung & \multicolumn{3}{p{12cm}|}{Der Compute Node und der Testclient befinden sich im selben Netzwerk.} \\ \hline
Testschritte & \multicolumn{3}{p{12cm}|}{
- Compute Node starten (Strom anschliessen)\newline
- 3 Minuten Sekunden warten\newline
- Auf dem Testclient über Putty oder Shell mit dem Befehl \newline \grqq nmap -sn 192.168.1.30\grqq \ eingeben\newline
- Prüfen ob die Zuweisung gemäss Hostnamenkonzept richtig ist.} \\ \hline
Erwartetes Ergebnis & \multicolumn{3}{p{12cm}|}{Die Rückgabewerte (Hostnamen, IP- und MAC-Adresse) sollen identisch mit deren im Hostnamenkonzept sein.} \\\hline
\end{tabular}
\caption{Testfall K-022}
\label{Testfall K-022}
\end{table}


\begin{table}[H]
\centering
\begin{tabular}{|p{4cm}|p{4cm}|p{4cm}|p{4cm}|}
\hline
Bezeichnung & \textbf{K-023} & Compute Node c21 & Hostnamen / IP / MAC \\ \hline
Beschreibung & \multicolumn{3}{p{12cm}|}{Der Compute Node wird auf die zugewiesene IP Adresse, MAC Adresse und den Hostnamen geprüft.} \\ \hline
Testvoraussetzung & \multicolumn{3}{p{12cm}|}{Der Compute Node und der Testclient befinden sich im selben Netzwerk.} \\ \hline
Testschritte & \multicolumn{3}{p{12cm}|}{
- Compute Node starten (Strom anschliessen)\newline
- 3 Minuten Sekunden warten\newline
- Auf dem Testclient über Putty oder Shell mit dem Befehl \newline \grqq nmap -sn 192.168.1.31\grqq \ eingeben\newline
- Prüfen ob die Zuweisung gemäss Hostnamenkonzept richtig ist.} \\ \hline
Erwartetes Ergebnis & \multicolumn{3}{p{12cm}|}{Die Rückgabewerte (Hostnamen, IP- und MAC-Adresse) sollen identisch mit deren im Hostnamenkonzept sein.} \\\hline
\end{tabular}
\caption{Testfall K-023}
\label{Testfall K-023}
\end{table}


\begin{table}[H]
\centering
\begin{tabular}{|p{4cm}|p{4cm}|p{4cm}|p{4cm}|}
\hline
Bezeichnung & \textbf{K-024} & Compute Node c22 & Hostnamen / IP / MAC \\ \hline
Beschreibung & \multicolumn{3}{p{12cm}|}{Der Compute Node wird auf die zugewiesene IP Adresse, MAC Adresse und den Hostnamen geprüft.} \\ \hline
Testvoraussetzung & \multicolumn{3}{p{12cm}|}{Der Compute Node und der Testclient befinden sich im selben Netzwerk.} \\ \hline
Testschritte & \multicolumn{3}{p{12cm}|}{
- Compute Node starten (Strom anschliessen)\newline
- 3 Minuten Sekunden warten\newline
- Auf dem Testclient über Putty oder Shell mit dem Befehl \newline \grqq nmap -sn 192.168.1.32\grqq \ eingeben\newline
- Prüfen ob die Zuweisung gemäss Hostnamenkonzept richtig ist.} \\ \hline
Erwartetes Ergebnis & \multicolumn{3}{p{12cm}|}{Die Rückgabewerte (Hostnamen, IP- und MAC-Adresse) sollen identisch mit deren im Hostnamenkonzept sein.} \\\hline
\end{tabular}
\caption{Testfall K-024}
\label{Testfall K-024}
\end{table}


\begin{table}[H]
\centering
\begin{tabular}{|p{4cm}|p{4cm}|p{4cm}|p{4cm}|}
\hline
Bezeichnung & \textbf{K-025} & Compute Node c23 & Hostnamen / IP / MAC \\ \hline
Beschreibung & \multicolumn{3}{p{12cm}|}{Der Compute Node wird auf die zugewiesene IP Adresse, MAC Adresse und den Hostnamen geprüft.} \\ \hline
Testvoraussetzung & \multicolumn{3}{p{12cm}|}{Der Compute Node und der Testclient befinden sich im selben Netzwerk.} \\ \hline
Testschritte & \multicolumn{3}{p{12cm}|}{
- Compute Node starten (Strom anschliessen)\newline
- 3 Minuten Sekunden warten\newline
- Auf dem Testclient über Putty oder Shell mit dem Befehl \newline \grqq nmap -sn 192.168.1.33\grqq \ eingeben\newline
- Prüfen ob die Zuweisung gemäss Hostnamenkonzept richtig ist.} \\ \hline
Erwartetes Ergebnis & \multicolumn{3}{p{12cm}|}{Die Rückgabewerte (Hostnamen, IP- und MAC-Adresse) sollen identisch mit deren im Hostnamenkonzept sein.} \\\hline
\end{tabular}
\caption{Testfall K-025}
\label{Testfall K-025}
\end{table}


\begin{table}[H]
\centering
\begin{tabular}{|p{4cm}|p{4cm}|p{4cm}|p{4cm}|}
\hline
Bezeichnung & \textbf{K-026} & Compute Node c24 & Hostnamen / IP / MAC \\ \hline
Beschreibung & \multicolumn{3}{p{12cm}|}{Der Compute Node wird auf die zugewiesene IP Adresse, MAC Adresse und den Hostnamen geprüft.} \\ \hline
Testvoraussetzung & \multicolumn{3}{p{12cm}|}{Der Compute Node und der Testclient befinden sich im selben Netzwerk.} \\ \hline
Testschritte & \multicolumn{3}{p{12cm}|}{
- Compute Node starten (Strom anschliessen)\newline
- 3 Minuten Sekunden warten\newline
- Auf dem Testclient über Putty oder Shell mit dem Befehl \newline \grqq nmap -sn 192.168.1.34\grqq \ eingeben\newline
- Prüfen ob die Zuweisung gemäss Hostnamenkonzept richtig ist.} \\ \hline
Erwartetes Ergebnis & \multicolumn{3}{p{12cm}|}{Die Rückgabewerte (Hostnamen, IP- und MAC-Adresse) sollen identisch mit deren im Hostnamenkonzept sein.} \\\hline
\end{tabular}
\caption{Testfall K-026}
\label{Testfall K-026}
\end{table}


\begin{table}[H]
\centering
\begin{tabular}{|p{4cm}|p{4cm}|p{4cm}|p{4cm}|}
\hline
Bezeichnung & \textbf{K-027} & Compute Node c25 & Hostnamen / IP / MAC \\ \hline
Beschreibung & \multicolumn{3}{p{12cm}|}{Der Compute Node wird auf die zugewiesene IP Adresse, MAC Adresse und den Hostnamen geprüft.} \\ \hline
Testvoraussetzung & \multicolumn{3}{p{12cm}|}{Der Compute Node und der Testclient befinden sich im selben Netzwerk.} \\ \hline
Testschritte & \multicolumn{3}{p{12cm}|}{
- Compute Node starten (Strom anschliessen)\newline
- 3 Minuten Sekunden warten\newline
- Auf dem Testclient über Putty oder Shell mit dem Befehl \newline \grqq nmap -sn 192.168.1.35\grqq \ eingeben\newline
- Prüfen ob die Zuweisung gemäss Hostnamenkonzept richtig ist.} \\ \hline
Erwartetes Ergebnis & \multicolumn{3}{p{12cm}|}{Die Rückgabewerte (Hostnamen, IP- und MAC-Adresse) sollen identisch mit deren im Hostnamenkonzept sein.} \\\hline
\end{tabular}
\caption{Testfall K-027}
\label{Testfall K-027}
\end{table}


\begin{table}[H]
\centering
\begin{tabular}{|p{4cm}|p{4cm}|p{4cm}|p{4cm}|}
\hline
Bezeichnung & \textbf{K-028} & Compute Node c26 & Hostnamen / IP / MAC \\ \hline
Beschreibung & \multicolumn{3}{p{12cm}|}{Der Compute Node wird auf die zugewiesene IP Adresse, MAC Adresse und den Hostnamen geprüft.} \\ \hline
Testvoraussetzung & \multicolumn{3}{p{12cm}|}{Der Compute Node und der Testclient befinden sich im selben Netzwerk.} \\ \hline
Testschritte & \multicolumn{3}{p{12cm}|}{
- Compute Node starten (Strom anschliessen)\newline
- 3 Minuten Sekunden warten\newline
- Auf dem Testclient über Putty oder Shell mit dem Befehl \newline \grqq nmap -sn 192.168.1.36\grqq \ eingeben\newline
- Prüfen ob die Zuweisung gemäss Hostnamenkonzept richtig ist.} \\ \hline
Erwartetes Ergebnis & \multicolumn{3}{p{12cm}|}{Die Rückgabewerte (Hostnamen, IP- und MAC-Adresse) sollen identisch mit deren im Hostnamenkonzept sein.} \\\hline
\end{tabular}
\caption{Testfall K-028}
\label{Testfall K-028}
\end{table}


\begin{table}[H]
\centering
\begin{tabular}{|p{4cm}|p{4cm}|p{4cm}|p{4cm}|}
\hline
Bezeichnung & \textbf{K-029} & Compute Node c27 & Hostnamen / IP / MAC \\ \hline
Beschreibung & \multicolumn{3}{p{12cm}|}{Der Compute Node wird auf die zugewiesene IP Adresse, MAC Adresse und den Hostnamen geprüft.} \\ \hline
Testvoraussetzung & \multicolumn{3}{p{12cm}|}{Der Compute Node und der Testclient befinden sich im selben Netzwerk.} \\ \hline
Testschritte & \multicolumn{3}{p{12cm}|}{
- Compute Node starten (Strom anschliessen)\newline
- 3 Minuten Sekunden warten\newline
- Auf dem Testclient über Putty oder Shell mit dem Befehl \newline \grqq nmap -sn 192.168.1.37\grqq \ eingeben\newline
- Prüfen ob die Zuweisung gemäss Hostnamenkonzept richtig ist.} \\ \hline
Erwartetes Ergebnis & \multicolumn{3}{p{12cm}|}{Die Rückgabewerte (Hostnamen, IP- und MAC-Adresse) sollen identisch mit deren im Hostnamenkonzept sein.} \\\hline
\end{tabular}
\caption{Testfall K-029}
\label{Testfall K-029}
\end{table}


\begin{table}[H]
\centering
\begin{tabular}{|p{4cm}|p{4cm}|p{4cm}|p{4cm}|}
\hline
Bezeichnung & \textbf{K-030} & Compute Node c28 & Hostnamen / IP / MAC \\ \hline
Beschreibung & \multicolumn{3}{p{12cm}|}{Der Compute Node wird auf die zugewiesene IP Adresse, MAC Adresse und den Hostnamen geprüft.} \\ \hline
Testvoraussetzung & \multicolumn{3}{p{12cm}|}{Der Compute Node und der Testclient befinden sich im selben Netzwerk.} \\ \hline
Testschritte & \multicolumn{3}{p{12cm}|}{
- Compute Node starten (Strom anschliessen)\newline
- 3 Minuten Sekunden warten\newline
- Auf dem Testclient über Putty oder Shell mit dem Befehl \newline \grqq nmap -sn 192.168.1.38\grqq \ eingeben\newline
- Prüfen ob die Zuweisung gemäss Hostnamenkonzept richtig ist.} \\ \hline
Erwartetes Ergebnis & \multicolumn{3}{p{12cm}|}{Die Rückgabewerte (Hostnamen, IP- und MAC-Adresse) sollen identisch mit deren im Hostnamenkonzept sein.} \\\hline
\end{tabular}
\caption{Testfall K-030}
\label{Testfall K-030}
\end{table}


\begin{table}[H]
\centering
\begin{tabular}{|p{4cm}|p{4cm}|p{4cm}|p{4cm}|}
\hline
Bezeichnung & \textbf{K-031} & Compute Node c29 & Hostnamen / IP / MAC \\ \hline
Beschreibung & \multicolumn{3}{p{12cm}|}{Der Compute Node wird auf die zugewiesene IP Adresse, MAC Adresse und den Hostnamen geprüft.} \\ \hline
Testvoraussetzung & \multicolumn{3}{p{12cm}|}{Der Compute Node und der Testclient befinden sich im selben Netzwerk.} \\ \hline
Testschritte & \multicolumn{3}{p{12cm}|}{
- Compute Node starten (Strom anschliessen)\newline
- 3 Minuten Sekunden warten\newline
- Auf dem Testclient über Putty oder Shell mit dem Befehl \newline \grqq nmap -sn 192.168.1.39\grqq \ eingeben\newline
- Prüfen ob die Zuweisung gemäss Hostnamenkonzept richtig ist.} \\ \hline
Erwartetes Ergebnis & \multicolumn{3}{p{12cm}|}{Die Rückgabewerte (Hostnamen, IP- und MAC-Adresse) sollen identisch mit deren im Hostnamenkonzept sein.} \\\hline
\end{tabular}
\caption{Testfall K-031}
\label{Testfall K-031}
\end{table}


\begin{table}[H]
\centering
\begin{tabular}{|p{4cm}|p{4cm}|p{4cm}|p{4cm}|}
\hline
Bezeichnung & \textbf{K-032} & Compute Node c30 & Hostnamen / IP / MAC \\ \hline
Beschreibung & \multicolumn{3}{p{12cm}|}{Der Compute Node wird auf die zugewiesene IP Adresse, MAC Adresse und den Hostnamen geprüft.} \\ \hline
Testvoraussetzung & \multicolumn{3}{p{12cm}|}{Der Compute Node und der Testclient befinden sich im selben Netzwerk.} \\ \hline
Testschritte & \multicolumn{3}{p{12cm}|}{
- Compute Node starten (Strom anschliessen)\newline
- 3 Minuten Sekunden warten\newline
- Auf dem Testclient über Putty oder Shell mit dem Befehl \newline \grqq nmap -sn 192.168.1.40\grqq \ eingeben\newline
- Prüfen ob die Zuweisung gemäss Hostnamenkonzept richtig ist.} \\ \hline
Erwartetes Ergebnis & \multicolumn{3}{p{12cm}|}{Die Rückgabewerte (Hostnamen, IP- und MAC-Adresse) sollen identisch mit deren im Hostnamenkonzept sein.} \\\hline
\end{tabular}
\caption{Testfall K-032}
\label{Testfall K-032}
\end{table}



\begin{table}[H]
\centering
\begin{tabular}{|p{4cm}|p{4cm}|p{4cm}|p{4cm}|}
\hline
Bezeichnung & \textbf{K-033} & Compute Node c31 & Hostnamen / IP / MAC \\ \hline
Beschreibung & \multicolumn{3}{p{12cm}|}{Der Compute Node wird auf die zugewiesene IP Adresse, MAC Adresse und den Hostnamen geprüft.} \\ \hline
Testvoraussetzung & \multicolumn{3}{p{12cm}|}{Der Compute Node und der Testclient befinden sich im selben Netzwerk.} \\ \hline
Testschritte & \multicolumn{3}{p{12cm}|}{
- Compute Node starten (Strom anschliessen)\newline
- 3 Minuten Sekunden warten\newline
- Auf dem Testclient über Putty oder Shell mit dem Befehl \newline \grqq nmap -sn 192.168.1.41\grqq \ eingeben\newline
- Prüfen ob die Zuweisung gemäss Hostnamenkonzept richtig ist.} \\ \hline
Erwartetes Ergebnis & \multicolumn{3}{p{12cm}|}{Die Rückgabewerte (Hostnamen, IP- und MAC-Adresse) sollen identisch mit deren im Hostnamenkonzept sein.} \\\hline
\end{tabular}
\caption{Testfall K-033}
\label{Testfall K-033}
\end{table}


\begin{table}[H]
\centering
\begin{tabular}{|p{4cm}|p{4cm}|p{4cm}|p{4cm}|}
\hline
Bezeichnung & \textbf{K-034} & Compute Node c32 & Hostnamen / IP / MAC \\ \hline
Beschreibung & \multicolumn{3}{p{12cm}|}{Der Compute Node wird auf die zugewiesene IP Adresse, MAC Adresse und den Hostnamen geprüft.} \\ \hline
Testvoraussetzung & \multicolumn{3}{p{12cm}|}{Der Compute Node und der Testclient befinden sich im selben Netzwerk.} \\ \hline
Testschritte & \multicolumn{3}{p{12cm}|}{
- Compute Node starten (Strom anschliessen)\newline
- 3 Minuten Sekunden warten\newline
- Auf dem Testclient über Putty oder Shell mit dem Befehl \newline \grqq nmap -sn 192.168.1.42\grqq \ eingeben\newline
- Prüfen ob die Zuweisung gemäss Hostnamenkonzept richtig ist.} \\ \hline
Erwartetes Ergebnis & \multicolumn{3}{p{12cm}|}{Die Rückgabewerte (Hostnamen, IP- und MAC-Adresse) sollen identisch mit deren im Hostnamenkonzept sein.} \\\hline
\end{tabular}
\caption{Testfall K-034}
\label{Testfall K-034}
\end{table}


\begin{table}[H]
\centering
\begin{tabular}{|p{4cm}|p{4cm}|p{4cm}|p{4cm}|}
\hline
Bezeichnung & \textbf{K-035} & Compute Node c33 & Hostnamen / IP / MAC \\ \hline
Beschreibung & \multicolumn{3}{p{12cm}|}{Der Compute Node wird auf die zugewiesene IP Adresse, MAC Adresse und den Hostnamen geprüft.} \\ \hline
Testvoraussetzung & \multicolumn{3}{p{12cm}|}{Der Compute Node und der Testclient befinden sich im selben Netzwerk.} \\ \hline
Testschritte & \multicolumn{3}{p{12cm}|}{
- Compute Node starten (Strom anschliessen)\newline
- 3 Minuten Sekunden warten\newline
- Auf dem Testclient über Putty oder Shell mit dem Befehl \newline \grqq nmap -sn 192.168.1.43\grqq \ eingeben\newline
- Prüfen ob die Zuweisung gemäss Hostnamenkonzept richtig ist.} \\ \hline
Erwartetes Ergebnis & \multicolumn{3}{p{12cm}|}{Die Rückgabewerte (Hostnamen, IP- und MAC-Adresse) sollen identisch mit deren im Hostnamenkonzept sein.} \\\hline
\end{tabular}
\caption{Testfall K-035}
\label{Testfall K-035}
\end{table}


\begin{table}[H]
\centering
\begin{tabular}{|p{4cm}|p{4cm}|p{4cm}|p{4cm}|}
\hline
Bezeichnung & \textbf{K-036} & Compute Node c34 & Hostnamen / IP / MAC \\ \hline
Beschreibung & \multicolumn{3}{p{12cm}|}{Der Compute Node wird auf die zugewiesene IP Adresse, MAC Adresse und den Hostnamen geprüft.} \\ \hline
Testvoraussetzung & \multicolumn{3}{p{12cm}|}{Der Compute Node und der Testclient befinden sich im selben Netzwerk.} \\ \hline
Testschritte & \multicolumn{3}{p{12cm}|}{
- Compute Node starten (Strom anschliessen)\newline
- 3 Minuten Sekunden warten\newline
- Auf dem Testclient über Putty oder Shell mit dem Befehl \newline \grqq nmap -sn 192.168.1.44\grqq \ eingeben\newline
- Prüfen ob die Zuweisung gemäss Hostnamenkonzept richtig ist.} \\ \hline
Erwartetes Ergebnis & \multicolumn{3}{p{12cm}|}{Die Rückgabewerte (Hostnamen, IP- und MAC-Adresse) sollen identisch mit deren im Hostnamenkonzept sein.} \\\hline
\end{tabular}
\caption{Testfall K-036}
\label{Testfall K-036}
\end{table}


\begin{table}[H]
\centering
\begin{tabular}{|p{4cm}|p{4cm}|p{4cm}|p{4cm}|}
\hline
Bezeichnung & \textbf{K-037} & Compute Node c35 & Hostnamen / IP / MAC \\ \hline
Beschreibung & \multicolumn{3}{p{12cm}|}{Der Compute Node wird auf die zugewiesene IP Adresse, MAC Adresse und den Hostnamen geprüft.} \\ \hline
Testvoraussetzung & \multicolumn{3}{p{12cm}|}{Der Compute Node und der Testclient befinden sich im selben Netzwerk.} \\ \hline
Testschritte & \multicolumn{3}{p{12cm}|}{
- Compute Node starten (Strom anschliessen)\newline
- 3 Minuten Sekunden warten\newline
- Auf dem Testclient über Putty oder Shell mit dem Befehl \newline \grqq nmap -sn 192.168.1.45\grqq \ eingeben\newline
- Prüfen ob die Zuweisung gemäss Hostnamenkonzept richtig ist.} \\ \hline
Erwartetes Ergebnis & \multicolumn{3}{p{12cm}|}{Die Rückgabewerte (Hostnamen, IP- und MAC-Adresse) sollen identisch mit deren im Hostnamenkonzept sein.} \\\hline
\end{tabular}
\caption{Testfall K-037}
\label{Testfall K-037}
\end{table}


\begin{table}[H]
\centering
\begin{tabular}{|p{4cm}|p{4cm}|p{4cm}|p{4cm}|}
\hline
Bezeichnung & \textbf{K-038} & Compute Node c36 & Hostnamen / IP / MAC \\ \hline
Beschreibung & \multicolumn{3}{p{12cm}|}{Der Compute Node wird auf die zugewiesene IP Adresse, MAC Adresse und den Hostnamen geprüft.} \\ \hline
Testvoraussetzung & \multicolumn{3}{p{12cm}|}{Der Compute Node und der Testclient befinden sich im selben Netzwerk.} \\ \hline
Testschritte & \multicolumn{3}{p{12cm}|}{
- Compute Node starten (Strom anschliessen)\newline
- 3 Minuten Sekunden warten\newline
- Auf dem Testclient über Putty oder Shell mit dem Befehl \newline \grqq nmap -sn 192.168.1.46\grqq \ eingeben\newline
- Prüfen ob die Zuweisung gemäss Hostnamenkonzept richtig ist.} \\ \hline
Erwartetes Ergebnis & \multicolumn{3}{p{12cm}|}{Die Rückgabewerte (Hostnamen, IP- und MAC-Adresse) sollen identisch mit deren im Hostnamenkonzept sein.} \\\hline
\end{tabular}
\caption{Testfall K-038}
\label{Testfall K-038}
\end{table}


\begin{table}[H]
\centering
\begin{tabular}{|p{4cm}|p{4cm}|p{4cm}|p{4cm}|}
\hline
Bezeichnung & \textbf{K-039} & Compute Node c37 & Hostnamen / IP / MAC \\ \hline
Beschreibung & \multicolumn{3}{p{12cm}|}{Der Compute Node wird auf die zugewiesene IP Adresse, MAC Adresse und den Hostnamen geprüft.} \\ \hline
Testvoraussetzung & \multicolumn{3}{p{12cm}|}{Der Compute Node und der Testclient befinden sich im selben Netzwerk.} \\ \hline
Testschritte & \multicolumn{3}{p{12cm}|}{
- Compute Node starten (Strom anschliessen)\newline
- 3 Minuten Sekunden warten\newline
- Auf dem Testclient über Putty oder Shell mit dem Befehl \newline \grqq nmap -sn 192.168.1.47\grqq \ eingeben\newline
- Prüfen ob die Zuweisung gemäss Hostnamenkonzept richtig ist.} \\ \hline
Erwartetes Ergebnis & \multicolumn{3}{p{12cm}|}{Die Rückgabewerte (Hostnamen, IP- und MAC-Adresse) sollen identisch mit deren im Hostnamenkonzept sein.} \\\hline
\end{tabular}
\caption{Testfall K-039}
\label{Testfall K-039}
\end{table}


\begin{table}[H]
\centering
\begin{tabular}{|p{4cm}|p{4cm}|p{4cm}|p{4cm}|}
\hline
Bezeichnung & \textbf{K-040} & Compute Node c38 & Hostnamen / IP / MAC \\ \hline
Beschreibung & \multicolumn{3}{p{12cm}|}{Der Compute Node wird auf die zugewiesene IP Adresse, MAC Adresse und den Hostnamen geprüft.} \\ \hline
Testvoraussetzung & \multicolumn{3}{p{12cm}|}{Der Compute Node und der Testclient befinden sich im selben Netzwerk.} \\ \hline
Testschritte & \multicolumn{3}{p{12cm}|}{
- Compute Node starten (Strom anschliessen)\newline
- 3 Minuten Sekunden warten\newline
- Auf dem Testclient über Putty oder Shell mit dem Befehl \newline \grqq nmap -sn 192.168.1.48\grqq \ eingeben\newline
- Prüfen ob die Zuweisung gemäss Hostnamenkonzept richtig ist.} \\ \hline
Erwartetes Ergebnis & \multicolumn{3}{p{12cm}|}{Die Rückgabewerte (Hostnamen, IP- und MAC-Adresse) sollen identisch mit deren im Hostnamenkonzept sein.} \\\hline
\end{tabular}
\caption{Testfall K-040}
\label{Testfall K-040}
\end{table}


\begin{table}[H]
\centering
\begin{tabular}{|p{4cm}|p{4cm}|p{4cm}|p{4cm}|}
\hline
Bezeichnung & \textbf{K-041} & Compute Node c39 & Hostnamen / IP / MAC \\ \hline
Beschreibung & \multicolumn{3}{p{12cm}|}{Der Compute Node wird auf die zugewiesene IP Adresse, MAC Adresse und den Hostnamen geprüft.} \\ \hline
Testvoraussetzung & \multicolumn{3}{p{12cm}|}{Der Compute Node und der Testclient befinden sich im selben Netzwerk.} \\ \hline
Testschritte & \multicolumn{3}{p{12cm}|}{
- Compute Node starten (Strom anschliessen)\newline
- 3 Minuten Sekunden warten\newline
- Auf dem Testclient über Putty oder Shell mit dem Befehl \newline \grqq nmap -sn 192.168.1.49\grqq \ eingeben\newline
- Prüfen ob die Zuweisung gemäss Hostnamenkonzept richtig ist.} \\ \hline
Erwartetes Ergebnis & \multicolumn{3}{p{12cm}|}{Die Rückgabewerte (Hostnamen, IP- und MAC-Adresse) sollen identisch mit deren im Hostnamenkonzept sein.} \\\hline
\end{tabular}
\caption{Testfall K-041}
\label{Testfall K-041}
\end{table}


\begin{table}[H]
\centering
\begin{tabular}{|p{4cm}|p{4cm}|p{4cm}|p{4cm}|}
\hline
Bezeichnung & \textbf{K-042} & Compute Node c40 & Hostnamen / IP / MAC \\ \hline
Beschreibung & \multicolumn{3}{p{12cm}|}{Der Compute Node wird auf die zugewiesene IP Adresse, MAC Adresse und den Hostnamen geprüft.} \\ \hline
Testvoraussetzung & \multicolumn{3}{p{12cm}|}{Der Compute Node und der Testclient befinden sich im selben Netzwerk.} \\ \hline
Testschritte & \multicolumn{3}{p{12cm}|}{
- Compute Node starten (Strom anschliessen)\newline
- 3 Minuten Sekunden warten\newline
- Auf dem Testclient über Putty oder Shell mit dem Befehl \newline \grqq nmap -sn 192.168.1.50\grqq \ eingeben\newline
- Prüfen ob die Zuweisung gemäss Hostnamenkonzept richtig ist.} \\ \hline
Erwartetes Ergebnis & \multicolumn{3}{p{12cm}|}{Die Rückgabewerte (Hostnamen, IP- und MAC-Adresse) sollen identisch mit deren im Hostnamenkonzept sein.} \\\hline
\end{tabular}
\caption{Testfall K-042}
\label{Testfall K-042}
\end{table}



\begin{table}[H]
\centering
\begin{tabular}{|p{4cm}|p{4cm}|p{4cm}|p{4cm}|}
\hline
Bezeichnung & \textbf{K-043} & Compute Node c41 & Hostnamen / IP / MAC \\ \hline
Beschreibung & \multicolumn{3}{p{12cm}|}{Der Compute Node wird auf die zugewiesene IP Adresse, MAC Adresse und den Hostnamen geprüft.} \\ \hline
Testvoraussetzung & \multicolumn{3}{p{12cm}|}{Der Compute Node und der Testclient befinden sich im selben Netzwerk.} \\ \hline
Testschritte & \multicolumn{3}{p{12cm}|}{
- Compute Node starten (Strom anschliessen)\newline
- 3 Minuten Sekunden warten\newline
- Auf dem Testclient über Putty oder Shell mit dem Befehl \newline \grqq nmap -sn 192.168.1.51\grqq \ eingeben\newline
- Prüfen ob die Zuweisung gemäss Hostnamenkonzept richtig ist.} \\ \hline
Erwartetes Ergebnis & \multicolumn{3}{p{12cm}|}{Die Rückgabewerte (Hostnamen, IP- und MAC-Adresse) sollen identisch mit deren im Hostnamenkonzept sein.} \\\hline
\end{tabular}
\caption{Testfall K-043}
\label{Testfall K-043}
\end{table}


\begin{table}[H]
\centering
\begin{tabular}{|p{4cm}|p{4cm}|p{4cm}|p{4cm}|}
\hline
Bezeichnung & \textbf{K-044} & Compute Node c42 & Hostnamen / IP / MAC \\ \hline
Beschreibung & \multicolumn{3}{p{12cm}|}{Der Compute Node wird auf die zugewiesene IP Adresse, MAC Adresse und den Hostnamen geprüft.} \\ \hline
Testvoraussetzung & \multicolumn{3}{p{12cm}|}{Der Compute Node und der Testclient befinden sich im selben Netzwerk.} \\ \hline
Testschritte & \multicolumn{3}{p{12cm}|}{
- Compute Node starten (Strom anschliessen)\newline
- 3 Minuten Sekunden warten\newline
- Auf dem Testclient über Putty oder Shell mit dem Befehl \newline \grqq nmap -sn 192.168.1.52\grqq \ eingeben\newline
- Prüfen ob die Zuweisung gemäss Hostnamenkonzept richtig ist.} \\ \hline
Erwartetes Ergebnis & \multicolumn{3}{p{12cm}|}{Die Rückgabewerte (Hostnamen, IP- und MAC-Adresse) sollen identisch mit deren im Hostnamenkonzept sein.} \\\hline
\end{tabular}
\caption{Testfall K-044}
\label{Testfall K-044}
\end{table}


\begin{table}[H]
\centering
\begin{tabular}{|p{4cm}|p{4cm}|p{4cm}|p{4cm}|}
\hline
Bezeichnung & \textbf{K-045} & Compute Node c43 & Hostnamen / IP / MAC \\ \hline
Beschreibung & \multicolumn{3}{p{12cm}|}{Der Compute Node wird auf die zugewiesene IP Adresse, MAC Adresse und den Hostnamen geprüft.} \\ \hline
Testvoraussetzung & \multicolumn{3}{p{12cm}|}{Der Compute Node und der Testclient befinden sich im selben Netzwerk.} \\ \hline
Testschritte & \multicolumn{3}{p{12cm}|}{
- Compute Node starten (Strom anschliessen)\newline
- 3 Minuten Sekunden warten\newline
- Auf dem Testclient über Putty oder Shell mit dem Befehl \newline \grqq nmap -sn 192.168.1.53\grqq \ eingeben\newline
- Prüfen ob die Zuweisung gemäss Hostnamenkonzept richtig ist.} \\ \hline
Erwartetes Ergebnis & \multicolumn{3}{p{12cm}|}{Die Rückgabewerte (Hostnamen, IP- und MAC-Adresse) sollen identisch mit deren im Hostnamenkonzept sein.} \\\hline
\end{tabular}
\caption{Testfall K-045}
\label{Testfall K-045}
\end{table}


\begin{table}[H]
\centering
\begin{tabular}{|p{4cm}|p{4cm}|p{4cm}|p{4cm}|}
\hline
Bezeichnung & \textbf{K-046} & Compute Node c44 & Hostnamen / IP / MAC \\ \hline
Beschreibung & \multicolumn{3}{p{12cm}|}{Der Compute Node wird auf die zugewiesene IP Adresse, MAC Adresse und den Hostnamen geprüft.} \\ \hline
Testvoraussetzung & \multicolumn{3}{p{12cm}|}{Der Compute Node und der Testclient befinden sich im selben Netzwerk.} \\ \hline
Testschritte & \multicolumn{3}{p{12cm}|}{
- Compute Node starten (Strom anschliessen)\newline
- 3 Minuten Sekunden warten\newline
- Auf dem Testclient über Putty oder Shell mit dem Befehl \newline \grqq nmap -sn 192.168.1.54\grqq \ eingeben\newline
- Prüfen ob die Zuweisung gemäss Hostnamenkonzept richtig ist.} \\ \hline
Erwartetes Ergebnis & \multicolumn{3}{p{12cm}|}{Die Rückgabewerte (Hostnamen, IP- und MAC-Adresse) sollen identisch mit deren im Hostnamenkonzept sein.} \\\hline
\end{tabular}
\caption{Testfall K-046}
\label{Testfall K-046}
\end{table}


\begin{table}[H]
\centering
\begin{tabular}{|p{4cm}|p{4cm}|p{4cm}|p{4cm}|}
\hline
Bezeichnung & \textbf{K-047} & Compute Node c45 & Hostnamen / IP / MAC \\ \hline
Beschreibung & \multicolumn{3}{p{12cm}|}{Der Compute Node wird auf die zugewiesene IP Adresse, MAC Adresse und den Hostnamen geprüft.} \\ \hline
Testvoraussetzung & \multicolumn{3}{p{12cm}|}{Der Compute Node und der Testclient befinden sich im selben Netzwerk.} \\ \hline
Testschritte & \multicolumn{3}{p{12cm}|}{
- Compute Node starten (Strom anschliessen)\newline
- 3 Minuten Sekunden warten\newline
- Auf dem Testclient über Putty oder Shell mit dem Befehl \newline \grqq nmap -sn 192.168.1.55\grqq \ eingeben\newline
- Prüfen ob die Zuweisung gemäss Hostnamenkonzept richtig ist.} \\ \hline
Erwartetes Ergebnis & \multicolumn{3}{p{12cm}|}{Die Rückgabewerte (Hostnamen, IP- und MAC-Adresse) sollen identisch mit deren im Hostnamenkonzept sein.} \\\hline
\end{tabular}
\caption{Testfall K-047}
\label{Testfall K-047}
\end{table}

\begin{table}[H]
\centering
\begin{tabular}{|p{4cm}|p{4cm}|p{4cm}|p{4cm}|}
\hline
Bezeichnung & \textbf{K-048} & NAS & Erreichbarkeit \\ \hline
Beschreibung & \multicolumn{3}{p{12cm}|}{Das NAS soll auf die Erreichbarkeit geprüft werden.} \\ \hline
Testvoraussetzung & \multicolumn{3}{p{12cm}|}{Das NAS ist am Netzwerk angeschlossen.} \\ \hline
Testschritte & \multicolumn{3}{p{12cm}|}{} \\ \hline
Erwartetes Ergebnis & \multicolumn{3}{p{12cm}|}{Das NAS antwortet auf den Ping befehl mit: \grqq 2 packets transmitted, 2 received, 0\% packet loss\grqq } \\\hline
\end{tabular}
\caption{Testfall K-048}
\label{Testfall K-048}
\end{table}


\subsection{Integrationstests}

\begin{table}[H]
\centering
\begin{tabular}{|p{4cm}|p{4cm}|p{4cm}|p{4cm}|}
\hline
Bezeichnung & \textbf{I-001} & Compute Nodes & Internet Zugang \\ \hline
Beschreibung & \multicolumn{3}{p{12cm}|}{Es wird getestet ob alle Computenodes auf das Internet zugreifen können.} \\ \hline
Testvoraussetzung & \multicolumn{3}{p{12cm}|}{Das Routing ist eingerichtet.} \\ \hline
Testschritte & \multicolumn{3}{p{12cm}|}{
- Den Befehl \grqq  pdsh -w c[1-45] ping -c 1 google.de\grqq \ auf dem Management Node eingeben.} \\ \hline
Erwartetes Ergebnis & \multicolumn{3}{p{12cm}|}{Der Rückgabewert muss pro Compute Node die Ausgabe \grqq 1 packets transmitted, 1 received, 0\% packet loss\grqq beinhalten. Damit der Test erfolgreich war.
} \\\hline
\end{tabular}
\caption{Testfall I-001}
\label{Testfall I-001}
\end{table}

\begin{table}[H]
\centering
\begin{tabular}{|p{4cm}|p{4cm}|p{4cm}|p{4cm}|}
\hline
Bezeichnung & \textbf{I-002} & NAS & Mountpoint \\ \hline
Beschreibung & \multicolumn{3}{p{12cm}|}{Es wird getestet ob der Management Node  automatisch bei einem Systemstart sich mit dem Netzwerkshare verbindet.} \\ \hline
Testvoraussetzung & \multicolumn{3}{p{12cm}|}{Der fstab-Eintrag muss vorhanden sein.} \\ \hline
Testschritte & \multicolumn{3}{p{12cm}|}{
- Management Node starten.\newline
- auf dem Management Node anmelden.\newline
- Den Befehl \grqq ls -l /media/nebula\_data/ | wc -l \grqq \ eingeben.
} \\ \hline
Erwartetes Ergebnis & \multicolumn{3}{p{12cm}|}{Die Ausgabe des Befehls muss einen grösseren Wert als \grqq 1\grqq zurückgeben.} \\\hline
\end{tabular}
\caption{Testfall I-002}
\label{Testfall I-002}
\end{table}

\begin{table}[H]
\centering
\begin{tabular}{|p{4cm}|p{4cm}|p{4cm}|p{4cm}|}
\hline
Bezeichnung & \textbf{I-003} & NAS / Compute Nodes & Mountpoint \\ \hline
Beschreibung & \multicolumn{3}{p{12cm}|}{Es wird getestet ob die Compute Nodes  automatisch bei einem Systemstart sich mit dem Netzwerkshare verbinden.} \\ \hline
Testvoraussetzung & \multicolumn{3}{p{12cm}|}{Der fstab-Eintrag muss vorhanden sein.} \\ \hline
Testschritte & \multicolumn{3}{p{12cm}|}{
- Compute Nodes starten.\newline
- auf dem Management Node anmelden.\newline
- Den Befehl \grqq pdsh -w c[1-45] ls -l /media/nebula\_data/ | wc -l \grqq \ eingeben.
} \\ \hline
Erwartetes Ergebnis & \multicolumn{3}{p{12cm}|}{Die Ausgabe des Befehls muss jewils einen grösseren Wert als \grqq 1\grqq zurückgeben.} \\\hline
\end{tabular}
\caption{Testfall I-003}
\label{Testfall I-003}
\end{table}

\begin{table}[H]
\centering
\begin{tabular}{|p{4cm}|p{4cm}|p{4cm}|p{4cm}|}
\hline
Bezeichnung & \textbf{I-004} & Cluster & Wiederaufbau \\ \hline
Beschreibung & \multicolumn{3}{p{12cm}|}{Es wird getestet ob der Cluster in einem Raum innerhalb von 15 Minuten wiederaufgebaut werden kann.} \\ \hline
Testvoraussetzung & \multicolumn{3}{p{12cm}|}{Cluster ist kompakt aufgebaut} \\ \hline
Testschritte & \multicolumn{3}{p{12cm}|}{
- Cluster herunterfahren.\newline
- Externe physische Verbindungen trennen.\newline
- Cluster in ein anderes Zimmer verschieben und wiederaufbauen (selbes Netzwerk).
} \\ \hline
Erwartetes Ergebnis & \multicolumn{3}{p{12cm}|}{Der Wiederaufbau soll nicht länger als 15 Minuten dauern.} \\\hline
\end{tabular}
\caption{Testfall I-004}
\label{Testfall I-004}
\end{table}

\subsection{Systemtests}
\begin{table}[H]
\centering
\begin{tabular}{|p{4cm}|p{4cm}|p{4cm}|p{4cm}|}
\hline
Bezeichnung & \textbf{S-001} & Compute Nodes & CPU Last \\ \hline
Beschreibung & \multicolumn{3}{p{12cm}|}{Es wird geprüft ob die CPU während des Schürfens der Kryptowährungen zu über 90\% beansprucht wird. } \\ \hline
Testvoraussetzung & \multicolumn{3}{p{12cm}|}{Der Cluster muss am Schürfen einer Kryptowährung sein. Dazu muss er bereits 15 Minuten am schürfen sein, bevor mit dem Test begonnen werden kann.} \\ \hline
Testschritte & \multicolumn{3}{p{12cm}|}{
Die Tests finden vom Management Node aus statt.\newline
- Den Befehl \grqq pdsh -w c[1-45] uptime\grqq \ eingeben. \newline
- Alternativ kann man sich auf jeden Compute Node anmelden und den Befehl \grqq top\grqq \ eingeben. 
} \\ \hline
Erwartetes Ergebnis & \multicolumn{3}{p{12cm}|}{Die Loadaverage-Werte überschreiten jeweils den Wert 1, die Ausgabe sollte in etwa so aussehen: \newline  load average: 4.18, 2.44, 1.16} \\\hline
\end{tabular}
\caption{Testfall S-001}
\label{Testfall S-001}
\end{table}

\begin{table}[H]
\centering
\begin{tabular}{|p{4cm}|p{4cm}|p{4cm}|p{4cm}|}
\hline
Bezeichnung & \textbf{S-002} & Nodes & Nagios Monitoring \\ \hline
Beschreibung & \multicolumn{3}{p{12cm}|}{Es wird geprüft ob alle Nodes von Nagios überwacht werden. } \\ \hline
Testvoraussetzung & \multicolumn{3}{p{12cm}|}{Alle Nodes müssen einmal in Betrieb gewesen sein} \\ \hline
Testschritte & \multicolumn{3}{p{12cm}|}{
- \url{http://nebula/nagios} aufrufen. \newline
- Bei Nagios anmelden. \newline
- Auf den Reiter \grqq Hosts\grqq klicken.
} \\ \hline
Erwartetes Ergebnis & \multicolumn{3}{p{12cm}|}{Es werden alle Nodes aufgelistet. Unabhängig des Status ob der Node in Betrieb ist oder nicht.} \\\hline
\end{tabular}
\caption{Testfall S-002}
\label{Testfall S-002}
\end{table}

\begin{table}[H]
\centering
\begin{tabular}{|p{4cm}|p{4cm}|p{4cm}|p{4cm}|}
\hline
Bezeichnung & \textbf{S-003} & Nodes & Nagios Alarmierung \\ \hline
Beschreibung & \multicolumn{3}{p{12cm}|}{Bei aufgetretenen Problemen soll Nagios eine E-Mail versenden.} \\ \hline
Testvoraussetzung & \multicolumn{3}{p{12cm}|}{Der Cluster muss in Betrieb sein.} \\ \hline
Testschritte & \multicolumn{3}{p{12cm}|}{
- Strom von Node c1 trennen. \newline
- 5 Minuten warten. \newline
- Email prüfen (christoph.amrein86@gmail.com)
} \\ \hline
Erwartetes Ergebnis & \multicolumn{3}{p{12cm}|}{Es trifft eine Alarmierungs-Email ein.} \\\hline
\end{tabular}
\caption{Testfall S-004}
\label{Testfall S-003}
\end{table}

\begin{table}[H]
\centering
\begin{tabular}{|p{4cm}|p{4cm}|p{4cm}|p{4cm}|}
\hline
Bezeichnung & \textbf{S-004} & Nodes & Ganglia Monitoring \\ \hline
Beschreibung & \multicolumn{3}{p{12cm}|}{Es wird geprüft ob alle Nodes von Ganglia gemonitored werden. } \\ \hline
Testvoraussetzung & \multicolumn{3}{p{12cm}|}{Alle Nodes müssen einmal in Betrieb gewesen sein} \\ \hline
Testschritte & \multicolumn{3}{p{12cm}|}{
- \url{http://nebula/ganglia} aufrufen. \newline
- Quelle \grqq  Nebula\grqq auswählen.
} \\ \hline
Erwartetes Ergebnis & \multicolumn{3}{p{12cm}|}{Es werden alle Nodes aufgelistet. } \\\hline
\end{tabular}
\caption{Testfall S-004}
\label{Testfall S-004}
\end{table}

\begin{table}[H]
\centering
\begin{tabular}{|p{4cm}|p{4cm}|p{4cm}|p{4cm}|}
\hline
Bezeichnung & \textbf{S-005} & Nodes & Cluster Jobs \\ \hline
Beschreibung & \multicolumn{3}{p{12cm}|}{Es wird geprüft ob Jobs auf dem Cluster laufen. } \\ \hline
Testvoraussetzung & \multicolumn{3}{p{12cm}|}{Der Cluster muss in Betrieb sein.} \\ \hline
Testschritte & \multicolumn{3}{p{12cm}|}{
Auf dem Management Node sind folgende Befehle abzusetzen: \newline
- \grqq mpicc -O3 /opt/ohpc/pub/examples/mpi/hello.c\grqq \newline
- \grqq srun -n 8 -N 2 --pty /bin/bash\grqq \newline
- \grqq squeue\grqq
} \\ \hline
Erwartetes Ergebnis & \multicolumn{3}{p{12cm}|}{Der Job wird in der Queue angezeigt und ist 2 Nodes zugewiesen. } \\\hline
\end{tabular}
\caption{Testfall S-005}
\label{Testfall S-005}
\end{table}


\begin{table}[H]
\centering
\begin{tabular}{|p{4cm}|p{4cm}|p{4cm}|p{4cm}|}
\hline
Bezeichnung & \textbf{S-006} & Nodes & Schürfen \\ \hline
Beschreibung & \multicolumn{3}{p{12cm}|}{Es wird geprüft ob über den Cluster Kryptowährung über einen Job geschürft werden kann. } \\ \hline
Testvoraussetzung & \multicolumn{3}{p{12cm}|}{Der Cluster muss in Betrieb sein.} \\ \hline
Testschritte & \multicolumn{3}{p{12cm}|}{
Auf dem Management Node sind folgende Befehle als root abzusetzen: \newline
- \grqq cd /opt/miners/tkinjo\grqq \newline
- \grqq srun --nodes=40-45 --ntasks=40 --cpus-per-task=4 ./cpuminer  -a cryptonight -o stratum+tcp://xdn.pool.minergate.com:45620 -u x -p x\grqq \newline
} \\ \hline
Erwartetes Ergebnis & \multicolumn{3}{p{12cm}|}{Es soll direkt in der Shell die Ausgabe des Miners ausgegeben werden. } \\\hline
\end{tabular}
\caption{Testfall S-006}
\label{Testfall S-006}
\end{table}