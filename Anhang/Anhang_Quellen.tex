% !TEX root = ../Diplombericht.tex
\section{Quellenverzeichnis}
\begin{table}[H]
\begin{tabular}[t]{p{10cm}|p{6cm}}
\hline
\rowcolor{heading} \textbf{Namen der Quelle} & \textbf{Titel und Bemerkung} \\\hline
\textbf{Wikipedia} \newline
\url{https://en.wikipedia.org/wiki/Comparison\_of\_cluster\_software} & Cluster Software Vergleichstabelle  \\\hline
\textbf{HPC Today} \newline
\url{http://www.hpctoday.com/best-practices/tinytitan-a-raspberry-pi-computing-based-cluster/} & Installationsanleitung und Beschreibung der HPC Lösung TinyTitan  \\\hline
\textbf{Jordi Corbilla von Thundax Software} \newline
\url{http://thundaxsoftware.blogspot.ch/2016/07/creating-raspberry-pi-3-cluster.html} & Komplette Installationsanleitung einer Noname Cluster Lösung \\\hline
\textbf{Benutzer Sakaki auf Github} \newline
\url{https://github.com/sakaki-/gentoo-on-rpi3-64bit} & Repository des Gentoo Images und Installationsanleitung \\\hline
\textbf{CentOS} \newline
\url{http://mirror.centos.org/altarch/7.4.1708/isos/aarch64/} & Image Repository von CentOS \\\hline
\textbf{raspberrypi.org} \newline
\url{https://www.raspberrypi.org/documentation/hardware/raspberrypi/bootmodes/net_tutorial.md} & Installationsanleitung zu PXE / Netzwerkboot \\\hline
\textbf{Fedora} \newline
\url{https://fedoraproject.org/wiki/Architectures/ARM} & Fedora Image für Raspberry PI's und Installationsanleitung dazu \\\hline
\textbf{Benutzer Uli Middelberg auf Github} \newline
\url{https://github.com/umiddelb/aarch64/wiki/Install-CentOS-7-on-your-favourite-ARMv8-ARM64-AArch64-board} & Beschreibung und Anleitung der Umgehungslösung für die Installation von CentOS auf den Raspberry PI's\\\hline
\end{tabular}
\caption{Quellenverzeichnis}
\end{table}
