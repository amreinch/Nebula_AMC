% !TEX root = ../Diplombericht.tex
\section{Testprotokoll}
\subsection{Komponententests}
\begin{table}[H]
\centering
\begin{tabular}{p{4.5cm}p{11.5cm}}
\hline
\cellcolor{heading}\textbf{Test-ID:} & \textbf{K-001} \\\hline
\cellcolor{heading}\textbf{Testobjekt} & Management Node \\\hline
\cellcolor{heading}\textbf{Testschritte} & 
- Management Node starten (Strom anschliessen).\newline
- 30 Sekunden warten.\newline
- Auf dem Testclient über Putty oder Shell mit dem Befehl \newline \grqq nmap -sn 192.168.1.10\grqq \ eingeben.\newline
- Prüfen, ob die Zuweisung gemäss Hostnamenkonzept richtig ist. \\\hline
\cellcolor{heading}\textbf{Erwartetes Ergebnis} & Hostname = nebula \newline
IP = 192.168.1.10 \newline
MAC = B8:27:EB:32:A9:1C \\\hline
\cellcolor{heading}\textbf{Tatsächliches Ergebnis} &
Nmap scan report for nebula.home (192.168.1.10)\newline
MAC Address: B8:27:EB:32:A9:1C (Raspberry Pi Foundation)\newline
Nmap done: 1 IP address (1 host up) scanned in 0.26 seconds  \\\hline
\cellcolor{heading}\textbf{Tester} & Christoph Amrein  \\\hline
\cellcolor{heading}\textbf{Datum des Tests} & 21.05.2018  \\\hline
\cellcolor{heading}\textbf{Testergebnis \newline (Fehlerklasse)} & 1 - Fehlerfrei, die Erwartungen sind erfüllt. \\\hline
\cellcolor{heading}\textbf{Fehlerbeschreibung} &   \\\hline
\end{tabular}
\caption{K-001 Protokoll}
\end{table}

\begin{table}[H]
\centering
\begin{tabular}{p{4.5cm}p{11.5cm}}
\hline
\cellcolor{heading}\textbf{Test-ID:} & \textbf{K-002} \\\hline
\cellcolor{heading}\textbf{Testobjekt} & Management Node \\\hline
\cellcolor{heading}\textbf{Testschritte} & 
- Management Node starten (Strom anschliessen).\newline
- 30 Sekunden warten.\newline
- Auf dem Testclient über Putty oder Shell den Befehl\newline \grqq ssh root@nebula\grqq \  eingeben. \newline
- Passwort eingeben. \\ \hline
\cellcolor{heading}\textbf{Erwartetes Ergebnis} & Die SSH Verbindung auf den Management Node hat funktioniert und man ist als root-Benutzer angemeldet.  \\\hline
\cellcolor{heading}\textbf{Tatsächliches Ergebnis} & login as: root \newline
root@nebula's password: \newline
Last login: Fri May 18 17:40:34 2018 from desktop-rrq1k7v.home \\\hline
\cellcolor{heading}\textbf{Tester} & Christoph Amrein  \\\hline
\cellcolor{heading}\textbf{Datum des Tests} & 21.05.2018  \\\hline
\cellcolor{heading}\textbf{Testergebnis \newline (Fehlerklasse)} & 1 - Fehlerfrei, die Erwartungen sind erfüllt. \\\hline
\cellcolor{heading}\textbf{Fehlerbeschreibung} &   \\\hline
\end{tabular}
\caption{K-002 Protokoll}
\end{table}


\begin{table}[H]
\centering
\begin{tabular}{p{4.5cm}p{11.5cm}}
\hline
\cellcolor{heading}\textbf{Test-ID:} & \textbf{K-003} \\\hline
\cellcolor{heading}\textbf{Testobjekt} & Compute Node c1 \\\hline
\cellcolor{heading}\textbf{Testschritte} & 
- Compute Node starten (Strom anschliessen).\newline
- 3 Minuten warten.\newline
- Auf dem Testclient über Putty oder Shell mit dem Befehl \newline \grqq nmap -sn 192.168.1.11\grqq \ eingeben.\newline
- Prüfen, ob die Zuweisung gemäss Hostnamenkonzept richtig ist. \\\hline
\cellcolor{heading}\textbf{Erwartetes Ergebnis} & Hostname = c1 \newline
IP = 192.168.1.11 \newline
MAC =  B8:27:EB:32:39:A7 \\\hline
\cellcolor{heading}\textbf{Tatsächliches Ergebnis} &
Nmap scan report for c1.home (192.168.1.11) \newline
MAC Address: B8:27:EB:32:39:A7 (Raspberry Pi Foundation) \newline
Nmap done: 1 IP address (1 host up) scanned in 0.09 seconds  \\\hline
\cellcolor{heading}\textbf{Tester} & Christoph Amrein  \\\hline
\cellcolor{heading}\textbf{Datum des Tests} & 21.05.2018  \\\hline
\cellcolor{heading}\textbf{Testergebnis \newline (Fehlerklasse)} & 1 - Fehlerfrei, die Erwartungen sind erfüllt. \\\hline
\cellcolor{heading}\textbf{Fehlerbeschreibung} &   \\\hline
\end{tabular}
\caption{K-003 Protokoll}
\end{table}

\begin{table}[H]
\centering
\begin{tabular}{p{4.5cm}p{11.5cm}}
\hline
\cellcolor{heading}\textbf{Test-ID:} & \textbf{K-004} \\\hline
\cellcolor{heading}\textbf{Testobjekt} & Compute Node c2 \\\hline
\cellcolor{heading}\textbf{Testschritte} & 
- Compute Node starten (Strom anschliessen).\newline
- 3 Minuten warten.\newline
- Auf dem Testclient über Putty oder Shell mit dem Befehl \newline \grqq nmap -sn 192.168.1.12\grqq \ eingeben.\newline
- Prüfen, ob die Zuweisung gemäss Hostnamenkonzept richtig ist. \\\hline
\cellcolor{heading}\textbf{Erwartetes Ergebnis} & Hostname = c2 \newline
IP = 192.168.1.12 \newline
MAC = B8:27:EB:2E:A3:D1 \\\hline
\cellcolor{heading}\textbf{Tatsächliches Ergebnis} &
Nmap scan report for c2.home (192.168.1.12) \newline
MAC Address: B8:27:EB:2E:A3:D1 (Raspberry Pi Foundation)\newline
Nmap done: 1 IP address (1 host up) scanned in 0.09 seconds \\\hline
\cellcolor{heading}\textbf{Tester} & Christoph Amrein  \\\hline
\cellcolor{heading}\textbf{Datum des Tests} & 22.05.2018  \\\hline
\cellcolor{heading}\textbf{Testergebnis \newline (Fehlerklasse)} & 1 - Fehlerfrei, die Erwartungen sind erfüllt. \\\hline
\cellcolor{heading}\textbf{Fehlerbeschreibung} &   \\\hline
\end{tabular}
\caption{K-004 Protokoll}
\end{table}


\begin{table}[H]
\centering
\begin{tabular}{p{4.5cm}p{11.5cm}}
\hline
\cellcolor{heading}\textbf{Test-ID:} & \textbf{K-005} \\\hline
\cellcolor{heading}\textbf{Testobjekt} & Compute Node c3 \\\hline
\cellcolor{heading}\textbf{Testschritte} & 
- Compute Node starten (Strom anschliessen).\newline
- 3 Minuten warten.\newline
- Auf dem Testclient über Putty oder Shell mit dem Befehl \newline \grqq nmap -sn 192.168.1.13\grqq \ eingeben.\newline
- Prüfen, ob die Zuweisung gemäss Hostnamenkonzept richtig ist. \\\hline
\cellcolor{heading}\textbf{Erwartetes Ergebnis} & Hostname = c3 \newline
IP = 192.168.1.13 \newline
MAC = B8:27:EB:50:45:3F \\\hline
\cellcolor{heading}\textbf{Tatsächliches Ergebnis} &
Nmap scan report for c3.home (192.168.1.13) \newline
MAC Address: B8:27:EB:50:45:3F (Raspberry Pi Foundation) \newline
Nmap done: 1 IP address (1 host up) scanned in 0.09 seconds  \\\hline
\cellcolor{heading}\textbf{Tester} & Christoph Amrein  \\\hline
\cellcolor{heading}\textbf{Datum des Tests} & 22.05.2018  \\\hline
\cellcolor{heading}\textbf{Testergebnis \newline (Fehlerklasse)} & 1 - Fehlerfrei, die Erwartungen sind erfüllt. \\\hline
\cellcolor{heading}\textbf{Fehlerbeschreibung} &   \\\hline
\end{tabular}
\caption{K-005 Protokoll}
\end{table}

\begin{table}[H]
\centering
\begin{tabular}{p{4.5cm}p{11.5cm}}
\hline
\cellcolor{heading}\textbf{Test-ID:} & \textbf{K-006} \\\hline
\cellcolor{heading}\textbf{Testobjekt} & Compute Node c4 \\\hline
\cellcolor{heading}\textbf{Testschritte} & 
- Compute Node starten (Strom anschliessen).\newline
- 3 Minuten warten.\newline
- Auf dem Testclient über Putty oder Shell mit dem Befehl \newline \grqq nmap -sn 192.168.1.14\grqq \ eingeben.\newline
- Prüfen, ob die Zuweisung gemäss Hostnamenkonzept richtig ist. \\\hline
\cellcolor{heading}\textbf{Erwartetes Ergebnis} & Hostname = c4 \newline
IP = 192.168.1.14 \newline
MAC = B8:27:EB:0D:E6:25 \\\hline
\cellcolor{heading}\textbf{Tatsächliches Ergebnis} &
Nmap scan report for c4.home (192.168.1.14) \newline
MAC Address: B8:27:EB:0D:E6:25 (Raspberry Pi Foundation) \newline
Nmap done: 1 IP address (1 host up) scanned in 0.10 seconds  \\\hline
\cellcolor{heading}\textbf{Tester} & Christoph Amrein  \\\hline
\cellcolor{heading}\textbf{Datum des Tests} & 22.05.2018  \\\hline
\cellcolor{heading}\textbf{Testergebnis \newline (Fehlerklasse)} & 1 - Fehlerfrei, die Erwartungen sind erfüllt. \\\hline
\cellcolor{heading}\textbf{Fehlerbeschreibung} &   \\\hline
\end{tabular}
\caption{K-006 Protokoll}
\end{table}

\begin{table}[H]
\centering
\begin{tabular}{p{4.5cm}p{11.5cm}}
\hline
\cellcolor{heading}\textbf{Test-ID:} & \textbf{K-007} \\\hline
\cellcolor{heading}\textbf{Testobjekt} & Compute Node c5 \\\hline
\cellcolor{heading}\textbf{Testschritte} & 
- Compute Node starten (Strom anschliessen).\newline
- 3 Minuten warten.\newline
- Auf dem Testclient über Putty oder Shell mit dem Befehl \newline \grqq nmap -sn 192.168.1.15\grqq \ eingeben.\newline
- Prüfen, ob die Zuweisung gemäss Hostnamenkonzept richtig ist. \\\hline
\cellcolor{heading}\textbf{Erwartetes Ergebnis} & Hostname = c5 \newline
IP = 192.168.1.15 \newline
MAC = B8:27:EB:3E:96:B5 \\\hline
\cellcolor{heading}\textbf{Tatsächliches Ergebnis} &
Nmap scan report for c5.home (192.168.1.15) \newline
MAC Address: B8:27:EB:3E:96:B5 (Raspberry Pi Foundation) \newline
Nmap done: 1 IP address (1 host up) scanned in 0.09 seconds  \\\hline
\cellcolor{heading}\textbf{Tester} & Christoph Amrein  \\\hline
\cellcolor{heading}\textbf{Datum des Tests} & 22.05.2018  \\\hline
\cellcolor{heading}\textbf{Testergebnis \newline (Fehlerklasse)} & 1 - Fehlerfrei, die Erwartungen sind erfüllt. \\\hline
\cellcolor{heading}\textbf{Fehlerbeschreibung} &   \\\hline
\end{tabular}
\caption{K-007 Protokoll}
\end{table}


\begin{table}[H]
\centering
\begin{tabular}{p{4.5cm}p{11.5cm}}
\hline
\cellcolor{heading}\textbf{Test-ID:} & \textbf{K-008} \\\hline
\cellcolor{heading}\textbf{Testobjekt} & Compute Node c6 \\\hline
\cellcolor{heading}\textbf{Testschritte} & 
- Compute Node starten (Strom anschliessen).\newline
- 3 Minuten warten.\newline
- Auf dem Testclient über Putty oder Shell mit dem Befehl \newline \grqq nmap -sn 192.168.1.16\grqq \ eingeben.\newline
- Prüfen, ob die Zuweisung gemäss Hostnamenkonzept richtig ist. \\\hline
\cellcolor{heading}\textbf{Erwartetes Ergebnis} & Hostname = c6 \newline
IP = 192.168.1.16 \newline
MAC = B8:27:EB:EE:77:DA \\\hline
\cellcolor{heading}\textbf{Tatsächliches Ergebnis} &
Nmap scan report for c6.home (192.168.1.16) \newline
MAC Address: B8:27:EB:EE:77:DA (Raspberry Pi Foundation) \newline
Nmap done: 1 IP address (1 host up) scanned in 0.09 seconds  \\\hline
\cellcolor{heading}\textbf{Tester} & Christoph Amrein  \\\hline
\cellcolor{heading}\textbf{Datum des Tests} & 22.05.2018  \\\hline
\cellcolor{heading}\textbf{Testergebnis \newline (Fehlerklasse)} & 1 - Fehlerfrei, die Erwartungen sind erfüllt. \\\hline
\cellcolor{heading}\textbf{Fehlerbeschreibung} &   \\\hline
\end{tabular}
\caption{K-008 Protokoll}
\end{table}

\begin{table}[H]
\centering
\begin{tabular}{p{4.5cm}p{11.5cm}}
\hline
\cellcolor{heading}\textbf{Test-ID:} & \textbf{K-009} \\\hline
\cellcolor{heading}\textbf{Testobjekt} & Compute Node c7 \\\hline
\cellcolor{heading}\textbf{Testschritte} & 
- Compute Node starten (Strom anschliessen).\newline
- 3 Minuten warten.\newline
- Auf dem Testclient über Putty oder Shell mit dem Befehl \newline \grqq nmap -sn 192.168.1.17\grqq \ eingeben.\newline
- Prüfen, ob die Zuweisung gemäss Hostnamenkonzept richtig ist. \\\hline
\cellcolor{heading}\textbf{Erwartetes Ergebnis} & Hostname = c7 \newline
IP = 192.168.1.17 \newline
MAC = B8:27:EB:21:63:E6 \\\hline
\cellcolor{heading}\textbf{Tatsächliches Ergebnis} &
Nmap scan report for c7.home (192.168.1.17) \newline
MAC Address: B8:27:EB:21:63:E6 (Raspberry Pi Foundation) \newline
Nmap done: 1 IP address (1 host up) scanned in 0.09 seconds  \\\hline
\cellcolor{heading}\textbf{Tester} & Christoph Amrein  \\\hline
\cellcolor{heading}\textbf{Datum des Tests} & 22.05.2018  \\\hline
\cellcolor{heading}\textbf{Testergebnis \newline (Fehlerklasse)} & 1 - Fehlerfrei, die Erwartungen sind erfüllt. \\\hline
\cellcolor{heading}\textbf{Fehlerbeschreibung} &   \\\hline
\end{tabular}
\caption{K-009 Protokoll}
\end{table}


\begin{table}[H]
\centering
\begin{tabular}{p{4.5cm}p{11.5cm}}
\hline
\cellcolor{heading}\textbf{Test-ID:} & \textbf{K-010} \\\hline
\cellcolor{heading}\textbf{Testobjekt} & Compute Node c8 \\\hline
\cellcolor{heading}\textbf{Testschritte} & 
- Compute Node starten (Strom anschliessen).\newline
- 3 Minuten warten.\newline
- Auf dem Testclient über Putty oder Shell mit dem Befehl \newline \grqq nmap -sn 192.168.1.18\grqq \ eingeben.\newline
- Prüfen, ob die Zuweisung gemäss Hostnamenkonzept richtig ist. \\\hline
\cellcolor{heading}\textbf{Erwartetes Ergebnis} & Hostname = c8 \newline
IP = 192.168.1.18 \newline
MAC = B8:27:EB:2E:2E:CC \\\hline
\cellcolor{heading}\textbf{Tatsächliches Ergebnis} &
Nmap scan report for c8.home (192.168.1.18) \newline
MAC Address: B8:27:EB:2E:2E:CC (Raspberry Pi Foundation) \newline
Nmap done: 1 IP address (1 host up) scanned in 0.09 seconds  \\\hline
\cellcolor{heading}\textbf{Tester} & Christoph Amrein  \\\hline
\cellcolor{heading}\textbf{Datum des Tests} & 22.05.2018  \\\hline
\cellcolor{heading}\textbf{Testergebnis \newline (Fehlerklasse)} & 1 - Fehlerfrei, die Erwartungen sind erfüllt. \\\hline
\cellcolor{heading}\textbf{Fehlerbeschreibung} &   \\\hline
\end{tabular}
\caption{K-010 Protokoll}
\end{table}

\begin{table}[H]
\centering
\begin{tabular}{p{4.5cm}p{11.5cm}}
\hline
\cellcolor{heading}\textbf{Test-ID:} & \textbf{K-011} \\\hline
\cellcolor{heading}\textbf{Testobjekt} & Compute Node c9 \\\hline
\cellcolor{heading}\textbf{Testschritte} & 
- Compute Node starten (Strom anschliessen).\newline
- 3 Minuten warten.\newline
- Auf dem Testclient über Putty oder Shell mit dem Befehl \newline \grqq nmap -sn 192.168.1.19\grqq \ eingeben.\newline
- Prüfen, ob die Zuweisung gemäss Hostnamenkonzept richtig ist. \\\hline
\cellcolor{heading}\textbf{Erwartetes Ergebnis} & Hostname = c9 \newline
IP = 192.168.1.19 \newline
MAC = B8:27:EB:17:32:96 \\\hline
\cellcolor{heading}\textbf{Tatsächliches Ergebnis} &
Nmap scan report for galaxy-a5-2017.home (192.168.1.19) \newline
MAC Address: B8:27:EB:17:32:96 (Raspberry Pi Foundation) \newline
Nmap done: 1 IP address (1 host up) scanned in 0.09 seconds  \\\hline
\cellcolor{heading}\textbf{Tester} & Christoph Amrein  \\\hline
\cellcolor{heading}\textbf{Datum des Tests} & 22.05.2018  \\\hline
\cellcolor{heading}\textbf{Testergebnis \newline (Fehlerklasse)} & 1 - Fehlerfrei, die Erwartungen sind erfüllt. \\\hline
\cellcolor{heading}\textbf{Fehlerbeschreibung} &   \\\hline
\end{tabular}
\caption{K-011 Protokoll}
\end{table}

\begin{table}[H]
\centering
\begin{tabular}{p{4.5cm}p{11.5cm}}
\hline
\cellcolor{heading}\textbf{Test-ID:} & \textbf{K-012} \\\hline
\cellcolor{heading}\textbf{Testobjekt} & Compute Node c10 \\\hline
\cellcolor{heading}\textbf{Testschritte} & 
- Compute Node starten (Strom anschliessen).\newline
- 3 Minuten warten.\newline
- Auf dem Testclient über Putty oder Shell mit dem Befehl \newline \grqq nmap -sn 192.168.1.20\grqq \ eingeben.\newline
- Prüfen, ob die Zuweisung gemäss Hostnamenkonzept richtig ist. \\\hline
\cellcolor{heading}\textbf{Erwartetes Ergebnis} & Hostname = c10 \newline
IP = 192.168.1.20 \newline
MAC = B8:27:EB:B2:1C:A9 \\\hline
\cellcolor{heading}\textbf{Tatsächliches Ergebnis} &
Nmap scan report for c10.home (192.168.1.20) \newline
MAC Address: B8:27:EB:B2:1C:A9 (Raspberry Pi Foundation) \newline
Nmap done: 1 IP address (1 host up) scanned in 0.09 seconds  \\\hline
\cellcolor{heading}\textbf{Tester} & Christoph Amrein  \\\hline
\cellcolor{heading}\textbf{Datum des Tests} & 22.05.2018  \\\hline
\cellcolor{heading}\textbf{Testergebnis \newline (Fehlerklasse)} & 1 - Fehlerfrei, die Erwartungen sind erfüllt. \\\hline
\cellcolor{heading}\textbf{Fehlerbeschreibung} &   \\\hline
\end{tabular}
\caption{K-012 Protokoll}
\end{table}

\begin{table}[H]
\centering
\begin{tabular}{p{4.5cm}p{11.5cm}}
\hline
\cellcolor{heading}\textbf{Test-ID:} & \textbf{K-013} \\\hline
\cellcolor{heading}\textbf{Testobjekt} & Compute Node c11 \\\hline
\cellcolor{heading}\textbf{Testschritte} & 
- Compute Node starten (Strom anschliessen).\newline
- 3 Minuten warten.\newline
- Auf dem Testclient über Putty oder Shell mit dem Befehl \newline \grqq nmap -sn 192.168.1.21\grqq \ eingeben.\newline
- Prüfen, ob die Zuweisung gemäss Hostnamenkonzept richtig ist. \\\hline
\cellcolor{heading}\textbf{Erwartetes Ergebnis} & Hostname = c11 \newline
IP = 192.168.1.21 \newline
MAC = B8:27:EB:AF:63:1F \\\hline
\cellcolor{heading}\textbf{Tatsächliches Ergebnis} &
Nmap scan report for c11.home (192.168.1.21) \newline
MAC Address: B8:27:EB:AF:63:1F (Raspberry Pi Foundation) \newline
Nmap done: 1 IP address (1 host up) scanned in 0.09 seconds  \\\hline
\cellcolor{heading}\textbf{Tester} & Christoph Amrein  \\\hline
\cellcolor{heading}\textbf{Datum des Tests} & 22.05.2018  \\\hline
\cellcolor{heading}\textbf{Testergebnis \newline (Fehlerklasse)} & 1 - Fehlerfrei, die Erwartungen sind erfüllt. \\\hline
\cellcolor{heading}\textbf{Fehlerbeschreibung} &   \\\hline
\end{tabular}
\caption{K-013 Protokoll}
\end{table}

\begin{table}[H]
\centering
\begin{tabular}{p{4.5cm}p{11.5cm}}
\hline
\cellcolor{heading}\textbf{Test-ID:} & \textbf{K-014} \\\hline
\cellcolor{heading}\textbf{Testobjekt} & Compute Node c12 \\\hline
\cellcolor{heading}\textbf{Testschritte} & 
- Compute Node starten (Strom anschliessen).\newline
- 3 Minuten warten.\newline
- Auf dem Testclient über Putty oder Shell mit dem Befehl \newline \grqq nmap -sn 192.168.1.22\grqq \ eingeben.\newline
- Prüfen, ob die Zuweisung gemäss Hostnamenkonzept richtig ist. \\\hline
\cellcolor{heading}\textbf{Erwartetes Ergebnis} & Hostname = c12 \newline
IP = 192.168.1.22 \newline
MAC = B8:27:EB:43:00:2C \\\hline
\cellcolor{heading}\textbf{Tatsächliches Ergebnis} &
Nmap scan report for c12.home (192.168.1.22) \newline
MAC Address: B8:27:EB:43:00:2C (Raspberry Pi Foundation) \newline
Nmap done: 1 IP address (1 host up) scanned in 0.10 seconds  \\\hline
\cellcolor{heading}\textbf{Tester} & Christoph Amrein  \\\hline
\cellcolor{heading}\textbf{Datum des Tests} & 22.05.2018  \\\hline
\cellcolor{heading}\textbf{Testergebnis \newline (Fehlerklasse)} & 1 - Fehlerfrei, die Erwartungen sind erfüllt. \\\hline
\cellcolor{heading}\textbf{Fehlerbeschreibung} &   \\\hline
\end{tabular}
\caption{K-014 Protokoll}
\end{table}

\begin{table}[H]
\centering
\begin{tabular}{p{4.5cm}p{11.5cm}}
\hline
\cellcolor{heading}\textbf{Test-ID:} & \textbf{K-015} \\\hline
\cellcolor{heading}\textbf{Testobjekt} & Compute Node c13 \\\hline
\cellcolor{heading}\textbf{Testschritte} & 
- Compute Node starten (Strom anschliessen).\newline
- 3 Minuten warten.\newline
- Auf dem Testclient über Putty oder Shell mit dem Befehl \newline \grqq nmap -sn 192.168.1.23\grqq \ eingeben.\newline
- Prüfen, ob die Zuweisung gemäss Hostnamenkonzept richtig ist. \\\hline
\cellcolor{heading}\textbf{Erwartetes Ergebnis} & Hostname = c13 \newline
IP = 192.168.1.23 \newline
MAC = B8:27:EB:13:7B:18 \\\hline
\cellcolor{heading}\textbf{Tatsächliches Ergebnis} &
Nmap scan report for c13.home (192.168.1.23) \newline
MAC Address: B8:27:EB:13:7B:18 (Raspberry Pi Foundation) \newline
Nmap done: 1 IP address (1 host up) scanned in 0.09 seconds  \\\hline
\cellcolor{heading}\textbf{Tester} & Christoph Amrein  \\\hline
\cellcolor{heading}\textbf{Datum des Tests} & 22.05.2018  \\\hline
\cellcolor{heading}\textbf{Testergebnis \newline (Fehlerklasse)} & 1 - Fehlerfrei, die Erwartungen sind erfüllt. \\\hline
\cellcolor{heading}\textbf{Fehlerbeschreibung} &   \\\hline
\end{tabular}
\caption{K-015 Protokoll}
\end{table}

\begin{table}[H]
\centering
\begin{tabular}{p{4.5cm}p{11.5cm}}
\hline
\cellcolor{heading}\textbf{Test-ID:} & \textbf{K-016} \\\hline
\cellcolor{heading}\textbf{Testobjekt} & Compute Node c14 \\\hline
\cellcolor{heading}\textbf{Testschritte} & 
- Compute Node starten (Strom anschliessen).\newline
- 3 Minuten warten.\newline
- Auf dem Testclient über Putty oder Shell mit dem Befehl \newline \grqq nmap -sn 192.168.1.24\grqq \ eingeben.\newline
- Prüfen, ob die Zuweisung gemäss Hostnamenkonzept richtig ist. \\\hline
\cellcolor{heading}\textbf{Erwartetes Ergebnis} & Hostname = c14 \newline
IP = 192.168.1.24 \newline
MAC = B8:27:EB:43:CD:29 \\\hline
\cellcolor{heading}\textbf{Tatsächliches Ergebnis} &
Nmap scan report for c14.home (192.168.1.24) \newline
MAC Address: B8:27:EB:43:CD:29 (Raspberry Pi Foundation) \newline
Nmap done: 1 IP address (1 host up) scanned in 0.09 seconds  \\\hline
\cellcolor{heading}\textbf{Tester} & Christoph Amrein  \\\hline
\cellcolor{heading}\textbf{Datum des Tests} & 22.05.2018  \\\hline
\cellcolor{heading}\textbf{Testergebnis \newline (Fehlerklasse)} & 1 - Fehlerfrei, die Erwartungen sind erfüllt. \\\hline
\cellcolor{heading}\textbf{Fehlerbeschreibung} &   \\\hline
\end{tabular}
\caption{K-016 Protokoll}
\end{table}

\begin{table}[H]
\centering
\begin{tabular}{p{4.5cm}p{11.5cm}}
\hline
\cellcolor{heading}\textbf{Test-ID:} & \textbf{K-017} \\\hline
\cellcolor{heading}\textbf{Testobjekt} & Compute Node c15 \\\hline
\cellcolor{heading}\textbf{Testschritte} & 
- Compute Node starten (Strom anschliessen).\newline
- 3 Minuten warten.\newline
- Auf dem Testclient über Putty oder Shell mit dem Befehl \newline \grqq nmap -sn 192.168.1.25\grqq \ eingeben.\newline
- Prüfen, ob die Zuweisung gemäss Hostnamenkonzept richtig ist. \\\hline
\cellcolor{heading}\textbf{Erwartetes Ergebnis} & Hostname = c15 \newline
IP = 192.168.1.25 \newline
MAC = B8:27:EB:FF:C7:56 \\\hline
\cellcolor{heading}\textbf{Tatsächliches Ergebnis} &
Nmap scan report for c15.home (192.168.1.25) \newline
MAC Address: B8:27:EB:FF:C7:56 (Raspberry Pi Foundation) \newline
Nmap done: 1 IP address (1 host up) scanned in 0.09 seconds  \\\hline
\cellcolor{heading}\textbf{Tester} & Christoph Amrein  \\\hline
\cellcolor{heading}\textbf{Datum des Tests} & 22.05.2018  \\\hline
\cellcolor{heading}\textbf{Testergebnis \newline (Fehlerklasse)} & 1 - Fehlerfrei, die Erwartungen sind erfüllt. \\\hline
\cellcolor{heading}\textbf{Fehlerbeschreibung} &   \\\hline
\end{tabular}
\caption{K-017 Protokoll}
\end{table}

\begin{table}[H]
\centering
\begin{tabular}{p{4.5cm}p{11.5cm}}
\hline
\cellcolor{heading}\textbf{Test-ID:} & \textbf{K-018} \\\hline
\cellcolor{heading}\textbf{Testobjekt} & Compute Node c16 \\\hline
\cellcolor{heading}\textbf{Testschritte} & 
- Compute Node starten (Strom anschliessen).\newline
- 3 Minuten warten.\newline
- Auf dem Testclient über Putty oder Shell mit dem Befehl \newline \grqq nmap -sn 192.168.1.26\grqq \ eingeben.\newline
- Prüfen, ob die Zuweisung gemäss Hostnamenkonzept richtig ist. \\\hline
\cellcolor{heading}\textbf{Erwartetes Ergebnis} & Hostname = c16 \newline
IP = 192.168.1.26 \newline
MAC = B8:27:EB:CE:98:66 \\\hline
\cellcolor{heading}\textbf{Tatsächliches Ergebnis} &
Nmap scan report for c16.home (192.168.1.26) \newline
MAC Address: B8:27:EB:CE:98:66 (Raspberry Pi Foundation) \newline
Nmap done: 1 IP address (1 host up) scanned in 0.10 seconds \\\hline
\cellcolor{heading}\textbf{Tester} & Christoph Amrein  \\\hline
\cellcolor{heading}\textbf{Datum des Tests} & 22.05.2018  \\\hline
\cellcolor{heading}\textbf{Testergebnis \newline (Fehlerklasse)} & 1 - Fehlerfrei, die Erwartungen sind erfüllt. \\\hline
\cellcolor{heading}\textbf{Fehlerbeschreibung} &   \\\hline
\end{tabular}
\caption{K-018 Protokoll}
\end{table}

\begin{table}[H]
\centering
\begin{tabular}{p{4.5cm}p{11.5cm}}
\hline
\cellcolor{heading}\textbf{Test-ID:} & \textbf{K-019} \\\hline
\cellcolor{heading}\textbf{Testobjekt} & Compute Node c17 \\\hline
\cellcolor{heading}\textbf{Testschritte} & 
- Compute Node starten (Strom anschliessen).\newline
- 3 Minuten warten.\newline
- Auf dem Testclient über Putty oder Shell mit dem Befehl \newline \grqq nmap -sn 192.168.1.27\grqq \ eingeben.\newline
- Prüfen, ob die Zuweisung gemäss Hostnamenkonzept richtig ist. \\\hline
\cellcolor{heading}\textbf{Erwartetes Ergebnis} & Hostname = c17 \newline
IP = 192.168.1.27 \newline
MAC = B8:27:EB:5D:63:34 \\\hline
\cellcolor{heading}\textbf{Tatsächliches Ergebnis} &
Nmap scan report for c17.home (192.168.1.27) \newline
MAC Address: B8:27:EB:5D:63:34 (Raspberry Pi Foundation) \newline
Nmap done: 1 IP address (1 host up) scanned in 0.09 seconds  \\\hline
\cellcolor{heading}\textbf{Tester} & Christoph Amrein  \\\hline
\cellcolor{heading}\textbf{Datum des Tests} & 22.05.2018  \\\hline
\cellcolor{heading}\textbf{Testergebnis \newline (Fehlerklasse)} & 1 - Fehlerfrei, die Erwartungen sind erfüllt. \\\hline
\cellcolor{heading}\textbf{Fehlerbeschreibung} &   \\\hline
\end{tabular}
\caption{K-019 Protokoll}
\end{table}

\begin{table}[H]
\centering
\begin{tabular}{p{4.5cm}p{11.5cm}}
\hline
\cellcolor{heading}\textbf{Test-ID:} & \textbf{K-020} \\\hline
\cellcolor{heading}\textbf{Testobjekt} & Compute Node c18 \\\hline
\cellcolor{heading}\textbf{Testschritte} & 
- Compute Node starten (Strom anschliessen).\newline
- 3 Minuten warten.\newline
- Auf dem Testclient über Putty oder Shell mit dem Befehl \newline \grqq nmap -sn 192.168.1.28\grqq \ eingeben.\newline
- Prüfen, ob die Zuweisung gemäss Hostnamenkonzept richtig ist. \\\hline
\cellcolor{heading}\textbf{Erwartetes Ergebnis} & Hostname = c18 \newline
IP = 192.168.1.28 \newline
MAC = B8:27:EB:91:3E:0F \\\hline
\cellcolor{heading}\textbf{Tatsächliches Ergebnis} &
Nmap scan report for c18.home (192.168.1.28) \newline
MAC Address: B8:27:EB:91:3E:0F (Raspberry Pi Foundation) \newline
Nmap done: 1 IP address (1 host up) scanned in 0.08 seconds \\\hline
\cellcolor{heading}\textbf{Tester} & Christoph Amrein  \\\hline
\cellcolor{heading}\textbf{Datum des Tests} & 22.05.2018  \\\hline
\cellcolor{heading}\textbf{Testergebnis \newline (Fehlerklasse)} & 1 - Fehlerfrei, die Erwartungen sind erfüllt. \\\hline
\cellcolor{heading}\textbf{Fehlerbeschreibung} &   \\\hline
\end{tabular}
\caption{K-020 Protokoll}
\end{table}

\begin{table}[H]
\centering
\begin{tabular}{p{4.5cm}p{11.5cm}}
\hline
\cellcolor{heading}\textbf{Test-ID:} & \textbf{K-021} \\\hline
\cellcolor{heading}\textbf{Testobjekt} & Compute Node c19 \\\hline
\cellcolor{heading}\textbf{Testschritte} & 
- Compute Node starten (Strom anschliessen).\newline
- 3 Minuten warten.\newline
- Auf dem Testclient über Putty oder Shell mit dem Befehl \newline \grqq nmap -sn 192.168.1.29\grqq \ eingeben.\newline
- Prüfen, ob die Zuweisung gemäss Hostnamenkonzept richtig ist. \\\hline
\cellcolor{heading}\textbf{Erwartetes Ergebnis} & Hostname = c19 \newline
IP = 192.168.1.29 \newline
MAC = B8:27:EB:F4:65:EC \\\hline
\cellcolor{heading}\textbf{Tatsächliches Ergebnis} &
Nmap scan report for c19.home (192.168.1.29) \newline
MAC Address: B8:27:EB:F4:65:EC (Raspberry Pi Foundation) \newline
Nmap done: 1 IP address (1 host up) scanned in 0.09 seconds  \\\hline
\cellcolor{heading}\textbf{Tester} & Christoph Amrein  \\\hline
\cellcolor{heading}\textbf{Datum des Tests} & 22.05.2018  \\\hline
\cellcolor{heading}\textbf{Testergebnis \newline (Fehlerklasse)} & 1 - Fehlerfrei, die Erwartungen sind erfüllt. \\\hline
\cellcolor{heading}\textbf{Fehlerbeschreibung} &   \\\hline
\end{tabular}
\caption{K-021 Protokoll}
\end{table}

\begin{table}[H]
\centering
\begin{tabular}{p{4.5cm}p{11.5cm}}
\hline
\cellcolor{heading}\textbf{Test-ID:} & \textbf{K-022} \\\hline
\cellcolor{heading}\textbf{Testobjekt} & Compute Node c20 \\\hline
\cellcolor{heading}\textbf{Testschritte} & 
- Compute Node starten (Strom anschliessen).\newline
- 3 Minuten warten.\newline
- Auf dem Testclient über Putty oder Shell mit dem Befehl \newline \grqq nmap -sn 192.168.1.30\grqq \ eingeben.\newline
- Prüfen, ob die Zuweisung gemäss Hostnamenkonzept richtig ist. \\\hline
\cellcolor{heading}\textbf{Erwartetes Ergebnis} & Hostname = c20 \newline
IP = 192.168.1.30 \newline
MAC = B8:27:EB:3E:AB:DC \\\hline
\cellcolor{heading}\textbf{Tatsächliches Ergebnis} &
Nmap scan report for c20.home (192.168.1.30) \newline
MAC Address: B8:27:EB:3E:AB:DC (Raspberry Pi Foundation) \newline
Nmap done: 1 IP address (1 host up) scanned in 0.10 seconds  \\\hline
\cellcolor{heading}\textbf{Tester} & Christoph Amrein  \\\hline
\cellcolor{heading}\textbf{Datum des Tests} & 22.05.2018  \\\hline
\cellcolor{heading}\textbf{Testergebnis \newline (Fehlerklasse)} & 1 - Fehlerfrei, die Erwartungen sind erfüllt. \\\hline
\cellcolor{heading}\textbf{Fehlerbeschreibung} &   \\\hline
\end{tabular}
\caption{K-022 Protokoll}
\end{table}

\begin{table}[H]
\centering
\begin{tabular}{p{4.5cm}p{11.5cm}}
\hline
\cellcolor{heading}\textbf{Test-ID:} & \textbf{K-023} \\\hline
\cellcolor{heading}\textbf{Testobjekt} & Compute Node c21 \\\hline
\cellcolor{heading}\textbf{Testschritte} & 
- Compute Node starten (Strom anschliessen).\newline
- 3 Minuten warten.\newline
- Auf dem Testclient über Putty oder Shell mit dem Befehl \newline \grqq nmap -sn 192.168.1.31\grqq \ eingeben.\newline
- Prüfen, ob die Zuweisung gemäss Hostnamenkonzept richtig ist. \\\hline
\cellcolor{heading}\textbf{Erwartetes Ergebnis} & Hostname = c21 \newline
IP = 192.168.1.31 \newline
MAC = B8:27:EB:66:60:F6 \\\hline
\cellcolor{heading}\textbf{Tatsächliches Ergebnis} &
Nmap scan report for c21.home (192.168.1.31) \newline
MAC Address: B8:27:EB:66:60:F6 (Raspberry Pi Foundation) \newline
Nmap done: 1 IP address (1 host up) scanned in 0.09 seconds  \\\hline
\cellcolor{heading}\textbf{Tester} & Christoph Amrein  \\\hline
\cellcolor{heading}\textbf{Datum des Tests} & 22.05.2018  \\\hline
\cellcolor{heading}\textbf{Testergebnis \newline (Fehlerklasse)} & 1 - Fehlerfrei, die Erwartungen sind erfüllt. \\\hline
\cellcolor{heading}\textbf{Fehlerbeschreibung} &   \\\hline
\end{tabular}
\caption{K-023 Protokoll}
\end{table}

\begin{table}[H]
\centering
\begin{tabular}{p{4.5cm}p{11.5cm}}
\hline
\cellcolor{heading}\textbf{Test-ID:} & \textbf{K-024} \\\hline
\cellcolor{heading}\textbf{Testobjekt} & Compute Node c22 \\\hline
\cellcolor{heading}\textbf{Testschritte} & 
- Compute Node starten (Strom anschliessen).\newline
- 3 Minuten warten.\newline
- Auf dem Testclient über Putty oder Shell mit dem Befehl \newline \grqq nmap -sn 192.168.1.32\grqq \ eingeben.\newline
- Prüfen, ob die Zuweisung gemäss Hostnamenkonzept richtig ist. \\\hline
\cellcolor{heading}\textbf{Erwartetes Ergebnis} & Hostname = c22 \newline
IP = 192.168.1.32 \newline
MAC = B8:27:EB:37:3F:74 \\\hline
\cellcolor{heading}\textbf{Tatsächliches Ergebnis} &
Nmap scan report for c22.home (192.168.1.32) \newline
MAC Address: B8:27:EB:37:3F:74 (Raspberry Pi Foundation) \newline
Nmap done: 1 IP address (1 host up) scanned in 0.09 seconds  \\\hline
\cellcolor{heading}\textbf{Tester} & Christoph Amrein  \\\hline
\cellcolor{heading}\textbf{Datum des Tests} & 22.05.2018  \\\hline
\cellcolor{heading}\textbf{Testergebnis \newline (Fehlerklasse)} & 1 - Fehlerfrei, die Erwartungen sind erfüllt. \\\hline
\cellcolor{heading}\textbf{Fehlerbeschreibung} &   \\\hline
\end{tabular}
\caption{K-024 Protokoll}
\end{table}

\begin{table}[H]
\centering
\begin{tabular}{p{4.5cm}p{11.5cm}}
\hline
\cellcolor{heading}\textbf{Test-ID:} & \textbf{K-025} \\\hline
\cellcolor{heading}\textbf{Testobjekt} & Compute Node c23 \\\hline
\cellcolor{heading}\textbf{Testschritte} & 
- Compute Node starten (Strom anschliessen).\newline
- 3 Minuten warten.\newline
- Auf dem Testclient über Putty oder Shell mit dem Befehl \newline \grqq nmap -sn 192.168.1.33\grqq \ eingeben.\newline
- Prüfen, ob die Zuweisung gemäss Hostnamenkonzept richtig ist. \\\hline
\cellcolor{heading}\textbf{Erwartetes Ergebnis} & Hostname = c23 \newline
IP = 192.168.1.33 \newline
MAC = B8:27:EB:18:5E:F0 \\\hline
\cellcolor{heading}\textbf{Tatsächliches Ergebnis} &
Nmap scan report for c23.home (192.168.1.33) \newline
MAC Address: B8:27:EB:18:5E:F0 (Raspberry Pi Foundation) \newline
Nmap done: 1 IP address (1 host up) scanned in 0.09 seconds  \\\hline
\cellcolor{heading}\textbf{Tester} & Christoph Amrein  \\\hline
\cellcolor{heading}\textbf{Datum des Tests} & 22.05.2018  \\\hline
\cellcolor{heading}\textbf{Testergebnis \newline (Fehlerklasse)} & 1 - Fehlerfrei, die Erwartungen sind erfüllt. \\\hline
\cellcolor{heading}\textbf{Fehlerbeschreibung} &   \\\hline
\end{tabular}
\caption{K-025 Protokoll}
\end{table}

\begin{table}[H]
\centering
\begin{tabular}{p{4.5cm}p{11.5cm}}
\hline
\cellcolor{heading}\textbf{Test-ID:} & \textbf{K-026} \\\hline
\cellcolor{heading}\textbf{Testobjekt} & Compute Node c24 \\\hline
\cellcolor{heading}\textbf{Testschritte} & 
- Compute Node starten (Strom anschliessen).\newline
- 3 Minuten warten.\newline
- Auf dem Testclient über Putty oder Shell mit dem Befehl \newline \grqq nmap -sn 192.168.1.34\grqq \ eingeben.\newline
- Prüfen, ob die Zuweisung gemäss Hostnamenkonzept richtig ist. \\\hline
\cellcolor{heading}\textbf{Erwartetes Ergebnis} & Hostname = c24 \newline
IP = 192.168.1.34 \newline
MAC = B8:27:EB:B0:23:B8 \\\hline
\cellcolor{heading}\textbf{Tatsächliches Ergebnis} &
Nmap scan report for c24.home (192.168.1.34) \newline
MAC Address: B8:27:EB:B0:23:B8 (Raspberry Pi Foundation) \newline
Nmap done: 1 IP address (1 host up) scanned in 0.09 seconds  \\\hline
\cellcolor{heading}\textbf{Tester} & Christoph Amrein  \\\hline
\cellcolor{heading}\textbf{Datum des Tests} & 22.05.2018  \\\hline
\cellcolor{heading}\textbf{Testergebnis \newline (Fehlerklasse)} & 1 - Fehlerfrei, die Erwartungen sind erfüllt. \\\hline
\cellcolor{heading}\textbf{Fehlerbeschreibung} &   \\\hline
\end{tabular}
\caption{K-026 Protokoll}
\end{table}


\begin{table}[H]
\centering
\begin{tabular}{p{4.5cm}p{11.5cm}}
\hline
\cellcolor{heading}\textbf{Test-ID:} & \textbf{K-027} \\\hline
\cellcolor{heading}\textbf{Testobjekt} & Compute Node c25 \\\hline
\cellcolor{heading}\textbf{Testschritte} & 
- Compute Node starten (Strom anschliessen).\newline
- 3 Minuten warten.\newline
- Auf dem Testclient über Putty oder Shell mit dem Befehl \newline \grqq nmap -sn 192.168.1.35\grqq \ eingeben.\newline
- Prüfen, ob die Zuweisung gemäss Hostnamenkonzept richtig ist. \\\hline
\cellcolor{heading}\textbf{Erwartetes Ergebnis} & Hostname = c25 \newline
IP = 192.168.1.35 \newline
MAC = B8:27:EB:BE:C4:94 \\\hline
\cellcolor{heading}\textbf{Tatsächliches Ergebnis} &
Nmap scan report for c25.home (192.168.1.35) \newline
MAC Address: B8:27:EB:BE:C4:94 (Raspberry Pi Foundation) \newline
Nmap done: 1 IP address (1 host up) scanned in 0.09 seconds  \\\hline
\cellcolor{heading}\textbf{Tester} & Christoph Amrein  \\\hline
\cellcolor{heading}\textbf{Datum des Tests} & 22.05.2018  \\\hline
\cellcolor{heading}\textbf{Testergebnis \newline (Fehlerklasse)} & 1 - Fehlerfrei, die Erwartungen sind erfüllt. \\\hline
\cellcolor{heading}\textbf{Fehlerbeschreibung} &   \\\hline
\end{tabular}
\caption{K-027 Protokoll}
\end{table}

\begin{table}[H]
\centering
\begin{tabular}{p{4.5cm}p{11.5cm}}
\hline
\cellcolor{heading}\textbf{Test-ID:} & \textbf{K-028} \\\hline
\cellcolor{heading}\textbf{Testobjekt} & Compute Node c26 \\\hline
\cellcolor{heading}\textbf{Testschritte} & 
- Compute Node starten (Strom anschliessen).\newline
- 3 Minuten warten.\newline
- Auf dem Testclient über Putty oder Shell mit dem Befehl \newline \grqq nmap -sn 192.168.1.36\grqq \ eingeben.\newline
- Prüfen, ob die Zuweisung gemäss Hostnamenkonzept richtig ist. \\\hline
\cellcolor{heading}\textbf{Erwartetes Ergebnis} & Hostname = c26 \newline
IP = 192.168.1.36 \newline
MAC = B8:27:EB:FB:FF:57 \\\hline
\cellcolor{heading}\textbf{Tatsächliches Ergebnis} &
Nmap scan report for c26.home (192.168.1.36)\newline
MAC Address: B8:27:EB:FB:FF:57 (Raspberry Pi Foundation) \newline
Nmap done: 1 IP address (1 host up) scanned in 0.09 seconds  \\\hline
\cellcolor{heading}\textbf{Tester} & Christoph Amrein  \\\hline
\cellcolor{heading}\textbf{Datum des Tests} & 22.05.2018  \\\hline
\cellcolor{heading}\textbf{Testergebnis \newline (Fehlerklasse)} & 1 - Fehlerfrei, die Erwartungen sind erfüllt. \\\hline
\cellcolor{heading}\textbf{Fehlerbeschreibung} &   \\\hline
\end{tabular}
\caption{K-028 Protokoll}
\end{table}

\begin{table}[H]
\centering
\begin{tabular}{p{4.5cm}p{11.5cm}}
\hline
\cellcolor{heading}\textbf{Test-ID:} & \textbf{K-029} \\\hline
\cellcolor{heading}\textbf{Testobjekt} & Compute Node c27 \\\hline
\cellcolor{heading}\textbf{Testschritte} & 
- Compute Node starten (Strom anschliessen).\newline
- 3 Minuten warten.\newline
- Auf dem Testclient über Putty oder Shell mit dem Befehl \newline \grqq nmap -sn 192.168.1.37\grqq \ eingeben.\newline
- Prüfen, ob die Zuweisung gemäss Hostnamenkonzept richtig ist. \\\hline
\cellcolor{heading}\textbf{Erwartetes Ergebnis} & Hostname = c27 \newline
IP = 192.168.1.37 \newline
MAC = B8:27:EB:4E:EC:CE \\\hline
\cellcolor{heading}\textbf{Tatsächliches Ergebnis} &
Nmap scan report for c27.home (192.168.1.37) \newline
MAC Address: B8:27:EB:4E:EC:CE (Raspberry Pi Foundation) \newline
Nmap done: 1 IP address (1 host up) scanned in 0.10 seconds  \\\hline
\cellcolor{heading}\textbf{Tester} & Christoph Amrein  \\\hline
\cellcolor{heading}\textbf{Datum des Tests} & 22.05.2018  \\\hline
\cellcolor{heading}\textbf{Testergebnis \newline (Fehlerklasse)} & 1 - Fehlerfrei, die Erwartungen sind erfüllt. \\\hline
\cellcolor{heading}\textbf{Fehlerbeschreibung} &   \\\hline
\end{tabular}
\caption{K-029 Protokoll}
\end{table}

\begin{table}[H]
\centering
\begin{tabular}{p{4.5cm}p{11.5cm}}
\hline
\cellcolor{heading}\textbf{Test-ID:} & \textbf{K-030} \\\hline
\cellcolor{heading}\textbf{Testobjekt} & Compute Node c28 \\\hline
\cellcolor{heading}\textbf{Testschritte} & 
- Compute Node starten (Strom anschliessen).\newline
- 3 Minuten warten.\newline
- Auf dem Testclient über Putty oder Shell mit dem Befehl \newline \grqq nmap -sn 192.168.1.38\grqq \ eingeben.\newline
- Prüfen, ob die Zuweisung gemäss Hostnamenkonzept richtig ist. \\\hline
\cellcolor{heading}\textbf{Erwartetes Ergebnis} & Hostname = c28 \newline
IP = 192.168.1.38 \newline
MAC = B8:27:EB:43:1C:35 \\\hline
\cellcolor{heading}\textbf{Tatsächliches Ergebnis} &
Nmap scan report for c28.home (192.168.1.38) \newline
MAC Address: B8:27:EB:43:1C:35 (Raspberry Pi Foundation)\newline
Nmap done: 1 IP address (1 host up) scanned in 0.09 seconds  \\\hline
\cellcolor{heading}\textbf{Tester} & Christoph Amrein  \\\hline
\cellcolor{heading}\textbf{Datum des Tests} & 22.05.2018  \\\hline
\cellcolor{heading}\textbf{Testergebnis \newline (Fehlerklasse)} & 1 - Fehlerfrei, die Erwartungen sind erfüllt. \\\hline
\cellcolor{heading}\textbf{Fehlerbeschreibung} &   \\\hline
\end{tabular}
\caption{K-030 Protokoll}
\end{table}


\begin{table}[H]
\centering
\begin{tabular}{p{4.5cm}p{11.5cm}}
\hline
\cellcolor{heading}\textbf{Test-ID:} & \textbf{K-031} \\\hline
\cellcolor{heading}\textbf{Testobjekt} & Compute Node c29 \\\hline
\cellcolor{heading}\textbf{Testschritte} & 
- Compute Node starten (Strom anschliessen).\newline
- 3 Minuten warten.\newline
- Auf dem Testclient über Putty oder Shell mit dem Befehl \newline \grqq nmap -sn 192.168.1.39\grqq \ eingeben.\newline
- Prüfen, ob die Zuweisung gemäss Hostnamenkonzept richtig ist. \\\hline
\cellcolor{heading}\textbf{Erwartetes Ergebnis} & Hostname = c29 \newline
IP = 192.168.1.39 \newline
MAC = B8:27:EB:DC:74:5F \\\hline
\cellcolor{heading}\textbf{Tatsächliches Ergebnis} &
Nmap scan report for c29.home (192.168.1.39) \newline
MAC Address: B8:27:EB:DC:74:5F (Raspberry Pi Foundation) \newline
Nmap done: 1 IP address (1 host up) scanned in 0.09 seconds  \\\hline
\cellcolor{heading}\textbf{Tester} & Christoph Amrein  \\\hline
\cellcolor{heading}\textbf{Datum des Tests} & 22.05.2018  \\\hline
\cellcolor{heading}\textbf{Testergebnis \newline (Fehlerklasse)} & 1 - Fehlerfrei, die Erwartungen sind erfüllt. \\\hline
\cellcolor{heading}\textbf{Fehlerbeschreibung} &   \\\hline
\end{tabular}
\caption{K-031 Protokoll}
\end{table}

\begin{table}[H]
\centering
\begin{tabular}{p{4.5cm}p{11.5cm}}
\hline
\cellcolor{heading}\textbf{Test-ID:} & \textbf{K-032} \\\hline
\cellcolor{heading}\textbf{Testobjekt} & Compute Node c30 \\\hline
\cellcolor{heading}\textbf{Testschritte} & 
- Compute Node starten (Strom anschliessen).\newline
- 3 Minuten warten.\newline
- Auf dem Testclient über Putty oder Shell mit dem Befehl \newline \grqq nmap -sn 192.168.1.40\grqq \ eingeben.\newline
- Prüfen, ob die Zuweisung gemäss Hostnamenkonzept richtig ist. \\\hline
\cellcolor{heading}\textbf{Erwartetes Ergebnis} & Hostname = c30 \newline
IP = 192.168.1.40 \newline
MAC = B8:27:EB:D1:DE:2F \\\hline
\cellcolor{heading}\textbf{Tatsächliches Ergebnis} &
Nmap scan report for c30.home (192.168.1.40) \newline
MAC Address: B8:27:EB:D1:DE:2F (Raspberry Pi Foundation) \newline
Nmap done: 1 IP address (1 host up) scanned in 0.09 seconds  \\\hline
\cellcolor{heading}\textbf{Tester} & Christoph Amrein  \\\hline
\cellcolor{heading}\textbf{Datum des Tests} & 22.05.2018  \\\hline
\cellcolor{heading}\textbf{Testergebnis \newline (Fehlerklasse)} & 1 - Fehlerfrei, die Erwartungen sind erfüllt. \\\hline
\cellcolor{heading}\textbf{Fehlerbeschreibung} &   \\\hline
\end{tabular}
\caption{K-032 Protokoll}
\end{table}

\begin{table}[H]
\centering
\begin{tabular}{p{4.5cm}p{11.5cm}}
\hline
\cellcolor{heading}\textbf{Test-ID:} & \textbf{K-033} \\\hline
\cellcolor{heading}\textbf{Testobjekt} & Compute Node c31 \\\hline
\cellcolor{heading}\textbf{Testschritte} & 
- Compute Node starten (Strom anschliessen).\newline
- 3 Minuten warten.\newline
- Auf dem Testclient über Putty oder Shell mit dem Befehl \newline \grqq nmap -sn 192.168.1.41\grqq \ eingeben.\newline
- Prüfen, ob die Zuweisung gemäss Hostnamenkonzept richtig ist. \\\hline
\cellcolor{heading}\textbf{Erwartetes Ergebnis} & Hostname = c31 \newline
IP = 192.168.1.41 \newline
MAC = B8:27:EB:5E:90:34 \\\hline
\cellcolor{heading}\textbf{Tatsächliches Ergebnis} &
Nmap scan report for c31.home (192.168.1.41) \newline
MAC Address: B8:27:EB:5E:90:34 (Raspberry Pi Foundation) \newline
Nmap done: 1 IP address (1 host up) scanned in 0.10 seconds  \\\hline
\cellcolor{heading}\textbf{Tester} & Christoph Amrein  \\\hline
\cellcolor{heading}\textbf{Datum des Tests} & 22.05.2018  \\\hline
\cellcolor{heading}\textbf{Testergebnis \newline (Fehlerklasse)} & 1 - Fehlerfrei, die Erwartungen sind erfüllt. \\\hline
\cellcolor{heading}\textbf{Fehlerbeschreibung} &   \\\hline
\end{tabular}
\caption{K-033 Protokoll}
\end{table}

\begin{table}[H]
\centering
\begin{tabular}{p{4.5cm}p{11.5cm}}
\hline
\cellcolor{heading}\textbf{Test-ID:} & \textbf{K-034} \\\hline
\cellcolor{heading}\textbf{Testobjekt} & Compute Node c32 \\\hline
\cellcolor{heading}\textbf{Testschritte} & 
- Compute Node starten (Strom anschliessen).\newline
- 3 Minuten warten.\newline
- Auf dem Testclient über Putty oder Shell mit dem Befehl \newline \grqq nmap -sn 192.168.1.42\grqq \ eingeben.\newline
- Prüfen, ob die Zuweisung gemäss Hostnamenkonzept richtig ist. \\\hline
\cellcolor{heading}\textbf{Erwartetes Ergebnis} & Hostname = c32 \newline
IP = 192.168.1.42 \newline
MAC = B8:27:EB:DE:80:24 \\\hline
\cellcolor{heading}\textbf{Tatsächliches Ergebnis} &
Nmap scan report for c32.home (192.168.1.42) \newline
MAC Address: B8:27:EB:DE:80:24 (Raspberry Pi Foundation) \newline
Nmap done: 1 IP address (1 host up) scanned in 0.09 seconds  \\\hline
\cellcolor{heading}\textbf{Tester} & Christoph Amrein  \\\hline
\cellcolor{heading}\textbf{Datum des Tests} & 22.05.2018  \\\hline
\cellcolor{heading}\textbf{Testergebnis \newline (Fehlerklasse)} & 1 - Fehlerfrei, die Erwartungen sind erfüllt. \\\hline
\cellcolor{heading}\textbf{Fehlerbeschreibung} &   \\\hline
\end{tabular}
\caption{K-034 Protokoll}
\end{table}

\begin{table}[H]
\centering
\begin{tabular}{p{4.5cm}p{11.5cm}}
\hline
\cellcolor{heading}\textbf{Test-ID:} & \textbf{K-035} \\\hline
\cellcolor{heading}\textbf{Testobjekt} & Compute Node c33 \\\hline
\cellcolor{heading}\textbf{Testschritte} & 
- Compute Node starten (Strom anschliessen).\newline
- 3 Minuten warten.\newline
- Auf dem Testclient über Putty oder Shell mit dem Befehl \newline \grqq nmap -sn 192.168.1.43\grqq \ eingeben.\newline
- Prüfen, ob die Zuweisung gemäss Hostnamenkonzept richtig ist. \\\hline
\cellcolor{heading}\textbf{Erwartetes Ergebnis} & Hostname = c33 \newline
IP = 192.168.1.43 \newline
MAC = B8:27:EB:A4:79:6F \\\hline
\cellcolor{heading}\textbf{Tatsächliches Ergebnis} &
Nmap scan report for c33.home (192.168.1.43) \newline
MAC Address: B8:27:EB:A4:79:6F (Raspberry Pi Foundation) \newline
Nmap done: 1 IP address (1 host up) scanned in 0.09 seconds  \\\hline
\cellcolor{heading}\textbf{Tester} & Christoph Amrein  \\\hline
\cellcolor{heading}\textbf{Datum des Tests} & 22.05.2018  \\\hline
\cellcolor{heading}\textbf{Testergebnis \newline (Fehlerklasse)} & 1 - Fehlerfrei, die Erwartungen sind erfüllt. \\\hline
\cellcolor{heading}\textbf{Fehlerbeschreibung} &   \\\hline
\end{tabular}
\caption{K-035 Protokoll}
\end{table}

\begin{table}[H]
\centering
\begin{tabular}{p{4.5cm}p{11.5cm}}
\hline
\cellcolor{heading}\textbf{Test-ID:} & \textbf{K-036} \\\hline
\cellcolor{heading}\textbf{Testobjekt} & Compute Node c34 \\\hline
\cellcolor{heading}\textbf{Testschritte} & 
- Compute Node starten (Strom anschliessen).\newline
- 3 Minuten warten.\newline
- Auf dem Testclient über Putty oder Shell mit dem Befehl \newline \grqq nmap -sn 192.168.1.44\grqq \ eingeben.\newline
- Prüfen, ob die Zuweisung gemäss Hostnamenkonzept richtig ist. \\\hline
\cellcolor{heading}\textbf{Erwartetes Ergebnis} & Hostname = c34 \newline
IP = 192.168.1.44 \newline
MAC = B8:27:EB:0A:4D:C7 \\\hline
\cellcolor{heading}\textbf{Tatsächliches Ergebnis} &
Nmap scan report for c34.home (192.168.1.44) \newline
MAC Address: B8:27:EB:0A:4D:C7 (Raspberry Pi Foundation) \newline
Nmap done: 1 IP address (1 host up) scanned in 0.09 seconds  \\\hline
\cellcolor{heading}\textbf{Tester} & Christoph Amrein  \\\hline
\cellcolor{heading}\textbf{Datum des Tests} & 22.05.2018  \\\hline
\cellcolor{heading}\textbf{Testergebnis \newline (Fehlerklasse)} & 1 - Fehlerfrei, die Erwartungen sind erfüllt. \\\hline
\cellcolor{heading}\textbf{Fehlerbeschreibung} &   \\\hline
\end{tabular}
\caption{K-036 Protokoll}
\end{table}

\begin{table}[H]
\centering
\begin{tabular}{p{4.5cm}p{11.5cm}}
\hline
\cellcolor{heading}\textbf{Test-ID:} & \textbf{K-037} \\\hline
\cellcolor{heading}\textbf{Testobjekt} & Compute Node c35 \\\hline
\cellcolor{heading}\textbf{Testschritte} & 
- Compute Node starten (Strom anschliessen).\newline
- 3 Minuten warten.\newline
- Auf dem Testclient über Putty oder Shell mit dem Befehl \newline \grqq nmap -sn 192.168.1.45\grqq \ eingeben.\newline
- Prüfen, ob die Zuweisung gemäss Hostnamenkonzept richtig ist. \\\hline
\cellcolor{heading}\textbf{Erwartetes Ergebnis} & Hostname = c35 \newline
IP = 192.168.1.45 \newline
MAC = B8:27:EB:5C:53:5F \\\hline
\cellcolor{heading}\textbf{Tatsächliches Ergebnis} &
Note: Host seems down. If it is really up, but blocking our ping probes, try -Pn \newline
Nmap done: 1 IP address (0 hosts up) scanned in 0.45 seconds  \\\hline
\cellcolor{heading}\textbf{Tester} & Christoph Amrein  \\\hline
\cellcolor{heading}\textbf{Datum des Tests} & 22.05.2018  \\\hline
\cellcolor{heading}\textbf{Testergebnis \newline (Fehlerklasse)} & 3 - Das Problem sollte innerhalb von 6 Monaten behoben werden. \\\hline
\cellcolor{heading}\textbf{Fehlerbeschreibung} & Der Compute Node ist defekt und kann nicht mehr in Betrieb genommen werden. \\\hline
\end{tabular}
\caption{K-037 Protokoll}
\end{table}

\begin{table}[H]
\centering
\begin{tabular}{p{4.5cm}p{11.5cm}}
\hline
\cellcolor{heading}\textbf{Test-ID:} & \textbf{K-038} \\\hline
\cellcolor{heading}\textbf{Testobjekt} & Compute Node c36 \\\hline
\cellcolor{heading}\textbf{Testschritte} & 
- Compute Node starten (Strom anschliessen).\newline
- 3 Minuten warten.\newline
- Auf dem Testclient über Putty oder Shell mit dem Befehl \newline \grqq nmap -sn 192.168.1.46\grqq \ eingeben.\newline
- Prüfen, ob die Zuweisung gemäss Hostnamenkonzept richtig ist. \\\hline
\cellcolor{heading}\textbf{Erwartetes Ergebnis} & Hostname = c36 \newline
IP = 192.168.1.46 \newline
MAC = B8:27:EB:F7:AF:C2 \\\hline
\cellcolor{heading}\textbf{Tatsächliches Ergebnis} &
Nmap scan report for c36.home (192.168.1.46) \newline
MAC Address: B8:27:EB:F7:AF:C2 (Raspberry Pi Foundation) \newline
Nmap done: 1 IP address (1 host up) scanned in 0.09 seconds  \\\hline
\cellcolor{heading}\textbf{Tester} & Christoph Amrein  \\\hline
\cellcolor{heading}\textbf{Datum des Tests} & 22.05.2018  \\\hline
\cellcolor{heading}\textbf{Testergebnis \newline (Fehlerklasse)} & 1 - Fehlerfrei, die Erwartungen sind erfüllt. \\\hline
\cellcolor{heading}\textbf{Fehlerbeschreibung} &   \\\hline
\end{tabular}
\caption{K-038 Protokoll}
\end{table}

\begin{table}[H]
\centering
\begin{tabular}{p{4.5cm}p{11.5cm}}
\hline
\cellcolor{heading}\textbf{Test-ID:} & \textbf{K-039} \\\hline
\cellcolor{heading}\textbf{Testobjekt} & Compute Node c37 \\\hline
\cellcolor{heading}\textbf{Testschritte} & 
- Compute Node starten (Strom anschliessen).\newline
- 3 Minuten warten.\newline
- Auf dem Testclient über Putty oder Shell mit dem Befehl \newline \grqq nmap -sn 192.168.1.47\grqq \ eingeben.\newline
- Prüfen, ob die Zuweisung gemäss Hostnamenkonzept richtig ist. \\\hline
\cellcolor{heading}\textbf{Erwartetes Ergebnis} & Hostname = c37 \newline
IP = 192.168.1.47 \newline
MAC = B8:27:EB:CE:BA:ED \\\hline
\cellcolor{heading}\textbf{Tatsächliches Ergebnis} &
Nmap scan report for c37.home (192.168.1.47) \newline
MAC Address: B8:27:EB:CE:BA:ED (Raspberry Pi Foundation) \newline
Nmap done: 1 IP address (1 host up) scanned in 0.09 seconds  \\\hline
\cellcolor{heading}\textbf{Tester} & Christoph Amrein  \\\hline
\cellcolor{heading}\textbf{Datum des Tests} & 22.05.2018  \\\hline
\cellcolor{heading}\textbf{Testergebnis \newline (Fehlerklasse)} & 1 - Fehlerfrei, die Erwartungen sind erfüllt. \\\hline
\cellcolor{heading}\textbf{Fehlerbeschreibung} &   \\\hline
\end{tabular}
\caption{K-039 Protokoll}
\end{table}

\begin{table}[H]
\centering
\begin{tabular}{p{4.5cm}p{11.5cm}}
\hline
\cellcolor{heading}\textbf{Test-ID:} & \textbf{K-040} \\\hline
\cellcolor{heading}\textbf{Testobjekt} & Compute Node c38 \\\hline
\cellcolor{heading}\textbf{Testschritte} & 
- Compute Node starten (Strom anschliessen).\newline
- 3 Minuten warten.\newline
- Auf dem Testclient über Putty oder Shell mit dem Befehl \newline \grqq nmap -sn 192.168.1.48\grqq \ eingeben.\newline
- Prüfen, ob die Zuweisung gemäss Hostnamenkonzept richtig ist. \\\hline
\cellcolor{heading}\textbf{Erwartetes Ergebnis} & Hostname = c38 \newline
IP = 192.168.1.48 \newline
MAC = B8:27:EB:59:38:3C \\\hline
\cellcolor{heading}\textbf{Tatsächliches Ergebnis} &
Nmap scan report for c38.home (192.168.1.48) \newline
MAC Address: B8:27:EB:59:38:3C (Raspberry Pi Foundation) \newline
Nmap done: 1 IP address (1 host up) scanned in 0.09 seconds  \\\hline
\cellcolor{heading}\textbf{Tester} & Christoph Amrein  \\\hline
\cellcolor{heading}\textbf{Datum des Tests} & 22.05.2018  \\\hline
\cellcolor{heading}\textbf{Testergebnis \newline (Fehlerklasse)} & 1 - Fehlerfrei, die Erwartungen sind erfüllt. \\\hline
\cellcolor{heading}\textbf{Fehlerbeschreibung} &   \\\hline
\end{tabular}
\caption{K-040 Protokoll}
\end{table}

\begin{table}[H]
\centering
\begin{tabular}{p{4.5cm}p{11.5cm}}
\hline
\cellcolor{heading}\textbf{Test-ID:} & \textbf{K-041} \\\hline
\cellcolor{heading}\textbf{Testobjekt} & Compute Node c39 \\\hline
\cellcolor{heading}\textbf{Testschritte} & 
- Compute Node starten (Strom anschliessen).\newline
- 3 Minuten warten.\newline
- Auf dem Testclient über Putty oder Shell mit dem Befehl \newline \grqq nmap -sn 192.168.1.49\grqq \ eingeben.\newline
- Prüfen, ob die Zuweisung gemäss Hostnamenkonzept richtig ist. \\\hline
\cellcolor{heading}\textbf{Erwartetes Ergebnis} & Hostname = c39 \newline
IP = 192.168.1.49 \newline
MAC = B8:27:EB:99:BB:8E\\\hline
\cellcolor{heading}\textbf{Tatsächliches Ergebnis} &
Nmap scan report for c39.home (192.168.1.49) \newline
MAC Address: B8:27:EB:99:BB:8E (Raspberry Pi Foundation) \newline
Nmap done: 1 IP address (1 host up) scanned in 0.09 seconds  \\\hline
\cellcolor{heading}\textbf{Tester} & Christoph Amrein  \\\hline
\cellcolor{heading}\textbf{Datum des Tests} & 22.05.2018  \\\hline
\cellcolor{heading}\textbf{Testergebnis \newline (Fehlerklasse)} & 1 - Fehlerfrei, die Erwartungen sind erfüllt. \\\hline
\cellcolor{heading}\textbf{Fehlerbeschreibung} &   \\\hline
\end{tabular}
\caption{K-041 Protokoll}
\end{table}

\begin{table}[H]
\centering
\begin{tabular}{p{4.5cm}p{11.5cm}}
\hline
\cellcolor{heading}\textbf{Test-ID:} & \textbf{K-042} \\\hline
\cellcolor{heading}\textbf{Testobjekt} & Compute Node c40 \\\hline
\cellcolor{heading}\textbf{Testschritte} & 
- Compute Node starten (Strom anschliessen).\newline
- 3 Minuten warten.\newline
- Auf dem Testclient über Putty oder Shell mit dem Befehl \newline \grqq nmap -sn 192.168.1.50\grqq \ eingeben.\newline
- Prüfen, ob die Zuweisung gemäss Hostnamenkonzept richtig ist. \\\hline
\cellcolor{heading}\textbf{Erwartetes Ergebnis} & Hostname = c40 \newline
IP = 192.168.1.50 \newline
MAC = B8:27:EB:8F:7A:0D\\\hline
\cellcolor{heading}\textbf{Tatsächliches Ergebnis} &
Nmap scan report for c40.home (192.168.1.50) \newline
MAC Address: B8:27:EB:8F:7A:0D (Raspberry Pi Foundation) \newline
Nmap done: 1 IP address (1 host up) scanned in 0.10 seconds  \\\hline
\cellcolor{heading}\textbf{Tester} & Christoph Amrein  \\\hline
\cellcolor{heading}\textbf{Datum des Tests} & 22.05.2018  \\\hline
\cellcolor{heading}\textbf{Testergebnis \newline (Fehlerklasse)} & 1 - Fehlerfrei, die Erwartungen sind erfüllt. \\\hline
\cellcolor{heading}\textbf{Fehlerbeschreibung} &   \\\hline
\end{tabular}
\caption{K-042 Protokoll}
\end{table}


\begin{table}[H]
\centering
\begin{tabular}{p{4.5cm}p{11.5cm}}
\hline
\cellcolor{heading}\textbf{Test-ID:} & \textbf{K-043} \\\hline
\cellcolor{heading}\textbf{Testobjekt} & Compute Node c41 \\\hline
\cellcolor{heading}\textbf{Testschritte} & 
- Compute Node starten (Strom anschliessen).\newline
- 3 Minuten warten.\newline
- Auf dem Testclient über Putty oder Shell mit dem Befehl \newline \grqq nmap -sn 192.168.1.51\grqq \ eingeben.\newline
- Prüfen, ob die Zuweisung gemäss Hostnamenkonzept richtig ist. \\\hline
\cellcolor{heading}\textbf{Erwartetes Ergebnis} & Hostname = c41 \newline
IP = 192.168.1.51 \newline
MAC = B8:27:EB:DE:C9:69 \\\hline
\cellcolor{heading}\textbf{Tatsächliches Ergebnis} &
Nmap scan report for c41.home (192.168.1.51) \newline
MAC Address: B8:27:EB:DE:C9:69 (Raspberry Pi Foundation) \newline
Nmap done: 1 IP address (1 host up) scanned in 0.09 seconds  \\\hline
\cellcolor{heading}\textbf{Tester} & Christoph Amrein  \\\hline
\cellcolor{heading}\textbf{Datum des Tests} & 22.05.2018  \\\hline
\cellcolor{heading}\textbf{Testergebnis \newline (Fehlerklasse)} & 1 - Fehlerfrei, die Erwartungen sind erfüllt. \\\hline
\cellcolor{heading}\textbf{Fehlerbeschreibung} &   \\\hline
\end{tabular}
\caption{K-043 Protokoll}
\end{table}


\begin{table}[H]
\centering
\begin{tabular}{p{4.5cm}p{11.5cm}}
\hline
\cellcolor{heading}\textbf{Test-ID:} & \textbf{K-044} \\\hline
\cellcolor{heading}\textbf{Testobjekt} & Compute Node c42 \\\hline
\cellcolor{heading}\textbf{Testschritte} & 
- Compute Node starten (Strom anschliessen).\newline
- 3 Minuten warten.\newline
- Auf dem Testclient über Putty oder Shell mit dem Befehl \newline \grqq nmap -sn 192.168.1.52\grqq \ eingeben.\newline
- Prüfen, ob die Zuweisung gemäss Hostnamenkonzept richtig ist. \\\hline
\cellcolor{heading}\textbf{Erwartetes Ergebnis} & Hostname = c42 \newline
IP = 192.168.1.52 \newline
MAC = B8:27:EB:7E:6F:48 \\\hline
\cellcolor{heading}\textbf{Tatsächliches Ergebnis} &
Nmap scan report for c42.home (192.168.1.52) \newline
MAC Address: B8:27:EB:7E:6F:48 (Raspberry Pi Foundation) \newline
Nmap done: 1 IP address (1 host up) scanned in 0.09 seconds  \\\hline
\cellcolor{heading}\textbf{Tester} & Christoph Amrein  \\\hline
\cellcolor{heading}\textbf{Datum des Tests} & 22.05.2018  \\\hline
\cellcolor{heading}\textbf{Testergebnis \newline (Fehlerklasse)} & 1 - Fehlerfrei, die Erwartungen sind erfüllt. \\\hline
\cellcolor{heading}\textbf{Fehlerbeschreibung} &   \\\hline
\end{tabular}
\caption{K-044 Protokoll}
\end{table}


\begin{table}[H]
\centering
\begin{tabular}{p{4.5cm}p{11.5cm}}
\hline
\cellcolor{heading}\textbf{Test-ID:} & \textbf{K-045} \\\hline
\cellcolor{heading}\textbf{Testobjekt} & Compute Node c43 \\\hline
\cellcolor{heading}\textbf{Testschritte} & 
- Compute Node starten (Strom anschliessen).\newline
- 3 Minuten warten.\newline
- Auf dem Testclient über Putty oder Shell mit dem Befehl \newline \grqq nmap -sn 192.168.1.53\grqq \ eingeben.\newline
- Prüfen, ob die Zuweisung gemäss Hostnamenkonzept richtig ist. \\\hline
\cellcolor{heading}\textbf{Erwartetes Ergebnis} & Hostname = c43 \newline
IP = 192.168.1.53 \newline
MAC = B8:27:EB:5D:DD:FE \\\hline
\cellcolor{heading}\textbf{Tatsächliches Ergebnis} &
nmap scan report for c43.home (192.168.1.53)  \newline
MAC Address: B8:27:EB:5D:DD:FE (Raspberry Pi Foundation)\newline
Nmap done: 1 IP address (1 host up) scanned in 0.09 seconds  \\\hline
\cellcolor{heading}\textbf{Tester} & Christoph Amrein  \\\hline
\cellcolor{heading}\textbf{Datum des Tests} & 22.05.2018  \\\hline
\cellcolor{heading}\textbf{Testergebnis \newline (Fehlerklasse)} & 1 - Fehlerfrei, die Erwartungen sind erfüllt. \\\hline
\cellcolor{heading}\textbf{Fehlerbeschreibung} &   \\\hline
\end{tabular}
\caption{K-045 Protokoll}
\end{table}

\begin{table}[H]
\centering
\begin{tabular}{p{4.5cm}p{11.5cm}}
\hline
\cellcolor{heading}\textbf{Test-ID:} & \textbf{K-046} \\\hline
\cellcolor{heading}\textbf{Testobjekt} & Compute Node c44 \\\hline
\cellcolor{heading}\textbf{Testschritte} & 
- Compute Node starten (Strom anschliessen).\newline
- 3 Minuten warten.\newline
- Auf dem Testclient über Putty oder Shell mit dem Befehl \newline \grqq nmap -sn 192.168.1.54\grqq \ eingeben.\newline
- Prüfen, ob die Zuweisung gemäss Hostnamenkonzept richtig ist. \\\hline
\cellcolor{heading}\textbf{Erwartetes Ergebnis} & Hostname = c44 \newline
IP = 192.168.1.54 \newline
MAC = 	B8:27:EB:A6:6D:4D \\\hline
\cellcolor{heading}\textbf{Tatsächliches Ergebnis} &
Nmap scan report for c44.home (192.168.1.54) \newline
MAC Address: B8:27:EB:A6:6D:4D (Raspberry Pi Foundation) \newline
Nmap done: 1 IP address (1 host up) scanned in 0.10 seconds  \\\hline
\cellcolor{heading}\textbf{Tester} & Christoph Amrein  \\\hline
\cellcolor{heading}\textbf{Datum des Tests} & 22.05.2018  \\\hline
\cellcolor{heading}\textbf{Testergebnis \newline (Fehlerklasse)} & 1 - Fehlerfrei, die Erwartungen sind erfüllt. \\\hline
\cellcolor{heading}\textbf{Fehlerbeschreibung} &   \\\hline
\end{tabular}
\caption{K-046 Protokoll}
\end{table}


\begin{table}[H]
\centering
\begin{tabular}{p{4.5cm}p{11.5cm}}
\hline
\cellcolor{heading}\textbf{Test-ID:} & \textbf{K-047} \\\hline
\cellcolor{heading}\textbf{Testobjekt} & Compute Node c45 \\\hline
\cellcolor{heading}\textbf{Testschritte} & 
- Compute Node starten (Strom anschliessen).\newline
- 3 Minuten warten.\newline
- Auf dem Testclient über Putty oder Shell mit dem Befehl \newline \grqq nmap -sn 192.168.1.55\grqq \ eingeben.\newline
- Prüfen, ob die Zuweisung gemäss Hostnamenkonzept richtig ist. \\\hline
\cellcolor{heading}\textbf{Erwartetes Ergebnis} & Hostname = c45 \newline
IP = 192.168.1.55 \newline
MAC = B8:27:EB:0C:63:10 \\\hline
\cellcolor{heading}\textbf{Tatsächliches Ergebnis} &
Nmap scan report for c45.home (192.168.1.55) \newline
MAC Address: B8:27:EB:0C:63:10 (Raspberry Pi Foundation) \newline
Nmap done: 1 IP address (1 host up) scanned in 0.09 seconds  \\\hline
\cellcolor{heading}\textbf{Tester} & Christoph Amrein  \\\hline
\cellcolor{heading}\textbf{Datum des Tests} & 22.05.2018  \\\hline
\cellcolor{heading}\textbf{Testergebnis \newline (Fehlerklasse)} & 1 - Fehlerfrei, die Erwartungen sind erfüllt. \\\hline
\cellcolor{heading}\textbf{Fehlerbeschreibung} &   \\\hline
\end{tabular}
\caption{K-047 Protokoll}
\end{table}


\begin{table}[H]
\centering
\begin{tabular}{p{4.5cm}p{11.5cm}}
\hline
\cellcolor{heading}\textbf{Test-ID:} & \textbf{K-048} \\\hline
\cellcolor{heading}\textbf{Testobjekt} & NAS \\\hline
\cellcolor{heading}\textbf{Testschritte} & 
Den Befehl \grqq ping 192.168.1.129\grqq \ \ auf dem Testclient ausführen und 2 Versuche abwarten. \\\hline
\cellcolor{heading}\textbf{Erwartetes Ergebnis} & Das NAS antwortet auf den Ping befehl mit: \grqq 2 packets transmitted, 2 received, 0\% packet loss\grqq. \\\hline
\cellcolor{heading}\textbf{Tatsächliches Ergebnis} &
PING 192.168.1.129 (192.168.1.129) 56(84) bytes of data. \newline
64 bytes from 192.168.1.129: icmp\_seq=1 ttl=64 time=0.539 ms \newline
64 bytes from 192.168.1.129: icmp\_seq=2 ttl=64 time=0.434 ms \newline
--- 192.168.1.129 ping statistics --- \newline
2 packets transmitted, 2 received, 0\% packet loss, time 1001ms 
rtt min/avg/max/mdev = 0.434/0.486/0.539/0.056 ms \\\hline
\cellcolor{heading}\textbf{Tester} & Christoph Amrein  \\\hline
\cellcolor{heading}\textbf{Datum des Tests} & 22.05.2018  \\\hline
\cellcolor{heading}\textbf{Testergebnis \newline (Fehlerklasse)} & 1 - Fehlerfrei, die Erwartungen sind erfüllt. \\\hline
\cellcolor{heading}\textbf{Fehlerbeschreibung} &   \\\hline
\end{tabular}
\caption{K-048 Protokoll}
\end{table}
\newpage
\subsection{Integrationstests}
\begin{longtable}{p{4.5cm}p{11.5cm}}
\hline
\cellcolor{heading}\textbf{Test-ID:} & \textbf{I-001} \\\hline
\cellcolor{heading}\textbf{Testobjekt} & Compute Nodes \\\hline
\cellcolor{heading}\textbf{Testschritte} & 
Den Befehl \grqq  pdsh -w c[1-45] ping -c 1 google.de\grqq \ auf dem Management Node eingeben. \\\hline 
\cellcolor{heading}\textbf{Erwartetes Ergebnis} & Der Rückgabewert muss pro Compute Node die Ausgabe \grqq 1 packets transmitted, 1 received, 0\% packet loss\grqq beinhalten. Damit der Test erfolgreich war. \\\hline
\cellcolor{heading}\textbf{Tatsächliches Ergebnis} &
--- google.de ping statistics --- \newline
c1: 1 packets transmitted, 1 received, 0\% packet loss, time 0ms \newline
c2: 1 packets transmitted, 1 received, 0\% packet loss, time 0ms \newline
c3: 1 packets transmitted, 1 received, 0\% packet loss, time 0ms \newline
c4: 1 packets transmitted, 1 received, 0\% packet loss, time 0ms \newline
c5: 1 packets transmitted, 1 received, 0\% packet loss, time 0ms \newline
c6: 1 packets transmitted, 1 received, 0\% packet loss, time 0ms \newline
c7: 1 packets transmitted, 1 received, 0\% packet loss, time 0ms \newline
c8: 1 packets transmitted, 1 received, 0\% packet loss, time 0ms \newline
c9: 1 packets transmitted, 1 received, 0\% packet loss, time 0ms \newline
c10: 1 packets transmitted, 1 received, 0\% packet loss, time 0ms \newline
c11: 1 packets transmitted, 1 received, 0\% packet loss, time 0ms \newline
c12: 1 packets transmitted, 1 received, 0\% packet loss, time 0ms \newline
c13: 1 packets transmitted, 1 received, 0\% packet loss, time 0ms \newline
c14: 1 packets transmitted, 1 received, 0\% packet loss, time 0ms \newline
c15: 1 packets transmitted, 1 received, 0\% packet loss, time 0ms \newline
c16: 1 packets transmitted, 1 received, 0\% packet loss, time 0ms \newline
c17: 1 packets transmitted, 1 received, 0\% packet loss, time 0ms \newline 
c18: 1 packets transmitted, 1 received, 0\% packet loss, time 0ms \newline
c19: 1 packets transmitted, 1 received, 0\% packet loss, time 0ms \newline
c20: 1 packets transmitted, 1 received, 0\% packet loss, time 0ms \newline
c21: 1 packets transmitted, 1 received, 0\% packet loss, time 0ms \newline
c22: 1 packets transmitted, 1 received, 0\% packet loss, time 0ms \newline
c23: 1 packets transmitted, 1 received, 0\% packet loss, time 0ms \newline
c24: 1 packets transmitted, 1 received, 0\% packet loss, time 0ms \newline 
c25: 1 packets transmitted, 1 received, 0\% packet loss, time 0ms \newline
c26: 1 packets transmitted, 1 received, 0\% packet loss, time 0ms \newline
\\\hline
\cellcolor{heading}\textbf{Tatsächliches Ergebnis} & 
c27: 1 packets transmitted, 1 received, 0\% packet loss, time 0ms \newline
c28: 1 packets transmitted, 1 received, 0\% packet loss, time 0ms \newline
c29: 1 packets transmitted, 1 received, 0\% packet loss, time 0ms \newline
c30: 1 packets transmitted, 1 received, 0\% packet loss, time 0ms \newline
c31: 1 packets transmitted, 1 received, 0\% packet loss, time 0ms \newline
c32: 1 packets transmitted, 1 received, 0\% packet loss, time 0ms \newline
c33: 1 packets transmitted, 1 received, 0\% packet loss, time 0ms \newline
c34: 1 packets transmitted, 1 received, 0\% packet loss, time 0ms \newline
c35: ssh: connect to host c35 port 22: No route to host\newline
c36: 1 packets transmitted, 1 received, 0\% packet loss, time 0ms \newline
c37: 1 packets transmitted, 1 received, 0\% packet loss, time 0ms \newline
c38: 1 packets transmitted, 1 received, 0\% packet loss, time 0ms \newline
c39: 1 packets transmitted, 1 received, 0\% packet loss, time 0ms \newline
c40: 1 packets transmitted, 1 received, 0\% packet loss, time 0ms \newline
c41: 1 packets transmitted, 1 received, 0\% packet loss, time 0ms \newline
c42: 1 packets transmitted, 1 received, 0\% packet loss, time 0ms \newline
c43: 1 packets transmitted, 1 received, 0\% packet loss, time 0ms \newline
c44: 1 packets transmitted, 1 received, 0\% packet loss, time 0ms \newline
c45: 1 packets transmitted, 1 received, 0\% packet loss, time 0ms \newline  \\\hline
\cellcolor{heading}\textbf{Tester} & Christoph Amrein  \\\hline
\cellcolor{heading}\textbf{Datum des Tests} & 22.05.2018  \\\hline
\cellcolor{heading}\textbf{Testergebnis \newline (Fehlerklasse)} & 1 - Fehlerfrei, die Erwartungen sind erfüllt. \\\hline
\cellcolor{heading}\textbf{Fehlerbeschreibung} &  Der Compute Node c35 wurde bereits bei den Komponententests als fehlerhaft erkannt. Deshalb ist das Egebnis wie erwartet. \\\hline
\caption{I-001 Protokoll}
\end{longtable}

\begin{table}[H]
\centering
\begin{tabular}{p{4.5cm}p{11.5cm}}
\hline
\cellcolor{heading}\textbf{Test-ID:} & \textbf{I-002} \\\hline
\cellcolor{heading}\textbf{Testobjekt} & NAS / Mountpoint / Management Node \\\hline
\cellcolor{heading}\textbf{Testschritte} & 
- Management Node starten.\newline
- Auf dem Management Node anmelden.\newline
- Den Befehl \grqq ls -l /media/nebula\_data/ | wc -l\grqq \ eingeben. \\\hline
\cellcolor{heading}\textbf{Erwartetes Ergebnis} & Die Ausgabe des Befehls muss einen grösseren Wert als \grqq 1\grqq zurückgeben. \\\hline
\cellcolor{heading}\textbf{Tatsächliches Ergebnis} &
17  \\\hline
\cellcolor{heading}\textbf{Tester} & Christoph Amrein  \\\hline
\cellcolor{heading}\textbf{Datum des Tests} & 22.05.2018  \\\hline
\cellcolor{heading}\textbf{Testergebnis \newline (Fehlerklasse)} & 1 - Fehlerfrei, die Erwartungen sind erfüllt. \\\hline
\cellcolor{heading}\textbf{Fehlerbeschreibung} &   \\\hline
\end{tabular}
\caption{I-002 Protokoll}
\end{table}

\begin{table}[H]
\centering
\begin{tabular}{p{4.5cm}p{11.5cm}}
\hline
\cellcolor{heading}\textbf{Test-ID:} & \textbf{I-003} \\\hline
\cellcolor{heading}\textbf{Testobjekt} & NAS / Mountpoint / Compute Nodes\\\hline
\cellcolor{heading}\textbf{Testschritte} & 
- Compute Nodes starten.\newline
- auf dem Management Node anmelden.\newline
- Den Befehl \grqq pdsh -w c[1-45] ls -l /media/nebula\_data/ | wc -l\grqq \  eingeben. \\\hline
\cellcolor{heading}\textbf{Erwartetes Ergebnis} & Die Ausgabe des Befehls muss jeweils einen grösseren Wert als \grqq 1\grqq zurückgeben. \\\hline
\cellcolor{heading}\textbf{Tatsächliches Ergebnis} &
Alle Compute Nodes haben folgendes ausgegeben. Die Ausgabe ist gekürzt. \newline
c1: 17 \newline
c2: 17 \newline
c3: 17 \newline
c4: 17 \newline
c5: 17 \newline
c6: 17 \newline
c7: 17 \newline
.. \newline
c35: ssh: connect to host c35 port 22: No route to host
pdsh@nebula: c35: ssh exited with exit code 255\newline
c36: 17 \newline
.. \newline
c43: 17 \newline
c44: 17 \newline
c45: 17 \\\hline
\cellcolor{heading}\textbf{Tester} & Christoph Amrein  \\\hline
\cellcolor{heading}\textbf{Datum des Tests} & 22.05.2018  \\\hline
\cellcolor{heading}\textbf{Testergebnis \newline (Fehlerklasse)} & 1 - Fehlerfrei, die Erwartungen sind erfüllt. \\\hline
\cellcolor{heading}\textbf{Fehlerbeschreibung} & Der Compute Node c35 wurde bereits bei den Komponententests als fehlerhaft erkannt. Deshalb ist das Ergebnis wie erwartet.  \\\hline
\end{tabular}
\caption{I-003 Protokoll}
\end{table}

\begin{table}[H]
\centering
\begin{tabular}{p{4.5cm}p{11.5cm}}
\hline
\cellcolor{heading}\textbf{Test-ID:} & \textbf{I-004} \\\hline
\cellcolor{heading}\textbf{Testobjekt} & Cluster \\\hline
\cellcolor{heading}\textbf{Testschritte} & 
- Cluster herunterfahren.\newline
- Externe physische Verbindungen trennen.\newline
- Cluster in ein anderes Zimmer verschieben und wiederaufbauen (selbes Netzwerk). \\\hline
\cellcolor{heading}\textbf{Erwartetes Ergebnis} & Der Wiederaufbau soll nicht länger als 15 Minuten dauern. \\\hline
\cellcolor{heading}\textbf{Tatsächliches Ergebnis} &
Der Cluster konnte innerhalb von 12 Minuten wiederaufgebaut werden. \\\hline
\cellcolor{heading}\textbf{Tester} & Christoph Amrein  \\\hline
\cellcolor{heading}\textbf{Datum des Tests} & 22.05.2018  \\\hline
\cellcolor{heading}\textbf{Testergebnis \newline (Fehlerklasse)} & 1 - Fehlerfrei, die Erwartungen sind erfüllt. \\\hline
\cellcolor{heading}\textbf{Fehlerbeschreibung} &   \\\hline
\end{tabular}
\caption{I-004 Protokoll}
\end{table}

\begin{table}[H]
\centering
\begin{tabular}{p{4.5cm}p{11.5cm}}
\hline
\cellcolor{heading}\textbf{Test-ID:} & \textbf{I-005} \\\hline
\cellcolor{heading}\textbf{Testobjekt} & Cluster \\\hline
\cellcolor{heading}\textbf{Testschritte} & 
- NAS ausschalten. \newline
- Festplatte 1 abhängen. \newline
- Prüfen ob Die Datei test.txt auf der Festplatte 2 ist. \newline
- Festplatte 1 anhängen. \newline
- Die obenstehenden Punkte mit der jeweils anderen Festplatte wiederholen. \\\hline
\cellcolor{heading}\textbf{Erwartetes Ergebnis} & Die test.txt Datei ist auf beiden Festplatten vorhanden. \\\hline
\cellcolor{heading}\textbf{Tatsächliches Ergebnis} &
Die test.txt Datei war auf beiden Festplatten zu finden. \\\hline
\cellcolor{heading}\textbf{Tester} & Christoph Amrein  \\\hline
\cellcolor{heading}\textbf{Datum des Tests} & 22.05.2018  \\\hline
\cellcolor{heading}\textbf{Testergebnis \newline (Fehlerklasse)} & 1 - Fehlerfrei, die Erwartungen sind erfüllt. \\\hline
\cellcolor{heading}\textbf{Fehlerbeschreibung} &   \\\hline
\end{tabular}
\caption{I-005 Protokoll}
\end{table}

\newpage
\subsection{Systemtests}
\begin{longtable}{p{4.5cm}p{11.5cm}}
\hline
\cellcolor{heading}\textbf{Test-ID:} & \textbf{S-001} \\\hline
\cellcolor{heading}\textbf{Testobjekt} & Compute Nodes / CPU Last\\\hline
\cellcolor{heading}\textbf{Testschritte} & 
Die Tests finden vom Management Node aus statt.\newline
- Den Befehl \grqq pdsh -w c[1-45] uptime\grqq \ eingeben. \newline 
- Alternativ kann man sich auf jeden Compute Node anmelden und den Befehl \grqq top\grqq \ eingeben. \\\hline
\cellcolor{heading}\textbf{Erwartetes Ergebnis} & 
Die Loadaverage-Werte überschreiten jeweils den Wert 1, die Ausgabe sollte in etwa so aussehen: \newline  load average: 4.18, 2.44, 1.16 \\\hline
\cellcolor{heading}\textbf{Tatsächliches Ergebnis} &
c1:  16:42:23 up  1:02,  0 users,  load average: 4.26, 4.41, 3.29 \newline
c2:  16:42:23 up  1:29,  0 users,  load average: 4.25, 4.26, 3.20 \newline
c3:  16:42:23 up  1:02,  0 users,  load average: 4.38, 4.32, 3.25 \newline
c4:  16:42:23 up  1:29,  0 users,  load average: 4.50, 4.35, 3.25 \newline
c5:  16:42:23 up  1:02,  0 users,  load average: 4.42, 4.30, 3.22 \newline
c6:  16:42:23 up  1:29,  0 users,  load average: 4.26, 4.38, 3.25 \newline
c7:  16:42:23 up  1:29,  0 users,  load average: 5.01, 4.46, 3.25 \newline
c8:  16:42:23 up  1:02,  0 users,  load average: 4.54, 4.42, 3.29 \newline
c9:  16:42:24 up  1:02,  0 users,  load average: 4.68, 4.38, 3.23 \newline
c10:  16:42:23 up  1:29,  0 users,  load average: 4.84, 4.54, 3.31 \newline
c11:  16:42:24 up  1:02,  0 users,  load average: 4.36, 4.38, 3.23 \newline
c12:  16:42:23 up  1:02,  0 users,  load average: 4.30, 4.22, 3.18 \newline
c13:  16:42:23 up  1:29,  0 users,  load average: 4.42, 4.36, 3.28 \newline
c14:  16:42:23 up  1:29,  0 users,  load average: 4.38, 4.28, 3.18 \newline
c15:  16:42:23 up  1:29,  0 users,  load average: 4.61, 4.53, 3.31 \newline
c16:  16:42:23 up  1:02,  0 users,  load average: 4.43, 4.39, 3.22 \newline
c17:  16:42:23 up  1:29,  0 users,  load average: 4.29, 4.28, 3.21 \newline
c18:  16:42:23 up  1:29,  0 users,  load average: 4.47, 4.34, 3.24 \newline
c19:  16:42:23 up  1:29,  0 users,  load average: 4.69, 4.42, 3.26 \newline
c20:  16:42:23 up  1:29,  0 users,  load average: 4.71, 4.42, 3.26 \newline
c21:  16:42:23 up  1:29,  0 users,  load average: 4.91, 4.46, 3.22 \newline
c22:  16:42:23 up  1:29,  0 users,  load average: 4.64, 4.34, 3.23 \newline
c23:  16:42:23 up  1:29,  0 users,  load average: 4.47, 4.32, 3.25 \newline
 \\\hline
\cellcolor{heading}\textbf{Tatsächliches Ergebnis} &
c24:  16:42:23 up  1:29,  0 users,  load average: 4.36, 4.30, 3.23 \newline
c25:  16:42:23 up  1:02,  0 users,  load average: 4.34, 4.31, 3.24 \newline
c26:  16:42:23 up  1:29,  0 users,  load average: 4.54, 4.43, 3.26 \newline
c27:  16:42:23 up  1:02,  0 users,  load average: 4.82, 4.45, 3.27 \newline
c28:  16:42:23 up  1:02,  0 users,  load average: 4.48, 4.38, 3.24 \newline
c29:  16:42:23 up  1:29,  0 users,  load average: 4.28, 4.32, 3.22 \newline
c30:  16:42:23 up  1:29,  0 users,  load average: 4.37, 4.31, 3.24 \newline
c31:  16:42:23 up  1:02,  0 users,  load average: 4.48, 4.36, 3.24 \newline
c32:  16:42:23 up  1:29,  0 users,  load average: 4.27, 4.26, 3.19 \newline
c33:  16:42:27 up  1:02,  0 users,  load average: 4.62, 4.42, 3.30 \newline
c34:  16:42:28 up  1:02,  0 users,  load average: 4.52, 4.27, 3.16 \newline
c35: ssh: connect to host c35 port 22: No route to host \newline
c36:  16:42:23 up  1:02,  0 users,  load average: 4.82, 4.45, 3.27 \newline
c37:  16:42:23 up  1:02,  0 users,  load average: 4.48, 4.38, 3.24 \newline
c38:  16:42:28 up  1:29,  0 users,  load average: 4.28, 4.18, 3.12 \newline
c39:  16:42:28 up  1:02,  0 users,  load average: 4.28, 4.40, 3.29 \newline
c40:  16:42:28 up  1:02,  0 users,  load average: 4.58, 4.51, 3.34 \newline
c41:  16:42:28 up 20 min,  0 users,  load average: 4.70, 4.40, 3.20 \newline
c42:  16:42:28 up  1:02,  0 users,  load average: 4.55, 4.38, 3.26 \newline
c43:  16:42:28 up  1:29,  0 users,  load average: 4.82, 4.49, 3.30 \newline
c44:  16:42:23 up  1:29,  0 users,  load average: 4.47, 4.34, 3.24 \newline
c45:  16:42:23 up  1:29,  0 users,  load average: 4.82, 4.45, 3.27 
\\\hline
\cellcolor{heading}\textbf{Tester} & Christoph Amrein  \\\hline
\cellcolor{heading}\textbf{Datum des Tests} & 22.05.2018  \\\hline
\cellcolor{heading}\textbf{Testergebnis \newline (Fehlerklasse)} & 1 - Fehlerfrei, die Erwartungen sind erfüllt. \\\hline
\cellcolor{heading}\textbf{Fehlerbeschreibung} & Der Compute Node c35 wurde bereits bei den Komponententests als fehlerhaft erkannt. Deshalb ist das Ergebnis wie erwartet.  \\\hline
\caption{S-001 Protokoll}
\end{longtable}

\begin{table}[H]
\centering
\begin{tabular}{p{4.5cm}p{11.5cm}}
\hline
\cellcolor{heading}\textbf{Test-ID:} & \textbf{S-002} \\\hline
\cellcolor{heading}\textbf{Testobjekt} & Nodes / Nagios Monitoring\\\hline
\cellcolor{heading}\textbf{Testschritte} & 
- \url{http://nebula/nagios} aufrufen. \newline
- Bei Nagios anmelden. \newline
- Auf den Reiter \grqq Hosts\grqq \  klicken. \\\hline
\cellcolor{heading}\textbf{Erwartetes Ergebnis} & Es werden alle Nodes aufgelistet. Unabhängig des Status, ob der Node in Betrieb ist oder nicht. \\\hline
\cellcolor{heading}\textbf{Tatsächliches Ergebnis} &
Alle Nodes sind aufgelistet und haben den Status UP, mit Ausnahme des Compute Nodes c35. \\\hline
\cellcolor{heading}\textbf{Tester} & Christoph Amrein  \\\hline
\cellcolor{heading}\textbf{Datum des Tests} & 22.05.2018  \\\hline
\cellcolor{heading}\textbf{Testergebnis \newline (Fehlerklasse)} & 1 - Fehlerfrei, die Erwartungen sind erfüllt. \\\hline
\cellcolor{heading}\textbf{Fehlerbeschreibung} & Der Compute Node c35 wurde bereits bei den Komponententests als fehlerhaft erkannt. Deshalb ist das Ergebnis wie erwartet.  \\\hline
\end{tabular}
\caption{S-002 Protokoll}
\end{table}

\begin{table}[H]
\centering
\begin{tabular}{p{4.5cm}p{11.5cm}}
\hline
\cellcolor{heading}\textbf{Test-ID:} & \textbf{S-003} \\\hline
\cellcolor{heading}\textbf{Testobjekt} & Nodes / Nagios Alarmierung\\\hline
\cellcolor{heading}\textbf{Testschritte} & 
- Strom von Node c1 trennen. \newline
- 5 Minuten warten. \newline
- Email prüfen (christoph.amrein86@gmail.com) \\\hline
\cellcolor{heading}\textbf{Erwartetes Ergebnis} & Es trifft eine Alarmierungs-Email ein. \\\hline
\cellcolor{heading}\textbf{Tatsächliches Ergebnis} &
E-Mail ist eingetroffen. \newline
***** Nagios ***** \newline
Notification Type: PROBLEM \newline
Host: c1 \newline
State: DOWN \newline
Address: 192.168.1.11 \newline
Info: CRITICAL - Host Unreachable (192.168.1.11) \newline
Date/Time: Sat May 26 17:59:10 CEST 2018 \newline
 \\\hline
\cellcolor{heading}\textbf{Tester} & Christoph Amrein  \\\hline
\cellcolor{heading}\textbf{Datum des Tests} & 26.05.2018  \\\hline
\cellcolor{heading}\textbf{Testergebnis \newline (Fehlerklasse)} & 1 - Fehlerfrei, die Erwartungen sind erfüllt. \\\hline
\cellcolor{heading}\textbf{Fehlerbeschreibung} &\\\hline
\end{tabular}
\caption{S-003 Protokoll}
\end{table}

\begin{table}[H]
\centering
\begin{tabular}{p{4.5cm}p{11.5cm}}
\hline
\cellcolor{heading}\textbf{Test-ID:} & \textbf{S-004} \\\hline
\cellcolor{heading}\textbf{Testobjekt} & Nodes / Ganglia Monitoring\\\hline
\cellcolor{heading}\textbf{Testschritte} & 
- \url{http://nebula/ganglia} aufrufen. \newline
- Quelle \grqq  Nebula\grqq auswählen. \\\hline
\cellcolor{heading}\textbf{Erwartetes Ergebnis} & Es werden alle Nodes aufgelistet. \\\hline
\cellcolor{heading}\textbf{Tatsächliches Ergebnis} &
Alle Nodes sind aufgelistet, jedoch nicht wie gewünscht gemonitored. \\\hline
\cellcolor{heading}\textbf{Tester} & Christoph Amrein  \\\hline
\cellcolor{heading}\textbf{Datum des Tests} & 26.05.2018  \\\hline
\cellcolor{heading}\textbf{Testergebnis \newline (Fehlerklasse)} & 3 - Kleiner Mangel, muss innerhalb von 6 Monaten gelöst werden. \\\hline
\cellcolor{heading}\textbf{Fehlerbeschreibung} & Es wird nur jeweils ein Node mit aktuellen Aufzeichnungen dargestellt. Es sollten aber alle Nodes aktive Aufzeichnungen haben.\\\hline
\end{tabular}
\caption{S-004 Protokoll}
\end{table}

\begin{table}[H]
\centering
\begin{tabular}{p{4.5cm}p{11.5cm}}
\hline
\cellcolor{heading}\textbf{Test-ID:} & \textbf{S-005} \\\hline
\cellcolor{heading}\textbf{Testobjekt} & Nodes / Cluster Jobs\\\hline
\cellcolor{heading}\textbf{Testschritte} & 
Auf dem Management Node sind folgende Befehle abzusetzen: \newline
- \grqq mpicc -O3 /opt/ohpc/pub/examples/mpi/hello.c\grqq \newline
- \grqq srun -n 8 -N 2 --pty /bin/bash\grqq \newline
- \grqq squeue\grqq \\\hline
\cellcolor{heading}\textbf{Erwartetes Ergebnis} & Der Job wird in der Queue angezeigt und ist 2 zufälligen Compute Nodes zugewiesen. \\\hline
\cellcolor{heading}\textbf{Tatsächliches Ergebnis} &
Die Spalten wurden gekürzt. \newline
[root@nebula tmp]\# squeue \newline
JOBID \quad USER \quad ST \quad TIME \quad NODES \quad NODELIST(REASON) \newline 123  \qquad \ \ root \qquad R \quad \ 6:14 \quad \ \ 2 \qquad \qquad c[36-37]
 \\\hline
\cellcolor{heading}\textbf{Tester} & Christoph Amrein  \\\hline
\cellcolor{heading}\textbf{Datum des Tests} & 26.05.2018  \\\hline
\cellcolor{heading}\textbf{Testergebnis \newline (Fehlerklasse)} & 1 - Fehlerfrei, die Erwartungen sind erfüllt. \\\hline
\cellcolor{heading}\textbf{Fehlerbeschreibung} & \\\hline
\end{tabular}
\caption{S-005 Protokoll}
\end{table}

\begin{table}[H]
\centering
\begin{tabular}{p{4.5cm}p{11.5cm}}
\hline
\cellcolor{heading}\textbf{Test-ID:} & \textbf{S-006} \\\hline
\cellcolor{heading}\textbf{Testobjekt} & Nodes / Schürfen\\\hline
\cellcolor{heading}\textbf{Testschritte} & 
Auf dem Management Node sind folgende Befehle als root abzusetzen: \newline
- \grqq cd /opt/miners/tkinjo\grqq \newline
- \grqq srun --nodes=40-45 --ntasks=40 --cpus-per-task=4 ./cpuminer  -a cryptonight -o stratum+tcp://xdn.pool.minergate.com:45620 -u x -p x\grqq
\\\hline
\cellcolor{heading}\textbf{Erwartetes Ergebnis} & Es soll direkt in der Shell die Ausgabe des Miners ausgegeben werden. \\\hline
\cellcolor{heading}\textbf{Tatsächliches Ergebnis} &
Es erscheint die gewünsche Ausgabe, es sind 40 aktive Nodes zu sehen, welche am Schürfen sind.
 \\\hline
\cellcolor{heading}\textbf{Tester} & Christoph Amrein  \\\hline
\cellcolor{heading}\textbf{Datum des Tests} & 26.05.2018  \\\hline
\cellcolor{heading}\textbf{Testergebnis \newline (Fehlerklasse)} & 1 - Fehlerfrei, die Erwartungen sind erfüllt. \\\hline
\cellcolor{heading}\textbf{Fehlerbeschreibung} & \\\hline
\end{tabular}
\caption{S-006 Protokoll}
\end{table}


\begin{table}[H]
\centering
\begin{tabular}{p{4.5cm}p{11.5cm}}
\hline
\cellcolor{heading}\textbf{Test-ID:} & \textbf{S-007} \\\hline
\cellcolor{heading}\textbf{Testobjekt} & Wallet / Minergate\\\hline
\cellcolor{heading}\textbf{Testschritte} & 
- Auf \url{minergate.com} anmelden. \newline
- Auf Dashboard klicken. \newline
- Prüfen ob die Coins unter DigitalNote angezeigt werden.
\\\hline
\cellcolor{heading}\textbf{Erwartetes Ergebnis} & Es soll eine grössere Zahl als 0 unter DigitalNote erscheinen. \\\hline
\cellcolor{heading}\textbf{Tatsächliches Ergebnis} &
Es werden über 62 XDN angezeigt.
 \\\hline
\cellcolor{heading}\textbf{Tester} & Christoph Amrein  \\\hline
\cellcolor{heading}\textbf{Datum des Tests} & 26.05.2018  \\\hline
\cellcolor{heading}\textbf{Testergebnis \newline (Fehlerklasse)} & 1 - Fehlerfrei, die Erwartungen sind erfüllt. \\\hline
\cellcolor{heading}\textbf{Fehlerbeschreibung} & \\\hline
\end{tabular}
\caption{S-007 Protokoll}
\end{table}