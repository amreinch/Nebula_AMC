% !TEX root = ../Diplombericht.tex

\section{Installation Management Node}
Die Serverinstallation des Management Nodes wird nach dem offiziellen OpenHPC Guide mit einigen Abweichungen durchgeführt. Die Installation basiert auf der Anleitung CentOS 7.4 aarch64 Install guide with Warewulf + Slurm \footnote{\url{http://openhpc.community/downloads/}}. Die Installation wird hauptsächlich mit dem ROOT-Benutzer durchgeführt, falls nichts anderes erwähnt wird.

\subsection{Variablen Definition}
Durch die Installation hinweg werden die folgenden Variablen verwendet:

\begin{longtable}{| p{0.5cm} | p{3cm} | p{8.5cm} | p{4cm} |} 
\hline
\rowcolor{heading} \textbf{Nr.} & \textbf{Variable} & \textbf{Beschreibung} &\textbf{Werte} \\\hline
1 & sms\_name & Hostname des Managementhosts & nebula \\\hline
2 & sms\_ip & IP Adresse des Managementhosts & 192.168.1.10 \\\hline
3 & sms\_eth\_internal & Ethernet Interface & eth0 \\\hline
4 & ntp\_server[0] \newline ntp\_server[1] \newline ntp\_server[2] \newline ntp\_server[3] & Zeitserver (Array) & server 0.ch.pool.ntp.org \newline server 1.ch.pool.ntp.org \newline server 2.ch.pool.ntp.org \newline server 3.ch.pool.ntp.org  \\\hline
5 & num\_computes & Anzahl Compute Nodes & 45 \\\hline
6 &  c\_ip[0] \newline  c\_ip[1] \newline c\_ip[2] \newline c\_ip[3] \newline c\_ip[4] \newline c\_ip[5] \newline c\_ip[6] \newline c\_ip[7] \newline c\_ip[8] \newline c\_ip[9] \newline c\_ip[10] \newline c\_ip[11] \newline c\_ip[12] \newline c\_ip[13] \newline c\_ip[14] \newline c\_ip[15] \newline c\_ip[16] \newline c\_ip[17] \newline c\_ip[18] & IP Adressen der Compute Nodes (Array) & 192.168.1.11 \newline 192.168.1.12 \newline 192.168.1.13 \newline 192.168.1.14 \newline 192.168.1.15 \newline 192.168.1.16 \newline 192.168.1.17 \newline 192.168.1.18 \newline 192.168.1.19 \newline 192.168.1.20 \newline 192.168.1.21 \newline 192.168.1.22 \newline 192.168.1.23 \newline 192.168.1.24 \newline 192.168.1.25 \newline 192.168.1.26 \newline 192.168.1.27 \newline 192.168.1.28 \newline 192.168.1.29 \\\hline 
\rowcolor{heading} \textbf{Nr.} & \textbf{Variable} & \textbf{Beschreibung} &\textbf{Werte} \\\hline
6 & c\_ip[19] \newline c\_ip[20] \newline c\_ip[21] \newline c\_ip[22] \newline c\_ip[23] \newline c\_ip[24]  \newline c\_ip[25] \newline c\_ip[26] \newline c\_ip[27] \newline c\_ip[28] \newline c\_ip[29]  \newline c\_ip[30] \newline c\_ip[31] \newline c\_ip[32] \newline c\_ip[33] \newline c\_ip[34] \newline c\_ip[35] \newline c\_ip[36] \newline c\_ip[37] \newline c\_ip[38] \newline c\_ip[39] \newline c\_ip[40] \newline c\_ip[41] \newline c\_ip[42] \newline c\_ip[43] \newline c\_ip[44] & IP Adressen der Compute Nodes (Array) &  192.168.1.30 \newline 192.168.1.31 \newline 192.168.1.32 \newline  192.168.1.33 \newline 192.168.1.34 \newline 192.168.1.35 \newline 192.168.1.36 \newline 192.168.1.37 \newline 192.168.1.38 \newline 192.168.1.39 \newline 192.168.1.40 \newline 192.168.1.41 \newline 192.168.1.42 \newline 192.168.1.43 \newline 192.168.1.44 \newline 192.168.1.45 \newline 192.168.1.46 \newline 192.168.1.47 \newline 192.168.1.48 \newline 192.168.1.49 \newline 192.168.1.50 \newline 192.168.1.51 \newline 192.168.1.52 \newline 192.168.1.53 \newline 192.168.1.54 \newline 192.168.1.55 \\\hline 
7 &  c\_name[0] \newline  c\_name[1] \newline c\_name[2] \newline c\_name[3] \newline c\_name[4] \newline c\_name[5] \newline c\_name[6] \newline c\_name[7] \newline c\_name[8] \newline c\_name[9] \newline c\_name[10] \newline c\_name[11] \newline c\_name[12]  & Hostnamen der Compute Nodes (Array) & c1 \newline c2 \newline c3 \newline c4 \newline c5 \newline c6 \newline c7 \newline c8 \newline c9 \newline c10 \newline c11 \newline c12 \newline c13 \\\hline 
\rowcolor{heading} \textbf{Nr.} & \textbf{Variable} & \textbf{Beschreibung} &\textbf{Werte} \\\hline
7 & c\_name[13] \newline c\_name[14] \newline c\_name[15] \newline c\_name[16] \newline c\_name[17] \newline c\_name[18] \newline c\_name[19] \newline c\_name[20] \newline c\_name[21] \newline c\_name[22] \newline c\_name[23] \newline c\_name[24] \newline c\_name[25] \newline c\_name[26] \newline c\_name[27] c\_name[28] \newline  c\_name[29] \newline c\_name[30] \newline c\_name[31] \newline c\_name[32] \newline c\_name[33] \newline c\_name[34] \newline c\_name[35] \newline c\_name[36] \newline c\_name[37] \newline c\_name[38] \newline c\_name[39] \newline c\_name[40] \newline c\_name[41] \newline c\_name[42] \newline c\_name[43] \newline c\_name[44] & Hostnamen der Compute Nodes (Array) &   c14 \newline c15 \newline c16 \newline c17 \newline c18 \newline c19 \newline c20 \newline c21 \newline c22 \newline c23 \newline c24 \newline c25 \newline c26 \newline c27 \newline c28 \newline c29 \newline c30 \newline c31 \newline c32 \newline c33 \newline c34 \newline c35 \newline c36 \newline c37 \newline c38 \newline c39 \newline c40 \newline c41 \newline c42 \newline c43 \newline c44 \newline c45 \newline \newline \newline \newline \newline \newline \newline \\\hline 
\rowcolor{heading} \textbf{Nr.} & \textbf{Variable} & \textbf{Beschreibung} &\textbf{Werte} \\\hline
8 & c\_mac[0] \newline  c\_mac[1] \newline c\_mac[2] \newline c\_mac[3] \newline c\_mac[4] \newline c\_mac[5] \newline c\_mac[6] \newline c\_mac[7] \newline c\_mac[8] \newline c\_mac[9] \newline  c\_mac[10] \newline c\_mac[11] \newline c\_mac[12] \newline c\_mac[13] \newline c\_mac[14] \newline c\_mac[15] \newline c\_mac[16] \newline c\_mac[17] \newline c\_mac[18] \newline c\_mac[19] \newline c\_mac[20] \newline c\_mac[21] \newline c\_mac[22] \newline c\_mac[23] \newline c\_mac[24] \newline c\_mac[25] \newline c\_mac[26] \newline c\_mac[27] \newline c\_mac[28] \newline c\_mac[29] \newline c\_mac[30] \newline c\_mac[31] \newline c\_mac[32] \newline c\_mac[33] \newline c\_mac[34] \newline c\_mac[35] \newline c\_mac[36] \newline c\_mac[37] \newline c\_mac[38] & MAC Adressen der Compute Nodes (Array) & B8:27:EB:32:39:A7 \newline B8:27:EB:2E:A3:D1 \newline B8:27:EB:50:45:3F \newline B8:27:EB:0D:E6:25 \newline B8:27:EB:3E:96:B5 \newline B8:27:EB:EE:77:DA \newline B8:27:EB:21:63:E6 \newline B8:27:EB:2E:2E:CC \newline B8:27:EB:17:32:96 \newline B8:27:EB:B2:1C:A9 \newline B8:27:EB:AF:63:1F \newline B8:27:EB:43:00:2C \newline B8:27:EB:13:7B:18 \newline B8:27:EB:43:CD:29 \newline B8:27:EB:FF:C7:56 \newline B8:27:EB:CE:98:66 \newline B8:27:EB:5D:63:34 \newline B8:27:EB:91:3E:0F \newline B8:27:EB:F4:65:EC \newline B8:27:EB:3E:AB:DC \newline B8:27:EB:66:60:F6 \newline B8:27:EB:37:3F:74 \newline B8:27:EB:18:5E:F0 \newline B8:27:EB:B0:23:B8 \newline B8:27:EB:BE:C4:94 \newline B8:27:EB:FB:FF:57 \newline B8:27:EB:4E:EC:CE \newline 	B8:27:EB:43:1C:35 \newline B8:27:EB:DC:74:5F \newline B8:27:EB:D1:DE:2F \newline B8:27:EB:5E:90:34 \newline B8:27:EB:DE:80:24 \newline B8:27:EB:A4:79:6F \newline B8:27:EB:0A:4D:C7 \newline B8:27:EB:5C:53:5F \newline B8:27:EB:F7:AF:C2 \newline B8:27:EB:CE:BA:ED \newline B8:27:EB:59:38:3C \newline B8:27:EB:99:BB:8E \\\hline
\rowcolor{heading} \textbf{Nr.} & \textbf{Variable} & \textbf{Beschreibung} &\textbf{Werte} \\\hline
8 & c\_mac[39] \newline c\_mac[40] \newline c\_mac[41] \newline c\_mac[42] \newline c\_mac[43] \newline c\_mac[44] & MAC Adressen der Compute Nodes (Array) &  B8:27:EB:8F:7A:0D \newline B8:27:EB:DE:C9:69 \newline B8:27:EB:7E:6F:48 \newline B8:27:EB:5D:DD:FE
\newline B8:27:EB:A6:6D:4D \newline B8:27:EB:0C:63:10\\\hline 
\caption{Variablen Definition}
\end{longtable}

\subsection{Basiskonfiguration}
Dem Management Node muss ein eindeutiger Hostname zugewiesen werden. Hierbei werden die untenstehenden Kommandos abgesetzt:

\begin{lstlisting}
echo ${sms_ip} ${sms_name} >> /etc/hosts
hostnamectl set-hostname ${sms_name}
\end{lstlisting}

Für die folgenden Arbeiten müssen die Firewall und SELinux deaktiviert werden. DHCP wird aufgrund der Verwendung von dnsmasq deaktivert.

\begin{lstlisting}
systemctl disable firewalld
systemctl stop firewalld
systemctl disable dhcpd
systemctl stop dhcpd
echo 0 > /selinux/enforce
\end{lstlisting}

\subsection{OpenHPC Komponenten installieren}
Es wird das OpenHPC Repository für die Installation der OpenHPC Komponenten benötigt. Dieses muss installiert werden:

\begin{lstlisting}
yum install http://build.openhpc.community/OpenHPC:/1.3/CentOS_7/aarch64/ohpc-release-1.3-1.el7.aarch64.rpm
\end{lstlisting}

Von nun an können die benötigten Pakete über den Red Hat Package Manager (RPM-Paketmanager) installiert werden.

Durch das neu angehängte Repository können nun die folgenden Pakete installiert werden:

\begin{lstlisting}
yum -y install ohpc-base
yum -y install ohpc-warewulf
\end{lstlisting}

Weiterhin spielt die Zeit eine wichtige Rolle zwischen der Kommunikation des Managementhosts und der Compute Nodes. Dazu müssen die bereits vorhandenen Einträge der Zeitserver in der ntp.conf entfernt werden.


\begin{lstlisting}
systemctl disable chronyd.service
systemctl stop chronyd.service
systemctl enable ntpd.service
sed -i '/^server/ d' /etc/ntp.conf
echo "server ${ntp_server[0]}" >> /etc/ntp.conf
echo "server ${ntp_server[1]}" >> /etc/ntp.conf
echo "server ${ntp_server[2]}" >> /etc/ntp.conf
echo "server ${ntp_server[3]}" >> /etc/ntp.conf
systemctl restart ntpd
\end{lstlisting}

Nun wird der Slurmcontroller installiert. Dieser dient dazu, die Jobs auf die Compute Nodes zu verteilen. Dieser steuert die Jobverwaltung und muss installiert und eingerichtet werden. Hierbei wird die Konfiguration für 45 Compute Nodes vorgenommen.

\begin{lstlisting}
yum -y install ohpc-slurm-server
perl -pi -e "s/ControlMachine=\S+/ControlMachine=${sms_name}/" /etc/slurm/slurm.conf
sed -i '/^NodeName/ d' /etc/slurm/slurm.conf
sed -i '/^Partition/ d' /etc/slurm/slurm.conf
echo "NodeName=c[1-45] CoresPerSocket=1 ThreadsPerCore=1 SocketsPerBoard=4 State=UNKNOWN" >> /etc/slurm/slurm.conf
echo "PartitionName=normal Nodes=c[1-45] Default=YES MaxTime=INFINITE State=UP" >> /etc/slurm/slurm.conf
\end{lstlisting}

\subsection{Netzwerkboot einrichten}
Das Dateisystem und die Boot-Partition werden in zwei verschiedenen Verzeichnissen erstellt und abgelegt.

Definieren des Dateisystem-Verzeichnisses und das Centos7.4 Basis-Dateisystem erstellen.
\begin{lstlisting}
export CHROOT=/opt/ohpc/admin/images/centos7.4
wwmkchroot centos-7 $CHROOT
\end{lstlisting}

Anstelle von DHCPD muss der networking Service verwendet werden. Deshalb wird dieser aktiviert und gestartet.
\begin{lstlisting}
systemctl enable networking
systemctl start networking
\end{lstlisting}

Nun wird das Paket dnsmaq installiert. Dadurch wird die Verteilung des Betriebssystems an eine gewisse IP Range und an ein Subnetz gewährleistet. Geräte ausserhalb der Range und des Subnetzes sollen das Betriebssystem nicht erhalten dürfen.

\begin{lstlisting}
yum -y install dnsmasq
\end{lstlisting}

Nach der Installation kann die Konfiguration von dnsmasq vorgenommen werden. Dazu werden die Datei /etc/dnsmasq.conf angepasst und die Datei /etc/dnsmasq.d/host.allow erstellt.

In der dnsmasq.conf Datei wird das Konfigurationsverzeichnis für die erlaubten Hosts erstellt, zudem wird eine IP Range angegeben, alle Geräte innerhalb dieser Range dürfen den Kernel des Betriebssystems aus dem Verzeichnis /tftboot erhalten.  
\begin{lstlisting}
[root@nebula etc]# cat /etc/dnsmasq.conf
conf-dir=/etc/dnsmasq.d
port=0
dhcp-range=192.168.1.10,192.168.1.50,12h
log-dhcp
enable-tftp
tftp-root=/tftpboot
pxe-service=0,"Raspberry Pi Boot"
\end{lstlisting}

Falls sich andere Geräte in dieser IP Range befinden, wird dies durch die /etc/dnsmasq.d/host.allow Datei abgefangen. Dort sind alle Compute Nodes mit MAC Adresse, Hostnamen und IP Adresse hinterlegt.

Die host.allow Datei kann in einer for Schlaufe mit den definierten Variablen einfach abgefüllt werden.

\begin{lstlisting}
for ((i=0; i<45; i++)); do echo "dhcp-host=${c_mac[$i]};${c_name[$i]};${c_ip[$i]}" >> /etc/dnsmasq.d/host.allow; done
\end{lstlisting}

Das Ergebnis muss wie folgt aussehen:
\begin{lstlisting}
[root@nebula etc]# cat /etc/dnsmasq.d/host.allow
dhcp-host=B8:27:EB:32:39:A7,c1,192.168.1.11
dhcp-host=B8:27:EB:2E:A3:D1,c2,192.168.1.12
dhcp-host=B8:27:EB:50:45:3F,c3,192.168.1.13
dhcp-host=B8:27:EB:0D:E6:25,c4,192.168.1.14
dhcp-host=B8:27:EB:3E:96:B5,c5,192.168.1.15
dhcp-host=B8:27:EB:EE:77:DA,c6,192.168.1.16
dhcp-host=B8:27:EB:21:63:E6,c7,192.168.1.17
dhcp-host=B8:27:EB:2E:2E:CC,c8,192.168.1.18
dhcp-host=B8:27:EB:17:32:96,c9,192.168.1.19
dhcp-host=B8:27:EB:B2:1C:A9,c10,192.168.1.20
dhcp-host=B8:27:EB:AF:63:1F,c11,192.168.1.21
dhcp-host=B8:27:EB:43:00:2C,c12,192.168.1.22
dhcp-host=B8:27:EB:13:7B:18,c13,192.168.1.23
dhcp-host=B8:27:EB:43:CD:29,c14,192.168.1.24
dhcp-host=B8:27:EB:FF:C7:56,c15,192.168.1.25
dhcp-host=B8:27:EB:CE:98:66,c16,192.168.1.26
dhcp-host=B8:27:EB:5D:63:34,c17,192.168.1.27
dhcp-host=B8:27:EB:91:3E:0F,c18,192.168.1.28
dhcp-host=B8:27:EB:F4:65:EC,c19,192.168.1.29
dhcp-host=B8:27:EB:3E:AB:DC,c20,192.168.1.30
dhcp-host=B8:27:EB:66:60:F6,c21,192.168.1.31
dhcp-host=B8:27:EB:37:3F:74,c22,192.168.1.32
dhcp-host=B8:27:EB:18:5E:F0,c23,192.168.1.33
dhcp-host=B8:27:EB:B0:23:B8,c24,192.168.1.34
dhcp-host=B8:27:EB:BE:C4:94,c25,192.168.1.35
dhcp-host=B8:27:EB:FB:FF:57,c26,192.168.1.36
dhcp-host=B8:27:EB:4E:EC:CE,c27,192.168.1.37
dhcp-host=B8:27:EB:43:1C:35,c28,192.168.1.38
dhcp-host=B8:27:EB:DC:74:5F,c29,192.168.1.39
dhcp-host=B8:27:EB:D1:DE:2F,c30,192.168.1.40
dhcp-host=B8:27:EB:5E:90:34,c31,192.168.1.41
dhcp-host=B8:27:EB:DE:80:24,c32,192.168.1.42
dhcp-host=B8:27:EB:A4:79:6F,c33,192.168.1.43
dhcp-host=B8:27:EB:0A:4D:C7,c34,192.168.1.44
dhcp-host=B8:27:EB:5C:53:5F,c35,192.168.1.45
dhcp-host=B8:27:EB:F7:AF:C2,c36,192.168.1.46
dhcp-host=B8:27:EB:CE:BA:ED,c37,192.168.1.47
dhcp-host=B8:27:EB:59:38:3C,c38,192.168.1.48
dhcp-host=B8:27:EB:99:BB:8E,c39,192.168.1.49
dhcp-host=B8:27:EB:8F:7A:0D,c40,192.168.1.50
dhcp-host=B8:27:EB:DE:C9:69,c41,192.168.1.51
dhcp-host=B8:27:EB:7E:6F:48,c42,192.168.1.52
dhcp-host=B8:27:EB:5D:DD:FE,c43,192.168.1.53
dhcp-host=B8:27:EB:A6:6D:4D,c44,192.168.1.54
dhcp-host=B8:27:EB:0C:63:10,c45,192.168.1.55
\end{lstlisting}

Boot-Verzeichnis erstellen. Aus diesem Verzeichnis wird der Kernel den Compute Nodes angeboten und übermittelt. Dabei wird das /boot Verzeichnis des Management Nodes in dieses Verzeichnis kopiert, da es sich um das selbe Betriebssystem handelt.

\begin{lstlisting}
[root@nebula /]# mkdir /tftpboot
[root@nebula /]# chmod 777 /tftpboot
[root@nebula /]# cp -r /boot /tftboot
\end{lstlisting}

Dabei muss darauf geachtet werden, dass die cmdline.txt Datei im /tftboot Verzeichnis folgenden Eintrag erhält. Dieser sagt aus, dass der Compute Node sein /root Verzeichnis vom Management Node unter dem Pfad /opt/ohpc/admin/images/centos7.4 beziehen soll.

\begin{lstlisting}
[root@nebula etc]# cat /tftpboot/cmdline.txt
dwc_otg.lpm_enable=0 root=/dev/nfs nfsroot=192.168.1.10:/opt/ohpc/admin/images/centos7.4,vers=3 rw ip=dhcp rootwait elevator=deadline
\end{lstlisting}

Zudem wird die config.txt angepasst. Hierbei wird die Taktrate der CPU auf 1400 MHz hochgetaktet. Dadurch steigt die Mining Performance von einer Hashrate 1.6 H/s auf 1.9 H/s.

\begin{lstlisting}
[root@nebula /]# echo "arm_freq=1400" > /tftpboot/config.txt
\end{lstlisting}


Den dnsmaq Service aktivieren und starten.
\begin{lstlisting}
[root@nebula /]# systemctl enable dnsmasq.service
[root@nebula /]# systemctl restart dnsmasq.service
\end{lstlisting}

OpenHPC Pakete für die Compute Nodes installieren.
\begin{lstlisting}
[root@nebula /]# yum -y --installroot=$CHROOT install ohpc-base-compute
\end{lstlisting}

Damit die Compute Nodes über den Hostnamen angesprochen werden können, wird eine Domain Name System (DNS) Konfiguration benötigt. Die Auflösung findet über folgende Einstellung statt. Dabei wird vom Management Node die resolv.conf Datei übernommen und auf das Dateisystem der Compute Nodes kopiert:
\begin{lstlisting}
[root@nebula /]# cp -p /etc/resolv.conf $CHROOT/etc/resolv.conf
\end{lstlisting}

Die zusätzlichen Pakete werden für die Kommunikation mit dem Slurmcontroller und der Zeitsynchronisierung benötigt.

\begin{lstlisting}
[root@nebula /]# yum -y --installroot=$CHROOT install ohpc-slurm-client
[root@nebula /]# yum -y --installroot=$CHROOT install ntp
\end{lstlisting}

Beim Booten der Compute Nodes muss das Dateisystem gemountet werden. Dafür werden fstab Einträge erstellt. Ausserdem wird auf der letzten Zeile der gemeinsame Netzwerkshare mit NFS angehängt.

\begin{lstlisting}
[root@nebula /]# echo "${sms_ip}:/home /home nfs nfsvers=3,nodev,nosuid,noatime 0 0" >> $CHROOT/etc/fstab
[root@nebula /]# echo "${sms_ip}:/opt/ohpc/pub /opt/ohpc/pub nfs nfsvers=3,nodev,noatime 0 0" >> $CHROOT/etc/fstab
[root@nebula /]# echo "192.168.1.129:/volume1/nebula /media/nebula_data nfs" >> $CHROOT/etc/fstab
\end{lstlisting}

Zudem müssen gewisse Verzeichnisse vom Management Node aus exportiert werden.

\begin{lstlisting}
[root@nebula /]# echo "/home *(rw,no_subtree_check,fsid=10,no_root_squash)" >> /etc/exports
[root@nebula /]# echo "echo "/opt/ohpc/pub *(ro,no_subtree_check,fsid=11)" >> /etc/exports
[root@nebula /]# echo "/opt/ohpc/admin/images/centos7.4 *(rw,sync,no_subtree_check,no_root_squash)" /etc/exports
[root@nebula /]# exportfs -a
[root@nebula /]# systemctl restart nfs-server
[root@nebula /]# systemctl enable nfs-server
\end{lstlisting}

Damit die Zeitsynchronisierung über den Management Node läuft, wird folgendes umgesetzt:

\begin{lstlisting}
[root@nebula /]# chroot $CHROOT systemctl enable ntpd
[root@nebula /]# echo "server ${sms_ip}" >> $CHROOT/etc/ntp.conf
\end{lstlisting}

Die Compute Nodes können nun mit Strom versorgt werden. Dabei wird auf dem Management Node auf die /var/log/messages Logdatei geachtet. Dort sind alle Einträge des Netzwerkboots zusammengeschlossen. Dabei müssen folgende Einträge erscheinen:

\begin{lstlisting}
May 17 20:31:34 nebula dnsmasq-dhcp[382]: 653460281 next server: 192.168.1.10
May 17 20:31:34 nebula dnsmasq-dhcp[382]: 653460281 sent size:  1 option: 53 message-type  2
May 17 20:31:34 nebula dnsmasq-dhcp[382]: 653460281 sent size:  4 option: 54 server-identifier  192.168.1.10
May 17 20:31:34 nebula dnsmasq-dhcp[382]: 653460281 sent size:  4 option: 51 lease-time  12h
May 17 20:31:34 nebula dnsmasq-dhcp[382]: 653460281 sent size:  4 option: 58 T1  6h
May 17 20:31:34 nebula dnsmasq-dhcp[382]: 653460281 sent size:  4 option: 59 T2  10h30m
May 17 20:31:34 nebula dnsmasq-dhcp[382]: 653460281 sent size:  4 option:  1 netmask  255.255.255.0
May 17 20:31:34 nebula dnsmasq-dhcp[382]: 653460281 sent size:  4 option: 28 broadcast  192.168.1.255
May 17 20:31:34 nebula dnsmasq-dhcp[382]: 653460281 sent size:  9 option: 60 vendor-class  50:58:45:43:6c:69:65:6e:74
May 17 20:31:34 nebula dnsmasq-dhcp[382]: 653460281 sent size: 17 option: 97 client-machine-id  00:44:44:44:44:44:44:44:44:44:44:44:44:44...
May 17 20:31:34 nebula dnsmasq-dhcp[382]: 653460281 sent size: 32 option: 43 vendor-encap  06:01:03:0a:04:00:50:58:45:09:14:00:00:11...
May 17 20:31:34 nebula dnsmasq-dhcp[382]: 653460281 available DHCP range: 192.168.1.10 -- 192.168.1.50
May 17 20:31:34 nebula dnsmasq-dhcp[382]: 653460281 vendor class: PXEClient:Arch:00000:UNDI:002001
..
..
..
May 17 20:31:35 nebula dnsmasq-tftp[382]: sent /tftpboot/bootcode.bin to 192.168.1.46
May 17 20:31:35 nebula dnsmasq-tftp[382]: file /tftpboot/autoboot.txt not found
May 17 20:31:35 nebula dnsmasq-tftp[382]: sent /tftpboot/config.txt to 192.168.1.38
May 17 20:31:35 nebula dnsmasq-tftp[382]: file /tftpboot/recovery.elf not found
May 17 20:31:35 nebula dnsmasq-tftp[382]: file /tftpboot/beb21ca9/start.elf not found
May 17 20:31:35 nebula dnsmasq-tftp[382]: file /tftpboot/75a66d4d/start.elf not found
May 17 20:31:35 nebula dnsmasq-tftp[382]: file /tftpboot/d0185ef0/start.elf not found
May 17 20:31:35 nebula dnsmasq-tftp[382]: file /tftpboot/10dec969/start.elf not found
May 17 20:31:35 nebula dnsmasq-tftp[382]: file /tftpboot/d4373f74/start.elf not found
May 17 20:31:35 nebula dnsmasq-tftp[382]: file /tftpboot/0643cd29/start.elf not found
May 17 20:31:35 nebula dnsmasq-tftp[382]: file /tftpboot/bootsig.bin not found
May 17 20:31:35 nebula dnsmasq-tftp[382]: sent /tftpboot/bootcode.bin to 192.168.1.40
May 17 20:31:35 nebula dnsmasq-tftp[382]: file /tftpboot/bootsig.bin not found
May 17 20:31:35 nebula dnsmasq-tftp[382]: sent /tftpboot/bootcode.bin to 192.168.1.16
May 17 20:31:35 nebula dnsmasq-tftp[382]: file /tftpboot/bootsig.bin not found
May 17 20:31:35 nebula dnsmasq-tftp[382]: sent /tftpboot/bootcode.bin to 192.168.1.27
..
..
..
May 17 20:32:56 nebula dnsmasq-dhcp[382]: 925035958 next server: 192.168.1.10
May 17 20:32:56 nebula dnsmasq-dhcp[382]: 925035958 sent size:  1 option: 53 message-type  2
May 17 20:32:56 nebula dnsmasq-dhcp[382]: 925035958 sent size:  4 option: 54 server-identifier  192.168.1.10
May 17 20:32:56 nebula dnsmasq-dhcp[382]: 925035958 sent size:  4 option: 51 lease-time  12h
May 17 20:32:56 nebula dnsmasq-dhcp[382]: 925035958 sent size:  4 option: 58 T1  6h
May 17 20:32:56 nebula dnsmasq-dhcp[382]: 925035958 sent size:  4 option: 59 T2  10h30m
May 17 20:32:56 nebula dnsmasq-dhcp[382]: 925035958 sent size:  4 option:  1 netmask  255.255.255.0
May 17 20:32:56 nebula dnsmasq-dhcp[382]: 925035958 sent size:  4 option: 28 broadcast  192.168.1.255
May 17 20:32:56 nebula dnsmasq-dhcp[382]: 925035958 sent size:  4 option:  3 router  192.168.1.10
May 17 20:32:56 nebula dnsmasq-dhcp[382]: 925035958 sent size:  3 option: 12 hostname  c12
May 17 20:32:56 nebula dnsmasq-dhcp[382]: 925035958 available DHCP range: 192.168.1.10 -- 192.168.1.50
May 17 20:32:56 nebula rpc.mountd[495]: authenticated mount request from 192.168.1.22:999 for /opt/ohpc/admin/images/centos7.4 (/opt/ohpc/admin/images/centos7.4)
\end{lstlisting}

Durch die letzte Zeile (authenticated mount request from 192.168.1.22) ist ersichtlich, dass der Compute Node c12 gestartet wurde und nach dem Betriebssystem angefragt hat und dieses mounten will.

\subsection{Netzwerkshare einbinden}

Der Netzwerkshare wird über NFS auf dem Management Node eingebunden. Dies wird mit einem Eintrag in /etc/fstab erledigt. Dadurch ist bei einem Neustart der Netzwerkshare automatisch wieder verbunden.
\begin{lstlisting}
[root@nebula /]# echo "192.168.1.129:/volume1/nebula /media/nebula_data nfs" >> /etc/fstab
\end{lstlisting}

\subsection{Registration Compute Nodes}

Die Basisinstallation ist soweit abgeschlossen. Nun werden die Compute Nodes registriert. Ab diesem Zeitpunkt kennt der Management Node jeden Compute Node. Die Compute Nodes werden wie folgt in den Datastore aufgenommen:

\begin{lstlisting}
[root@nebula /]# echo "GATEWAYDEV=${eth_internal}" > /tmp/network.$$
[root@nebula /]# wwsh -y file import /tmp/network.$$ --name network
[root@nebula /]# wwsh -y file set network --path /etc/sysconfig/network --mode=0644 --uid=0
[root@nebula /]# for ((i=0; i<$num_computes; i++)) ; do
[root@nebula /]# wwsh -y node new ${c_name[i]} --ipaddr=${c_ip[i]} --hwaddr=${c_mac[i]} -D ${eth_internal}
done
\end{lstlisting}
\newpage
\subsection{Monitoring installieren}

Das Nagios Monitoring wird wie folgt installiert. Dabei werden alle Grundüberwachungen installiert und können Out of the Box verwendet werden:

\begin{lstlisting}
[root@nebula /]# yum -y install ohpc-nagios
[root@nebula /]# yum -y --installroot=$CHROOT install nagios-plugins-all-ohpc nrpe-ohpc
[root@nebula /]# chroot $CHROOT systemctl enable nrpe
[root@nebula /]# perl -pi -e "s/^allowed_hosts=/# allowed_hosts=/" $CHROOT/etc/nagios/nrpe.cfg
[root@nebula /]# echo "nrpe 5666/tcp # NRPE" >> $CHROOT/etc/services
[root@nebula /]# echo "nrpe : ${sms_ip} : ALLOW" >> $CHROOT/etc/hosts.allow
[root@nebula /]# echo "nrpe : ALL : DENY" >> $CHROOT/etc/hosts.allow
[root@nebula /]# chroot $CHROOT /usr/sbin/useradd -c "NRPE user for the NRPE service" -d /var/run/nrpe -r -g nrpe -s /sbin/nologin nrpe
[root@nebula /]# chroot $CHROOT /usr/sbin/groupadd -r nrpe
[root@nebula /]# mv /etc/nagios/conf.d/services.cfg.example /etc/nagios/conf.d/services.cfg
[root@nebula /]# mv /etc/nagios/conf.d/hosts.cfg.example /etc/nagios/conf.d/hosts.cfg
[root@nebula /]# for ((i=0; i<$num_computes; i++)) ; do
perl -pi -e "s/HOSTNAME$(($i+1))/${c_name[$i]}/ || s/HOST$(($i+1))_IP/${c_ip[$i]}/" \
[root@nebula /]# /etc/nagios/conf.d/hosts.cfg
done
[root@nebula /]# perl -pi -e "s/ \/bin\/mail/ \/usr\/bin\/mailx/g" /etc/nagios/objects/commands.cfg
[root@nebula /]# perl -pi -e "s/nagios\@localhost/root\@${sms_name}/" /etc/nagios/objects/contacts.cfg
[root@nebula /]# echo command[check_ssh]=/usr/lib64/nagios/plugins/check_ssh localhost >> $CHROOT/etc/nagios/nrpe.cfg
[root@nebula /]# htpasswd -bc /etc/nagios/passwd nagiosadmin ${Eigenes_Passwort}
[root@nebula /]# chkconfig nagios on
[root@nebula /]# systemctl start nagios
[root@nebula /]# chmod u+s `which ping
\end{lstlisting}

Nagios kann nun eingesetzt werden und ist über http://nebula/nagios erreichbar.

Das Performance Monitoring wird mit Ganglia realisiert, welches sich wie folgt installieren lässt:

\begin{lstlisting}
[root@nebula /]# yum -y install ohpc-ganglia
[root@nebula /]# yum -y --installroot=$CHROOT install ganglia-gmond-ohpc
[root@nebula /]# cp /opt/ohpc/pub/examples/ganglia/gmond.conf /etc/ganglia/gmond.conf
[root@nebula /]# perl -pi -e "s/<sms>/${sms_name}/" /etc/ganglia/gmond.conf
[root@nebula /]# cp /etc/ganglia/gmond.conf $CHROOT/etc/ganglia/gmond.conf
echo "gridname MySite" >> /etc/ganglia/gmetad.conf
[root@nebula /]# systemctl enable gmond
[root@nebula /]# systemctl enable gmetad
[root@nebula /]# systemctl start gmond
[root@nebula /]# systemctl start gmetad
[root@nebula /]# chroot $CHROOT systemctl enable gmond
[root@nebula /]# systemctl try-restart httpd
\end{lstlisting}

Ganglia kann nun eingesetzt werden und ist über http://nebula/ganglia erreichbar.

\subsection{Miner installieren}
Es wird die cpuminer-multi Version von tkinjo verwendet. Diese ist mit dem ARMv8 Prozessor kompatibel und der Miner kann kompiliert werden.

Die folgenden Schritte sind für die Installation notwendig:

Verzeichnis erstellen, in dem der Miner installiert werden soll.
\begin{lstlisting}
[root@nebula /]# mkdir  -p /opt/miners
\end{lstlisting}

Das tkinjo cpuminer-multi Repository klonen. Dabei wird das /opt/miners/tkinjo Verzeichnis erstellt.
\begin{lstlisting}
[root@nebula /]# cd /opt/miners
[root@nebula miners]# git clone https://github.com/tkinjo1985/cpuminer-multi.git tkinjo
Klone nach 'tkinjo'...
remote: Counting objects: 3805, done.
remote: Total 3805 (delta 0), reused 0 (delta 0), pack-reused 3805
Empfange Objekte: 100% (3805/3805), 18.98 MiB | 3.77 MiB/s, done.
Loese Unterschiede auf: 100% (2589/2589), done.
\end{lstlisting}

Die geklonte Version wurde noch modifiziert, so dass die jeweiligen Hostnamen beim Schürfen angegeben werden. Dazu wird die /opt/miners/tkinjo/cpu-miner.c wie folgt angepasst. Dies kann mit git apply cpu-miner.c.diff installiert werden:
\begin{lstlisting}
cat  /opt/miners/tkinjo/cpu-miner.c.diff
diff --git a/cpu-miner.c b/cpu-miner.c
index 0620d15..0d8c5e2 100644
--- a/cpu-miner.c
+++ b/cpu-miner.c
@@ -281,6 +281,11 @@ char *opt_api_allow = NULL;
 int opt_api_remote = 0;
 int opt_api_listen = 4048; /* 0 to disable */

+#ifndef MAX_HOST_LEN
+#define MAX_HOST_LEN 0xff
+char local_hostname[MAX_HOST_LEN];
+#endif
+
 #ifdef HAVE_GETOPT_LONG
 #include <getopt.h>
 #else
@@ -1011,6 +1016,7 @@ out:
 #define YAY "yay!!!"
 #define BOO "booooo"

+
 static int share_result(int result, struct work *work, const char *reason)
 {
        const char *flag;
@@ -1052,14 +1058,14 @@ static int share_result(int result, struct work *work, const char *reason)
        case ALGO_PLUCK:
        case ALGO_SCRYPTJANE:
                sprintf(s, hashrate >= 1e6 ? "%.0f" : "%.2f", hashrate);
-               applog(LOG_NOTICE, "accepted: %lu/%lu (%s), %s H/s %s",
-                       accepted_count, accepted_count + rejected_count,
+               applog(LOG_NOTICE, "[" CL_LBL "%s" CL_N "] accepted: %lu/%lu (%s), %s H/s %s",
+                       local_hostname, accepted_count, accepted_count + rejected_count,
                        suppl, s, flag);
                break;
        default:
                sprintf(s, hashrate >= 1e6 ? "%.0f" : "%.2f", hashrate / 1000.0);
-               applog(LOG_NOTICE, "accepted: %lu/%lu (%s), %s kH/s %s",
-                       accepted_count, accepted_count + rejected_count,
+               applog(LOG_NOTICE, "[" CL_LBL "%s" CL_N "] accepted: %lu/%lu (%s), %s kH/s %s",
+                       local_hostname, accepted_count, accepted_count + rejected_count,
                        suppl, s, flag);
                break;
        }
@@ -2388,12 +2394,12 @@ static void *miner_thread(void *userdata)
                        case ALGO_CRYPTONIGHT:
                        case ALGO_PLUCK:
                        case ALGO_SCRYPTJANE:
-                               applog(LOG_INFO, "CPU #%d: %.2f H/s", thr_id, thr_hashrates[thr_id]);
+                               applog(LOG_INFO, "[" CL_LBL "%s" CL_N "] CPU #%d: %.2f H/s", local_hostname, thr_id, thr_hashrates[thr_id]);
                                break;
                        default:
                                sprintf(s, thr_hashrates[thr_id] >= 1e6 ? "%.0f" : "%.2f",
                                                thr_hashrates[thr_id] / 1e3);
-                               applog(LOG_INFO, "CPU #%d: %s kH/s", thr_id, s);
+                               applog(LOG_INFO, "[" CL_LBL "%s" CL_N "] CPU #%d: %s kH/s", local_hostname, thr_id, s);
                                break;
                        }
                        tm_rate_log = time(NULL);
@@ -2409,11 +2415,11 @@ static void *miner_thread(void *userdata)
                                case ALGO_AXIOM:
                                case ALGO_SCRYPTJANE:
                                        sprintf(s, "%.3f", hashrate);
-                                       applog(LOG_NOTICE, "Total: %s H/s", s);
+                                       applog(LOG_NOTICE, "[" CL_LBL "%s" CL_N "] Total: %s H/s", local_hostname, s);
                                        break;
                                default:
                                        sprintf(s, hashrate >= 1e6 ? "%.0f" : "%.2f", hashrate / 1000);
-                                       applog(LOG_NOTICE, "Total: %s kH/s", s);
+                                       applog(LOG_NOTICE, "[" CL_LBL "%s" CL_N "] Total: %s kH/s", local_hostname, s);
                                        break;
                                }
                                global_hashrate = (uint64_t) hashrate;
@@ -3338,6 +3344,8 @@ int main(int argc, char *argv[]) {
        long flags;
        int i, err;

+  gethostname(local_hostname, MAX_HOST_LEN*sizeof(char));
+
        pthread_mutex_init(&applog_lock, NULL);

        show_credits();
\end{lstlisting}

Ohne Anpassung sieht die Ausgabe wie folgt aus:
\begin{lstlisting}
[2018-05-19 14:09:03] CPU #3: 1.86 H/s
[2018-05-19 14:09:03] CPU #2: 1.86 H/s
[2018-05-19 14:09:03] CPU #3: 1.86 H/s
[2018-05-19 14:09:03] CPU #1: 1.86 H/s
[2018-05-19 14:09:03] CPU #3: 1.86 H/s
[2018-05-19 14:09:03] CPU #0: 1.86 H/s
[2018-05-19 14:09:03] CPU #2: 1.86 H/s
[2018-05-19 14:09:03] CPU #2: 1.85 H/s
[2018-05-19 14:09:03] CPU #2: 1.86 H/s
[2018-05-19 14:09:03] CPU #3: 1.85 H/s
[2018-05-19 14:09:03] CPU #3: 1.86 H/s
[2018-05-19 14:09:03] CPU #0: 1.85 H/s
[2018-05-19 14:09:03] CPU #0: 1.85 H/s
[2018-05-19 14:09:03] CPU #0: 1.85 H/s
[2018-05-19 14:09:03] CPU #2: 1.85 H/s
[2018-05-19 14:09:03] CPU #1: 1.85 H/s
[2018-05-19 14:09:03] CPU #2: 1.85 H/s
\end{lstlisting}
Durch die Modifikation werden die Compute Node Namen ausgegeben:
\begin{lstlisting}
[2018-05-19 14:06:33] [c26] CPU #2: 1.86 H/s
[2018-05-19 14:06:33] [c14] CPU #1: 1.86 H/s
[2018-05-19 14:06:33] [c1] CPU #1: 1.86 H/s
[2018-05-19 14:06:33] [c30] CPU #1: 1.85 H/s
[2018-05-19 14:06:33] [c9] CPU #3: 1.86 H/s
[2018-05-19 14:06:33] [c21] CPU #1: 1.86 H/s
[2018-05-19 14:06:33] [c9] CPU #2: 1.85 H/s
[2018-05-19 14:06:33] [c21] CPU #2: 1.86 H/s
[2018-05-19 14:06:33] [c14] CPU #3: 1.85 H/s
[2018-05-19 14:06:33] [c1] CPU #2: 1.85 H/s
[2018-05-19 14:06:33] [c39] CPU #2: 1.85 H/s
[2018-05-19 14:06:33] [c39] CPU #0: 1.85 H/s
[2018-05-19 14:06:33] [c13] CPU #2: 1.84 H/s
[2018-05-19 14:06:33] [c13] CPU #0: 1.84 H/s
[2018-05-19 14:06:33] [c4] CPU #3: 1.85 H/s
\end{lstlisting}
Die geklonte Version kann nun kompiliert werden. Dabei kann die Warnung: \grqq Implizite Deklaration der Funktion\grqq \ ignoriert werden. Zur Übersicht wurde die Ausgabe unten gekürzt:
\begin{lstlisting}
[root@nebula miners]# cd tkinjo/
[root@nebula tkinjo]#./build.sh
make: *** Keine Regel, um >>clean<< zu erstellen.  Schluss.
clean
configure.ac:15: installing './compile'
configure.ac:4: installing './config.guess'
configure.ac:4: installing './config.sub'
configure.ac:9: installing './install-sh'
configure.ac:9: installing './missing'
Makefile.am: installing './depcomp'
checking build system type... aarch64-unknown-linux-gnu
checking host system type... aarch64-unknown-linux-gnu
checking target system type... aarch64-unknown-linux-gnu
checking for a BSD-compatible install... /usr/bin/install -c
checking whether build environment is sane... yes
checking for a thread-safe mkdir -p... /usr/bin/mkdir -p
In file included from algo/../scryptjane/scrypt-jane-romix.h:2:0,
                 from algo/scrypt-jane.c:24:
algo/../scryptjane/scrypt-jane-chacha.h: In Funktion >>scrypt_test_mix<<:
algo/../scryptjane/scrypt-jane-chacha.h:125:20: Warnung: Implizite Deklaration der Funktion >>detect_cpu<< [-Wimplicit-function-declaration]
  size_t cpuflags = detect_cpu();
                    ^~~~~~~~~~
make[2]: Leaving directory `/opt/miners/tkinjo'
make[1]: Leaving directory `/opt/miners/tkinjo'
\end{lstlisting}
 
