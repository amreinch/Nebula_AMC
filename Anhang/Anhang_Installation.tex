% !TEX root = ../Diplombericht.tex

\section{Installation Managementnode}
Die Serverinstallation des Managementnodes wird nach dem offiziellen OpenHPC Guide mit einigen Abweichungen durchgeführt. Die Installation basiert auf der Anleitung CentOS 7.4 aarch64 Install guide with Warewulf + Slurm. 

\subsection{Variablen Definition}
Durch die Installation hinweg werden die folgenden Variablen verwendet.



\begin{longtable}{| p{0.8cm} | p{3cm} | p{7cm} | p{5.2cm} |} 
\hline
\rowcolor{heading} \textbf{Nr.} & \textbf{Variable} & \textbf{Beschreibung} &\textbf{Werte} \\\hline
1 & sms\_name & Hostname des Managementhosts & nebula \\\hline
2 & sms\_ip & IP Adresse des Managementhosts & 192.168.1.10 \\\hline
3 & sms\_eth\_internal & Ethernet Interface & eth0 \\\hline
4 & internal\_netmask & Netzmaske des Ethernet Interfaces & 255.255.255.0 \\\hline
5 & ntp\_server[0] \newline ntp\_server[1] \newline ntp\_server[2] \newline ntp\_server[3] & Zeitserver (Array) & server 0.ch.pool.ntp.org \newline server 1.ch.pool.ntp.org \newline server 2.ch.pool.ntp.org \newline server 3.ch.pool.ntp.org  \\\hline
6 & num\_computes & Anzahl Computenodes & 45 \\\hline
7 &  c\_ip[0] \newline  c\_ip[1] \newline c\_ip[2] \newline c\_ip[3] \newline c\_ip[4] \newline c\_ip[5] \newline c\_ip[6] \newline c\_ip[7] \newline c\_ip[8] \newline c\_ip[9] \newline c\_ip[10] \newline c\_ip[11] \newline c\_ip[12] \newline c\_ip[13] \newline c\_ip[14] \newline c\_ip[15] \newline c\_ip[16] \newline c\_ip[17] \newline c\_ip[18] \newline c\_ip[19] & IP Adressen der Computenodes (Array) & 192.168.1.11 \newline 192.168.1.12 \newline 192.168.1.13 \newline 192.168.1.14 \newline 192.168.1.15 \newline 192.168.1.16 \newline 192.168.1.17 \newline 192.168.1.18 \newline 192.168.1.19 \newline 192.168.1.20 \newline 192.168.1.21 \newline 192.168.1.22 \newline 192.168.1.23 \newline 192.168.1.24 \newline 192.168.1.25 \newline 192.168.1.26 \newline 192.168.1.27 \newline 192.168.1.28 \newline 192.168.1.29 \newline 192.168.1.30 \\\hline 
\rowcolor{heading} \textbf{Nr.} & \textbf{Variable} & \textbf{Beschreibung} &\textbf{Werte} \\\hline
7 & c\_ip[20] \newline c\_ip[21] \newline c\_ip[22] \newline c\_ip[23] \newline c\_ip[24]  \newline c\_ip[25] \newline c\_ip[26] \newline c\_ip[27] \newline c\_ip[28] \newline c\_ip[29]  \newline c\_ip[30] \newline c\_ip[31] \newline c\_ip[32] \newline c\_ip[33] \newline c\_ip[34] \newline c\_ip[35] \newline c\_ip[36] \newline c\_ip[37] \newline c\_ip[38] \newline c\_ip[39] \newline c\_ip[40] \newline c\_ip[41] \newline c\_ip[42] \newline c\_ip[43] \newline c\_ip[44] & IP Adressen der Computenodes (Array) & 192.168.1.31 \newline  192.168.1.32 \newline  192.168.1.33 \newline 192.168.1.34 \newline 192.168.1.35 \newline 192.168.1.36 \newline 192.168.1.37 \newline 192.168.1.38 \newline 192.168.1.39 \newline 192.168.1.40 \newline 192.168.1.41 \newline 192.168.1.42 \newline 192.168.1.43 \newline 192.168.1.44 \newline 192.168.1.45 \newline 192.168.1.46 \newline 192.168.1.47 \newline 192.168.1.48 \newline 192.168.1.49 \newline 192.168.1.50 \newline 192.168.1.51 \newline 192.168.1.52 \newline 192.168.1.53 \newline 192.168.1.54 \newline 192.168.1.55 \\\hline 
\caption{Variablen Definition}
\end{longtable}